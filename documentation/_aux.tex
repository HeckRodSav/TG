\MexerDepois{Alterar}

O objetivo principal do projeto é a concepção de um sistema de localização relativa para um foguete de sondagem atmosférica.
Este sistema servirá para guiar um grupo de busca em campo na tarefa de localizar o veículo após o aterrizagem.


O direcionamento da busca seria feita utilizando o método de ângulo de chegada relativo ao sinal oriundo da telemetria do foguete.
O sinal será recebido por uma matriz de antenas, viabilizando os cálculos de ângulo.


O objetivo secundário do projeto é comparar a performance deste sistema com a performance de outro sistema de localização relativa, que utiliza coordenadas geográficas para os cálculos.
Este outro sistema se baseia em cálculos de azimute entre as coordenadas do veículo e do grupo de busca.

Por se tratar de um sistema com dados processados no veículo, se houverem problemas internos com o GPS de bordo, estes cálculos são inviabilizados.
Outro problema desde sistema é relativo à precisão da coordenada geográfica, fazendo com que, à certa distância, a precisão da direção perca sua confiabilidade.

Por outro lado, a versão utilizando o angulo de chegada não depende dos dados transmitidos no sinal, bem como poderá funcionar bem a distâncias mais curtas.






% , um sistema que "aponte" a direção a se seguir, e comparar a performance deste com um sistema que apontaria na mesma direção, mas baseado em cálculos de Azimuth relativo usando duas coordenadas de GPS, do foguete e da equipe de busca.
% Eu já construí esse sistema baseado em GPS, ele funciona até que bem, mas a taxa de atualização do GPS que eu tenho é bem ruim lenta, devo usar isso como parte da minha justificativa pra concepção do projeto.