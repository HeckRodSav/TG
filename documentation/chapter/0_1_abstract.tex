Upon launch, an atmospheric sounding rocket can land anywhere within its range, and conducting the search can be a major challenge without an effective location strategy.
Many of these vehicles rely on \acs{GNSS} location, such as \acs{GPS}, but they still rely on a telemetry system to ensure the correct transmission of geographic coordinates to the search team.
This work considers a scenario where the location information could not be decoded, but the \acs{RF} signal is still detectable, proposing the Angle of Arrival (\acs{AoA}) measurement to determine the search direction.
A simulation was built for a system that, based on the phase shift difference in a circular antenna array, is capable of determining the \acs{AoA} of the incident \ac{RF} signal.
During development, compatibility with various numerical resolution software programs was sought, particularly GNU Octave and MATLAB, which proved challenging given the limitations of using free software.
Antenna arrays with three, five, and seven antennas were considered. The geometry with fewer antennas showed lower overall accuracy compared to the others.
The simulations considered cases with different levels of \acs{AWGN} noise interference and considered scenarios with and without signal attenuation.
In all simulations, the R\textsuperscript{2} value was above \qty{75}{\percent} and, on average, above \qty{92}{\percent}.
These results indicate that the proposed method is effective in different contexts.

\paragraph*{\textit{Keywords:}} Angle of Arrival; Radio Frequency; Atmospheric Sounding Rockets; Numerical Resolution; Telemetry; Location.
