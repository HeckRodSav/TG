% \MexerDepois{Alterar}

O objetivo principal do projeto é a concepção de um sistema de localização relativa para um foguete de sondagem atmosférica.
Este sistema servirá para guiar um grupo de busca em campo na tarefa de localizar o veículo após o aterrizagem.


O direcionamento da busca seria feita utilizando o método de ângulo de chegada relativo ao sinal oriundo da telemetria do foguete.
O sinal será recebido por uma matriz de antenas, viabilizando os cálculos de ângulo.


O objetivo secundário do projeto é comparar a performance deste sistema com a performance de outro sistema de localização relativa, que utiliza coordenadas geográficas para os cálculos.
Este outro sistema se baseia em cálculos de azimute entre as coordenadas do veículo e do grupo de busca.

Por se tratar de um sistema com dados processados no veículo, se houverem problemas internos com o GPS de bordo, estes cálculos são inviabilizados.
Outro problema desde sistema é relativo à precisão da coordenada geográfica, fazendo com que, à certa distância, a precisão da direção perca sua confiabilidade.

Por outro lado, a versão utilizando o angulo de chegada não depende dos dados transmitidos no sinal, bem como poderá funcionar bem a distâncias mais curtas.






% , um sistema que "aponte" a direção a se seguir, e comparar a performance deste com um sistema que apontaria na mesma direção, mas baseado em cálculos de Azimuth relativo usando duas coordenadas de GPS, do foguete e da equipe de busca.
% Eu já construí esse sistema baseado em GPS, ele funciona até que bem, mas a taxa de atualização do GPS que eu tenho é bem ruim lenta, devo usar isso como parte da minha justificativa pra concepção do projeto.

Ao ser lançado, um foguete de sondagem atmosférica pode pousar em qualquer lugar dentro do seu raio de alcance, e realizar a busca pode se tornar um grande desafio sem uma estratégia de localização eficaz.
Muitos desses veículos contam com localização por \acs{GNSS}, como o \acs{GPS}, porém ainda dependem de um sistema de telemetria que garanta a correta transmissão das coordenadas geográficas à equipe de busca.
O presente trabalho considera o cenário onde a informação de localização não pôde ser decodificada, porém o sinal de \acs{RF} ainda é detectável, propondo a aferição de \acf{AoA} para determinar a direção de busca.
Foi construída a simulação de um sistema que, baseado na diferença de defasagens em uma malha circular de antenas, é capaz de determinar o \acs{AoA} do sinal \acs{RF} incidente.
Durante o desenvolvimento, foi almejada a compatibilidade com diferentes \textit{softwares} de resolução numérica, particularmente o GNU Octave e o MATLAB, o que se mostrou um desafio, considerando as limitações no uso de \textit{software} livre.
Foram consideradas malhas de antenas com três, cinco e sete antenas.
A geometria com menos antenas apresentou uma acurácia geral mais baixa comparada às demais.
As simulações contemplaram casos com diferentes níveis de interferência por ruído do tipo \acs{AWGN}, e consideraram cenários com e sem atenuação no sinal.
Em todas as simulações, o valor de R\textsuperscript{2} foi acima de \qty{75}{\percent} e, em média, acima de \qty{92}{\percent}.
Esses resultados indicam que o método proposto se mostra eficaz em diferentes contextos.

\paragraph*{Palavras-chave:} \textit{Angle of Arrival}; Radiofrequência; Foguetes de Sondagem Atmosférica; Resolução Numérica; Telemetria; Localização.

