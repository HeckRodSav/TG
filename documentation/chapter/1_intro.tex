\chapter{Introdução}

\section{Motivação}

% foguetes de sondagem, o que são? onde vivem?
% como achar depois?
% Um GNSS simples serve?
% Opção de dois dispositivos GPS, mas funciona sempre?
% Proposta de sistema baseado em AoA


Foguetes de sondagem atmosférica são veículos aeroespaciais sub-orbitais utilizados para levar sensores e experimentos científicos a altos níveis atmosféricos, com o intuito de realizar estudos e análises relacionados às diversas condições ali presentes \cite{isro}.
Estes veículos geralmente utilizam motor de propelente sólido, de um ou dois estágios, e são equipados com sistemas de controle, telemetria e recuperação, além de transportarem o experimento científico, denominado carga-paga \cite{esa, sabbatini2014esa}.
Algumas das vantagens desses veículos são o baixo custo e a menor necessidade de alcance pra sistemas de telemetria e rastreio, tendo em vista que não entram em órbita \cite{nasa}.

No contexto de foguetes de sondagem, existem competições que fomentam o desenvolvimento e competitividade em equipes universitárias de foguetemodelismo \cite{esra}.
Algumas dessas competições tem grande parte de suas categorias definidas nas bases de foguetes de sondagem, com apogeu de voo entre \SI{1}{\kilo\metre} e \SI{10}{\kilo\metre} de altura acima do nível do solo.
Nestes casos, a sequência de operações normal do foguete, apresentada na \autoref{fig:conops}, consiste em: ignição do primeiro estágio do motor, decolagem, período propulsionado, término de queima do primeiro estágio, desacoplamento do primeiro estágio, ignição do segundo estágio, segundo período propulsionado, término de queima do segundo estágio, início do período inercial balístico, apogeu, detecção do apogeu pelos sistemas embarcados e liberação do paraquedas piloto, liberação do paraquedas principal a certa altitude e finalmente o pouso \cite{esa, sabbatini2014esa}.
Desacoplamento do primeiro estágio, ignição e fase propulsionada do segundo somente se aplicam a foguetes de dois estágios.

\begin{figure}[h]
    \centering
    \caption{Caption}
    % \begin{center}
\begin{tikzpicture}

    % Variables


    \coordinate (bottomleft) at (-0.5,-0.5);
    \coordinate (topright) at (10.5,5);

    \def\coordref[#1](#2){%

        \coordinate(sysref) at (#2);

        \draw[#1, -latex] (sysref) ++(-0.4,-0.3) -- ++(0.9,0) node[midway, below]{$x$};
        \draw[#1, -latex] (sysref) ++(-0.3,-0.4) -- ++(0,0.9) node[midway, left]{$y$};
        \draw[#1, -latex] \centerarc(sysref)(-90:180:0.25);
        \draw[#1] (sysref) node{$+$}
    }

    % \draw[Red,dashed] (bottomleft) rectangle (topright);
    \clip (bottomleft) rectangle (topright);

    \coordinate (O) at (0,0);

    % \draw [help lines, dashed] (bottomleft) grid (topright); % desenha grid
    % \draw [red] (O) node[draw,cross out] {}; % marca pont(0,0)

    % \draw[Red] (4,2)
    %     node[draw, thick, shape=foguete, fill=cmyk_R!25, scale=4] {}
    %     node[draw, circle, inner sep=2pt] {}
    % ;

    % \shade[inner color=yellow,outer color=white] (6,0) rectangle +(2,1);

    % \draw
    %     (-0.5,0) -- (10.5,0)
    % ;


    \coordinate (left) at (-0.5,0);
    \coordinate (right) at (10.5,0);
    \coordinate (middle) at ($(left)!0.5!(right)$);
    \draw[thin, Black!50, path fading=west] (left) -- (middle);
    \draw[thin, Black!50, path fading=east] (middle) -- (right);
    % \shade[draw, inner color=Green!10,outer color=Red] (-0.5,0) -- (10.5,3);


    \draw[cmyk_K] (0,0) node[draw, shape=foguete, fill=cmyk_R, anchor=south, rotate=0] {};

    \draw[cmyk_K] (0.5,2) node[OrangeRed, rotate=170, anchor=south, inner sep=-2.5pt, xshift=0.25pt] {\Fire} node[draw, shape=foguete, fill=cmyk_R, anchor=south, rotate=-10] {};

    \draw[cmyk_K] (5,4) node[draw, shape=foguete, fill=cmyk_R, anchor=south, rotate=-95] {};

    \draw[cmyk_K] (9,0) node[draw, shape=foguete, fill=cmyk_R, anchor=east, rotate=-90] {};



    \begin{pgfonlayer}{background}    % select the background layer
        \clip (bottomleft) rectangle (10.5,0);
        \shade[inner color=SaddleBrown!50,outer color=White] (bottomleft) rectangle (10.5,0.5);
    \end{pgfonlayer}


\end{tikzpicture}
% \end{center}
    % \includegraphics{}
    \caption*{Fonte: Autor}
    \label{fig:conops}
\end{figure}

A partir do momento do pouso, o próximo objetivo nessas competições consiste em localizar o foguete, vários métodos podem ser empregados nessa situação, desde cores chamativas no veículo e paraquedas, até sinais sonoros.
Essas competições geralmente recomendam, e até exigem, a presença de um \ac{GNSS}, capaz de transmitir as coordenadas do veículo após o pouso para localização, como um \ac{GPS} \cite{irec}.

O processo de localização baseada em dados simples de \ac{GNSS}, latitude e longitude, pode se tornar mais complicado se o grupo de busca não tem certeza de como encontrar essas coordenadas.
Existem dispositivos de \ac{GNSS} portáteis, porém estes podem criar dificuldades na interface com os dados recebidos da telemetria do foguete.
Neste caso, seria possível desenvolver um dispositivo capaz de lidar diretamente com as informações de localização fornecidas pela telemetria e guiar o grupo de buscar na direção correta.

Os dados recebidos da telemetria ainda precisam de certo grau de confiabilidade para que sejam devidamente processados e tratados, o que pode ser um problema se o veículo está longe do grupo de busca ou o dispositivo de \ac{GNSS} a bordo não esteja apto a fornecer dados corretamente.
Nesse caso, ainda é possível buscar o foguete utilizando o próprio sinal da telemetria, independente dos dados transmitidos.







\section{Objetivos}

O principal objetivo deste trabalho é desenvolver e projetar um dispositivo portátil capaz de indicar a direção da origem de um sinal de \ac{RF} baseado em métodos de detecção de \ac{AoA}.

Como objetivo secundário, a análise comparativa com um sistema de utilidade semelhante, porém baseado inteiramente em coordenadas de \ac{GNSS}.

\section{Estrutura do documento}

\MexerDepois{Completar isso depois}
