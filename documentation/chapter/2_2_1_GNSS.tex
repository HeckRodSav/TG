\subsection{Direcionamento por coordenadas geográficas}\label{ssec:gnss}

Coordenadas geográficas são definidas por dois valores, latitude e longitude, associadas a coordenadas esféricas referenciadas a partir do centro da terra, assumindo o raio da coordenada como o raio médio da superfície do planeta, cerca de $R_\text{Terra} = \SI{6371E3}{\metre}$ \cite{palomaguitarrara, chrisveness}.
A latitude equivale à componente polar $\phi$ centralizada na linha do equador, enquanto a longitude equivale à componente $\theta$ centralizada no meridiano de Greenwich \cite{palomaguitarrara, henriquefleming2003}.

Conhecendo as coordenadas de dois pontos distintos A e B, é possível determinar seu ângulo de \textit{bearing} \ac{betab} relativo, referente ao norte, ou seja, o ângulo da direção a se seguir partindo do ponto A para chegar ao ponto B, a partir da direção norte no ponto de origem A \cite{henriquefleming2003}.

Sendo \textcolor{Green}{\ac{Ag}} e \textcolor{Blue}{\ac{Bg}} duas coordenadas geográficas, \textcolor{Green}{\ac{phiA}} e \textcolor{Blue}{\ac{phiB}} suas respectivas latitudes, e \textcolor{Green}{\ac{thetaA}} e \textcolor{Blue}{\ac{thetaB}} suas respectivas longitudes, conforme ilustrado na \autoref{fig:globo}.

\begin{figure}[htbp]
    \centering
    \caption{Representação geométrica de distância e ângulo em relação ao norte entre coordenadas geográficas \textcolor{Green}{\ac{Ag}} e \textcolor{Blue}{\ac{Bg}}.}
    \begin{tikzpicture}[scale=1,tdplot_main_coords,every node/.style={font = {\footnotesize\bfseries}}]
% https://latex.org/forum/viewtopic.php?t=25316

\def\rvec{3.5}

\def\phivecA{30}
\def\thetavecA{75}

\def\phivecB{45}
\def\thetavecB{15}

\def\auxOpacity{0.75}

\pgfmathsetmacro{\ax}{\rvec*sin(\phivecA)*cos(\thetavecA)}
\pgfmathsetmacro{\ay}{\rvec*sin(\phivecA)*sin(\thetavecA)}
\pgfmathsetmacro{\az}{\rvec*cos(\phivecA)}

\pgfmathsetmacro{\bx}{\rvec*sin(\phivecB)*cos(\thetavecB)}
\pgfmathsetmacro{\by}{\rvec*sin(\phivecB)*sin(\thetavecB)}
\pgfmathsetmacro{\bz}{\rvec*cos(\phivecB)}


\shadedraw[tdplot_screen_coords, ball color = white, opacity=0.25] (0,0) circle (\rvec);

%-----------------------
\coordinate (O) at (0,0,0);

\tdplotsetcoord{A}{\rvec}{\phivecA}{\thetavecA}

\tdplotsetcoord{B}{\rvec}{\phivecB}{\thetavecB}

\tdplotsetcoord{N}{\rvec}{0}{90}

%draw the main coordinate system axes
\draw[thick,-latex] (O) -- (1.35*\rvec,0,0) node[anchor=north east]{$x$};
\draw[thick,-latex] (O) -- (0,1.35*\rvec,0) node[anchor=north west]{$y$};
\draw[thick,-latex] (O) -- (0,0,1.35*\rvec) node[anchor=south]{$z$};

\draw[thick,-latex,opacity=0] (O) -- (-1.35*\rvec,0,0) node[anchor=south west]{$x$};
\draw[thick,-latex,opacity=0] (O) -- (0,-1.35*\rvec,0) node[anchor=south east]{$y$};
\draw[thick,-latex,opacity=0] (O) -- (0,0,-1.35*\rvec) node[anchor=north]{$z$};


\draw[-latex,very thick,opacity=\auxOpacity, color=cmyk_G] (O) -- (A) node[anchor=south west, opacity=1] {$\mathbf{A}$};
\draw[-latex,very thick,opacity=\auxOpacity, color=cmyk_B] (O) -- (B) node[anchor=south east, opacity=1] {$\mathbf{B}$};
\draw[very thick,color=Black] (N) node[anchor=west] {$\mathbf{N}$};


\draw[dashed, opacity=0.15] (\rvec,0,0) arc (0:360:\rvec);
\draw[thin] (\rvec,0,0) arc (0:135:\rvec);
\draw[thin] (\rvec,0,0) arc (0:-45:\rvec);

\draw[dashed, color=cmyk_G, opacity=\auxOpacity] (O) -- (Axy) -- (A);
\draw[dashed, color=cmyk_B, opacity=\auxOpacity] (O) -- (Bxy) -- (B);

\draw[dotted, color=cmyk_G, opacity=\auxOpacity] (Ax) -- (Axy);
\draw[dotted, color=cmyk_G, opacity=\auxOpacity] (Ay) -- (Axy);

\draw[dotted, color=cmyk_B, opacity=\auxOpacity] (Bx) -- (Bxy);
\draw[dotted, color=cmyk_B, opacity=\auxOpacity] (By) -- (Bxy);

% \pause


\tdplotdrawarc[color=cmyk_G, opacity=\auxOpacity]{(O)}{0.75*\rvec}{0}{\thetavecA}{anchor=north}{$\theta_A$}

\tdplotdrawarc[color=cmyk_B, opacity=\auxOpacity]{(O)}{0.5*\rvec}{0}{\thetavecB}{anchor=north east}{$\theta_B$}

% \pause


\tdplotsetthetaplanecoords{\thetavecA}
\tdplotdrawarc[color=cmyk_G, opacity=\auxOpacity, tdplot_rotated_coords]{(O)}{0.75*\rvec}{90}{\phivecA}{anchor=west}{$\phi_A$}

\tdplotsetthetaplanecoords{\thetavecB}
\tdplotdrawarc[color=cmyk_B, opacity=\auxOpacity, tdplot_rotated_coords]{(O)}{0.5*\rvec}{90}{\phivecB}{anchor=east}{$\phi_B$}

% \pause

\tdplotsetthetaplanecoords{\thetavecB}
\draw[dashed, cmyk_B, opacity=0.15, tdplot_rotated_coords] (\rvec,0,0) arc (180:-40:-\rvec);
\draw[thin, cmyk_B, opacity=0.25, tdplot_rotated_coords] (\rvec,0,0) arc (0:140:\rvec);
\draw[thin, cmyk_B, opacity=0.25, tdplot_rotated_coords] (\rvec,0,0) arc (360:320:\rvec);
\draw[very thick, color=cmyk_B, tdplot_rotated_coords] (\rvec,0,0) arc (0:\phivecB:\rvec);

\tdplotsetthetaplanecoords{\thetavecA}
\draw[dashed, cmyk_G, opacity=0.15, tdplot_rotated_coords] (\rvec,0,0) arc (180:-40:-\rvec);
\draw[thin, cmyk_G, opacity=0.25, tdplot_rotated_coords] (\rvec,0,0) arc (0:140:\rvec);
\draw[thin, cmyk_G, opacity=0.25, tdplot_rotated_coords] (\rvec,0,0) arc (360:320:\rvec);
\draw[very thick, color=cmyk_G, tdplot_rotated_coords] (\rvec,0,0) arc (0:\phivecA:\rvec);


\drawArc{\ax}{\ay}{\az}{\bx}{\by}{\bz}{\rvec}{anchor=south}{$d$}

\tdplotsetrotatedcoords{\phivecA}{\thetavecA}{0}

\def\centerarc[#1](#2)(#3:#4:#5)% Syntax: [draw options] (center) (initial angle:final angle:radius)
    { \draw[#1] ($(#2)+({#5*cos(#3)},{#5*sin(#3)})$) arc (#3:#4:#5) node[midway,anchor=east] {$\beta$}; }

\centerarc[cmyk_R, tdplot_rotated_coords](A)(292:214:0.4)

\end{tikzpicture}
    % \includegraphics{../pictures/globo.pdf}
    \caption*{Fonte: Autor.}
    \label{fig:globo}
\end{figure}

Calculam-se \ac{Deltaphi} e \ac{Deltatheta} conforme Equações \ref{eq:Deltaphi} e \ref{eq:Deltatheta}, respectivamente.

    \begin{equation}\label{eq:Deltaphi}
        \ac{Deltaphi} = \textcolor{Blue}{\ac{phiB}} - \textcolor{Green}{\ac{phiA}}
    \end{equation}
    \begin{equation}\label{eq:Deltatheta}
        \ac{Deltatheta} = \textcolor{Blue}{\ac{thetaB}} - \textcolor{Green}{\ac{thetaA}}
    \end{equation}

Através da lei dos haversines é possível obter a distância mínima $d$ entre as coordenadas, sobre a superfície, e também o ângulo de \textit{Bearing} \textcolor{Red}{\ac{betab}} formado no vértice \textcolor{Green}{\ac{Ag}} do triângulo esférico $\mathbf{N}\textcolor{Green}{\ac{Ag}}\textcolor{Blue}{\ac{Bg}}$ \cite{chrisveness}.
Para o cálculo de distância, os ângulos devem ser tratados em radianos.

\begin{equation}
    X = \cos\left(\textcolor{Blue}{\ac{thetaB}}\right)\cdot \sin\left(\ac{Deltaphi}\right)
\end{equation}
\begin{equation}
    Y = \cos\left(\textcolor{Green}{\ac{thetaA}}\right)\cdot\sin\left(\textcolor{Blue}{\ac{thetaB}}\right) - \sin\left(\textcolor{Green}{\ac{thetaA}}\right) \cdot \cos\left(\textcolor{Green}{\ac{thetaB}}\right) \cdot \cos\left(\ac{Deltaphi}\right)
\end{equation}
\begin{equation}
    Z = \sin^2\left(\frac{\ac{Deltatheta}}{2}\right) + \cos\left(\textcolor{Blue}{\ac{thetaB}}\right) \cdot \cos\left(\textcolor{Green}{\ac{thetaA}}\right) \cdot \sin^2\left(\frac{\ac{Deltaphi}}{2}\right)
\end{equation}
\begin{equation}
    \textcolor{Red}{\ac{betab}} = \arctan\left(\frac{X}{Y}\right) - \frac{\pi}{2}
\end{equation}
\begin{equation}
    \textcolor{Red}{\ac{dAB}} = R_\text{Terra} \cdot 2 \cdot \arctan\left(\frac{\sqrt{Z}}{\sqrt{1-Z}}\right)
\end{equation}

O ângulo \textcolor{Red}{\ac{betab}} calculado aqui é referente à direção cardeal Norte, assim, uma equipe de busca equipada com uma bússola simples seria capaz de seguir a direção correta.
A \autoref{fig:bearing} apresenta a aplicação desenvolvida por \citeauthor{chrisveness}, capaz de calcular o ângulo de \textit{Bearing} entre duas coordenadas, note que, neste caso, o ângulo referido é relacionado à direção cardinal Leste \cite{chrisveness}.

\begin{figure}[htbp]
    \centering
    \caption{Cálculo do ângulo de \textit{Bearing} \textcolor{Red}{\ac{betab}} entre as coordenadas dos Campi Santo André e São Bernardo do Campo da UFABC.}
    \includegraphics[width=0.7\textwidth]{../pictures/bearing.png}
    \caption*{Fonte: \citeauthor{chrisveness} 2019 \cite{chrisveness}}
    \label{fig:bearing}
\end{figure}