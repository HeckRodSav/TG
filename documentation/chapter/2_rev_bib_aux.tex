\subsection{Estimar \acs{AoA} utilizando malha de antenas}\label{ssec:aoa}

Analisando a defasagem de um sinal de \ac{RF} incidindo em uma malha de antenas, é possível estimar seu \acf{AoA}, ou seja, determinar a direção do emissor do sinal em relação ao sistema.
Este valor é calculado utilizando dados como a distância entre as antenas, o comprimento de onda \ac{lambda} do sinal e a velocidade da luz no meio, usualmente tomada como $\ac{c} = \SI{299792458,6 \pm 0,3}{\metre\per\second}$ no ar \cite{jennings1987continuity, bensky2016wireless, horst2021localization, Schssel2016AngleOA}.
A \autoref{eq:wavelength} apresenta a relação do comprimento de onda \ac{lambda} com a frequência \ac{f}, a frequência angular \ac{omega} e a velocidade da luz \ac{c}.

% Comprimento de onda
\begin{equation} \label{eq:wavelength}
    \ac{lambda} = \frac{\ac{c}}{\ac{f}} = \frac{2\pi \cdot \ac{c}}{\ac{omega}}
\end{equation}

% {Distância mínima}

Se um emissor de sinal estiver suficientemente distante, é possível considerar que a frente de onda tem um comportamento planar, essa característica simplifica as operações envolvidas.
A distância de Fraunhofer (\ac{dFran}) é a mínima para essa condição, ela define o início da região de \textit{far-field}, conforme apresentado na \autoref{eq:fraunhofer}, onde \ac{D} é a maior dimensão da antena emissora \cite{balanis2016antenna}.
Tomando $ \ac{D} = 2 \ac{lambda}$, para uma antena de dipolo, obtém-se $\ac{dFran} = 8 \ac{lambda} $.
A \autoref{fig:plana_0} ilustra o comportamento planar de uma frente de onda, com destaque na distância \ac{dFran}.

\begin{equation} \label{eq:fraunhofer}
    \ac{dFran} = \frac{2 \cdot \ac{D}^2}{\ac{lambda}} \quad \Rightarrow \quad \ac{dFran} = \frac{2 \cdot \left(2 \cdot \ac{lambda} \right)^2}{\ac{lambda}} = 8 \ac{lambda}
\end{equation}

\begin{figure}[htbp]
    \centering
    \caption{Característica de frente de onda a cada \ac{lambda} a partir da antena emissora, destaque para $\ac{dFran} = 8 \ac{lambda}$.}
        % \resizebox{\textwidth}{!}{%
    \begin{circuitikz}[american, voltage shift=0.5, line width=0.5]

        \def\wavelength{0.5}
        \def\d{0.5*\wavelength}

        \def\closeRange{1}
        \def\farRange{\closeRange+30}

        \coordinate (O) at (0,0);
        \coordinate (antenna) at (-\closeRange,0);
        % \draw [help lines, dashed] (-5,-3) grid (5,3); % desenha grid
        % \draw [red] (O) node[draw,cross out] {}; % marca pont(0,0)

        \draw[thick]
            (antenna) node[dinantenna, scale=0.75]{}
        ;

        % \draw (\closeRange-0.5,-4) rectangle (\farRange+0.1,4);
        \clip (-0.75,-1.5) rectangle (12.1,1.5);
        \foreach \x [evaluate={\z=int((\x+\closeRange));}] in {0,...,30} {
            \draw [gray, thin, opacity=0.5] (antenna) circle (\z*\wavelength);
            \draw [black]
            (antenna) ++ (\z*\wavelength,0)
            node[anchor=south, font = {\footnotesize\bfseries}, rotate=-90,scale=0.75]{$\z\lambda$}
            ++(0, -4)
            -- ++(0,8);
        }

        \draw [Red, thick] (antenna) ++ (8*\wavelength,-4) -- ++(0,8);

        % \foreach \x in {0,60,...,300} {
        %     \draw[thick] (\x:1 cm) -- (\x + 60:1 cm);

        %     \draw (\x + 30:1.732 cm) node[Gray, circ]{};
        %     \draw[Gray, dashed] (\x:1 cm) -- ++(\x: 0.9cm);
        %     \draw[Gray, dotted]
        %     %     % (\x:1 cm) arc (\x+240:\x+180:1cm)
        %         (\x:1 cm) arc [start angle=\x+120, delta angle=110, radius=1cm]
        %         (\x:1 cm) arc [start angle=\x+120, delta angle=-50, radius=1cm]
        %     ;
        % }

        % \draw (0,0) node [circ] {} node [below left,font={\scriptsize\bfseries}] {BS};
        % \draw[thick, densely dotted] (0,0) circle (1cm);

        % \draw[-latex] (0,0) -- (0:1cm) node[midway, below] {$R_c$};
        % \draw[-latex] (0,0) -- (90:0.866cm) node[midway, left] {$R$};

    \end{circuitikz}
  % }

    % \includegraphics{../pictures/plana_0.pdf}
    \caption*{Fonte: Autor.}
    \label{fig:plana_0}
\end{figure}

% {Trigonometria}

Tomando agora um par de antenas separadas por uma distância fixa \ac{d}, torna-se viável fazer a análise trigonométrica entre as antenas e a frente de onda incidente, onde essa distância \ac{d} será a hipotenusa do triângulo retângulo formado.
Para realizar esta análise, ainda é necessário conhecer uma segunda dimensão do triângulo retângulo envolvido, esta é obtida da defasagem \ac{DeltaPhi} dos sinais incidentes nas antenas, conforme apresentado na \autoref{eq:defasagem}.
A \autoref{fig:AoA} apresenta quatro casos de chegada do sinal de \ac{RF} em um par de antenas.


% Relacao trigonometrica de defasagem e angulo do sinal
\begin{equation} \label{eq:defasagem}
    \ac{d} \cdot
    \cos\left(\textcolor{Purple}{\ac{betak}}\right) =
    \ac{lambda} \cdot
    \frac{\ac{DeltaPhi}}{2 \pi}
\end{equation}


É importante ressaltar que um sistema com um único par de antenas não é suficiente para determinar completamente o \ac{thetaAoA}, já que o valor calculado de \textcolor{Purple}{\ac{betak}} é igual para casos simétricos em relação ao par de antenas, criando um caso de ambiguidade.
As Figuras \ref{fig:AoA:1} e \ref{fig:AoA:2} apresentam exemplos de diferentes valores de \ac{thetaAoA} para o mesmo valor de \textcolor{Purple}{\ac{betak}}.
Existem ainda dois casos notáveis, onde o sinal chega alinhado ao par de antenas ou perpendicular a elas, apresentados respectivamente nas Figuras \ref{fig:AoA:3} e \ref{fig:AoA:4}.

\begin{figure}
    \caption{Diferentes valores de \ac{thetaAoA} para sinal incidente em par de antenas, sistema com ângulo $\textcolor{cmyk_M}{\ac{alphak}}=\SI{20}{\degree}$ em relação à referência.}
    \label{fig:AoA}

    \hfill
    \begin{subfigure}[b]{0.45\textwidth}
        \centering
        \caption{$ \ac{thetaAoA} = \SI{60}{\degree} $, $ \textcolor{Purple}{\ac{betak}} = \SI{40}{\degree} $}
        % \resizebox{!}{0.7\textheight}{%
\begin{circuitikz}[american, voltage shift=0.5, line width=0.5, every node/.style={font = {\footnotesize\bfseries}}]

    \def\wavelength{3.5}
    \pgfmathsetmacro\d{0.5*\wavelength}

    \def\antennaAngle{20}
    \pgfmathsetmacro\signalAngle{\antennaAngle+40}

    \def\closeRange{9}
    \def\farRange{\closeRange+13}

	\def\NAntennas{3}
	\pgfmathsetmacro\AngleAntennas{360/\NAntennas}
	\def\ShiftAngleAntennas{-90}

	\pgfmathsetmacro\RhoAntennas{\d/(2*sin(180/\NAntennas))}

    \def\centerarc(#1)(#2:#3:#4)% Syntax: [draw options] (center) (initial angle:final angle:radius)
    { ($(#1)+({#4*cos(#2)},{#4*sin(#2)})$) arc (#2:#3:#4) }

    \def\coordref[#1](#2){%

        \coordinate(sysref) at (#2);

        \draw[#1, -latex] (sysref) ++(-0.4,-0.3) -- ++(0.9,0) node[midway, below]{$x$};
        \draw[#1, -latex] (sysref) ++(-0.3,-0.4) -- ++(0,0.9) node[midway, left]{$y$};
        \draw[#1, -latex] \centerarc(sysref)(-90:180:0.25);
        \draw[#1] (sysref) node{$+$}
    }

    \coordinate (bottomleft) at (-3.5,-1);
    \coordinate (topright) at (3.5,5);


    % \draw[Red,dashed] (bottomleft) rectangle (topright);
    \clip (bottomleft) rectangle (topright);

    \coordinate (O) at (0,0);
    \coordinate (sourceAntenna) at (\signalAngle:\closeRange*\wavelength);
    % \draw [help lines, dashed] (bottomleft) grid (topright); % desenha grid
    % \draw [red] (O) node[draw,cross out] {}; % marca pont(0,0)

    % Circulo de antenas
	% \draw[densely dotted, opacity=0.25] (O) ++(90:\RhoAntennas) circle (\RhoAntennas);

    % Linhas do sinal de fundo
    \foreach \x [evaluate={\y=int((\x+\closeRange));\z=int((\x+\closeRange));}] in {-3,...,3} {
        \draw [black!75, very thin]
        (sourceAntenna) ++ (\signalAngle:-\z*\wavelength)
            % node[anchor=west, font = {\footnotesize\bfseries}]{$\y\lambda$}
        ($(sourceAntenna) + (\signalAngle:-\z*\wavelength) + ({10*cos(\signalAngle+90)},{10*sin(\signalAngle+90)})$)
            --
        ($(sourceAntenna) + (\signalAngle:-\z*\wavelength) - ({10*cos(\signalAngle+90)},{10*sin(\signalAngle+90)})$)
        % \draw [gray, thin] (sourceAntenna) circle (\z)
        ;
    }

    % Antenas
    \draw[thick, cmyk_R] (O) node[dinantenna] (A00) {} ;
    % \draw[thick, cmyk_G, opacity=0.75] (O) ++(60:\d) node[dinantenna] (A0d) {} node [below] {$A_{k+2}$};
    \draw[thick, cmyk_B] (O) ++(\antennaAngle:\d) node[dinantenna] (Ad0) {} ;

    \draw[very thin, Black!50, -latex] % Desenha eixo X
        (-3,0) -- (3,0) node[below left] {$x$}
    ;

    % Ângulo alpha entre antenas
    \draw[thin, cmyk_M]
        \centerarc(O)(0:\antennaAngle:0.3)
        node [above, inner sep=3pt] {$\alpha$}
    ;


    % Desenha senoide de fundo
    \draw[Goldenrod, domain=-8:8, samples=100]
        (A00) ++(\signalAngle+90:0.5*\wavelength) coordinate(signalAux)
        plot[shift={(signalAux)}, rotate=\signalAngle]({\x},{cos(\x * pi * 2 / \wavelength r)})
    ;

    % Direção do sinal
    \draw[very thick, dashed, -latex, Goldenrod]
        % (A00) ++(1.5*\d,0) ++ (\signalAngle:-0.5*\d) -- coordinate(angleArrow) ++(\signalAngle:\d)
        (A00) ++(-2,0) ++ (\signalAngle:-0.25*\d) -- coordinate(angleArrow) ++ (\signalAngle:0.5*\d) --++(\signalAngle:0.25*\d)
    ;
    % Angulo Theta do sinal
    \draw[thin]
        (angleArrow) ++ (0.4, 0) node [below,inner sep=2pt] {$\theta_\text{\ac{AoA}}$}
        \centerarc(angleArrow)(0:\signalAngle:0.4)
    ;

    % Triangulo retângulo + quadradinho
    \draw[Black]

        (A00) --++($({\signalAngle-90}:{\d*sin(\signalAngle-\antennaAngle)})$) coordinate (pontoTriangulo) -- (Ad0) -- (A00)

        (pontoTriangulo)
          ++(\signalAngle:0.125)
        --++(\signalAngle+90:0.125)
        --++(\signalAngle+180:0.125)
    ;

    % Arco do angulo beta
    \draw[thin, Purple]
        (Ad0) ++ (180+\antennaAngle:0.4) node[above, inner sep=3pt] {$\beta$}
        \centerarc(Ad0)(180+\antennaAngle:180+\signalAngle:0.4)
    ;

    % Distânci d entre antenas
    \draw[latex-latex]
        ($(A00)+(0,1)$) -- ($(Ad0)+(0,1)$) node [midway, fill=white, circle, inner sep=1pt] {$d$}
    ;

    \newcommand\CircleRadius{3cm}
    %   \draw (0,0) circle (\CircleRadius);
    % special method of noting the position of a point
    \coordinate (P) at (50:\CircleRadius);

\end{circuitikz}
% }


        % \includegraphics{../pictures/AoA_1.pdf}
        \label{fig:AoA:1}
    \end{subfigure}
    \hfill
    \begin{subfigure}[b]{0.45\textwidth}
        \centering
        \caption{$ \ac{thetaAoA} = \SI{-20}{\degree} $, $ \textcolor{Purple}{\ac{betak}} = \SI{40}{\degree} $}
        \input{../pictures/AoA_2}
        % \includegraphics{../pictures/AoA_2.pdf}
        \label{fig:AoA:2}
    \end{subfigure}
    \hfill

    \vspace{\floatsep}

    \hfill
    \begin{subfigure}[b]{0.45\textwidth}
        \centering
        \caption{$ \ac{thetaAoA}=\SI{110}{\degree} $, $ \textcolor{Purple}{\ac{betak}} = \SI{90}{\degree} $}
        \input{../pictures/AoA_3}
        % \includegraphics{../pictures/AoA_3.pdf}
        \label{fig:AoA:3}
    \end{subfigure}
    \hfill
    \begin{subfigure}[b]{0.45\textwidth}
        \centering
        \caption{$ \ac{thetaAoA}=\SI{20}{\degree} $, $ \textcolor{Purple}{\ac{betak}} = \SI{0}{\degree} $}
        \input{../pictures/AoA_4}
        % \includegraphics{../pictures/AoA_4.pdf}
        \label{fig:AoA:4}
    \end{subfigure}
    \hfill

    \caption*{Fonte: Autor.}
\end{figure}

% {Distância entre antenas}

A escolha da distância \ac{d} entre as antenas deve ser feita de forma a otimizar a resolução da medida de defasagem, com a maior distância possível. Porém é necessário evitar ambiguidades na análise, por se tratar de um sinal periódico, o valor se repetirá a cada \ac{lambda}, e terá valores simétricos quando $\ac{d} > \sfrac{\ac{lambda}}{2}$, ilustrado na \autoref{fig:AoA_d:fail}.
Adota-se então $\ac{d}=\sfrac{\ac{lambda}}{2}$, conforme apresentado na \autoref{fig:AoA_d:ok} e \autoref{eq:distancia} \cite{bensky2016wireless, horst2021localization, Schssel2016AngleOA}.

\begin{figure}[htbp]
    \caption{Diferentes valores para \ac{d}.}
    \label{fig:AoA_d}

    \hfill
    \begin{subfigure}[b]{0.45\textwidth}
        \centering
        \caption{$\ac{d} > \sfrac{\ac{lambda}}{2}$}
        \input{../pictures/AoA_0_fail}
        % \includegraphics{../pictures/AoA_0_fail.pdf}
        \label{fig:AoA_d:fail}
    \end{subfigure}
    \hfill
    \begin{subfigure}[b]{0.45\textwidth}
        \centering
        \caption{$\ac{d} = \sfrac{\ac{lambda}}{2}$}
        \input{../pictures/AoA_0}
        % \includegraphics{../pictures/AoA_0.pdf}
        \label{fig:AoA_d:ok}
    \end{subfigure}
    \hfill

    \caption*{Fonte: Autor.}
\end{figure}

\begin{equation} \label{eq:distancia} % Distancia entre par de antenas
    \ac{d} = \frac{\ac{lambda}}{2}
\end{equation}

% % {Ambiguidade simétrica}

Para contornar a ambiguidade de simetria, é possível adicionar mais antenas à malha.
O conjunto de \ac{Nant} antenas deve respeitar a distância \ac{d} entre as antenas de um par, e pode ser disposto como um polígono regular com \ac{Nant} lados de tamanho \ac{d}, onde cada antena está em um vértice.
A \autoref{fig:antennas} apresenta exemplos dessa disposição de antenas, note que valores pares de \ac{Nant} implicam que existirão pares de antenas paralelos, que resultam em leituras redundantes.

\begin{figure}[htbp]
    \centering
    \caption{Diferentes distribuições de antenas.}
    \label{fig:antennas}

    \hfill
    \begin{subfigure}[c]{0.4\textwidth}
        \centering
        \caption{$ \ac{Nant} = 3 $.}
        % \input{../pictures/antennas_3}
        \includegraphics{../pictures/antennas_3.pdf}
        \label{fig:antennas:3}
    \end{subfigure}
    \hfill
    \begin{subfigure}[c]{0.575\textwidth}
        \centering
        \caption{$ \ac{Nant} = 6 $.}
        % % \resizebox{!}{0.7\textheight}{%
\begin{circuitikz}[american, voltage shift=0.5, line width=0.5,every node/.style={font = {\footnotesize\bfseries}}]

    \def\wavelength{8}
    \pgfmathsetmacro\d{0.5*\wavelength}

    \def\signalAngle{75}
    \def\antennaAngle{120}

    \def\closeRange{9}
    \def\farRange{\closeRange+13}

	\def\NAntennas{6}
	\pgfmathsetmacro\AngleAntennas{360/\NAntennas}
	\pgfmathsetmacro\ShiftAngleAntennas{-90+1.5*\AngleAntennas}

	\pgfmathsetmacro\RhoAntennas{\d/(2*sin(180/\NAntennas))}

    \def\centerarc(#1)(#2:#3:#4)% Syntax: [draw options] (center) (initial angle:final angle:radius)
    { ($(#1)+({#4*cos(#2)},{#4*sin(#2)})$) arc (#2:#3:#4) }

    \def\coordref[#1](#2){%

        \coordinate(sysref) at (#2);

        \draw[#1, -latex] (sysref) ++(-0.4,-0.3) -- ++(0.9,0) node[midway, below]{$x$};
        \draw[#1, -latex] (sysref) ++(-0.3,-0.4) -- ++(0,0.9) node[midway, left]{$y$};
        \draw[#1, -latex] \centerarc(sysref)(-90:180:0.25);
        \draw[#1] (sysref) node{$+$}
    }

    \coordinate (bottomleft) at (-4.5,-4.5);
    \coordinate (topright) at (4.5,4.5);


    % \draw[Red,dashed] (bottomleft) rectangle (topright);
    \clip (bottomleft) rectangle (topright);

    \coordinate (O) at (0,0);
    \coordinate (sourceAntenna) at (\signalAngle:\closeRange*\wavelength);
    % \draw [help lines, dashed] (bottomleft) grid (topright); % desenha grid
    % \draw [red] (O) node[draw,cross out] {}; % marca pont(0,0)

	\draw[densely dotted] (O) circle (\RhoAntennas);

    \draw[thick, cmyk_G] (O) ++(1*\AngleAntennas+\ShiftAngleAntennas:\RhoAntennas) node[dinantenna] (A1) {} node [above right] {$A_{1}$};
    \draw[thick, cmyk_B] (O) ++(2*\AngleAntennas+\ShiftAngleAntennas:\RhoAntennas) node[dinantenna] (A2) {} node [above left] {$A_{2}$};
    \draw[thick, cmyk_R] (O) ++(3*\AngleAntennas+\ShiftAngleAntennas:\RhoAntennas) node[dinantenna] (A3) {};
    \draw[thick, cmyk_C] (O) ++(4*\AngleAntennas+\ShiftAngleAntennas:\RhoAntennas) node[dinantenna] (A4) {};
    \draw[thick, cmyk_M] (O) ++(5*\AngleAntennas+\ShiftAngleAntennas:\RhoAntennas) node[dinantenna] (A5) {};
    \draw[thick, DarkGoldenrod] (O) ++(6*\AngleAntennas+\ShiftAngleAntennas:\RhoAntennas) node[dinantenna] (A6) {};

	\coordinate (A1_2) at ($(A1)!0.5!(A2)$);

	\draw[Black!25, dotted]
		(A1) --
		(A6) --
		(A5) --
		(A4) --
		(A3) --
		(A2)
	;

	\draw[Black!50, densely dotted]
		(O) --
		(A2) --
		(A1_2)
	;

	\draw
		(A1) --
		(O) --
		(A1_2) --
		(A1)
	;

	\draw
         (A1_2)
           ++(0:0.125)
         --++(-90:0.125)
         --++(+180:0.125)
	;

	\node at (O) {\tiny\textbullet};

	\draw
		(O) ++(90:0.3) node[left, inner sep=1.5pt] {$\textstyle \frac{\pi}{N_\text{ant}}$}
		\centerarc(O)(1*\AngleAntennas+\ShiftAngleAntennas:90:0.3)
	;


    % Distânci d entre antenas
    \draw[latex-latex]
        ($(A1)+(0,1)$) -- ($(A2)+(0,1)$) node [midway, fill=white, circle, inner sep=1pt] {$d$}
    ;

    \draw[decorate, decoration={brace, amplitude=5pt}, thin]
    ($(A1)+({1*\AngleAntennas+\ShiftAngleAntennas-90}:0.1)$)
    -- coordinate (brace)
    ($(O)+({1*\AngleAntennas+\ShiftAngleAntennas-90}:0.1)$)
    ;

    \draw (brace) ++({1*\AngleAntennas+\ShiftAngleAntennas-90}:5pt)
        node[anchor=north west, circle, fill=white, inner sep=1pt] {$\rho$}
    ;

	\draw[decorate, decoration={brace, amplitude=5pt}, thin]
    ($(A1_2)+({90}:0.1)$)
    -- coordinate (brace)
    ($(A1)+({90}:0.1)$)
    ;

    \draw (brace) ++({90}:5pt)
        node[anchor=south, circle, inner sep=1pt] {$\sfrac{d}{2}$}
    ;

    % \coordref[Black!25](3.5,0);

\end{circuitikz}
% }


        \includegraphics{../pictures/antennas_6.pdf}
        \label{fig:antennas:6}
    \end{subfigure}
    \hfill

    \vspace{\floatsep}

    \hfill
    \begin{subfigure}[c]{0.425\textwidth}
        \centering
        \caption{$ \ac{Nant} = 4 $.}
        % \input{../pictures/antennas_4}
        \includegraphics{../pictures/antennas_4.pdf}
        \label{fig:antennas:4}
    \end{subfigure}
    \hfill
    \begin{subfigure}[c]{0.525\textwidth}
        \centering
        \caption{$ \ac{Nant} = 5 $.}
        % \input{../pictures/antennas_5}
        \includegraphics{../pictures/antennas_5.pdf}
        \label{fig:antennas:5}
    \end{subfigure}
    \hfill

    \caption*{Fonte: Autor.}
\end{figure}

A \autoref{eq:raio:poligono} descreve o raio \ac{rho} do círculo que circunscreve o polígono regular de \ac{Nant} antenas.
Este raio equivale à distância das antenas em relação ao ponto central do polígono.

\begin{equation} \label{eq:raio:poligono} % Diametro do circulo que circunscreve poligono de antenas
	\ac{rho} = \frac{\ac{d}}{2\cdot \sin\left(\displaystyle\frac{\pi}{\ac{Nant}}\right)}
\end{equation}

Cada antena é identificada por um índice \ac{k}, conforme a \autoref{eq:ant:idx}, e tem sua coordenada espacial definida como um valor complexo descrito na \autoref{eq:ant:coord}.
Essas coordenadas são definidas como números complexos para simplificar a análise do ângulo \textcolor{cmyk_M}{\ac{alphak}} em que um par de antenas \textcolor{cmyk_B}{\ac{Ak}} e $\textcolor{cmyk_R}{A_{k+1}}$ se dispõe em relação à geometria do sistema, conforme \autoref{eq:ant:angle}.

\begin{equation} \label{eq:ant:idx} % Indices das antenas
	\ac{k} = \left\{1, 2, \dotsc, \ac{Nant}\right\}
\end{equation}

% Coordenada da antena
\begin{equation} \label{eq:ant:coord}
	\textcolor{cmyk_B}{\ac{Ak}} =
    \ac{rho}
    \cdot \exp\left(\ac{imath}\cdot \ac{k} \cdot \frac{2\pi}{\ac{Nant}}\right) =
    \left( \operatorname{\mathcal{Re}}\left( \textcolor{cmyk_B}{\ac{Ak}} \right), ~\operatorname{\mathcal{Im}}\left( \textcolor{cmyk_B}{\ac{Ak}} \right) \right) =
    \left( \ac{xk}, ~ \ac{yk} \right)
\end{equation}

% Angulo do par de antenas
\begin{equation} \label{eq:ant:angle}
	\textcolor{cmyk_M}{\ac{alphak}} = \arg\left( \textcolor{cmyk_B}{\ac{Ak}} - \textcolor{cmyk_R}{A_{k+1}} \right)
\end{equation}

% {Calcular fase}

Para calcular a fase em uma antena, é interessante representar o sinal recebido como um valor complexo.
Uma forma de obter o complexo de fase consiste em analisar a correlação do sinal incidente \ac{w} com sinais de referência de mesma frequência que o sinal de interesse, em um período completo, \autoref{eq:periodo}.
Os valors \ac{Ik} (em fase) e \ac{Qk} (em quadratura) são calculados respectivamente pela correlação com um cosseno, conforme \autoref{eq:in_phase}, e com um seno, conforme \autoref{eq:out_of_phase}.
A \autoref{eq:Z} apresenta o valor complexo \textcolor{cmyk_B}{\ac{Zk}} de fase para a antena \textcolor{cmyk_B}{\ac{Ak}}.


% Periodo do sinal
\begin{equation} \label{eq:periodo}
    \ac{T} = \frac{2\pi}{\ac{omega}} = \frac{1}{\ac{f}}
\end{equation}

% In phase
\begin{equation} \label{eq:in_phase}
    \ac{Ik} =
    \int\limits_0^{\ac{T}} \cos\left(\ac{omega} \cdot\tau\right)
    \cdot \ac{w}\left( \ac{xk}, ~\ac{yk}, ~\tau \right) \partial \tau
\end{equation}

% Out of phase
\begin{equation} \label{eq:out_of_phase}
    \ac{Qk} =
    \int\limits_0^{\ac{T}} \sin\left(\ac{omega}\cdot\tau\right)
    \cdot \ac{w}\left( \ac{xk}, ~\ac{yk}, ~\tau \right) \partial \tau
\end{equation}

% Fase na antena
\begin{equation} \label{eq:Z}
    \textcolor{cmyk_B}{\ac{Zk}} =
    \frac{\ac{omega}}{\pi}\cdot\left(\ac{Ik} + \ac{imath} \ac{Qk}\right)
\end{equation}

% {Calcular defasagem entre antenas}

O cálculo de defasagem de sinal em um par de antenas consiste na análise de diferença de fase dos valores \textcolor{cmyk_B}{\ac{Zk}} e $\textcolor{cmyk_R}{Z_{k+1}}$ do par de antenas \textcolor{cmyk_B}{\ac{Ak}} e $\textcolor{cmyk_R}{A_{k+1}}$, conforme apresentado na \autoref{eq:dephase}.
Obtido o valor de defasagem \ac{DeltaPhi} entre o par de antenas, finalmente é possível calcular o ângulo \textcolor{Purple}{\ac{betak}} através da \autoref{eq:beta}, note que a simplificação somente é possível com valor de $\ac{d} = \sfrac{\ac{lambda}}{2}$.
A \autoref{fig:AoA:geometria} apresenta a geometria do sistema destacando os valores de interesse na análise de um dos pares de antenas, tomando $\ac{Nant} = 3$.

% Defasagem entre antenas
\begin{equation} \label{eq:dephase}
    \ac{DeltaPhi} =
    \textcolor{cmyk_B}{\ac{Phik}} - \textcolor{cmyk_R}{\Phi_{k+1}} =
    \arg\left(\textcolor{cmyk_B}{\ac{Zk}}\right) - \arg\left(\textcolor{cmyk_R}{Z_{k+1}}\right) =
    \arg\left(\textcolor{cmyk_B}{\ac{Zk}} \cdot \overline{\textcolor{cmyk_R}{Z_{k+1}}}\right)
\end{equation}

% Angulo do sinal em relação ao par de antenas
\begin{equation} \label{eq:beta}
    \textcolor{Purple}{\ac{betak}} = \arccos\left(\frac{\cancel{\ac{lambda}}}{\cancel{d}}\cdot\frac{\ac{DeltaPhi}}{\cancel{2}\pi}\right)
\end{equation}

\begin{figure}[htbp]
    \centering
    \caption{Geometria geral do sistema com $\ac{Nant} = 3$.}
    \input{../pictures/AoA_geometria}
    \label{fig:AoA:geometria}
    \caption*{Fonte: Autor.}
\end{figure}

Para cada par de antenas, são calculados dois valores \ac{thetak} conforme a \autoref{eq:theta:pmk}, equivalentes a dois valores possíveis para o \ac{thetaAoA}.
A \autoref{eq:theta:conj} define o conjunto \ac{Theta} dos valores aferidos de \ac{thetak} para todos os pares de antenas do sistema, este conjunto sempre terá $2 \cdot \ac{Nant}$ elementos, dos quais, metade estão próximos do real valor de \ac{thetaAoA} e os demais são valores distintos do objetivo e entre si.

% Conjunto de angulos calculados
\begin{equation} \label{eq:theta:pmk}
	\ac{thetak} = \textcolor{cmyk_M}{\ac{alphak}}\pm \textcolor{Purple}{\ac{betak}}
\end{equation}

% Conjunto de angulos calculados
\begin{equation} \label{eq:theta:conj}
	\ac{Theta} = \left\{\ac{thetak} ~\middle\vert~ \forall \ac{k}\right\}
\end{equation}

% {Escolha dos possíveis angulos}

Com os possíveis valores de \ac{thetaAoA} obtidos, é necessário estimar qual o valor correto.
Para isso, é criada uma lista auxiliar \ac{ThetaQuanti}, quantizando os valores de \ac{Theta} em intervalos de tamanho \ac{delta}, descrito na \autoref{eq:delta:range}.
A \autoref{eq:theta:quant} descreve a operação de quantização dos valores de \ac{Theta}, que, por se tratar de um cálculo auxiliar, utiliza-se o arredondamento para o inteiro mais próximo.
A quantização implica que os valores de \ac{Theta} serão agrupados por faixas de largura \ac{delta}.

% range_angle
\begin{equation} \label{eq:delta:range}
    \ac{delta} = \frac{\pi}{2 \cdot \left( 1 + \ac{Nant} \right)}
\end{equation}

% Angulos normalizados
\begin{equation} \label{eq:theta:quant}
    \ac{ThetaQuanti} =
    \left\{\left\lfloor\frac{\theta}{\ac{delta}}\right\rceil\cdot\ac{delta} ~\middle\vert~ \forall \theta \in \ac{Theta}  \right\}
\end{equation}

Salvo casos com muito ruído, espera-se que alguns valores em \ac{ThetaQuanti} se repitam, partindo disso, calcula-se \ac{thetaMo}, a moda estatística destes valores, conforme \autoref{eq:theta:moda}.
Esse valor deverá estar próximo ao \ac{thetaAoA}, e será utilizado na filtragem dos valores aferidos em \ac{Theta}.

% Moda entre angulos normalizados
\begin{equation} \label{eq:theta:moda}
    \ac{thetaMo} = \operatorname{\mathcal{M_o}}\left( \ac{ThetaQuanti}  \right)
\end{equation}

O conjunto \ac{ThetaFiltro} contém itens de \ac{Theta} que estejam ao redor do valor \ac{thetaMo} calculado, num intervalo de \ac{delta} para mais ou para menos, conforme \autoref{eq:theta:filtro}.


% Filtra no intervalo
\begin{equation} \label{eq:theta:filtro}
    \ac{ThetaFiltro} = \left\{\theta \in \ac{Theta}  ~\middle\vert~
    \ac{thetaMo} - \ac{delta} \leq \theta \leq \ac{thetaMo} + \ac{delta}\right\}
\end{equation}

Finalmente obtém-se o valor de \ac{thetaAoA} pela mediana dos valores em \ac{ThetaFiltro}, conforme \autoref{eq:theta:mediana}.

% Mediana
\begin{equation} \label{eq:theta:mediana}
    \ac{thetaAoA} = \widetilde{\ac{ThetaFiltro}}
\end{equation}

