\subsection{AoA aux}


Analisando a defasagem de sinal \ac{RF} em um conjunto de antenas, é possível estimar seu \acf{AoA}, essa técnica pode ser utilizada para determinar a direção do emissor em relação à matriz de antenas utilizada.
Baseando-se em dados como a distância entre as antenas, o comprimento de onda $\lambda$ do sinal e a velocidade da luz no meio, usualmente tomada como $c = \SI{299792458,6 \pm 0,3}{\metre\per\second}$ no ar \cite{jennings1987continuity, bensky2016wireless, horst2021localization}.
A \autoref{eq:wavelength} apresenta a relação do comprimento de onda $\lambda$ com a frequência $f$, a frequência angular $\omega$ e a velocidade da luz $c$.

\begin{equation} \label{eq:wavelength} % Comprimento de onda
    \lambda = \frac{c}{f} = \frac{2\pi \cdot c}{\omega}
\end{equation}

Se um emissor de sinal estiver distante o suficiente, é possível considerar que a frente de onda tem um comportamento planar, essa característica simplifica as operações envolvidas.
A distância de Fraunhofer ($d_F$) é a mínima para essa condição, que define o início da região de \textit{far-field}, conforme apresentado na \autoref{eq:fraunhofer}, onde $D$ é a maior dimensão da antena \cite{balanis2016antenna}.
Tomando $D = 2  \lambda$, para uma antena de dipolo, obtém-se $d_F = 8 \lambda$.
A \autoref{fig:plana_0} ilustra o comportamento planar de uma frente de onda, com destaque na distância $d_F$.

\begin{equation} \label{eq:fraunhofer}
    d_F = \frac{2 \cdot D^2}{\lambda} \quad \Rightarrow \quad d_F = \frac{2 \cdot \left(2 \cdot \lambda \right)^2}{\lambda} = 8 \lambda
\end{equation}

\begin{figure}[htpb]
    \centering
    \caption{Característica de frente de onda a cada $\lambda$ a partir da antena emissora.}
        % \resizebox{\textwidth}{!}{%
    \begin{circuitikz}[american, voltage shift=0.5, line width=0.5]

        \def\wavelength{0.5}
        \def\d{0.5*\wavelength}

        \def\closeRange{1}
        \def\farRange{\closeRange+30}

        \coordinate (O) at (0,0);
        \coordinate (antenna) at (-\closeRange,0);
        % \draw [help lines, dashed] (-5,-3) grid (5,3); % desenha grid
        % \draw [red] (O) node[draw,cross out] {}; % marca pont(0,0)

        \draw[thick]
            (antenna) node[dinantenna, scale=0.75]{}
        ;

        % \draw (\closeRange-0.5,-4) rectangle (\farRange+0.1,4);
        \clip (-0.75,-1.5) rectangle (12.1,1.5);
        \foreach \x [evaluate={\z=int((\x+\closeRange));}] in {0,...,30} {
            \draw [gray, thin, opacity=0.5] (antenna) circle (\z*\wavelength);
            \draw [black]
            (antenna) ++ (\z*\wavelength,0)
            node[anchor=south, font = {\footnotesize\bfseries}, rotate=-90,scale=0.75]{$\z\lambda$}
            ++(0, -4)
            -- ++(0,8);
        }

        \draw [Red, thick] (antenna) ++ (8*\wavelength,-4) -- ++(0,8);

        % \foreach \x in {0,60,...,300} {
        %     \draw[thick] (\x:1 cm) -- (\x + 60:1 cm);

        %     \draw (\x + 30:1.732 cm) node[Gray, circ]{};
        %     \draw[Gray, dashed] (\x:1 cm) -- ++(\x: 0.9cm);
        %     \draw[Gray, dotted]
        %     %     % (\x:1 cm) arc (\x+240:\x+180:1cm)
        %         (\x:1 cm) arc [start angle=\x+120, delta angle=110, radius=1cm]
        %         (\x:1 cm) arc [start angle=\x+120, delta angle=-50, radius=1cm]
        %     ;
        % }

        % \draw (0,0) node [circ] {} node [below left,font={\scriptsize\bfseries}] {BS};
        % \draw[thick, densely dotted] (0,0) circle (1cm);

        % \draw[-latex] (0,0) -- (0:1cm) node[midway, below] {$R_c$};
        % \draw[-latex] (0,0) -- (90:0.866cm) node[midway, left] {$R$};

    \end{circuitikz}
  % }

    % \includegraphics{../pictures/plana_0.pdf}
    \caption*{Fonte: Autor.}
    \label{fig:plana_0}
\end{figure}


Tomando agora um par de antenas separadas por uma distância fixa $d$, torna-se viável fazer a análise trigonométrica entre as antenas e a frente de onda que chega, onde essa distância $d$ será a hipotenusa do triângulo.
A \autoref{fig:AoA} apresenta a disposição das antenas em quatro casos de chegada de sinal \ac{RF}.
Para realizar a análise, é necessário conhecer uma segunda dimensão do triângulo retângulo envolvido, esta é obtida analisando a defasagem entre o sinal das antenas, conforme apresentado na \autoref{eq:defasagem}.


\begin{equation} \label{eq:defasagem} % Relação trigonometrica de defasagem e angulo do sinal
    d \cdot \cos\left(\pm\beta\right) = \lambda \cdot \frac{\Delta\Phi}{2 \pi}
\end{equation}


É importante ressaltar que um sistemas com um único par de antenas não é capaz de determinar completamente o \ac{AoA}, já que o valor calculado de $\beta$ é igual para casos simétricos, a relação é apresentada nas Figuras \ref{fig:AoA:1} e \ref{fig:AoA:2}.
Existem ainda dois casos notáveis, onde o sinal chega alinhado com o par de antenas ou perpendicular a elas, apresentados respectivamente nas Figuras \ref{fig:AoA:3} e \ref{fig:AoA:4}.

\begin{figure}
    \caption{\ac{AoA} ($\Theta$) com par de antenas em diversas direções equivalentes, sistema com ângulo $\alpha=\SI{20}{\degree}$.}
    \label{fig:AoA}

    \hfill
    \begin{subfigure}[b]{0.45\textwidth}
        \centering
        \caption{$\Theta=\SI{60}{\degree}$, $\beta=\SI{40}{\degree}$}
        \input{../pictures/AoA_1_new}
        % \includegraphics{../pictures/AoA_1.pdf}
        \label{fig:AoA:1}
    \end{subfigure}
    \hfill
    \begin{subfigure}[b]{0.45\textwidth}
        \centering
        \caption{$\Theta=\SI{-20}{\degree}$, $\beta=\SI{40}{\degree}$}
        \input{../pictures/AoA_2_new}
        % \includegraphics{../pictures/AoA_2.pdf}
        \label{fig:AoA:2}
    \end{subfigure}
    \hfill

    \vspace{\floatsep}

    \hfill
    \begin{subfigure}[b]{0.45\textwidth}
        \centering
        \caption{$\Theta=\SI{110}{\degree}$, $\beta=\SI{90}{\degree}$}
        \input{../pictures/AoA_3_new}
        % \includegraphics{../pictures/AoA_3.pdf}
        \label{fig:AoA:3}
    \end{subfigure}
    \hfill
    \begin{subfigure}[b]{0.45\textwidth}
        \centering
        \caption{$\Theta=\SI{20}{\degree}$, $\beta=\SI{0}{\degree}$}
        \input{../pictures/AoA_4_new}
        % \includegraphics{../pictures/AoA_4.pdf}
        \label{fig:AoA:4}
    \end{subfigure}
    \hfill

    \caption*{Fonte: Autor.}
\end{figure}

A escolha da distância $d$ entre as antenas deve ser feita de forma a otimizar a resolução da medida de defasagem, com a maior distância possível. Porém é necessário evitar ambiguidades na análise, como o sinal é periódico, o valor se repetirá a cada $\lambda$, e terá valores simétricos quando $d > \sfrac{\lambda}{2}$, ilustrado na \autoref{fig:AoA_d:fail}.
Adota-se então $d=\sfrac{\lambda}{2}$, conforme apresentado na \autoref{fig:AoA_d:ok} \cite{bensky2016wireless, horst2021localization}.

\begin{figure}
    \caption{Diferentes valores para $d$.}
    \label{fig:AoA_d}

    \hfill
    \begin{subfigure}[b]{0.45\textwidth}
        \centering
        \caption{$d > \sfrac{\lambda}{2}$}
        % \input{../pictures/AoA_0_fail}
        \includegraphics{../pictures/AoA_0_fail.pdf}
        \label{fig:AoA_d:fail}
    \end{subfigure}
    \hfill
    \begin{subfigure}[b]{0.45\textwidth}
        \centering
        \caption{$d = \sfrac{\lambda}{2}$}
        % \input{../pictures/AoA_0}
        \includegraphics{../pictures/AoA_0.pdf}
        \label{fig:AoA_d:ok}
    \end{subfigure}
    \hfill

    \caption*{Fonte: Autor.}
\end{figure}


















\begin{figure}[H]
    \centering
    \input{../pictures/antennas}
    \caption{Agora sim 0}
\end{figure}

\begin{figure}[H]
    \centering
    \input{../pictures/AoA_2_alt}
    \caption{Agora sim}
\end{figure}


\begin{equation} % Distância entre par de antenas
    d = \frac{\lambda}{2}
\end{equation}

\begin{equation} % Diâmetro do circulo que circunscreve poligono de antenas
	\rho = \frac{d}{2\cdot \sin\left(\frac{\pi}{N_\text{ant}}\right)}
\end{equation}

\begin{equation} % Índices das antenas
	k = \left\{1, 2, \dotsc, N_\text{ant}\right\}
\end{equation}

\begin{equation} % Coordenada da antena
	A_k =
    \rho \cdot \exp\left(\imath\cdot k \cdot \frac{2\pi}{N_\text{ant}}\right) =
    \left( x_{A_k},~ y_{A_k} \right) =
    \left( \operatorname{\mathcal{Re}}\left( A_k \right), ~\operatorname{\mathcal{Im}}\left( A_k \right) \right)
\end{equation}

\begin{equation} % Período do sinal
    T = \frac{2\pi}{\omega} = \frac{1}{f}
\end{equation}

\begin{equation} % In phase
    I_k = \int\limits_0^T \cos\left(\omega\cdot\tau\right) \cdot w\left( \tau, ~x_{A_k}, ~y_{A_k} \right) \partial \tau
\end{equation}

\begin{equation} % Out of phase
    Q_k = \int\limits_0^T \sin\left(\omega\cdot\tau\right) \cdot w\left( \tau, ~x_{A_k}, ~y_{A_k} \right) \partial \tau
\end{equation}

\begin{equation} % Fase na antena
    Z_k = \frac{\omega}{\pi}\cdot\left(I_k + \imath Q_k\right)
\end{equation}

\begin{equation} % Defasagem entre antenas
    \Delta\Phi_{k} =
    \Phi_{k} - \Phi_{k+1} =
    \arg\left(Z_{k}\right) - \arg\left(Z_{k+1}\right) =
    \arg\left(Z_{k} \cdot \overline{Z_{k+1}}\right)
\end{equation}

\begin{equation} % Angulo do sinal em relação ao par de antenas
    \beta_{k} = \pm \arccos\left(\frac{\cancel{\lambda}}{\cancel{d}}\cdot\frac{\Delta\Phi_{k}}{\cancel{2}\pi}\right)
\end{equation}

\begin{equation} % Angulo do par de antenas
	\alpha_{k} = \arg\left( A_{k} - A_{k+1} \right)
\end{equation}

\begin{equation} % Conjunto de angulos calculados
	\theta_{\pm k} = \alpha_{k}\pm \beta_{k}
\end{equation}

\begin{equation} % Conjunto de angulos calculados
	\Theta = \left\{\theta_{\pm k}=\alpha_{k}\pm \beta_{k} ~\middle\vert~ \forall k\right\}
\end{equation}

\begin{equation} % range_angle
    \delta = \frac{\pi}{2 \cdot \left( 1 + N_\text{ant} \right)}
\end{equation}

\begin{equation} % Angulos normalizados
    \Theta_{\left\lfloor\bullet\right\rceil} =
    \left\{\left\lfloor\frac{\theta}{\delta}\right\rceil\cdot\delta ~\middle\vert~ \forall \theta \in \Theta  \right\}
\end{equation}

\begin{equation} % Moda entre angulos normalizados
    \theta_\mathcal{M_o} = \operatorname{\mathcal{M_o}}\left( \Theta_{\left\lfloor\bullet\right\rceil}  \right)
\end{equation}

\begin{equation} % Filtra no intervalo
    \Theta_\text{F} = \left\{\theta \in \Theta  ~\middle\vert~
    \theta_\mathcal{M_o} - \delta \leq \theta \leq \theta_\mathcal{M_o} + \delta\right\}
\end{equation}

\begin{equation} % Mediana
    \theta_\text{AoA} = \widetilde{\Theta_\text{F}}
\end{equation}

\begin{figure}[H]
    \centering
    \input{../pictures/simul_POLY_3_R_50.tex}
    \caption{Gráfico}
\end{figure}

\begin{figure}[H]
    \centering
    \input{../pictures/simul_POLY_3_R_50_SNR_1_ATT.tex}
    \caption{Gráfico}
\end{figure}

\begin{figure}[H]
    \centering
    \input{../pictures/simul_POLY_3_R_50_SNR_1.tex}
    \caption{Gráfico}
\end{figure}
