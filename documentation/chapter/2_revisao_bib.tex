\chapter{Revisão Bibliográfica}

\section{Fundamentação teórica}

\subsection{Direcionamento por coordenadas geográficas}

Coordenadas geográficas são definidas por dois valores, latitude e longitude, associadas a coordenadas esféricas referenciadas a partir do centro da terra, assumindo o raio da coordenada como o raio médio da superfície do planeta, cerca de $R_\text{Terra} = \SI{6371E3}{\metre}$ \cite{palomaguitarrara, chrisveness}.
A latitude equivale à componente polar $\phi$ centralizada na linha do equador, enquanto a longitude equivale à componente $\theta$ centralizado do meridiano de Greenwich \cite{palomaguitarrara, henriquefleming2003}.

Conhecendo as coordenadas de dois pontos distintos A e B, é possível determinar seu ângulo de \textit{bearing} $\beta$ relativo, referente ao norte, ou seja, o ângulo da direção a se seguir para chegar do ponto A ao ponto B a partir da direção do note no ponto de origem A \cite{henriquefleming2003}.

Sendo \textcolor{Green}{$\mathbf{A}$} e \textcolor{Blue}{$\mathbf{B}$} duas coordenadas geográficas, \textcolor{Green}{$\phi_A$} e \textcolor{Blue}{$\phi_B$} suas respectivas latitudes,  \textcolor{Green}{$\theta_A$} e \textcolor{Blue}{$\theta_B$} suas respectivas longitudes, conforme ilustrado na \autoref{fig:globo}.

\begin{figure}[htbp]
    \centering
    \caption{Representação geométrica de distância e ângulo em relação ao norte entre coordenadas geográficas \textcolor{Green}{$\mathbf{A}$} e \textcolor{Blue}{$\mathbf{B}$}.}
    % \begin{tikzpicture}[scale=1,tdplot_main_coords,every node/.style={font = {\footnotesize\bfseries}}]
% https://latex.org/forum/viewtopic.php?t=25316

\def\rvec{3.5}

\def\phivecA{30}
\def\thetavecA{75}

\def\phivecB{45}
\def\thetavecB{15}

\def\auxOpacity{0.75}

\pgfmathsetmacro{\ax}{\rvec*sin(\phivecA)*cos(\thetavecA)}
\pgfmathsetmacro{\ay}{\rvec*sin(\phivecA)*sin(\thetavecA)}
\pgfmathsetmacro{\az}{\rvec*cos(\phivecA)}

\pgfmathsetmacro{\bx}{\rvec*sin(\phivecB)*cos(\thetavecB)}
\pgfmathsetmacro{\by}{\rvec*sin(\phivecB)*sin(\thetavecB)}
\pgfmathsetmacro{\bz}{\rvec*cos(\phivecB)}


\shadedraw[tdplot_screen_coords, ball color = white, opacity=0.25] (0,0) circle (\rvec);

%-----------------------
\coordinate (O) at (0,0,0);

\tdplotsetcoord{A}{\rvec}{\phivecA}{\thetavecA}

\tdplotsetcoord{B}{\rvec}{\phivecB}{\thetavecB}

\tdplotsetcoord{N}{\rvec}{0}{90}

%draw the main coordinate system axes
\draw[thick,-latex] (O) -- (1.35*\rvec,0,0) node[anchor=north east]{$x$};
\draw[thick,-latex] (O) -- (0,1.35*\rvec,0) node[anchor=north west]{$y$};
\draw[thick,-latex] (O) -- (0,0,1.35*\rvec) node[anchor=south]{$z$};

\draw[thick,-latex,opacity=0] (O) -- (-1.35*\rvec,0,0) node[anchor=south west]{$x$};
\draw[thick,-latex,opacity=0] (O) -- (0,-1.35*\rvec,0) node[anchor=south east]{$y$};
\draw[thick,-latex,opacity=0] (O) -- (0,0,-1.35*\rvec) node[anchor=north]{$z$};


\draw[-latex,very thick,opacity=\auxOpacity, color=cmyk_G] (O) -- (A) node[anchor=south west, opacity=1] {$\mathbf{A}$};
\draw[-latex,very thick,opacity=\auxOpacity, color=cmyk_B] (O) -- (B) node[anchor=south east, opacity=1] {$\mathbf{B}$};
\draw[very thick,color=Black] (N) node[anchor=west] {$\mathbf{N}$};


\draw[dashed, opacity=0.15] (\rvec,0,0) arc (0:360:\rvec);
\draw[thin] (\rvec,0,0) arc (0:135:\rvec);
\draw[thin] (\rvec,0,0) arc (0:-45:\rvec);

\draw[dashed, color=cmyk_G, opacity=\auxOpacity] (O) -- (Axy) -- (A);
\draw[dashed, color=cmyk_B, opacity=\auxOpacity] (O) -- (Bxy) -- (B);

\draw[dotted, color=cmyk_G, opacity=\auxOpacity] (Ax) -- (Axy);
\draw[dotted, color=cmyk_G, opacity=\auxOpacity] (Ay) -- (Axy);

\draw[dotted, color=cmyk_B, opacity=\auxOpacity] (Bx) -- (Bxy);
\draw[dotted, color=cmyk_B, opacity=\auxOpacity] (By) -- (Bxy);

% \pause


\tdplotdrawarc[color=cmyk_G, opacity=\auxOpacity]{(O)}{0.75*\rvec}{0}{\thetavecA}{anchor=north}{$\theta_A$}

\tdplotdrawarc[color=cmyk_B, opacity=\auxOpacity]{(O)}{0.5*\rvec}{0}{\thetavecB}{anchor=north east}{$\theta_B$}

% \pause


\tdplotsetthetaplanecoords{\thetavecA}
\tdplotdrawarc[color=cmyk_G, opacity=\auxOpacity, tdplot_rotated_coords]{(O)}{0.75*\rvec}{90}{\phivecA}{anchor=west}{$\phi_A$}

\tdplotsetthetaplanecoords{\thetavecB}
\tdplotdrawarc[color=cmyk_B, opacity=\auxOpacity, tdplot_rotated_coords]{(O)}{0.5*\rvec}{90}{\phivecB}{anchor=east}{$\phi_B$}

% \pause

\tdplotsetthetaplanecoords{\thetavecB}
\draw[dashed, cmyk_B, opacity=0.15, tdplot_rotated_coords] (\rvec,0,0) arc (180:-40:-\rvec);
\draw[thin, cmyk_B, opacity=0.25, tdplot_rotated_coords] (\rvec,0,0) arc (0:140:\rvec);
\draw[thin, cmyk_B, opacity=0.25, tdplot_rotated_coords] (\rvec,0,0) arc (360:320:\rvec);
\draw[very thick, color=cmyk_B, tdplot_rotated_coords] (\rvec,0,0) arc (0:\phivecB:\rvec);

\tdplotsetthetaplanecoords{\thetavecA}
\draw[dashed, cmyk_G, opacity=0.15, tdplot_rotated_coords] (\rvec,0,0) arc (180:-40:-\rvec);
\draw[thin, cmyk_G, opacity=0.25, tdplot_rotated_coords] (\rvec,0,0) arc (0:140:\rvec);
\draw[thin, cmyk_G, opacity=0.25, tdplot_rotated_coords] (\rvec,0,0) arc (360:320:\rvec);
\draw[very thick, color=cmyk_G, tdplot_rotated_coords] (\rvec,0,0) arc (0:\phivecA:\rvec);


\drawArc{\ax}{\ay}{\az}{\bx}{\by}{\bz}{\rvec}{anchor=south}{$d$}

\tdplotsetrotatedcoords{\phivecA}{\thetavecA}{0}

\def\centerarc[#1](#2)(#3:#4:#5)% Syntax: [draw options] (center) (initial angle:final angle:radius)
    { \draw[#1] ($(#2)+({#5*cos(#3)},{#5*sin(#3)})$) arc (#3:#4:#5) node[midway,anchor=east] {$\beta$}; }

\centerarc[cmyk_R, tdplot_rotated_coords](A)(292:214:0.4)

\end{tikzpicture}
    \includegraphics{../pictures/globo.pdf}
    \caption*{Fonte: Autor.}
    \label{fig:globo}
\end{figure}

Calculam-se $\Delta_\phi$ e $\Delta_\theta$.

    \begin{equation}
        \Delta_\phi = \textcolor{Blue}{\phi_B} - \textcolor{Green}{\phi_A}
    \end{equation}
    \begin{equation}
        \Delta_\theta = \textcolor{Blue}{\theta_B} - \textcolor{Green}{\theta_A}
    \end{equation}

Através da lei dos haversines é possível obter a distância mínima $d$ entre as coordenadas, sobre a superfície, e também o ângulo de \textit{Bearing} \textcolor{Red}{$\beta$} formado no vértice \textcolor{Green}{$\mathbf{A}$} do triângulo esférico $\mathbf{N}$\textcolor{Green}{$\mathbf{A}$}\textcolor{Blue}{$\mathbf{B}$} \cite{chrisveness}.
Para o cálculo de distância, os ângulos devem ser tratados em radianos.

\begin{equation}
    X = \cos\left(\textcolor{Blue}{\theta_B}\right)\cdot \sin\left(\Delta_\phi\right)
\end{equation}
\begin{equation}
    Y = \cos\left(\textcolor{Green}{\theta_A}\right)\cdot\sin\left(\textcolor{Blue}{\theta_B}\right) - \sin\left(\textcolor{Green}{\theta_A}\right) \cdot \cos\left(\textcolor{Green}{\theta_B}\right) \cdot \cos\left(\Delta_\phi\right)
\end{equation}
\begin{equation}
    Z = \sin^2\left(\frac{\Delta_\theta}{2}\right) + \cos\left(\textcolor{Blue}{\theta_B}\right) \cdot \cos\left(\textcolor{Green}{\theta_A}\right) \cdot \sin^2\left(\frac{\Delta_\phi}{2}\right)
\end{equation}
\begin{equation}
    \textcolor{Red}{\beta} = \arctan\left(\frac{X}{Y}\right) - \frac{\pi}{2}
\end{equation}
\begin{equation}
    \textcolor{Red}{d} = R_\text{Terra} \cdot 2 \cdot \arctan\left(\frac{\sqrt{Z}}{\sqrt{1-Z}}\right)
\end{equation}

O ângulo \textcolor{Red}{$\beta$} calculado aqui é referente à direção cardeal Norte, assim, uma equipe de busca equipada com uma bússola simples seria capaz de seguir a direção correta.
A \autoref{fig:bearing} apresenta a aplicação desenvolvida por \citeauthor{chrisveness}, capaz de calcular o ângulo de \textit{Bearing} entre duas coordenadas, note que, neste caso, o ângulo referido é relacionado à direção cardial Leste \cite{chrisveness}.

\begin{figure}[htbp]
    \centering
    \caption{Cálculo do ângulo de \textit{Bearing} \textcolor{Red}{$\beta$} entre as coordenadas dos Campi Santo André e São Bernardo do Campo da UFABC.}
    \includegraphics[width=0.7\textwidth]{../pictures/bearing.png}
    \caption*{Fonte: \citeauthor{chrisveness} 2019 \cite{chrisveness}}
    \label{fig:bearing}
\end{figure}

\pagebreak
\subsection{Estimar \acs{AoA} utilizando matriz de antenas}

Analisando a defasagem de sinal \ac{RF} em um conjunto de antenas, é possível estimar seu \acf{AoA}, essa técnica pode ser utilizada para determinar a direção do emissor em relação à matriz de antenas utilizada.
Baseando-se em dados como a distância entre as antenas, o comprimento de onda $\lambda$ do sinal e a velocidade da luz no meio, usualmente tomada como $c = \SI{299792458,6 \pm 0,3}{\metre\per\second}$ no ar \cite{jennings1987continuity, bensky2016wireless, horst2021localization}.
A \autoref{eq:wavelength} apresenta a relação do comprimento de onda $\lambda$ com a frequência $f$, com a frequência angular $\omega$ e a velocidade da luz $c$.

\begin{equation}\label{eq:wavelength}
    \lambda = \frac{c}{f} = \frac{2\pi \cdot c}{\omega}
\end{equation}

Se um emissor estiver distante o bastante, é possível considerar que sua frente de onda tem um comportamento planar, essa característica simplifica as operações envolvidas.
A distância de Fraunhofer ($d_F$) é a mínima para essa condição, que define o início da região de \textit{far-field}, conforme apresentado na \autoref{eq:fraunhofer}, onde $D$ é a maior dimensão da antena \cite{balanis2016antenna}.
Tomando $D = 2  \lambda$, para uma antena de dipolo, obtém-se $d_F = 8 \lambda$.
A \autoref{fig:plana_0} ilustra o comportamento planar de uma frente de onda além de $d_F$.

\begin{equation}\label{eq:fraunhofer}
    d_F = \frac{2 \cdot D^2}{\lambda} \quad \Rightarrow \quad d_F = \frac{2 \cdot \left(2 \cdot \lambda \right)^2}{\lambda} = 8 \lambda
\end{equation}

\begin{figure}[htpb]
    \centering
    \caption{Característica de frente de onda a cada $\lambda$ a partir da antena.}
        % \resizebox{\textwidth}{!}{%
    \begin{circuitikz}[american, voltage shift=0.5, line width=0.5]

        \def\wavelength{0.5}
        \def\d{0.5*\wavelength}

        \def\closeRange{1}
        \def\farRange{\closeRange+30}

        \coordinate (O) at (0,0);
        \coordinate (antenna) at (-\closeRange,0);
        % \draw [help lines, dashed] (-5,-3) grid (5,3); % desenha grid
        % \draw [red] (O) node[draw,cross out] {}; % marca pont(0,0)

        \draw[thick]
            (antenna) node[dinantenna, scale=0.75]{}
        ;

        % \draw (\closeRange-0.5,-4) rectangle (\farRange+0.1,4);
        \clip (-0.75,-1.5) rectangle (12.1,1.5);
        \foreach \x [evaluate={\z=int((\x+\closeRange));}] in {0,...,30} {
            \draw [gray, thin, opacity=0.5] (antenna) circle (\z*\wavelength);
            \draw [black]
            (antenna) ++ (\z*\wavelength,0)
            node[anchor=south, font = {\footnotesize\bfseries}, rotate=-90,scale=0.75]{$\z\lambda$}
            ++(0, -4)
            -- ++(0,8);
        }

        \draw [Red, thick] (antenna) ++ (8*\wavelength,-4) -- ++(0,8);

        % \foreach \x in {0,60,...,300} {
        %     \draw[thick] (\x:1 cm) -- (\x + 60:1 cm);

        %     \draw (\x + 30:1.732 cm) node[Gray, circ]{};
        %     \draw[Gray, dashed] (\x:1 cm) -- ++(\x: 0.9cm);
        %     \draw[Gray, dotted]
        %     %     % (\x:1 cm) arc (\x+240:\x+180:1cm)
        %         (\x:1 cm) arc [start angle=\x+120, delta angle=110, radius=1cm]
        %         (\x:1 cm) arc [start angle=\x+120, delta angle=-50, radius=1cm]
        %     ;
        % }

        % \draw (0,0) node [circ] {} node [below left,font={\scriptsize\bfseries}] {BS};
        % \draw[thick, densely dotted] (0,0) circle (1cm);

        % \draw[-latex] (0,0) -- (0:1cm) node[midway, below] {$R_c$};
        % \draw[-latex] (0,0) -- (90:0.866cm) node[midway, left] {$R$};

    \end{circuitikz}
  % }

    % \includegraphics{../pictures/plana_0.pdf}
    \caption*{Fonte: Autor.}
    \label{fig:plana_0}
\end{figure}

Tomando agora um par de antenas separadas por uma distância fixa $d$, torna-se viável fazer a análise trigonométrica entre as antenas e a frente de onda que chega, onde essa distância $d$ será a hipotenusa do triângulo.
A \autoref{fig:AoA} apresenta a disposição das antenas em dois casos de chegada de sinal \ac{RF}.
Para realizar a análise, é necessário conhecer uma segunda dimensão do triângulo retângulo envolvido, esta é obtida analisando a defasagem entre o sinal das antenas, conforme apresentado na \autoref{eq:defasagem}.

\begin{equation}\label{eq:defasagem}
    d \cdot \sin\left(\beta\right) = \lambda \cdot \frac{\Delta\Phi}{2 \pi} \quad \Rightarrow \quad \beta = \arcsin \left(\frac{\lambda}{d} \cdot \frac{\Delta\Phi}{2 \pi}\right)
\end{equation}

É importante ressaltar que um sistemas com um único par de antenas não é capaz de determinar completamente o \ac{AoA}, já que o valor calculado de $\beta$ é igual para casos simétricos, a relação é apresentada nas Figuras \ref{fig:AoA:1} e \ref{fig:AoA:2}.
Existem ainda dois casos notáveis, onde o sinal chega alinhado com a linha das com as antenas ou perpendicular a ela, apresentados respectivamente nas Figuras \ref{fig:AoA:3} e \ref{fig:AoA:4}.

\begin{figure}
    \caption{\ac{AoA} com par de antenas em diversas direções equivalentes.}
    \label{fig:AoA}

    \hfill
    \begin{subfigure}[b]{0.45\textwidth}
        \centering
        \caption{$\beta=\SI{60}{\degree}$}
        % \resizebox{!}{0.7\textheight}{%
\begin{circuitikz}[american, voltage shift=0.5, line width=0.5, every node/.style={font = {\footnotesize\bfseries}}]

    \def\wavelength{3.5}
    \pgfmathsetmacro\d{0.5*\wavelength}

    \def\antennaAngle{20}
    \pgfmathsetmacro\signalAngle{\antennaAngle+40}

    \def\closeRange{9}
    \def\farRange{\closeRange+13}

	\def\NAntennas{3}
	\pgfmathsetmacro\AngleAntennas{360/\NAntennas}
	\def\ShiftAngleAntennas{-90}

	\pgfmathsetmacro\RhoAntennas{\d/(2*sin(180/\NAntennas))}

    \def\centerarc(#1)(#2:#3:#4)% Syntax: [draw options] (center) (initial angle:final angle:radius)
    { ($(#1)+({#4*cos(#2)},{#4*sin(#2)})$) arc (#2:#3:#4) }

    \def\coordref[#1](#2){%

        \coordinate(sysref) at (#2);

        \draw[#1, -latex] (sysref) ++(-0.4,-0.3) -- ++(0.9,0) node[midway, below]{$x$};
        \draw[#1, -latex] (sysref) ++(-0.3,-0.4) -- ++(0,0.9) node[midway, left]{$y$};
        \draw[#1, -latex] \centerarc(sysref)(-90:180:0.25);
        \draw[#1] (sysref) node{$+$}
    }

    \coordinate (bottomleft) at (-3.5,-1);
    \coordinate (topright) at (3.5,5);


    % \draw[Red,dashed] (bottomleft) rectangle (topright);
    \clip (bottomleft) rectangle (topright);

    \coordinate (O) at (0,0);
    \coordinate (sourceAntenna) at (\signalAngle:\closeRange*\wavelength);
    % \draw [help lines, dashed] (bottomleft) grid (topright); % desenha grid
    % \draw [red] (O) node[draw,cross out] {}; % marca pont(0,0)

    % Circulo de antenas
	% \draw[densely dotted, opacity=0.25] (O) ++(90:\RhoAntennas) circle (\RhoAntennas);

    % Linhas do sinal de fundo
    \foreach \x [evaluate={\y=int((\x+\closeRange));\z=int((\x+\closeRange));}] in {-3,...,3} {
        \draw [black!75, very thin]
        (sourceAntenna) ++ (\signalAngle:-\z*\wavelength)
            % node[anchor=west, font = {\footnotesize\bfseries}]{$\y\lambda$}
        ($(sourceAntenna) + (\signalAngle:-\z*\wavelength) + ({10*cos(\signalAngle+90)},{10*sin(\signalAngle+90)})$)
            --
        ($(sourceAntenna) + (\signalAngle:-\z*\wavelength) - ({10*cos(\signalAngle+90)},{10*sin(\signalAngle+90)})$)
        % \draw [gray, thin] (sourceAntenna) circle (\z)
        ;
    }

    % Antenas
    \draw[thick, cmyk_R] (O) node[dinantenna] (A00) {} ;
    % \draw[thick, cmyk_G, opacity=0.75] (O) ++(60:\d) node[dinantenna] (A0d) {} node [below] {$A_{k+2}$};
    \draw[thick, cmyk_B] (O) ++(\antennaAngle:\d) node[dinantenna] (Ad0) {} ;

    \draw[very thin, Black!50, -latex] % Desenha eixo X
        (-3,0) -- (3,0) node[below left] {$x$}
    ;

    % Ângulo alpha entre antenas
    \draw[thin, cmyk_M]
        \centerarc(O)(0:\antennaAngle:0.3)
        node [above, inner sep=3pt] {$\alpha$}
    ;


    % Desenha senoide de fundo
    \draw[Goldenrod, domain=-8:8, samples=100]
        (A00) ++(\signalAngle+90:0.5*\wavelength) coordinate(signalAux)
        plot[shift={(signalAux)}, rotate=\signalAngle]({\x},{cos(\x * pi * 2 / \wavelength r)})
    ;

    % Direção do sinal
    \draw[very thick, dashed, -latex, Goldenrod]
        % (A00) ++(1.5*\d,0) ++ (\signalAngle:-0.5*\d) -- coordinate(angleArrow) ++(\signalAngle:\d)
        (A00) ++(-2,0) ++ (\signalAngle:-0.25*\d) -- coordinate(angleArrow) ++ (\signalAngle:0.5*\d) --++(\signalAngle:0.25*\d)
    ;
    % Angulo Theta do sinal
    \draw[thin]
        (angleArrow) ++ (0.4, 0) node [below,inner sep=2pt] {$\theta_\text{\ac{AoA}}$}
        \centerarc(angleArrow)(0:\signalAngle:0.4)
    ;

    % Triangulo retângulo + quadradinho
    \draw[Black]

        (A00) --++($({\signalAngle-90}:{\d*sin(\signalAngle-\antennaAngle)})$) coordinate (pontoTriangulo) -- (Ad0) -- (A00)

        (pontoTriangulo)
          ++(\signalAngle:0.125)
        --++(\signalAngle+90:0.125)
        --++(\signalAngle+180:0.125)
    ;

    % Arco do angulo beta
    \draw[thin, Purple]
        (Ad0) ++ (180+\antennaAngle:0.4) node[above, inner sep=3pt] {$\beta$}
        \centerarc(Ad0)(180+\antennaAngle:180+\signalAngle:0.4)
    ;

    % Distânci d entre antenas
    \draw[latex-latex]
        ($(A00)+(0,1)$) -- ($(Ad0)+(0,1)$) node [midway, fill=white, circle, inner sep=1pt] {$d$}
    ;

    \newcommand\CircleRadius{3cm}
    %   \draw (0,0) circle (\CircleRadius);
    % special method of noting the position of a point
    \coordinate (P) at (50:\CircleRadius);

\end{circuitikz}
% }


        % \includegraphics{../pictures/AoA_1.pdf}
        \label{fig:AoA:1}
    \end{subfigure}
    \hfill
    \begin{subfigure}[b]{0.45\textwidth}
        \centering
        \caption{$\beta=\SI{60}{\degree}$}
        \input{../pictures/AoA_2}
        % \includegraphics{../pictures/AoA_2.pdf}
        \label{fig:AoA:2}
    \end{subfigure}
    \hfill

    \hfill
    \begin{subfigure}[b]{0.45\textwidth}
        \centering
        \caption{$\beta=\SI{90}{\degree}$}
        \input{../pictures/AoA_3}
        % \includegraphics{../pictures/AoA_3.pdf}
        \label{fig:AoA:3}
    \end{subfigure}
    \hfill
    \begin{subfigure}[b]{0.45\textwidth}
        \centering
        \caption{$\beta=\SI{0}{\degree}$}
        \input{../pictures/AoA_4}
        % \includegraphics{../pictures/AoA_4.pdf}
        \label{fig:AoA:4}
    \end{subfigure}
    \hfill

    \caption*{Fonte: Autor.}
\end{figure}

A escolha da distância $d$ entre as antenas deve ser feita de forma a otimizar a otimizar a resolução da medida de defasagem, com a maior distância possível, porém é necessário evitar ambiguidades na análise, como o sinal é periódico, o valor se repetirá a cada $\lambda$, e terá valores simétricos quando $d > \sfrac{\lambda}{2}$, ilustrado na \autoref{fig:AoA_d:fail}.
Adota-se então $d=\sfrac{\lambda}{2}$, conforme apresentado na \autoref{fig:AoA_d:ok} \cite{bensky2016wireless, horst2021localization}.

\begin{figure}
    \caption{Diferentes valores para $d$.}
    \label{fig:AoA_d}

    \hfill
    \begin{subfigure}[b]{0.45\textwidth}
        \centering
        \caption{$d > \sfrac{\lambda}{2}$}
        \input{../pictures/AoA_0_fail}
        % \includegraphics{../pictures/AoA_0_fail.pdf}
        \label{fig:AoA_d:fail}
    \end{subfigure}
    \hfill
    \begin{subfigure}[b]{0.45\textwidth}
        \centering
        \caption{$d = \sfrac{\lambda}{2}$}
        \input{../pictures/AoA_0}
        % \includegraphics{../pictures/AoA_0.pdf}
        \label{fig:AoA_d:ok}
    \end{subfigure}
    \hfill

    \caption*{Fonte: Autor.}
\end{figure}

Para contornar a ambiguidade gerada pela simetria no sistema, é possível adicionar uma terceira antena, de forma que não esteja alinhada com as duas primeiras.
Um exemplo é apresentado na \autoref{fig:AoA_5}.

\begin{figure}[htbp]
    \centering
    \caption{Possível disposição de matriz de antenas.}
        % \resizebox{!}{0.7\textheight}{%
    \begin{circuitikz}[american, voltage shift=0.5, line width=0.5,every node/.style={font = {\footnotesize\bfseries}}]

        \def\wavelength{4}
        \def\d{0.5*\wavelength}


        \def\antennaAngle{240}
        \def\closeRange{9}
        \def\farRange{\closeRange+13}

        \def\centerarc[#1](#2)(#3:#4:#5)(#6)(#7)% Syntax: [draw options] (center) (initial angle:final angle:radius)
        { \draw[#1] ($(#2)+({#5*cos(#3)},{#5*sin(#3)})$) arc (#3:#4:#5) node[midway,anchor=#7] {#6}; }


        \coordinate (O) at (0,0);
        \coordinate (antenna) at (\antennaAngle:\closeRange*\wavelength);
        % \draw [help lines, dashed] (-5,-3) grid (5,3); % desenha grid
        % \draw [red] (O) node[draw,cross out] {}; % marca pont(0,0) 
        
        % \draw (-6.8,-4) rectangle (6.8,4);
        \clip (-6.8,-4) rectangle (6.8,4);

        % \draw[thick]
        %     (antenna) node[dinantenna]{}
        % ;
        
        \foreach \x [evaluate={\y=int((\x+\closeRange));\z=int((\x+\closeRange)*\wavelength);}] in {-3,...,3} {
            \draw [black, thin] 
            (antenna) ++ (\antennaAngle:-\z)
                % node[anchor=west, font = {\footnotesize\bfseries}]{$\y\lambda$}
            ($(antenna) + (\antennaAngle:-\z) + ({10*cos(\antennaAngle+90)},{10*sin(\antennaAngle+90)})$)
                -- 
            ($(antenna) + (\antennaAngle:-\z) - ({10*cos(\antennaAngle+90)},{10*sin(\antennaAngle+90)})$);
            % \draw [gray, thin] (antenna) circle (\z);
        }
        
        \draw[thick]
            (0,0)  node[Green, dinantenna] (A00) {}
            (0,\d) node[Blue,  dinantenna] (A0d) {}
            (\d,0) node[Red,   dinantenna] (Ad0) {}
        ;

        \draw[very thick, dashed, -latex]
            (A00) ++(-\d,0) coordinate(aux) ++(\antennaAngle:0.5*\d) -- ++(\antennaAngle:-\d)
        ;

        
        % \draw[Goldenrod, domain=-8:8, samples=100] plot[shift={(aux)}, rotate=\antennaAngle]({\x},{sin(\x * pi * 2 / \wavelength r)});

        \draw[thin, Red, opacity=0.5]
            (A00) ++ ($({\antennaAngle-90}:{\d*sin(\antennaAngle)})$) -- (Ad0) -- (A00)

            ($({\antennaAngle-90}:{\d*sin(\antennaAngle)})$) 
              ++(\antennaAngle+180:0.25)
            --++(\antennaAngle-90:0.25)
            --++(\antennaAngle:0.25)
        ;


        \centerarc[thin, Red, opacity=0.5](A00)(\antennaAngle+90:360:0.4)($\beta_{0d}$)(north)

        
        \draw[thin, Blue, opacity=0.5]
            (A00) ++ ($({\antennaAngle+90}:{\d*cos(\antennaAngle)})$) -- (A0d) -- (A00)

            ($({\antennaAngle+90}:{\d*cos(\antennaAngle)})$) 
              ++(\antennaAngle+180:0.25)
            --++(\antennaAngle+90:0.25)
            --++(\antennaAngle:0.25)    
        ;

        \centerarc[thin, Blue, opacity=0.5](A00)(\antennaAngle-90:90:0.4)($\beta_{d0}$)(north east)

        \draw[latex-latex]
            ($(A00)+(0,1)$) -- ($(Ad0)+(0,1)$) node [midway, fill=white] {$d$}
        ;
        


        % \foreach \x in {0,60,...,300} {
        %     \draw[thick] (\x:1 cm) -- (\x + 60:1 cm);
            
        %     \draw (\x + 30:1.732 cm) node[Gray, circ]{};
        %     \draw[Gray, dashed] (\x:1 cm) -- ++(\x: 0.9cm);
        %     \draw[Gray, dotted]
        %     %     % (\x:1 cm) arc (\x+240:\x+180:1cm)
        %         (\x:1 cm) arc [start angle=\x+120, delta angle=110, radius=1cm]
        %         (\x:1 cm) arc [start angle=\x+120, delta angle=-50, radius=1cm]
        %     ;
        % }
    
        % \draw (0,0) node [circ] {} node [below left,font={\scriptsize\bfseries}] {BS};
        % \draw[thick, densely dotted] (0,0) circle (1cm);
        
        % \draw[-latex] (0,0) -- (0:1cm) node[midway, below] {$R_c$};
        % \draw[-latex] (0,0) -- (90:0.866cm) node[midway, left] {$R$};
            
    \end{circuitikz}
  % }


    % \includegraphics{../pictures/AoA_5.pdf}
    \caption*{Fonte: Autor.}
    \label{fig:AoA_5}
\end{figure}

Para obter o valor de defasagem entre antenas, é interessante representar o sinal recebido como um valor complexo.
Uma forma de obter o valor complexo é realizando a integração de período completo no produto do sinal recebido na antena por uma senoide ou cossenoide de mesma frequência.
Nas Equações \ref{eq:S} e \ref{eq:C}, $S$ e $C$ são respectivamente proporcionais às componentes em fase e em quadratura do sinal $w$ recebido na antena, $k$ é uma função auxiliar que garante frequência correta em todos os operandos.
É válido notar que $w$ e $k$ são funções de várias variáveis.

\begin{equation}\label{eq:S}
    S = \int_0^T w(t) \cdot \sin(k(t)) \partial t
\end{equation}

\begin{equation}\label{eq:C}
    C = \int_0^T w(t) \cdot \cos(k(t)) \partial t
\end{equation}

Finalmente, na \autoref{eq:Z}, $Z$ é o valor complexo associada ao sinal recebido em na antena.

\begin{equation}\label{eq:Z}
    Z_{x,y} = 2\cdot(S + \imath C)
\end{equation}

A defasagem entre um par de antenas é dado pelo ângulo do valor resultante da multiplicação do valor complexo da primeira antena pelo complexo conjugado da segunda antena, conforme apresentado na \autoref{eq:phase}.

\begin{equation}\label{eq:phase}
    \Delta\Phi_{x,y} = \arg(Z_{0,0}\cdot Z^*_{x,y})
\end{equation}

Com este valor é possível estimar valor de $\beta$ no intervalo $\SI{0}{\degree} \leq \beta \leq \SI{180}{\degree}$.

Utilizando a terceira antena perpendicular ao primeiro par e alinhada com uma das antenas iniciais, conforme \autoref{fig:AoA_5}, é possível estimar o valor de $\beta$ no intervalo $\SI{0}{\degree} \leq \beta \leq \SI{360}{\degree}$.
Cada par de antena pode indicar o valor da coordenada geométrica associada ao eixo que a caracteriza, conforme indicado na \autoref{eq:componente}.
Essa propriedade somente é válida nessa geometria.

\begin{equation}\label{eq:componente}
    \text{componente}_{x,y} = -\frac{\Delta_{x,y}}{\pi}\cdot\frac{\cancel{\lambda}}{\cancel{d \cdot 2}} = -\frac{\Delta_{x,y}}{\pi}
\end{equation}


\pagebreak

\begin{figure}[H]
    \centering
    \input{../pictures/antennas}
    \caption{Agora sim 0}
\end{figure}

\begin{figure}[H]
    \centering
    \input{../pictures/AoA_2_alt}
    \caption{Agora sim}
\end{figure}

\begin{equation} % Comprimento de onda
    \lambda = \frac{c}{f} = \frac{2\pi \cdot c}{\omega}
\end{equation}

\begin{equation} % Distância entre par de antenas
    d = \frac{\lambda}{2}
\end{equation}

\begin{equation} % Diâmetro do circulo que circunscreve poligono de antenas
	\rho = \frac{d}{2\cdot \sin\left(\frac{\pi}{N_\text{ant}}\right)}
\end{equation}

\begin{equation} % Índices das antenas
	k = \left\{1, 2, \dotsc, N_\text{ant}\right\}
\end{equation}

\begin{equation} % Coordenada da antena
	A_k =
    \rho \cdot \exp\left(\imath\cdot k \cdot \frac{2\pi}{N_\text{ant}}\right) =
    \left( x_{A_k},~ y_{A_k} \right) =
    \left( \operatorname{\mathcal{Re}}\left( A_k \right), ~\operatorname{\mathcal{Im}}\left( A_k \right) \right)
\end{equation}

\begin{equation} % Período do sinal
    T = \frac{2\pi}{\omega} = \frac{1}{f}
\end{equation}

\begin{equation} % Relação trigonometrica de defasagem e angulo do sinal
    d \cdot \cos\left(\beta\right) = \lambda \cdot \frac{\Delta\Phi}{2 \pi}
\end{equation}

\begin{equation} % In phase
    I_k = \int\limits_0^T \cos\left(\omega\cdot\tau\right) \cdot w\left( \tau, ~x_{A_k}, ~y_{A_k} \right) \partial \tau
\end{equation}

\begin{equation} % Out of phase
    Q_k = \int\limits_0^T \sin\left(\omega\cdot\tau\right) \cdot w\left( \tau, ~x_{A_k}, ~y_{A_k} \right) \partial \tau
\end{equation}

\begin{equation} % Fase na antena
    Z_k = \frac{\omega}{\pi}\cdot\left(I_k + \imath Q_k\right)
\end{equation}

\begin{equation} % Defasagem entre antenas
    \Delta\Phi_{k} =
    \Phi_{k} - \Phi_{k+1} =
    \arg\left(Z_{k}\right) - \arg\left(Z_{k+1}\right) =
    \arg\left(Z_{k} \cdot \overline{Z_{k+1}}\right)
\end{equation}

\begin{equation} % Angulo do sinal em relação ao par de antenas
    \beta_{k} = \pm \arccos\left(\frac{\cancel{\lambda}}{\cancel{d}}\cdot\frac{\Delta\Phi_{k}}{\cancel{2}\pi}\right)
\end{equation}

\begin{equation} % Angulo do par de antenas
	\alpha_{k} = \arg\left( A_{k} - A_{k+1} \right)
\end{equation}

\begin{equation} % Conjunto de angulos calculados
	\theta_{\pm k} = \alpha_{k}\pm \beta_{k}
\end{equation}

\begin{equation} % Conjunto de angulos calculados
	\Theta = \left\{\theta_{\pm k}=\alpha_{k}\pm \beta_{k} ~\middle\vert~ \forall k\right\}
\end{equation}

\begin{equation} % range_angle
    \delta = \frac{\pi}{4\cdot N_\text{ant}}
\end{equation}

\begin{equation} % Angulos normalizados
    \Theta_{\left\lfloor\bullet\right\rceil} = \left\{\frac{\left\lfloor\theta\cdot \delta\cdot 10 \right\rceil}{\delta\cdot 10} ~\middle\vert~ \forall \theta \in \Theta  \right\}
\end{equation}

\begin{equation} % Moda entre angulos normalizados
    \theta_\mathcal{M_o} = \operatorname{\mathcal{M_o}}\left( \Theta_{\left\lfloor\bullet\right\rceil}  \right)
\end{equation}

\begin{equation} % Filtra no intervalo
    \Theta_\text{F} = \left\{\theta \in \Theta  ~\middle\vert~
    \theta_\mathcal{M_o} - \delta \leq \theta \leq \theta_\mathcal{M_o} + \delta\right\}
\end{equation}

\begin{equation} % Mediana
    \theta_\text{AoA} = \widetilde{\Theta_\text{F}}
\end{equation}

\begin{figure}[H]
    \centering
    \input{../pictures/graph2.tex}
    \caption{Gráfico}
\end{figure}

\begin{figure}[H]
    \centering
    \input{../pictures/graph2_copy.tex}
    \caption{Gráfico}
\end{figure}



\section{Trabalhos relacionados}

Em seu trabalho, \citeauthor{horst2021localization} \cite{horst2021localization} analisa dois algoritmos de detecção de \ac{AoA}, realizando as análises em ambientes internos e utilizando matrizes de antenas.
O primeiro método analisado consiste em uma aproximação do ângulo, feita utilizando um software fornecido pela Texas Instruments, fabricante do hardware utilizado.
Já o segundo método, baseia-se na construção matemática do \ac{AoA} baseado na diferença de fase instantânea do sinal entre as antenas do sistema, uma abordagem semelhante à proposta neste trabalho.
Os resultados obtidos indicam que o método de aproximação teve melhor acurácia nos valores de ângulo.

A proposta de \citeauthor{zeaiter:hal-03693641} \cite{zeaiter:hal-03693641} busca validar a performance da detecção de \ac{AoA} em ambiente fechado, realizando a análise em diferentes modulações, larguras de canal e fatores de espalhamento.
Também propõe que, ao combinar de seu algoritmo de localização de \ac{AoA} com a função de autocorrelação, é possível analisar os dados de dois sinais recebidos simultaneamente.

Outro trabalho de \citeauthor{zeaiter:hal-03932846} \cite{zeaiter:hal-03932846} consiste um uma aproximação do \ac{AoA} utilizando um método de autocorrelação em um sinal \ac{LoRa} de baixa potência.
Seu objetivo consiste em detectar o sinal \ac{LoRa} operando em transmissão de baixa potência, caso onde a vida útil da bateria do sistema transmissor é estendida.
O algoritmo apresentado busca picos de autocorrelação no sinal recebido, além de utilizar \ac{FFT} para denotá-los e melhorar a \ac{SNR}.
Quando um pico é detectado, o algoritmo é capaz de encontrar o \ac{AoA}.

% O trabalho de \citeauthor{aernouts2020combining} \cite{aernouts2020combining} combina o método de filtro de partículas às medidas TDoA e \ac{AoA} obtidas em ambiente urbano denso.
% A performance é analisada de maneira comparativa à estimativa de TDoA e a um trabalho anterior baseado em combinação de matrizes.
% Seus resultados indicam um erro médio estimado de \SI{199}{\metre} sem o \ac{AoA}.

\citeauthor{bnilam20172d} \cite{bnilam20172d} propõe uma técnica que, sem qualquer informação prévia de largura de banda, consegue estimar \ac{AoA} do sinal recebido.
O sistema proposto consiste em uma \ac{UCA} seguida de um filtro transversal, também utiliza de vetores especiais de largura de banda variável junto com um estimador de relação sinal-ruído térmico para determinar simultaneamente \ac{AoA} e largura de banda do sinal recebido.

Em outro trabalho, \citeauthor{bnilam2017adaptive} \cite{bnilam2017adaptive} estudam a possibilidade de estimar \ac{AoA} para transceptores de \ac{IoT} em ambiente interno.
Também propõe um modelo probabilístico adaptativo que opera no modelo de estimativa de \ac{AoA}, incrementando sua performance.
Seus resultados indicam que estes métodos superam a performance de modelos probabilísticos estáticos tradicionais, tanto em acurácia de localização quanto em estabilidade no valor obtido.

Neste trabalho, \citeauthor{bnilam2019low} \cite{bnilam2019low} propõe um dispositivo de baixo custo capaz de estimar o \ac{AoA}, de forma que seja viável sua utilização em dispositivos de \ac{IoT}.
O dispositivo consiste numa conversão de vários \ac{SDR} individuais de baixo custo num único \ac{SDR} com múltiplos canais de \ac{RF}.
Seus resultados experimentais indicam que o dispositivo é capaz de estimar valores de \ac{AoA} de forma estável e acurada.

A proposta de \citeauthor{bnilam2020angle} \cite{bnilam2020angle} neste trabalho consiste em um novo algoritmo para determinação de \ac{AoA} chamado ANGLE (\textit{ANGular Location Estimation}), baseado em modelos probabilísticos para a resposta do sinal recebido.
Sua proposta ainda sugere duas versões do método, para o caso de amostragem única e de decomposição de subespaço, como utilizado no algoritmo MUSIC (\textit{MUltiple SIgnal Classification}).

\citeauthor{bnilam2020lora} \cite{bnilam2020lora} apresenta neste trabalho uma abordagem mais amigável para estimativa de \ac{AoA} em redes \ac{LoRa}.
O sistema proposto, denominado LoRay (\ac{LoRa} array) é composto por hardware e software preparados para fazer a estimativa de \ac{AoA} em ambiente urbano, onde o sistema foi validado.
O hardware utilizado foi descrito em um trabalho anterior \cite{bnilam2019low}.
Este sistema apresentou resultados estáveis e acurados para estimativa de \ac{AoA} tanto nos casos \ac{LoS} e quanto nos \ac{NLoS}.

% \citeauthor{steckel2018low} \cite{steckel2018low}

% \citeauthor{du2018long} \cite{du2018long}

Em seu trabalho, \citeauthor{niculescu2003ad} \cite{niculescu2003ad} propõe métodos para detecção de posição e orientação em cada nó de uma rede \textit{ad hoc}.
A proposta parte de possíveis problemas relacionados a utilização de \ac{GPS} em ambiente fechado

