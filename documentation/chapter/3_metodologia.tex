\chapter{Metodologia}\label{cap:metodologia}

Neste capítulo, são explorados os métodos utilizados para a construção do trabalho proposto.


\section{Simulação}

A construção da simulação partiu de uma abordagem físico-matemática, definindo o sinal \ac{w} como uma função de onda relativa ao tempo e ao espaço, analisando seus valores incidindo em cada antena \ac{Ak} e comparando as defasagens \ac{DeltaPhi} entre os diferentes pares de antenas.
Para simplificar a construção da simulação, foram utilizadas funções paramétricas, descritas na presente seção.

\subsection{Parâmetros envolvidos}

Com o objetivo de garantir a coerência entre as partes da simulação, vários parâmetros foram utilizados, definindo detalhes em relação às operações matemáticas e às formas de registro dos valores calculados.
Estes parâmetros são divididos entre os que recebem valores numéricos, booleanos ou matrizes numéricas.

Os parâmetros numéricos são:
\begin{itemize}
	\item \lstinline|amp_w|, amplitude desejada para o sinal;
	\item \lstinline|ang_w|, direção do emissor do sinal, equivalente ao ângulo \ac{thetaAoA} de chegada do sinal em relação à malha de antenas;
	\item \lstinline|angle_Z_A_x_B|, ângulo relativo \ac{betak} para par de antenas;
	\item \lstinline|d|, distância \ac{d} entre par de antenas da malha;
	\item \lstinline|choose_angle|, ângulo \ac{thetaAoA} final calculado pelo sistema;
	\item \lstinline|interval|, indica os limites para a geração de imagem da simulação;
	\item \lstinline|lambda_w|, comprimento de onda \ac{lambda};
	\item \lstinline|N_antenas|, quantidade \ac{Nant} de antenas da malha;
	\item \lstinline|omega_w|, frequência angular \ac{omega};
	\item \lstinline|phase_w|, fase $\phi$ do sinal no emissor;
	\item \lstinline|Rho|, raio \ac{rho} do polígono que dispõe as antenas na malha;
	\item \lstinline|r_w|, distância que o emissor de sinal está da coordenada $(0,~0)$ do sistema;
	\item \lstinline|range_step|, largura em graus do passo na simulação.
	\item \lstinline|resolution|, relativo à quantidade de pontos utilizados na aproximação numérica do cálculo de correlação;
	\item \lstinline|SNR|, valor da \ac{SNR} linear;
	\item \lstinline|SNR_dB|, valor da \ac{SNR} em \si{\decibel};
	\item \lstinline|t_w|, tempo $t$ associado ao instante de aferição do sinal;
	\item \lstinline|x_w| ou \lstinline|y_w|, coordenada $x$ ou $y$ no espaço para aferição do sinal \ac{w};
	\item \lstinline|Z_antenna|, \lstinline|Z_antenna_A| ou \lstinline|Z_antenna_B|, valor complexo, coordenada de antena;
	\item \lstinline|Z_phase_A| ou \lstinline|Z_phase_B|, valor complexo, fase \ac{Phik} de antena;
\end{itemize}

Os parâmetros booleanos são:
\begin{itemize}
	\item \lstinline|ATT|, indica se o sinal contará com atenuação por distância;
	\item \lstinline|C|, indica a utilização de componente cossenoidal na construção do sinal;
	\item \lstinline|CHG_PHI|, indica se a fase geral do sinal deve mudar ao longo da simulação;
	\item \lstinline|CHG_R|, indica se a distância do emissor do sinal deverá mudar ao longo da simulação;
	\item \lstinline|CHG_THETA|, indica se o ângulo de origem do sinal deverá mudar ao longo da simulação;
	\item \lstinline|NOISE|, indica se o sinal contará com ruído;
	\item \lstinline|S|, indica a utilização de componente senoidal na construção do sinal;
	\item \lstinline|S_DAT|, indica se os pontos gerados pela simulação deverão ser salvos;
	\item \lstinline|S_GIF|, indica se a imagem gerada pela simulação deverá ser salva;
\end{itemize}

Os parâmetros de matrizes numéricas são:
\begin{itemize}
	\item \lstinline|ant_array|, coordenadas das antenas da malha;
	\item \lstinline|delta_A_x_B_array|, contendo o ângulo \ac{thetak} calculado por $\ac{alphak} + \ac{betak}$ aferido para cada par de antenas da malha;
	\item \lstinline|delta_B_x_A_array|, contendo o ângulo \ac{thetak} calculado por $\ac{alphak} - \ac{betak}$ aferido para cada par de antenas da malha;
	\item \lstinline|Z_phase_array|, matriz de valores numéricos complexos, contendo o sinal complexo aferido para cada antena da malha;
	\item \lstinline|z_plot|, estado corrente do sinal no espaço, utilizado na geração de imagem da simulação;
	\item \lstinline|Z_x_array|, valores complexos, contendo a defasagem \ac{DeltaPhi} aferido para cada par de antenas na malha;
\end{itemize}


\subsection{Funções auxiliares}

% argument_r
A primeira função a ser definida é \lstinline|argument_r|, que opera como auxiliar para normalização de argumento para as funções trigonométricas utilizadas nas análises, garantindo coerência em frequência angular e coordenadas espaciais.
Seus argumentos são, respectivamente, \lstinline|x_w|, \lstinline|y_w|, \lstinline|t_w|, \lstinline|ang_w|, \lstinline|r_w|, \lstinline|phase_w|, \lstinline|lambda_w| e \lstinline|omega_w|.
O \autoref{cod:argument_r} apresenta uma versão simplificada da função \lstinline|argument_r| desenvolvida.

\begin{lstfloat}[htbp]
	\centering
	\lstinputlisting[
			basicstyle=\ttfamily\small\setstretch{1},
			label=cod:argument_r,
			caption={Função \lstinline|argument_r|, simplificada.}
		]{../code/argument_r_alt.m}
	\caption*{Fonte: Autor.}
\end{lstfloat}

% Carregar bibliotecas

% ref_sin e ref_cos
Para determinar a fase do sinal \ac{w}, incidente em cada antena \ac{Ak}, calcula-se a correlação deste sinal com sinais de referência seno e cossenos, fornecidos respectivamente pelas funções \lstinline|ref_sin| e \lstinline|ref_cos|.
As duas funções recebem os mesmos argumentos, e estes são, respectivamente, \lstinline|t_w| e \lstinline|omega_w|.
Ambos os casos utilizam a função \lstinline|argument_r| para garantir coerência de frequência com o sinal incidente.
Os Códigos \ref{cod:ref_cos} e \ref{cod:ref_sin} apresentam, respectivamente, versões simplificadas das funções \lstinline|ref_cos| e \lstinline|ref_sin| desenvolvidas.


\begin{lstfloat}[htbp]
	\centering
	\lstinputlisting[
			basicstyle=\ttfamily\small\setstretch{1},
			label=cod:ref_cos,
			caption={Função \lstinline|ref_cos|, simplificada.}
		]{../code/ref_cos_alt.m}
	\caption*{Fonte: Autor.}
\end{lstfloat}

\begin{lstfloat}[htbp]
	\centering
	\lstinputlisting[
			basicstyle=\ttfamily\small\setstretch{1},
			label=cod:ref_sin,
			caption={Função \lstinline|ref_sin|, simplificada.}
		]{../code/ref_sin_alt.m}
	\caption*{Fonte: Autor.}
\end{lstfloat}


% signal_r
A próxima função construída foi \lstinline|signal_r|, que calcula o valor do sinal \ac{w} numa coordenada $(x,~y)$ e um instante $t$.
Considera-se que o sinal é composto pela soma de seno e cosseno, e que são determinadas a distância e a direção de sua fonte emissora.
Também é possível definir amplitude e fase na origem, além da presença de atenuação e ruído do tipo \ac{AWGN}.
Seus argumentos são, respectivamente, \lstinline|x_w|, \lstinline|y_w|, \lstinline|t_w|, \lstinline|amp_w|, \lstinline|ang_w|, \lstinline|r_w|, \lstinline|phase_w|, \lstinline|lambda_w|, \lstinline|omega_w|, \lstinline|S|, \lstinline|C|, \lstinline|NOISE|, \lstinline|SNR_dB| e \lstinline|ATT|.
É utilizada a função \lstinline|argument_r| para garantir coerência de frequência entre as componentes e com os sinais de referência utilizados no cálculo de correlação.
Para implementação do ruído, foi utilizada a função \lstinline|awgn|, no GNU Octave, é necessária a biblioteca \textit{communications}, porém para o MATLAB, não é necessário carregar bibliotecas \cite{awgnOctave, awgnMATLAB}.
O \autoref{cod:signal_r} apresenta uma versão simplificada da função \lstinline|signal_r| desenvolvida.

\begin{lstfloat}[htbp]
	\centering
	\lstinputlisting[
			basicstyle=\ttfamily\small\setstretch{1},
			label=cod:signal_r,
			caption={Função \lstinline|signal_r|, simplificada.}
		]{../code/signal_r_alt.m}
	\caption*{Fonte: Autor.}
\end{lstfloat}

% phase_z
A função \lstinline|phase_z| calcula o valor complexo de fase \ac{Zk} para a antena \ac{Ak} através da correlação pelos sinais de seno e cosseno.
Seus argumentos são, respectivamente, \lstinline|t|, \lstinline|Z_antenna|, \lstinline|amp_w|, \lstinline|ang_w|, \lstinline|r_w|, \lstinline|phase_w|, \lstinline|lambda_w|, \lstinline|omega_w|, \lstinline|S|, \lstinline|C|, \lstinline|NOISE|, \lstinline|SNR_dB| e \lstinline|ATT|.
O \autoref{cod:phase_z} apresenta uma versão simplificada da função \lstinline|phase_z| desenvolvida.

\begin{lstfloat}[htbp]
	\centering
	\lstinputlisting[
			basicstyle=\ttfamily\small\setstretch{1},
			label=cod:phase_z,
			caption={Função \lstinline|phase_z|, simplificada.}
		]{../code/phase_z_alt.m}
	\caption*{Fonte: Autor.}
\end{lstfloat}

% dephase_A_to_B
O cálculo do valor complexo de defasagem \ac{DeltaPhi}, o ângulo relativo \ac{betak} e o ângulo \ac{alphak} entre um par de antenas é realizado pela função \lstinline|dephase_A_to_B|.
Seus argumentos são, respectivamente, \lstinline|Z_phase_A| e \lstinline|Z_phase_B|.
O \autoref{cod:dephase_A_to_B} apresenta uma versão simplificada da função \lstinline|dephase_A_to_B| desenvolvida.

\begin{lstfloat}[htbp]
	\centering
	\lstinputlisting[
			basicstyle=\ttfamily\small\setstretch{1},
			label=cod:dephase_A_to_B,
			caption={Função \lstinline|dephase_A_to_B|, simplificada.}
		]{../code/dephase_A_to_B_alt.m}
	\caption*{Fonte: Autor.}
\end{lstfloat}

% deltas_A_B
Os ângulos \ac{thetak} para um par de antenas são calculados pela função \lstinline|deltas_A_B|.
Seus argumentos são, respectivamente, \lstinline|angle_Z_A_x_B|, \lstinline|Z_antenna_A| e \lstinline|Z_antenna_B|.
O \autoref{cod:deltas_A_B} apresenta uma versão simplificada da função \lstinline|deltas_A_B| desenvolvida.

\begin{lstfloat}[htbp]
	\centering
	\lstinputlisting[
			basicstyle=\ttfamily\small\setstretch{1},
			label=cod:deltas_A_B,
			caption={Função \lstinline|deltas_A_B|, simplificada.}
		]{../code/deltas_A_B_alt.m}
	\caption*{Fonte: Autor.}
\end{lstfloat}


% isoctave
A última função auxiliar desenvolvida foi \lstinline|isoctave|, que confere se a corrente simulação está sendo executada no GNU Octave, retornando um valor binário e não recebe qualquer parâmetro.
O \autoref{cod:isoctave} apresenta uma versão simplificada da função \lstinline|isoctave| desenvolvida.

\begin{lstfloat}[htbp]
	\centering
	\lstinputlisting[
			basicstyle=\ttfamily\small\setstretch{1},
			label=cod:isoctave,
			caption={Função \lstinline|isoctave|, simplificada.}
		]{../code/isoctave_alt.m}
	\caption*{Fonte: Autor.}
\end{lstfloat}



\subsection{Função de cálculo para \acs{AoA}}

% calc_AoA_polygon
A primeira grande função desenvolvida foi \lstinline|calc_AoA|, que é responsável pelo cálculo geral da simulação.
Inicialmente são calculadas as coordenadas das \ac{Nant} antenas e, em sequência, os valores de fase do sinal incidente \ac{w} em cada antena \ac{Ak}, então as defasagens entre os pares de antenas e finalmente a seleção do valor mais provável para \ac{thetaAoA}.
Seus argumentos são, respectivamente, \lstinline|amp_w|, \lstinline|ang_w|, \lstinline|r_w|, \lstinline|phase_w|, \lstinline|lambda_w|, \lstinline|omega_w|, \lstinline|S|, \lstinline|C|, \lstinline|NOISE|, \lstinline|SNR_dB|, \lstinline|ATT|, \lstinline|resolution|, \lstinline|d| e \lstinline|N_antenas|.
Nessa função também são definidas três subfunções auxiliares \lstinline|phase_z|, \lstinline|dephase_A_to_B| e \lstinline|deltas_A_B|.
O \autoref{cod:calc_AoA} apresenta uma versão simplificada da função \lstinline|calc_AoA| desenvolvida.
A \autoref{fig:AoA:fluxograma} apresenta a sequências de operações realizadas pela função.

\begin{lstfloat}[htbp]
	\centering
	\lstinputlisting[
			basicstyle=\ttfamily\small\setstretch{1},
			label=cod:calc_AoA,
			caption={Função \lstinline|calc_AoA|, simplificada.}
		]{../code/calc_AoA_alt.m}
	\caption*{Fonte: Autor.}
\end{lstfloat}


\begin{figure}[htbp]
    \centering
    \caption{Fluxograma de operações da função \lstinline|calc_AoA|.}
    \begin{tikzpicture}[node distance=1.75cm]
    \node (s) [startstop] {Início};
    \node (a1) [below of=s, io] {Aferir fase nas antenas};
    \node (a2) [below of=a1, process] {Calcular defasagem};
    \node (a3) [below of=a2, process] {Ajustar ângulos por pares};
    \node (a4) [below of=a3, process] {Quantizar valores};
    \node (a5) [right of=s, process, node distance=6cm] {Calcular moda};
    \node (a6) [below of=a5, process] {Filtrar valores plausíveis};
    \node (a7) [below of=a6, process] {Calcular mediana};
    \node (a8) [below of=a7, process] {Retornar resultado};
    \node (f)  [below of=a8, startstop] {Fim};

    % \node [anchor=south west, font = {\scriptsize\bfseries}, Red] at (a4.east) {N};
    % \node [anchor=south east, font = {\scriptsize\bfseries}, Green] at (a4.west) {S};

	\node (x1) [below of = s, ghost] {};

    \node [fit=(x1)] (fita1) {}; \draw [niceBrace] ([yshift=2.5pt]fita1.south west) -- ([yshift=-2.5pt]fita1.north west);
    \node [fit=(a2)] (fita2) {}; \draw [niceBrace] ([yshift=2.5pt]fita2.south west) -- ([yshift=-2.5pt]fita2.north west);
    \node [fit=(a3)] (fita3) {}; \draw [niceBrace] ([yshift=2.5pt]fita3.south west) -- ([yshift=-2.5pt]fita3.north west);
    \node [fit=(a4)] (fita4) {}; \draw [niceBrace] ([yshift=2.5pt]fita4.south west) -- ([yshift=-2.5pt]fita4.north west);

    \node [fit=(a5)] (fita5) {}; \draw [niceBrace] ([yshift=-2.5pt]fita5.north east) -- ([yshift=2.5pt]fita5.south east);
    \node [fit=(a6)] (fita6) {}; \draw [niceBrace] ([yshift=-2.5pt]fita6.north east) -- ([yshift=2.5pt]fita6.south east);
    \node [fit=(a7)] (fita7) {}; \draw [niceBrace] ([yshift=-2.5pt]fita7.north east) -- ([yshift=2.5pt]fita7.south east);
    \node [fit=(a8)] (fita8) {}; \draw [niceBrace] ([yshift=-2.5pt]fita8.north east) -- ([yshift=2.5pt]fita8.south east);

	\node [left of=x1, auxBlock, anchor=east] {\ac{Zk}};
	\node [left of=a2, auxBlock, anchor=east] {\ac{DeltaPhi}};
	\node [left of=a3, auxBlock, anchor=east] {\ac{Theta}};
	\node [left of=a4, auxBlock, anchor=east] {\ac{ThetaQuanti}};

    \node [right of=a5, auxBlock] {\ac{thetaMo}};
    \node [right of=a6, auxBlock] {\ac{ThetaFiltro}};
    \node [right of=a7, auxBlock] {$\widetilde{\ac{ThetaFiltro}}$};
    \node [right of=a8, auxBlock] {\ac{thetaAoA}};

    \draw [arrow] (s) -- (a1);
    \draw [arrow] (a1) -- (a2);
    \draw [arrow] (a2) -- (a3);
    \draw [arrow] (a3) -- (a4);
    % \draw [arrow] (a4) -- (a5);
    \draw [arrow] (a4) -| ($(a4)!0.5!(a5)$) |- (a5);
    % \draw [arrow] (a4) -- ([xshift=-.5cm]a4.west) |- (a2);
    \draw [arrow] (a5) -- (a6);
    \draw [arrow] (a6) -- (a7);
    \draw [arrow] (a7) -- (a8);
    \draw [arrow] (a8) -- (f);
\end{tikzpicture}
    \label{fig:AoA:fluxograma}
    \caption*{Fonte: Autor.}
\end{figure}

\subsection{Função de geração saída visual}

% generate_fig_polygon

A segunda grande função desenvolvida foi \lstinline|generate_fig|, que constrói a animação de saída da simulação, formada por dois gráficos.
O primeiro gráfico, à esquerda nas animações geradas, apresenta a disposição das antenas, os valores de fase para cada uma delas, os valores de defasagem entre os pares de antenas, todos os possíveis valores de \ac{thetak}, e finalmente o valor real e o escolhido para \ac{thetaAoA}.
O segundo gráfico, à direita nas animações geradas, apresenta a disposição das antenas e uma representação do sinal \ac{w} no espaço exibido.
Os valores exibidos são calculados pela função \lstinline|calc_AoA|.
Seus argumentos são, respectivamente, \lstinline|z_plot|, \lstinline|x_w|, \lstinline|y_w|, \lstinline|ang_w|, \lstinline|lambda_w|, \lstinline|interval|, \lstinline|Rho|, \lstinline|choose_angle|, \lstinline|ant_array|, \lstinline|Z_phase_array|, \lstinline|Z_x_array|, \lstinline|delta_A_x_B_array| e \lstinline|delta_B_x_A_array|.
A \autoref{fig:example:simul_POLY_3_R_50} ilustra os gráficos gerados pela função \lstinline|generate_fig|.

\begin{figure}[htbp]
	\centering
	\caption{Exemplo de quadro da animação de saída da função \lstinline|generate_fig|.}
	\includegraphics[width=0.9\textwidth]{../pictures/simul_POLY_3_R_50.png}
	\label{fig:example:simul_POLY_3_R_50}
	\caption*{Fonte: Autor, saída gráfica disponível em \href{https://github.com/HeckRodSav/TG/blob/main/documentation/pictures/POLY_3/simul_POLY_3_R_50.gif}{\underline{GitHub}}.}
\end{figure}

\subsection{Função geral da simulção}

% Definir função base xyt e variáveis

Finalmente a função responsável por juntar todas as partes é \lstinline|w_xyt|, a base para a simulação, ela invoca as funções \lstinline|calc_AoA| e \lstinline|generate_fig| com os devidos parâmetros, além de garantir que os arquivos gerados sejam salvos corretamente.
Seus argumentos são, respectivamente, \lstinline|NOISE|, \lstinline|ATT|, \lstinline|CHG_PHI|, \lstinline|CHG_R|, \lstinline|CHG_THETA|, \lstinline|S_GIF|, \lstinline|S_DAT|, \lstinline|SNR|, \lstinline|range_step| e \lstinline|N_antenas|.
Essa função também invoca a simulação do algoritmo de Gauss-Newton para as mesmas condições.

\section{Simulação do Algoritmo de Gauss-Newton}

Para análise comparatória, também foi desenvolvida uma simulação para o método de \ac{AoA} utilizando o algoritmo de Gauss-Newton, adaptando a proposta de \citeauthor{Horst2025BTLEAoA} \cite{Horst2025BTLEAoA}, referida na \autoref{sec:trabalhos_relacionados}.
Nesta adaptação, cada receptor é formado por um par de antenas, separados pela mesma distância \ac{d} utilizada na presente proposta, e o sistema como um todo utiliza \ac{Nant} receptores dispostos de maneira linear.
Considerando que a presente análise é planar, a componente de terceira dimensão foi desconsiderada.
O método realiza 5 iterações para convergir os resultados em cada ponto de análise.

O código desenvolvido está nos anexos, na \autoref{apdx:codigo:gn}.

