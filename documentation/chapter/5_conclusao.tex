\chapter{Conclusão}\label{cap:conclusao}

Foguetes de sondagem atmosférica podem pousar em qualquer lugar dentro do raio de alcance do voo, e recuperá-los pode ser inviável sem uma estratégia de localização eficaz.
Uma estratégia muito utilizada é a localização por \ac{GNSS}, por exemplo, o \ac{GPS}, contudo, esta ainda depende que a equipe de busca tenha acesso às próprias coordenadas geográficas e comunicação efetiva com o sistema embarcado do foguete.

O presente trabalho propõe a utilização de um sistema de localização baseada no sinal \ac{RF} emitido pelo veículo, e não pela informação contida nesse sinal, analisando as diferenças de defasagem do sinal incidente em uma malha de antenas, e assim calculando o \ac{AoA} deste sinal.

Durante a revisão bibliográfica, fundamentou-se a teoria aplicada nessa proposta.
Partindo de uma abordagem físico-matemática para analisar o sinal e a forma que a defasagem entre antenas pode ser utilizada para calcular a direção do emissor.
Também foram listadas algumas propostas que atuam de forma semelhante, analisando o sinal incidente em uma malha de antenas, que demonstra a relevância do método.
Além disso, foi realizado um breve levantamento sobre o método de direcionamento por coordenadas geográficas e o ângulo de \textit{bearing}, que guia uma equipe de busca ao veículo almejado.

Com base na fundamentação físico-matemática, foi desenvolvida uma simulação com o sinal de \ac{RF} incidente em uma malha de antenas.
Considerando que o foguete esteja em solo, assumiu-se um espaço de duas dimensões, porém ainda mantendo a possibilidade do veículo, emissor do sinal, se mover livremente em relação ao sistema de antenas.
As simulações foram construídas a partir dessas possibilidades, com o veículo circulando o sistema de antenas e se aproximando.

As simulações realizadas utilizaram geometrias de três, cinco e sete antenas.
A escolha dessas quantidades deu-se por questões geométricas, pois polígonos regulares com uma contagem par de lados terão lados paralelos, enquanto polígonos de lados ímpares não apresentam essa propriedade.
Os valores de R\textsuperscript{2} para as configurações simuladas foram todos acima de \qty{75}{\percent}, o que indica grande acurácia na modelagem proposta.
A comparação entre as geometrias estudadas indicou que o sistema com três antenas teve uma acurácia média menor que as geometrias com mais antenas.

% Comparando com os resultados válidos do algoritmo de \ac{GN}, os valores obtidos pelo sistema proposto se mostraram mais assertivos, particularmente em casos com ruído e atenuação.
% Possívelmente com mais iterações o algoritmo seria capaz de convegir para valores mais próximos aos corretos, porém os erros numéricos presentes podem ter corroborado com a divergência dos resultados.

As limitações impostas pelo uso de \textit{software} livre fizeram com que fossem utilizados métodos diferentes dos levantados na revisão bibliográfica, porém o método estatístico proposto se mostrou eficaz nos testes realizados.
Outra limitação foi relacionada à compatibilidade do código escrito, já que a sintaxe e algumas funções do MATLAB têm algumas diferenças em relação às equivalentes do GNU Octave.

Em conclusão, o trabalho aqui proposto se mostrou eficaz na determinação do \ac{AoA} para um sinal incidente em uma malha de antenas, garantindo um valor de R\textsuperscript{2} acima de \qty{75}{\percent} em todos os casos e valor médio de R\textsuperscript{2} acima de \qty{92}{\percent}.

Para trabalhos futuros, é possível analisar outras disposições de antenas no arranjo.
Apesar da formulação atual optar por polígonos regulares por simplicidade, a matemática utilizada calcula os ângulos de cada par de antenas individualmente, o que viabiliza outras disposições de antenas, que respeitem a distância entre antenas de um par.
Outras possibilidades englobam analisar mais classes de ruídos e até problemas de propagação multicaminho.
Além disso, a construção de um dispositivo eletrônico capaz de realizar a aferição de fase em uma malha de antenas poderá corroborar no levantamento de outros problemas a serem analisados e também validar a presente proposta.
