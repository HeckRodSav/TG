\newcommand{\impecavel}{%
{\color{slideBlue}m}%
{\color{slideCyan}a}%
{\color{slideTurquoise}r}%
{\color{slideGreen}a}%
{\color{slideYellow}v}%
{\color{slideOrange}i}%
{\color{slideRed}l}%
{\color{slidePink}h}%
{\color{slidePurple}o}%
{\color{slideBlue}s}%
{\color{slideCyan}o}%
}

\newcommand{\palette}{{\Huge
    {\color{slideBlue}$\blacksquare$}
    {\color{slideCyan}$\blacksquare$}
    {\color{slideTurquoise}$\blacksquare$}
    {\color{slideGreen}$\blacksquare$}
    {\color{slideYellow}$\blacksquare$}
    {\color{slideOrange}$\blacksquare$}
    {\color{slideRed}$\blacksquare$}
    {\color{slidePink}$\blacksquare$}
    {\color{slidePurple}$\blacksquare$}
}}

\lstset{
    language=matlab,
    morekeywords=[1]{deg2rad, rad2deg, mod},
    morekeywords=[2]{ref_sin, ref_cos, isoctave, generate_fig, signal_r, argument_r, calc_AoA, w_xyt, phase_z, dephase_A_to_B, deltas_A_B},
}


% \newcommand{\x}{$\bullet$}





% \newcommand\pgfmathsinandcos[3]{%
%   \pgfmathsetmacro#1{sin(#3)}%
%   \pgfmathsetmacro#2{cos(#3)}%
% }
% \newcommand\LongitudePlane[3][current plane]{%
%   \pgfmathsinandcos\sinEl\cosEl{#2} % elevation
%   \pgfmathsinandcos\sint\cost{#3} % azimuth
%   \tikzset{#1/.style={cm={\cost,\sint*\sinEl,0,\cosEl,(0,0)}}}
% }

% \newcommand\LatitudePlane[3][current plane]{%
%   \pgfmathsinandcos\sinEl\cosEl{#2} % elevation
%   \pgfmathsinandcos\sint\cost{#3} % latitude
%   \pgfmathsetmacro\yshift{\RadiusSphere*\cosEl*\sint}
%   \tikzset{#1/.style={cm={\cost,0,0,\cost*\sinEl,(0,\yshift)}}} %
% }
% \newcommand\NewLatitudePlane[4][current plane]{%
%   \pgfmathsinandcos\sinEl\cosEl{#3} % elevation
%   \pgfmathsinandcos\sint\cost{#4} % latitude
%   \pgfmathsetmacro\yshift{#2*\cosEl*\sint}
%   \tikzset{#1/.style={cm={\cost,0,0,\cost*\sinEl,(0,\yshift)}}} %
% }
% \newcommand\DrawLongitudeCircle[2][1]{
%   \LongitudePlane{\angEl}{#2}
%   \tikzset{current plane/.prefix style={scale=#1}}
%    % angle of "visibility"
%   \pgfmathsetmacro\angVis{atan(sin(#2)*cos(\angEl)/sin(\angEl))} %
%   \draw[current plane] (\angVis:1) arc (\angVis:\angVis+180:1);
%   \draw[current plane,opacity=0.4] (\angVis-180:1) arc (\angVis-180:\angVis:1);
% }
% \newcommand\DrawLongitudeArc[4][black]{
%   \LongitudePlane{\angEl}{#2}
%   \tikzset{current plane/.prefix style={scale=1}}
%   \pgfmathsetmacro\angVis{atan(sin(#2)*cos(\angEl)/sin(\angEl))} %
%   \pgfmathsetmacro\angA{mod(max(\angVis,#3),360)} %
%   \pgfmathsetmacro\angB{mod(min(\angVis+180,#4),360} %
%   \draw[current plane,#1,opacity=0.4] (#3:\RadiusSphere) arc (#3:#4:\RadiusSphere);
%   \draw[current plane,#1]  (\angA:\RadiusSphere) arc (\angA:\angB:\RadiusSphere);
% }%
% \newcommand\DrawLatitudeCircle[2][1]{
%   \LatitudePlane{\angEl}{#2}
%   \tikzset{current plane/.prefix style={scale=#1}}
%   \pgfmathsetmacro\sinVis{sin(#2)/cos(#2)*sin(\angEl)/cos(\angEl)}
%   % angle of "visibility"
%   \pgfmathsetmacro\angVis{asin(min(1,max(\sinVis,-1)))}
%   \draw[current plane] (\angVis:1) arc (\angVis:-\angVis-180:1);
%   \draw[current plane,opacity=0.4] (180-\angVis:1) arc (180-\angVis:\angVis:1);
% }

% \newcommand\DrawLatitudeArc[4][black]{
%   \LatitudePlane{\angEl}{#2}
%   \tikzset{current plane/.prefix style={scale=1}}
%   \pgfmathsetmacro\sinVis{sin(#2)/cos(#2)*sin(\angEl)/cos(\angEl)}
%   % angle of "visibility"
%   \pgfmathsetmacro\angVis{asin(min(1,max(\sinVis,-1)))}
%   \pgfmathsetmacro\angA{max(min(\angVis,#3),-\angVis-180)} %
%   \pgfmathsetmacro\angB{min(\angVis,#4)} %
%   \draw[current plane,#1,opacity=0.4] (#3:\RadiusSphere) arc (#3:#4:\RadiusSphere);
%   \draw[current plane,#1] (\angA:\RadiusSphere) arc (\angA:\angB:\RadiusSphere);
% }

% %% document-wide tikz options and styles

% \tikzset{%
%   >=latex, % option for nice arrows
%   inner sep=0pt,%
%   outer sep=2pt,%
%   mark coordinate/.style={inner sep=0pt,outer sep=0pt,minimum size=3pt,
%     fill=black,circle}%
% }




\usepackage{tikz-3dplot}

%Angle Definitions
%-----------------

%set the plot display orientation
%synatax: \tdplotsetdisplay{\theta_d}{\phi_d}
% \tdplotsetmaincoords{65}{110}
% \tdplotsetmaincoords{60}{135}
\tdplotsetmaincoords{54.736}{135}
% \tdplotsetmaincoords{60}{45}

% there's got to be a better way to do this.
\newcommand{\Normalize}[3]
{
    \pgfmathsetmacro{\normyn}{sqrt(#1*#1+#2*#2+#3*#3)}
    \pgfmathsetmacro{\normx}{#1/\normyn}\pgfmathsetmacro{\normy}{#2/\normyn}\pgfmathsetmacro{\normz}{#3/\normyn}
}

% calculate the counterclockwise angle of a vector of length 1 in the rotated xy plane.
\newcommand{\toAngle}[3]
{
    \tdplottransformrotmain{1}{0}{0}
    \pgfmathsetmacro\xa{acos(\tdplotresx *#1 + \tdplotresy* #2 + \tdplotresz* #3)}
    \tdplottransformrotmain{0}{1}{0}
    \pgfmathsetmacro\ya{acos(\tdplotresx *#1 + \tdplotresy* #2 + \tdplotresz* #3)}
    \pgfmathsetmacro\normySum{round(\xa+\ya)}
    \pgfmathsetmacro\normyDiff{round(\xa-\ya )}
    \ifthenelse{\lengthtest{\normySum pt = 270pt}} {
        \pgfmathsetmacro\normyAngle{\ya+90}
    }{
        \ifthenelse{\lengthtest{\normyDiff pt = -90pt}} {
            \pgfmathsetmacro\normyAngle{360-\xa }
        }{
            \pgfmathsetmacro\normyAngle{\xa }
        }
    }
}



\newcommand{\drawArc}[9]
{
    \Normalize{#1}{#2}{#3}
    \pgfmathsetmacro{\pax}{\normx}\pgfmathsetmacro{\pay}{\normy}    \pgfmathsetmacro{\paz}{\normz}

    \Normalize{#4}{#5}{#6}
    \pgfmathsetmacro{\pbx}{\normx}\pgfmathsetmacro{\pby}{\normy}    \pgfmathsetmacro{\pbz}{\normz}

    % take the cross product and normalize it
    \tdplotcrossprod(\pax,\pay,\paz)(\pbx,\pby,\pbz)

    % calculate the rotation that maps the z axis onto the cross product
    \tdplotsetrotatedcoords{atan2(\tdplotresy,\tdplotresx)}{atan2(sqrt(\tdplotresx*\tdplotresx+\tdplotresy*\tdplotresy),\tdplotresz)}{0.0}

    % calculate the counterclockwise angles from the rotated x axis to each vector, then order them increasing.
    \toAngle{\pax}{\pay}{\paz}
    \pgfmathsetmacro\xangle{\normyAngle}

    \toAngle{\pbx}{\pby}{\pbz}
    \pgfmathsetmacro\yangle{\normyAngle}

    \ifthenelse{\lengthtest{\xangle pt < \yangle pt}} {
        \pgfmathsetmacro\first{\xangle}
        \pgfmathsetmacro\second{\yangle}
    }{
        \pgfmathsetmacro\first{\yangle}
        \pgfmathsetmacro\second{\xangle}
    }
    % draw the arc at radius R from the first angle to the second

    \tdplotdrawarc[tdplot_rotated_coords, cmyk_R, very  thick, -latex,]{(0,0,0)}{#7}{\first }{\second }{#8}{#9}
}


% Blocos para fluxograma

% \tikzstyle{block} = [%
%     minimum width=4cm,
%     minimum height=1.2cm,
%     text centered,
% ]

% \tikzstyle{auxBlock} = [ %
%     block,
%     anchor=west,
%     node distance=2cm
% ]

% \tikzstyle{auxBlockPhantom} = [ %
%     block,
%     anchor=east,
%     node distance=2cm,
% ]

% \tikzstyle{ghost} = [%
%     block,
%     draw=none
% ]


% \tikzstyle{flowchart} = [%
%     block,
%     thick,
%     draw=black,
%     align=center,
% ]

% \tikzstyle{niceBrace} = [%
%     decoration={brace, raise=2pt, amplitude=5pt},
%     decorate,
%     thick
% ]

% \tikzstyle{startstop} = [%
%     flowchart,
%     rounded rectangle,
% ]

% \tikzstyle{io} = [%
%     flowchart,
%     trapezium,
%     trapezium stretches=true,
%     trapezium left angle=60,
%     trapezium right angle=120,
% ]

% \tikzstyle{process} = [%
%     flowchart,
%     rectangle,
% ]

% \tikzstyle{connection} = [%
%     flowchart,
%     circle,
%     minimum width=0.25cm,
%     minimum height=0.25cm,
% ]

% \tikzstyle{decision} = [%
%     flowchart,
%     diamond,
%     aspect=1.5
% ]

% \tikzstyle{loop} = [
%     flowchart,
%     signal,
%     signal to=west and east,
% ]

% \tikzstyle{arrow} = [thick,->,>=latex]
