\newcommand{\impecavel}{%
{\color{slideBlue}m}%
{\color{slideCyan}a}%
{\color{slideTurquoise}r}%
{\color{slideGreen}a}%
{\color{slideYellow}v}%
{\color{slideOrange}i}%
{\color{slideRed}l}%
{\color{slidePink}h}%
{\color{slidePurple}o}%
{\color{slideBlue}s}%
{\color{slideCyan}o}%
}

\newcommand{\palette}{{\Huge
    {\color{slideBlue}$\blacksquare$}
    {\color{slideCyan}$\blacksquare$}
    {\color{slideTurquoise}$\blacksquare$}
    {\color{slideGreen}$\blacksquare$}
    {\color{slideYellow}$\blacksquare$}
    {\color{slideOrange}$\blacksquare$}
    {\color{slideRed}$\blacksquare$}
    {\color{slidePink}$\blacksquare$}
    {\color{slidePurple}$\blacksquare$}
}}

\lstset{
    language=matlab,
    morekeywords=[1]{deg2rad, rad2deg, mod, polarplot, fullfile, isfolder, mkdir, arrayfun, awgn, copyfile, corr, complex, pkg, fflushframe2im, rgb2ind, imwrite},
    morekeywords=[2]{ref_sin, ref_cos, isoctave, generate_fig, signal_r, argument_r, calc_AoA_geometric, w_xyt, phase_z, dephase_A_to_B, deltas_A_B, bearing, copy_to_documentation, calc_AoA_music_ULA, calc_AoA_music_UCA, music_algorithm, music_spectrum, steering_linear, steering_circular, w_xyt_single, w_xyt_dat, w_xyt_auto, w_xyt_full_auto, settings},
}


% \newcommand{\x}{$\bullet$}


\newcommand{\coeficientesDedeterminacao}[1]{
    % \csvloop{separator=tab, respect percent=true, file={../data/POLY_3/simul_POLY_3_R_50_SNR_100.r2.dat}, command=\xdef\primeirinho{\csvcoli}\xdef\segundinho{\csvcoliii}}

    \edef\arquivo{#1}

    \csvreader[
        separator=tab,
        respect percent=true,
    ]{../data/\arquivo/simul_\arquivo_R_50.r2.dat}{}{
        \edef\cdGEOMETRICO{\csvcoli}
        \edef\cdMSC{\csvcoliii}
    }

    \csvreader[
        separator=tab,
        respect percent=true,
    ]{../data/\arquivo/simul_\arquivo_R_50_ATT.r2.dat}{}{
        \edef\cdAttGEOMETRICO{\csvcoli}
        \edef\cdAttMSC{\csvcoliii}
    }

    \csvreader[
        separator=tab,
        respect percent=true,
    ]{../data/\arquivo/simul_\arquivo_R_50_SNR_100.r2.dat}{}{
        \edef\cdCemGEOMETRICO{\csvcoli}
        \edef\cdCemMSC{\csvcoliii}
    }

    \csvreader[
        separator=tab,
        respect percent=true,
    ]{../data/\arquivo/simul_\arquivo_R_50_SNR_100_ATT.r2.dat}{}{
        \edef\cdCemAttGEOMETRICO{\csvcoli}
        \edef\cdCemAttMSC{\csvcoliii}
    }

    \csvreader[
        separator=tab,
        respect percent=true,
    ]{../data/\arquivo/simul_\arquivo_R_50_SNR_100.r2.dat}{}{
        \edef\cdCemGEOMETRICO{\csvcoli}
        \edef\cdCemMSC{\csvcoliii}
    }

    \csvreader[
        separator=tab,
        respect percent=true,
    ]{../data/\arquivo/simul_\arquivo_R_50_SNR_100_ATT.r2.dat}{}{
        \edef\cdCemAttGEOMETRICO{\csvcoli}
        \edef\cdCemAttMSC{\csvcoliii}
    }

    \csvreader[
        separator=tab,
        respect percent=true,
    ]{../data/\arquivo/simul_\arquivo_R_50_SNR_50.r2.dat}{}{
        \edef\cdCinquentaGEOMETRICO{\csvcoli}
        \edef\cdCinquentaMSC{\csvcoliii}
    }

    \csvreader[
        separator=tab,
        respect percent=true,
    ]{../data/\arquivo/simul_\arquivo_R_50_SNR_50_ATT.r2.dat}{}{
        \edef\cdCinquentaAttGEOMETRICO{\csvcoli}
        \edef\cdCinquentaAttMSC{\csvcoliii}
    }

    \csvreader[
        separator=tab,
        respect percent=true,
    ]{../data/\arquivo/simul_\arquivo_R_50_SNR_25.r2.dat}{}{
        \edef\cdVinteCincoGEOMETRICO{\csvcoli}
        \edef\cdVinteCincoMSC{\csvcoliii}
    }

    \csvreader[
        separator=tab,
        respect percent=true,
    ]{../data/\arquivo/simul_\arquivo_R_50_SNR_25_ATT.r2.dat}{}{
        \edef\cdVinteCincoAttGEOMETRICO{\csvcoli}
        \edef\cdVinteCincoAttMSC{\csvcoliii}
    }

    \csvreader[
        separator=tab,
        respect percent=true,
    ]{../data/\arquivo/simul_\arquivo_R_50_SNR_5.r2.dat}{}{
        \edef\cdCincoGEOMETRICO{\csvcoli}
        \edef\cdCincoMSC{\csvcoliii}
    }

    \csvreader[
        separator=tab,
        respect percent=true,
    ]{../data/\arquivo/simul_\arquivo_R_50_SNR_5_ATT.r2.dat}{}{
        \edef\cdCincoAttGEOMETRICO{\csvcoli}
        \edef\cdCincoAttMSC{\csvcoliii}
    }

    \csvreader[
        separator=tab,
        respect percent=true,
    ]{../data/\arquivo/simul_\arquivo_R_50_SNR_1.r2.dat}{}{
        \edef\cdUmGEOMETRICO{\csvcoli}
        \edef\cdUmMSC{\csvcoliii}
    }

    \csvreader[
        separator=tab,
        respect percent=true,
    ]{../data/\arquivo/simul_\arquivo_R_50_SNR_1_ATT.r2.dat}{}{
        \edef\cdUmAttGEOMETRICO{\csvcoli}
        \edef\cdUmAttMSC{\csvcoliii}
    }

    \begin{tabular}{@{}
        S[table-format = 2.0]
        S[table-format = 3.2, table-model-setup = \bfseries]
        S[table-format = 3.2, table-model-setup = \bfseries]
        S[table-format = 3.2, table-model-setup = \bfseries]
        S[table-format = 3.2, table-model-setup = \bfseries]
        @{}}
        \toprule
        {
            \multirow{2}{*}{\begin{tabular}[c]{@{}c@{}} \acs{SNR} \\ (\unit{\deci\bel})\end{tabular}}
        } & \multicolumn{4}{c}{R² (\unit{\percent}) } \\ \cmidrule(lr){2-5}
        { } &
            \multicolumn{2}{c}{Sem \acs{ATT}} & \multicolumn{2}{c}{Com \acs{ATT}} \\ \cmidrule(lr){2-3} \cmidrule(lr){4-5}
        { } & {
            Geométrico
        } & {
            \acs{MUSIC}\textsubscript{\acs{UCA}}
        } & {
            Geométrico
        } & {
            \acs{MUSIC}\textsubscript{\acs{UCA}}
        }\\\midrule
        \infinity & \bfseries \cdGEOMETRICO   & \bfseries \cdMSC   & \cdAttGEOMETRICO             & \cdAttMSC \\ % ∞
        20        & \cdCemGEOMETRICO          & \cdCemMSC          & \cdCemAttGEOMETRICO          & \cdCemAttMSC \\ % 100
        17        & \cdCinquentaGEOMETRICO    & \cdCinquentaMSC    & \cdCinquentaAttGEOMETRICO    & \cdCinquentaAttMSC \\ % 50
        14        & \cdVinteCincoGEOMETRICO   & \cdVinteCincoMSC   & \cdVinteCincoAttGEOMETRICO   & \cdVinteCincoAttMSC \\ % 25
        7         & \cdCincoGEOMETRICO        & \cdCincoMSC        & \cdCincoAttGEOMETRICO        & \cdCincoAttMSC \\ % 5
        0         & \bfseries \cdUmGEOMETRICO & \bfseries \cdUmMSC & \bfseries \cdUmAttGEOMETRICO & \bfseries \cdUmAttMSC \\ % 1
        \bottomrule
    \end{tabular}
}



% \newcommand\pgfmathsinandcos[3]{%
%   \pgfmathsetmacro#1{sin(#3)}%
%   \pgfmathsetmacro#2{cos(#3)}%
% }
% \newcommand\LongitudePlane[3][current plane]{%
%   \pgfmathsinandcos\sinEl\cosEl{#2} % elevation
%   \pgfmathsinandcos\sint\cost{#3} % azimuth
%   \tikzset{#1/.style={cm={\cost,\sint*\sinEl,0,\cosEl,(0,0)}}}
% }

% \newcommand\LatitudePlane[3][current plane]{%
%   \pgfmathsinandcos\sinEl\cosEl{#2} % elevation
%   \pgfmathsinandcos\sint\cost{#3} % latitude
%   \pgfmathsetmacro\yshift{\RadiusSphere*\cosEl*\sint}
%   \tikzset{#1/.style={cm={\cost,0,0,\cost*\sinEl,(0,\yshift)}}} %
% }
% \newcommand\NewLatitudePlane[4][current plane]{%
%   \pgfmathsinandcos\sinEl\cosEl{#3} % elevation
%   \pgfmathsinandcos\sint\cost{#4} % latitude
%   \pgfmathsetmacro\yshift{#2*\cosEl*\sint}
%   \tikzset{#1/.style={cm={\cost,0,0,\cost*\sinEl,(0,\yshift)}}} %
% }
% \newcommand\DrawLongitudeCircle[2][1]{
%   \LongitudePlane{\angEl}{#2}
%   \tikzset{current plane/.prefix style={scale=#1}}
%    % angle of "visibility"
%   \pgfmathsetmacro\angVis{atan(sin(#2)*cos(\angEl)/sin(\angEl))} %
%   \draw[current plane] (\angVis:1) arc (\angVis:\angVis+180:1);
%   \draw[current plane,opacity=0.4] (\angVis-180:1) arc (\angVis-180:\angVis:1);
% }
% \newcommand\DrawLongitudeArc[4][black]{
%   \LongitudePlane{\angEl}{#2}
%   \tikzset{current plane/.prefix style={scale=1}}
%   \pgfmathsetmacro\angVis{atan(sin(#2)*cos(\angEl)/sin(\angEl))} %
%   \pgfmathsetmacro\angA{mod(max(\angVis,#3),360)} %
%   \pgfmathsetmacro\angB{mod(min(\angVis+180,#4),360} %
%   \draw[current plane,#1,opacity=0.4] (#3:\RadiusSphere) arc (#3:#4:\RadiusSphere);
%   \draw[current plane,#1]  (\angA:\RadiusSphere) arc (\angA:\angB:\RadiusSphere);
% }%
% \newcommand\DrawLatitudeCircle[2][1]{
%   \LatitudePlane{\angEl}{#2}
%   \tikzset{current plane/.prefix style={scale=#1}}
%   \pgfmathsetmacro\sinVis{sin(#2)/cos(#2)*sin(\angEl)/cos(\angEl)}
%   % angle of "visibility"
%   \pgfmathsetmacro\angVis{asin(min(1,max(\sinVis,-1)))}
%   \draw[current plane] (\angVis:1) arc (\angVis:-\angVis-180:1);
%   \draw[current plane,opacity=0.4] (180-\angVis:1) arc (180-\angVis:\angVis:1);
% }

% \newcommand\DrawLatitudeArc[4][black]{
%   \LatitudePlane{\angEl}{#2}
%   \tikzset{current plane/.prefix style={scale=1}}
%   \pgfmathsetmacro\sinVis{sin(#2)/cos(#2)*sin(\angEl)/cos(\angEl)}
%   % angle of "visibility"
%   \pgfmathsetmacro\angVis{asin(min(1,max(\sinVis,-1)))}
%   \pgfmathsetmacro\angA{max(min(\angVis,#3),-\angVis-180)} %
%   \pgfmathsetmacro\angB{min(\angVis,#4)} %
%   \draw[current plane,#1,opacity=0.4] (#3:\RadiusSphere) arc (#3:#4:\RadiusSphere);
%   \draw[current plane,#1] (\angA:\RadiusSphere) arc (\angA:\angB:\RadiusSphere);
% }

% %% document-wide tikz options and styles

% \tikzset{%
%   >=latex, % option for nice arrows
%   inner sep=0pt,%
%   outer sep=2pt,%
%   mark coordinate/.style={inner sep=0pt,outer sep=0pt,minimum size=3pt,
%     fill=black,circle}%
% }




\usepackage{tikz-3dplot}

%Angle Definitions
%-----------------

%set the plot display orientation
%synatax: \tdplotsetdisplay{\theta_d}{\phi_d}
% \tdplotsetmaincoords{65}{110}
% \tdplotsetmaincoords{60}{135}
\tdplotsetmaincoords{54.736}{135}
% \tdplotsetmaincoords{60}{45}

% there's got to be a better way to do this.
\newcommand{\Normalize}[3]
{
    \pgfmathsetmacro{\normyn}{sqrt(#1*#1+#2*#2+#3*#3)}
    \pgfmathsetmacro{\normx}{#1/\normyn}\pgfmathsetmacro{\normy}{#2/\normyn}\pgfmathsetmacro{\normz}{#3/\normyn}
}

% calculate the counterclockwise angle of a vector of length 1 in the rotated xy plane.
\newcommand{\toAngle}[3]
{
    \tdplottransformrotmain{1}{0}{0}
    \pgfmathsetmacro\xa{acos(\tdplotresx *#1 + \tdplotresy* #2 + \tdplotresz* #3)}
    \tdplottransformrotmain{0}{1}{0}
    \pgfmathsetmacro\ya{acos(\tdplotresx *#1 + \tdplotresy* #2 + \tdplotresz* #3)}
    \pgfmathsetmacro\normySum{round(\xa+\ya)}
    \pgfmathsetmacro\normyDiff{round(\xa-\ya )}
    \ifthenelse{\lengthtest{\normySum pt = 270pt}} {
        \pgfmathsetmacro\normyAngle{\ya+90}
    }{
        \ifthenelse{\lengthtest{\normyDiff pt = -90pt}} {
            \pgfmathsetmacro\normyAngle{360-\xa }
        }{
            \pgfmathsetmacro\normyAngle{\xa }
        }
    }
}

\makeatletter

\pgfdeclareshape{foguete}{
  \nodeparts{}
  \savedmacro\fogueteparameters{%
	\pgfmathsetlengthmacro\unit{+0.125cm}%
    \addtosavedmacro\unit%
  }
  \anchor{center}{\pgfpointorigin}%
  \anchor{north}{\fogueteparameters%
    \pgfpoint{0*\unit}{4*\unit}}
  \anchor{south}{\fogueteparameters%
    \pgfpoint{0*\unit}{-4*\unit}}
  \anchor{east}{\fogueteparameters%
    \pgfpoint{0.5*\unit}{0\unit}}%
  \anchor{west}{\fogueteparameters%
    \pgfpoint{-0.5*\unit}{0\unit}}%
%   \anchor{north west}{\fogueteparameters%
%     \pgfpointpolar{135}{\radius*sqrt(2)}}
%   \anchor{south west}{\fogueteparameters%
%     \pgfpointpolar{225}{\radius*sqrt(2)}}
%   \anchor{north east}{\fogueteparameters%
%     \pgfpointpolar{45}{\radius+\outersep}}%
%   \anchor{south east}{\fogueteparameters%
%     \pgfpointpolar{315}{\radius+\outersep}}%
  \behindforegroundpath{%
    \fogueteparameters%
%     \pgfpathmoveto{\pgfpointpolar{135}{\radius}}%
%     \pgfpathlineto{\pgfpointpolar{135}{\radius*sqrt(2)}}%
%     \pgfpathmoveto{\pgfpointpolar{225}{\radius}}%
%     \pgfpathlineto{\pgfpointpolar{225}{\radius*sqrt(2)}}%
%     \pgfpathmoveto{\pgfpointpolar{0}{\radius}}%
%     \pgfpathlineto{\pgfpointpolar{0}{\radius*sqrt(2)}}%
    \pgfpathmoveto{\pgfpoint{0.5*\unit}{-2*\unit}}%
    \pgfpathlineto{\pgfpoint{0.5*\unit}{-4*\unit}}%
    \pgfusepath{stroke}
    %
    \pgfpathmoveto{\pgfpoint{-0.5*\unit}{-2*\unit}}%
    \pgfpathlineto{\pgfpoint{-0.5*\unit}{-4*\unit}}%
    \pgfusepath{stroke}
    %
    % \pgfpathmoveto{\pgfpoint{-0.5*\unit}{2*\unit}}%
    % \pgfpatharcaxes{0}{180}{\pgfpoint{0.5*\unit}{0*\unit}}{\pgfpoint{0*\unit}{2*\unit}}
    % % % \pgfpatharcaxes{180}{120}{\pgfpoint{1*\unit}{0*\unit}}{\pgfpoint{0*\unit}{3*\unit}}
    % % % \pgfpatharcaxes{60}{0}{\pgfpoint{1*\unit}{0*\unit}}{\pgfpoint{0*\unit}{3*\unit}}
    % \pgfusepath{stroke}
  }
  \backgroundpath{%
    \fogueteparameters%
    % \pgfsetplottension{0.9}
    % \pgfplothandlercurveto
    % \pgfplotstreamstart
    %     \pgfplotstreampoint{\pgfpoint{0*\unit}{4*\unit}}
    %     \pgfplotstreampoint{\pgfpoint{0.35*\unit}{3.2*\unit}}
    %     \pgfplotstreampoint{\pgfpoint{0.475*\unit}{2.5*\unit}}
    %     \pgfplotstreampoint{\pgfpoint{0.5*\unit}{2*\unit}}
    % \pgfplotstreamend
    \pgfpathmoveto{\pgfpoint{0.5*\unit}{2*\unit}}%
    \pgfpathlineto{\pgfpoint{0.5*\unit}{-2*\unit}}%
    \pgfpathlineto{\pgfpoint{1.5*\unit}{-2.5*\unit}}%
	\pgfpathlineto{\pgfpoint{1.5*\unit}{-3.5*\unit}}%
	\pgfpathlineto{\pgfpoint{0.5*\unit}{-4*\unit}}%
	\pgfpathlineto{\pgfpoint{-0.5*\unit}{-4*\unit}}%
	\pgfpathlineto{\pgfpoint{-1.5*\unit}{-3.5*\unit}}%
	\pgfpathlineto{\pgfpoint{-1.5*\unit}{-2.5*\unit}}%
	\pgfpathlineto{\pgfpoint{-0.5*\unit}{-2*\unit}}%
	\pgfpathlineto{\pgfpoint{-0.5*\unit}{2*\unit}}%
    \pgfpatharcaxes{180}{0}{\pgfpoint{0.5*\unit}{0*\unit}}{\pgfpoint{0*\unit}{2*\unit}}
    % \pgfplothandlercurveto
    % \pgfplotstreamstart
    %     \pgfplotstreampoint{\pgfpoint{-0.5*\unit}{2*\unit}}
    %     \pgfplotstreampoint{\pgfpoint{-0.475*\unit}{2.5*\unit}}
    %     \pgfplotstreampoint{\pgfpoint{-0.35*\unit}{3.2*\unit}}
    %     \pgfplotstreampoint{\pgfpoint{0*\unit}{4*\unit}}
    % \pgfplotstreamend
  }
}


\makeatother


\newcommand{\drawArc}[9]
{
    \Normalize{#1}{#2}{#3}
    \pgfmathsetmacro{\pax}{\normx}\pgfmathsetmacro{\pay}{\normy}    \pgfmathsetmacro{\paz}{\normz}

    \Normalize{#4}{#5}{#6}
    \pgfmathsetmacro{\pbx}{\normx}\pgfmathsetmacro{\pby}{\normy}    \pgfmathsetmacro{\pbz}{\normz}

    % take the cross product and normalize it
    \tdplotcrossprod(\pax,\pay,\paz)(\pbx,\pby,\pbz)

    % calculate the rotation that maps the z axis onto the cross product
    \tdplotsetrotatedcoords{atan2(\tdplotresy,\tdplotresx)}{atan2(sqrt(\tdplotresx*\tdplotresx+\tdplotresy*\tdplotresy),\tdplotresz)}{0.0}

    % calculate the counterclockwise angles from the rotated x axis to each vector, then order them increasing.
    \toAngle{\pax}{\pay}{\paz}
    \pgfmathsetmacro\xangle{\normyAngle}

    \toAngle{\pbx}{\pby}{\pbz}
    \pgfmathsetmacro\yangle{\normyAngle}

    \ifthenelse{\lengthtest{\xangle pt < \yangle pt}} {
        \pgfmathsetmacro\first{\xangle}
        \pgfmathsetmacro\second{\yangle}
    }{
        \pgfmathsetmacro\first{\yangle}
        \pgfmathsetmacro\second{\xangle}
    }
    % draw the arc at radius R from the first angle to the second

    \tdplotdrawarc[tdplot_rotated_coords, cmyk_R, very  thick, -latex,]{(0,0,0)}{#7}{\first }{\second }{#8}{#9}
}


% Blocos para fluxograma

\tikzstyle{block} = [%
    minimum width=5cm,
    minimum height=1.2cm,
    text centered,
]

\tikzstyle{auxBlock} = [ %
    block,
    minimum width=3cm,
    anchor=west,
    node distance=2cm
]

\tikzstyle{auxBlockPhantom} = [ %
    block,
    anchor=east,
    node distance=2cm,
]

\tikzstyle{ghost} = [%
    block,
    draw=none
]


\tikzstyle{flowchart} = [%
    block,
    thick,
    draw=black,
    align=center,
]

\tikzstyle{niceBrace} = [%
    decoration={brace, raise=2pt, amplitude=5pt},
    decorate,
    thick
]

\tikzstyle{startstop} = [%
    flowchart,
    rounded rectangle,
]

\tikzstyle{io} = [%
    flowchart,
    trapezium,
    trapezium stretches=true,
    trapezium left angle=60,
    trapezium right angle=120,
]

\tikzstyle{process} = [%
    flowchart,
    rectangle,
]

\tikzstyle{connection} = [%
    flowchart,
    circle,
    minimum width=0.25cm,
    minimum height=0.25cm,
]

\tikzstyle{decision} = [%
    flowchart,
    diamond,
    aspect=1.5
]

\tikzstyle{loop} = [
    flowchart,
    signal,
    signal to=west and east,
]

\tikzstyle{arrow} = [thick,->,>=latex]
