    % \resizebox{!}{0.7\textheight}{%
    \begin{circuitikz}[american, voltage shift=0.5, line width=0.5,every node/.style={font = {\footnotesize\bfseries}}]

        \def\wavelength{4}
        \def\d{0.5*\wavelength}


        \def\antennaAngle{120}
        \def\closeRange{9}
        \def\farRange{\closeRange+13}

        \def\centerarc[#1](#2)(#3:#4:#5)% Syntax: [draw options] (center) (initial angle:final angle:radius)
        { \draw[#1] ($(#2)+({#5*cos(#3)},{#5*sin(#3)})$) arc (#3:#4:#5) node[midway,anchor=west] {$\beta$}; }


        \coordinate (O) at (0,0);
        \coordinate (antenna) at (\antennaAngle:\closeRange*\wavelength);
        % \draw [help lines, dashed] (-3,-3) grid (3,3); % desenha grid
        % \draw [red] (O) node[draw,cross out] {}; % marca pont(0,0) 
        
        % \draw (5,1.25) rectangle (-1,-1.1);
        \clip (5,1.25) rectangle (-1,-1.1);

        
        % \draw[Goldenrod, domain=-3:3, samples=100] plot[shift={(-1,-1)}, rotate=30]({\x},{sin(\x * pi * 2 / \wavelength r)});
        \draw[Goldenrod, domain=-3:6, samples=50] 
            plot ({\x},{cos(\x * pi * 2 / \wavelength r)})
        ;
        
        \draw[Black, dashed, domain=-3:6, samples=2] 
            plot[Black, thin, dashed, samples=2] (\x,0)
        ;
        
        
        % \pause
        \draw[thick]
            (0,0)  node[Green, dinantenna] (A00) {}
            % (0,\d) node[Blue,  dinantenna] (A0d) {}
            % (\d,0) node[Red,   dinantenna] (Ad0) {}
        ;
        
        % \pause
        \draw[thick]
            % (0,0)  node[Green, dinantenna] (A00) {}
            % (0,\d) node[Blue,  dinantenna] (A0d) {}
            (\d,0) node[Red,   dinantenna] (Ad0) {}
        ;

        \draw[latex-latex]
            ($(A00)+(0,1.1)$) -- ++(\wavelength,0) node [midway, fill=white] {$\lambda$}
        ;
    % \visible<7->{
        \draw[latex-latex]
            ($(A00)+(0,-0.1)$) -- ++(0.5*\wavelength,0) node [midway, fill=white, fill opacity=0.75, anchor=north] {$d = \sfrac{\lambda}{2}$}
        ;
    % }
            
    \end{circuitikz}
  % }