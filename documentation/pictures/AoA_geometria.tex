% \resizebox{!}{0.7\textheight}{%
\begin{circuitikz}[american, voltage shift=0.5, line width=0.5, every node/.style={font = {\footnotesize\bfseries}}]

    \def\wavelength{5}
    \pgfmathsetmacro\d{0.5*\wavelength}

    \def\signalAngle{75}
    \def\antennaAngle{120}

    \def\closeRange{9}
    \def\farRange{\closeRange+13}

	\def\NAntennas{3}
	\pgfmathsetmacro\AngleAntennas{360/\NAntennas}
	\def\ShiftAngleAntennas{-90}

	\pgfmathsetmacro\RhoAntennas{\d/(2*sin(180/\NAntennas))}

    \def\centerarc(#1)(#2:#3:#4)% Syntax: [draw options] (center) (initial angle:final angle:radius)
    { ($(#1)+({#4*cos(#2)},{#4*sin(#2)})$) arc (#2:#3:#4) }

    \def\coordref[#1](#2){%

        \coordinate(sysref) at (#2);

        \draw[#1, -latex] (sysref) ++(-0.4,-0.3) -- ++(0.9,0) node[midway, below]{$x$};
        \draw[#1, -latex] (sysref) ++(-0.3,-0.4) -- ++(0,0.9) node[midway, left]{$y$};
        \draw[#1, -latex] \centerarc(sysref)(-90:180:0.25);
        \draw[#1] (sysref) node{$+$}
    }

    \coordinate (bottomleft) at (-6,-0.5);
    \coordinate (topright) at (6,6.5);


    % \draw[Red,dashed] (bottomleft) rectangle (topright);
    \clip (bottomleft) rectangle (topright);

    \coordinate (O) at (0,0);
    \coordinate (sourceAntenna) at (\signalAngle:\closeRange*\wavelength);
    % \draw [help lines, dashed] (bottomleft) grid (topright); % desenha grid
    % \draw [red] (O) node[draw,cross out] {}; % marca pont(0,0)

    % Circulo de antenas
	\draw[densely dotted, opacity=0.25] (O) ++(90:\RhoAntennas) circle (\RhoAntennas);

    % Linhas do sinal de fundo
    \foreach \x [evaluate={\y=int((\x+\closeRange));\z=int((\x+\closeRange));}] in {-3,...,3} {
        \draw [black!75, very thin]
        (sourceAntenna) ++ (\signalAngle:-\z*\wavelength)
            % node[anchor=west, font = {\footnotesize\bfseries}]{$\y\lambda$}
        ($(sourceAntenna) + (\signalAngle:-\z*\wavelength) + ({10*cos(\signalAngle+90)},{10*sin(\signalAngle+90)})$)
            --
        ($(sourceAntenna) + (\signalAngle:-\z*\wavelength) - ({10*cos(\signalAngle+90)},{10*sin(\signalAngle+90)})$)
        % \draw [gray, thin] (sourceAntenna) circle (\z)
        ;
    }

    % Antenas
    \draw[thick, cmyk_R] (O) node[dinantenna] (A00) {} node [below] {$A_{k+1}$};
    \draw[thick, cmyk_G, opacity=0.75] (O) ++(60:\d) node[dinantenna] (A0d) {} node [below] {$A_{k+2}$};
    \draw[thick, cmyk_B] (O) ++(\antennaAngle:\d) node[dinantenna] (Ad0) {} node [above left] {$A_{k}$};

    \draw[very thin, Black!50] % Desenha eixo X
        (-0.5*\d,0) -- (2*\d,0) node[right] {$x$}
    ;

    % Ângulo alpha entre antenas
    \draw[thin, cmyk_M]
        (0.3,0) node [above right, inner sep=1pt] {\ac{alphak}}
        \centerarc(O)(0:\antennaAngle:0.3)
    ;

    % Comprimento de onda
    \draw[latex-latex]
        (A00) ++(\signalAngle+90:\wavelength) coordinate(signalAux)
         -- node [midway, fill=white, circle, inner sep=1pt] {$\lambda$} ++(\signalAngle:\wavelength)
    ;

    % Desenha senoide de fundo
    \draw[Goldenrod, domain=-8:8, samples=100] plot[shift={(signalAux)}, rotate=\signalAngle]({\x},{cos(\x * pi * 2 / \wavelength r)});

    % Direção do sinal
    \draw[very thick, dashed, -latex, Goldenrod]
        (A00) ++(1.5*\d,0) ++ (\signalAngle:-0.5*\d) -- coordinate(angleArrow) ++(\signalAngle:\d)
    ;
    % Angulo Theta do sinal
    \draw[thin]
        \centerarc(angleArrow)(0:\signalAngle:0.4) node [midway,above right,inner sep=1pt] {\ac{thetaAoA}}
    ;

    % Triangulo retângulo + quadradinho
    \draw[Black]

        (A00) --++($({\signalAngle-90}:{\d*sin(\signalAngle-\antennaAngle)})$) coordinate (pontoTriangulo) -- (Ad0) -- (A00)

        (pontoTriangulo)
          ++(\signalAngle:0.125)
        --++(\signalAngle-90:0.125)
        --++(\signalAngle+180:0.125)
    ;
    % Arco do angulo beta
    \draw[thin, Purple] \centerarc(Ad0)(180+\antennaAngle:180+\signalAngle:0.4)
        (Ad0) ++ (180+\signalAngle:0.5) node[below right, inner sep=1pt, fill=white, fill opacity=0.75,] {\ac{betak}}
    ;

    % Distânci d entre antenas
    \draw[latex-latex]
        ($(A00)+(0,1)$) -- ($(Ad0)+(0,1)$) node [midway, fill=white, circle, inner sep=1pt] {\ac{d}}
    ;

    \newcommand\CircleRadius{3cm}
    % special method of noting the position of a point
    \coordinate (P) at (50:\CircleRadius);

    \draw[decorate, decoration={brace, amplitude=5pt}, thin]
    ($(pontoTriangulo)+({\signalAngle+90}:0.1)$)
    -- coordinate (brace)
    ($(Ad0)+({\signalAngle+90}:0.1)$)
    ;

    \draw (brace) ++({\signalAngle+90}:5pt)
        node[anchor=east] {$\displaystyle\ac{lambda} \cdot \frac{\ac{DeltaPhi}}{2 \pi}$}
    ;

    \coordref[Black!25](3,3);

    % \foreach \x in {0,60,...,300} {
    %     \draw[thick] (\x:1 cm) -- (\x + 60:1 cm);

    %     \draw (\x + 30:1.732 cm) node[Gray, circ]{};
    %     \draw[Gray, dashed] (\x:1 cm) -- ++(\x: 0.9cm);
    %     \draw[Gray, dotted]
    %     %     % (\x:1 cm) arc (\x+240:\x+180:1cm)
    %         (\x:1 cm) arc [start angle=\x+120, delta angle=110, radius=1cm]
    %         (\x:1 cm) arc [start angle=\x+120, delta angle=-50, radius=1cm]
    %     ;
    % }

    % \draw (0,0) node [circ] {} node [below left,font={\scriptsize\bfseries}] {BS};
    % \draw[thick, densely dotted] (0,0) circle (1cm);

    % \draw[-latex] (0,0) -- (0:1cm) node[midway, below] {$R_c$};
    % \draw[-latex] (0,0) -- (90:0.866cm) node[midway, left] {$R$};

\end{circuitikz}
% }

