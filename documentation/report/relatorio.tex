\documentclass[12pt, openany]{report}
% \documentclass[12pt, openany, showframe]{report}
% \documentclass[12pt,showframe, openleft]{report}

% https://titospadini.medium.com/sugestões-para-um-bom-trabalho-de-graduação-em-engenharia-81fcb0ae263d

%%%%%%%%%%%%%%%%%%%%%%%%%%%%%%%%%%%%%%%%%%%%%%%%%%%%%%%%%%%%%%%%%%%%%%%%%%%%%%%%
% TEMPLATE PARA DOCUMENTOS
% UFABC Rocket Design
%
% Para atualiza sua versão, acesse o link abaixo e depois volte aqui para atualizar
%
% https://www.overleaf.com/read/xbbzdqfgqmvp#ee6791
%
% Lista de revisões: (data ISO - nome)
%
% 2020-12-17 - Heitor Rodrigues Savegnago
% 2021-01-17 - Heitor Rodrigues Savegnago
% 2021-01-23 - Heitor Rodrigues Savegnago
% 2021-04-21 - Heitor Rodrigues Savegnago
% 2021-12-23 - Heitor Rodrigues Savegnago
% 2022-02-02 - Heitor Rodrigues Savegnago
% 2022-04-19 - Heitor Rodrigues Savegnago
% 2022-11-02 - Heitor Rodrigues Savegnago
% 2023-05-24 - Heitor Rodrigues Savegnago
% 2023-08-06 - Heitor Rodrigues Savegnago
% 2023-08-18 - Heitor Rodrigues Savegnago
% 2023-09-06 - Heitor Rodrigues Savegnago
% 2023-09-27 - Heitor Rodrigues Savegnago
% 2023-11-29 - Heitor Rodrigues Savegnago
% 2024-05-13 - Heitor Rodrigues Savegnago
% 2024-07-28 - Heitor Rodrigues Savegnago
% 2024-08-14 - Heitor Rodrigues Savegnago
% 2024-12-12 - Heitor Rodrigues Savegnago
%
%%%%%%%%%%%%%%%%%%%%%%%%%%%%%%%%%%%%%%%%%%%%%%%%%%%%%%%%%%%%%%%%%%%%%%%%%%%%%%%%

% Formatação geral do documento

    % Configuração de margens
    \ifx\standalone\undefined
        \usepackage[
            a4paper,
            twoside,
            top = 3cm,
            bottom = 2cm,
            left = 3cm,
            right = 2cm
        ]{geometry} % Margens do documento
    \fi

    % Fonte
        %\renewcommand{\rmdefault}{phv} % Arial
        \usepackage[lighttt]{lmodern} % Fonte Latin Modern
            % \renewcommand{\familydefault}{\sfdefault} % Estilo global Sans Serif

        % Subistituição de fonte para casos de erro
        \DeclareFontFamilySubstitution{TS1}{aer}{lmr}
        \DeclareFontFamilySubstitution{TS1}{aett}{lmtt}
        \DeclareFontFamilySubstitution{TS1}{aess}{lmss}

    % Espaçamento entre linhas
        \usepackage{setspace} % Espaçamento do texto
            \renewcommand{\baselinestretch}{1.15}

    % Correções de espaçamentos
        \usepackage{indentfirst} % Parágrafos indentados
            \setlength{\parindent}{1cm} % Define espaço de paragrafo como 1cm

        \raggedbottom % Corrige espaçamento de paragrafo

        \usepackage{microtype} % Ajusta detalhes menores nos espaçamentos da página para ficar mais agradável

    % Decorações da página
        \usepackage{fancyhdr} % Cabeçalhos e rodapés em páginas

    % Formatação de títulos
    	\newcommand{\titlefont}{\fontsize{18}{20}}
    	\newcommand{\sectionfont}{\fontsize{16}{20}}
    	\newcommand{\subsectionfont}{\fontsize{14}{20}}
    	\newcommand{\subsubsectionfont}{\fontsize{12}{20}}

        \usepackage{titlesec} % Configurações para seção
            \ifx\article\undefined
                \titleformat{\chapter}{\normalfont \titlefont \bfseries}{\thechapter}{1em}{}
                \titlespacing*{\chapter}{0pt}{3.5ex plus 1ex minus .2ex}{2.3ex plus .2ex}
            \fi
            \titleformat{\section}{\normalfont \sectionfont \bfseries}{\thesection}{1em}{}
            \titleformat{\subsection}{\normalfont \subsectionfont \bfseries}{\thesubsection}{1em}{}
            \titleformat{\subsubsection}{\normalfont \subsubsectionfont \bfseries}{\thesubsubsection}{1em}{}

        \usepackage{titletoc} % Configurações para sumário e listas de figuras e tabelas

            % Essas configurações estão modeladas para alinhar todas as numerações
            % à esquerda e títulos também alinhados à esquerda em outra margem

            % Configurações para sumário estilo ABNT
            \titlecontents{chapter} % Qual nível se refere
                [3.5em] % Espaço à esquerda, entre margem e título
                {\bigskip} % A cima da linha (geralmente espaçamento vertical)
                {\normalfont\normalsize\contentslabel[\bfseries\thecontentslabel]{3.5em}\bfseries\uppercase} % Formato da linha com numeração
                {\hspace*{-3.5em}\bfseries\uppercase} % Formato da linha sem numeração
                {\bfseries\dotfill\contentspage} % Formato do preenchimento entre título e número da página

            \titlecontents{section} % Qual nível se refere
                [3.5em] % Espaço à esquerda, entre margem e título
                {\smallskip} % A cima da linha (geralmente espaçamento)
                {\normalfont\normalsize\contentslabel[\bfseries\thecontentslabel]{3.5em}\bfseries} % Formato da linha com numeração
                {\hspace*{-3.5em}\bfseries} % Formato da linha sem numeração
                {\bfseries\dotfill\contentspage} % Formato do preenchimento entre título e número da página

            \dottedcontents{subsection}[3.5em]{}{3.5em}{0.44em}
            \dottedcontents{subsubsection}[3.5em]{}{3.5em}{0.44em}

            % Configuração para listas de figuras e tabelas
            \dottedcontents{figure}[2em]{\smallskip}{2em}{0.44em}
            \dottedcontents{table}[2em]{\smallskip}{2em}{0.44em}

    % Alteração de limite de níveis no sumário

        \setcounter{tocdepth}{4} % Níveis exibidos no sumário
        \setcounter{secnumdepth}{4} % Nível de números exibidos

    % Opções para tópicos de enumerate

        \usepackage[shortlabels]{enumitem} % Selecionar formato em itemize

%%%%%%%%%%%%%%%%%%%%%%%%%%%%%%%%%%%%%%%%%%%%%%%%%%%%%%%%%%%%%%%%%%%%%%%%%%%%%%%%

% Codificação de carecteres de entrada

    \usepackage{ae} % "Almost European"
    \usepackage[T1]{fontenc} % Caracteres especiais
    \usepackage[utf8]{inputenc} % Caracteres especiais

    \usepackage{fontawesome} % Símbolos especiais

%%%%%%%%%%%%%%%%%%%%%%%%%%%%%%%%%%%%%%%%%%%%%%%%%%%%%%%%%%%%%%%%%%%%%%%%%%%%%%%%

% Configuração de linguagem padrão do documento

    \usepackage[english, main=brazil]{babel} % Detalhes automáticos em Português
        % \selectlanguage{brazil}

        \AtBeginDocument{\renewcommand{\contentsname}{\centerline{Sumário}}}
        \AtBeginDocument{\renewcommand{\bibname}{Referências Bibliográficas}}
        \AtBeginDocument{\renewcommand{\listfigurename}{\centerline{Figuras}}}
        \AtBeginDocument{\renewcommand{\listtablename}{\centerline{Tabelas}}}
        \AtBeginDocument{\renewcommand{\lstlistlistingname}{\centerline{Códigos}}}
    % 	\AtBeginDocument{\renewcommand{\figurename}{Figura}}
    % 	\AtBeginDocument{\renewcommand{\tablename}{Tabela}}

    \usepackage{textcomp} % Suporte de caracteres especiais

    \usepackage{csquotes} % Opções de citação

%%%%%%%%%%%%%%%%%%%%%%%%%%%%%%%%%%%%%%%%%%%%%%%%%%%%%%%%%%%%%%%%%%%%%%%%%%%%%%%%

% Configurações de cores

    \usepackage{transparent} % Opções de transparência

    \usepackage[svgnames]{xcolor} % Define cores

        \definecolor{0DF}{HTML}{00DDFF}%
        \definecolor{0FD}{HTML}{00FFDD}%
        \definecolor{DF0}{HTML}{DDFF00}%
        \definecolor{FD0}{HTML}{FFDD00}%
        \definecolor{F0D}{HTML}{FF00DD}%
        \definecolor{D0F}{HTML}{DD00FF}%

        \definecolor{0BD}{HTML}{00BBDD}%
        \definecolor{0DB}{HTML}{00DDBB}%
        \definecolor{BD0}{HTML}{BBDD00}%
        \definecolor{DB0}{HTML}{DDBB00}%
        \definecolor{D0B}{HTML}{DD00BB}%
        \definecolor{B0D}{HTML}{BB00DD}%

        \definecolor{09B}{HTML}{0099BB}%
        \definecolor{0B9}{HTML}{00BB99}%
        \definecolor{9B0}{HTML}{99BB00}%
        \definecolor{B90}{HTML}{BB9900}%
        \definecolor{B09}{HTML}{BB0099}%
        \definecolor{90B}{HTML}{9900BB}%

        \definecolor{079}{HTML}{007799}%
        \definecolor{097}{HTML}{009977}%
        \definecolor{790}{HTML}{779900}%
        \definecolor{970}{HTML}{997700}%
        \definecolor{907}{HTML}{990077}%
        \definecolor{709}{HTML}{770099}%

        \definecolor{057}{HTML}{005577}%
        \definecolor{075}{HTML}{007755}%
        \definecolor{570}{HTML}{557700}%
        \definecolor{750}{HTML}{775500}%
        \definecolor{705}{HTML}{770055}%
        \definecolor{507}{HTML}{550077}%

        \definecolor{035}{HTML}{003355}%
        \definecolor{053}{HTML}{005533}%
        \definecolor{350}{HTML}{335500}%
        \definecolor{530}{HTML}{553300}%
        \definecolor{503}{HTML}{550033}%
        \definecolor{305}{HTML}{330055}%

        \definecolor{013}{HTML}{001133}%
        \definecolor{031}{HTML}{003311}%
        \definecolor{130}{HTML}{113300}%
        \definecolor{310}{HTML}{331100}%
        \definecolor{301}{HTML}{330011}%
        \definecolor{103}{HTML}{110033}%

        \definecolor{rgb_R}{rgb}{1,0,0}%
        \definecolor{rgb_G}{rgb}{0,1,0}%
        \definecolor{rgb_B}{rgb}{0,0,1}%
        \definecolor{rgb_M}{rgb}{1,0,1}%
        \definecolor{rgb_Y}{rgb}{1,1,0}%
        \definecolor{rgb_C}{rgb}{0,1,1}%
        \definecolor{rgb_W}{rgb}{1,1,1}%
        \definecolor{rgb_K}{rgb}{0,0,0}%

        \definecolor{cmyk_C}{cmyk}{1,0,0,0}%
        \definecolor{cmyk_M}{cmyk}{0,1,0,0}%
        \definecolor{cmyk_Y}{cmyk}{0,0,1,0}%
        \definecolor{cmyk_G}{cmyk}{1,0,1,0}%
        \definecolor{cmyk_B}{cmyk}{1,1,0,0}%
        \definecolor{cmyk_R}{cmyk}{0,1,1,0}%
        \definecolor{cmyk_K}{cmyk}{1,1,1,1}%
        \definecolor{cmyk_W}{cmyk}{0,0,0,0}%

        \definecolor{strs}			{rgb}	{0.9,	0.2,	0	}%
        \definecolor{coments}		{rgb}	{0,		0.5,	0	}%
        \definecolor{backcode}		{rgb}	{0.3,	0,		0.2	}%

        \newcommand{\MexerDepois}[1]{
            \vspace*{2em}
            {\huge\color{F0D}#1}
            \vspace*{2em}}
        \newcommand{\mexer}[1]{{\color{F0D}#1}}

        \newcommand{\showcolors}%Mostra tabela de cores
        {{\ttfamily
        		{\color{0DF}$\overset{\text{\tiny 0DF}}{\blacksquare}$}
        		{\color{0BD}$\overset{\text{\tiny 0BD}}{\blacksquare}$}
        		{\color{09B}$\overset{\text{\tiny 09B}}{\blacksquare}$}
        		{\color{079}$\overset{\text{\tiny 079}}{\blacksquare}$}
        		{\color{057}$\overset{\text{\tiny 057}}{\blacksquare}$}
        		{\color{035}$\overset{\text{\tiny 035}}{\blacksquare}$}
        		{\color{013}$\overset{\text{\tiny 013}}{\blacksquare}$}
        		\\
        		{\color{0FD}$\overset{\text{\tiny 0FD}}{\blacksquare}$}
        		{\color{0DB}$\overset{\text{\tiny 0DB}}{\blacksquare}$}
        		{\color{0B9}$\overset{\text{\tiny 0B9}}{\blacksquare}$}
        		{\color{097}$\overset{\text{\tiny 097}}{\blacksquare}$}
        		{\color{075}$\overset{\text{\tiny 075}}{\blacksquare}$}
        		{\color{053}$\overset{\text{\tiny 053}}{\blacksquare}$}
        		{\color{031}$\overset{\text{\tiny 031}}{\blacksquare}$}
        		\\
        		{\color{DF0}$\overset{\text{\tiny DF0}}{\blacksquare}$}
        		{\color{BD0}$\overset{\text{\tiny BD0}}{\blacksquare}$}
        		{\color{9B0}$\overset{\text{\tiny 9B0}}{\blacksquare}$}
        		{\color{790}$\overset{\text{\tiny 790}}{\blacksquare}$}
        		{\color{570}$\overset{\text{\tiny 570}}{\blacksquare}$}
        		{\color{350}$\overset{\text{\tiny 350}}{\blacksquare}$}
        		{\color{130}$\overset{\text{\tiny 130}}{\blacksquare}$}
        		\\
        		{\color{FD0}$\overset{\text{\tiny FD0}}{\blacksquare}$}
        		{\color{DB0}$\overset{\text{\tiny DB0}}{\blacksquare}$}
        		{\color{B90}$\overset{\text{\tiny B90}}{\blacksquare}$}
        		{\color{970}$\overset{\text{\tiny 970}}{\blacksquare}$}
        		{\color{750}$\overset{\text{\tiny 750}}{\blacksquare}$}
        		{\color{530}$\overset{\text{\tiny 530}}{\blacksquare}$}
        		{\color{310}$\overset{\text{\tiny 310}}{\blacksquare}$}
        		\\
        		{\color{F0D}$\overset{\text{\tiny F0D}}{\blacksquare}$}
        		{\color{D0B}$\overset{\text{\tiny D0B}}{\blacksquare}$}
        		{\color{B09}$\overset{\text{\tiny B09}}{\blacksquare}$}
        		{\color{907}$\overset{\text{\tiny 907}}{\blacksquare}$}
        		{\color{705}$\overset{\text{\tiny 705}}{\blacksquare}$}
        		{\color{503}$\overset{\text{\tiny 503}}{\blacksquare}$}
        		{\color{301}$\overset{\text{\tiny 301}}{\blacksquare}$}
        		\\
        		{\color{D0F}$\overset{\text{\tiny D0F}}{\blacksquare}$}
        		{\color{B0D}$\overset{\text{\tiny B0D}}{\blacksquare}$}
        		{\color{90B}$\overset{\text{\tiny 90B}}{\blacksquare}$}
        		{\color{709}$\overset{\text{\tiny 709}}{\blacksquare}$}
        		{\color{507}$\overset{\text{\tiny 507}}{\blacksquare}$}
        		{\color{305}$\overset{\text{\tiny 305}}{\blacksquare}$}
        		{\color{103}$\overset{\text{\tiny 103}}{\blacksquare}$}
        }}

%%%%%%%%%%%%%%%%%%%%%%%%%%%%%%%%%%%%%%%%%%%%%%%%%%%%%%%%%%%%%%%%%%%%%%%%%%%%%%%%

% Configurações do pacote listings para adição de código no documento

    \usepackage{listings}%Configura layout para mostrar codigos a partir de arquivo
        \AtBeginDocument{\renewcommand{\lstlistingname}{Código}}


        \lstdefinelanguage{JavaScript}{
            keywords={typeof, new, true, false, catch, function, return, null, catch, switch, var, if, in, while, do, else, case, break},
            ndkeywords={class, export, boolean, throw, implements, import, this},
            % ndkeywordstyle=\color{darkgray}\bfseries,
            identifierstyle=\color{black},
            sensitive=true,
            comment=[l]{//},
            morecomment=[s]{/*}{*/},
            morestring=[b]',
            morestring=[b]"
        }

        \lstset{% Configurando layout para mostrar códigos C++
            language=[11]C++,
            basicstyle=\ttfamily\small\setstretch{1},
            backgroundcolor=\color{backcode!5},
            stringstyle=\color{strs},
            commentstyle=\color{coments},
            keywordstyle=[1]\itshape\color{079},
            keywordstyle=[2]\color{907},
            keywordstyle=[3]\color{097},
            keywordstyle=[4]\bfseries\color{790},
            keywordstyle=[5]\color{709},
            keywordstyle=[6]\color{970},
            morekeywords=[1]{byte},
            morekeywords=[2]{},
            morekeywords=[3]{uint8_t, size_t, type},
            morekeywords=[4]{},
            numbers=left,
            numberstyle=\tiny,
            escapeinside={§}{§},
            tabsize=2,
            extendedchars=true,
            showspaces=false,
            showstringspaces=false,
            numberbychapter=false,
            emptylines=1,
            frame=L,
            firstnumber=auto,
            breaklines=true,
            breakautoindent=true,
            captionpos=t,
            float=htbp,
            xleftmargin=2em,
            inputencoding=utf8,
            %texcl=true,
            upquote=true,
            literate=%
                {á}{{\'a}}1 {à}{{\`a}}1 {ä}{{\"a}}1 {â}{{\^a}}1 {ã}{{\~a}}1 {å}{{\r{a}}}1
                {Á}{{\'A}}1 {À}{{\`A}}1 {Ä}{{\"A}}1 {Â}{{\^A}}1 {Ã}{{\~A}}1 {Å}{{\r{A}}}1
                {é}{{\'e}}1 {è}{{\`e}}1 {ë}{{\"e}}1 {ê}{{\^e}}1 {ẽ}{{\~e}}1
                {É}{{\'E}}1 {È}{{\`E}}1 {Ë}{{\"E}}1 {Ê}{{\^E}}1 {Ẽ}{{\~E}}1
                {í}{{\'i}}1 {ì}{{\`i}}1 {ï}{{\"i}}1 {î}{{\^i}}1 {ĩ}{{\~i}}1
                {Í}{{\'I}}1 {Ì}{{\`I}}1 {Ï}{{\"I}}1 {Î}{{\^I}}1 {Ĩ}{{\~I}}1
                {ó}{{\'o}}1 {ò}{{\`o}}1 {ö}{{\"o}}1 {ô}{{\^o}}1 {õ}{{\~o}}1 {ő}{{\H{o}}}1
                {Ó}{{\'O}}1 {Ò}{{\`O}}1 {Ö}{{\"O}}1 {Ô}{{\^O}}1 {Õ}{{\~O}}1 {Ő}{{\H{O}}}1
                {ú}{{\'u}}1 {ù}{{\`u}}1 {ü}{{\"u}}1 {û}{{\^u}}1 {ũ}{{\~u}}1 {ű}{{\H{u}}}1
                {Ú}{{\'U}}1 {Ù}{{\`U}}1 {Ü}{{\"U}}1 {Û}{{\^U}}1 {Ũ}{{\~U}}1 {Ű}{{\H{U}}}1
                {œ}{{\oe}}1 {Œ}{{\OE}}1 {æ}{{\ae}}1 {Æ}{{\AE}}1 {ß}{{\ss}}1
                {ç}{{\c{c}}}1 {Ç}{{\c{C}}}1
                {ñ}{{\~n}}1 {Ñ}{{\~N}}1
                {ø}{{\o}}1 {Ø}{{\O}}1
                {⁰}{{\textsuperscript{0}}}1
                {¹}{{\textsuperscript{1}}}1
                {²}{{\textsuperscript{2}}}1
                {³}{{\textsuperscript{3}}}1
                {⁴}{{\textsuperscript{4}}}1
                {⁵}{{\textsuperscript{5}}}1
                {⁶}{{\textsuperscript{6}}}1
                {⁷}{{\textsuperscript{7}}}1
                {⁸}{{\textsuperscript{8}}}1
                {⁹}{{\textsuperscript{9}}}1
                {°}{{\textdegree}}1
                {€}{{\euro}}1 {£}{{\pounds}}1
                {«}{{\guillemotleft}}1 {»}{{\guillemotright}}1
                {¿}{{?`}}1 {¡}{{!`}}1
        }

        \newcommand{\coda}[1]{{\color{057}\lstinline|#1|}}
        \newcommand{\code}[1]{{\color{075}\lstinline|#1|}}
        \newcommand{\codi}[1]{{\color{570}\lstinline|#1|}}
        \newcommand{\codo}[1]{{\color{750}\lstinline|#1|}}
        \newcommand{\codu}[1]{{\color{705}\lstinline|#1|}}
        \newcommand{\codw}[1]{{\color{507}\lstinline|#1|}}
        \newcommand{\codGuide}{
            \begin{center}
                \large{\coda{A}\\\code{E}\\\codi{I}\\\codo{O}\\\codu{U}\\\codw{W}

                \showcolors}
            \end{center}
        }

    % Configurações adicinais para o pacote titletoc
        \contentsuse{lstlisting}{lol}
        \dottedcontents{lstlisting}[2em]{\smallskip}{2em}{0.44em}

%%%%%%%%%%%%%%%%%%%%%%%%%%%%%%%%%%%%%%%%%%%%%%%%%%%%%%%%%%%%%%%%%%%%%%%%%%%%%%%%

% Adição de caracteres

    \usepackage[euler]{textgreek} % Caracteres gregos

    \usepackage{pmboxdraw} % Caracteres unicode

%%%%%%%%%%%%%%%%%%%%%%%%%%%%%%%%%%%%%%%%%%%%%%%%%%%%%%%%%%%%%%%%%%%%%%%%%%%%%%%%

% Pacotes de opções matemáticas

    \usepackage{amsmath, amssymb, xfrac, cancel} % símbolos matemáticos

    \usepackage{siunitx} % Comando \SI para unidades de medida
        \sisetup{locale = FR} % Utilizar virgular para marcação decimal
        \sisetup{separate-uncertainty = true}
        \sisetup{exponent-product=\ensuremath{\cdot}}
        \sisetup{separate-uncertainty=true}
        \sisetup{multi-part-units=single}
        \sisetup{group-separator = {}}
        \sisetup{detect-all}

        \DeclareSIUnit{\nothing}{{\relax}}
        \DeclareSIUnit{\var}{VAR}
        \DeclareSIUnit{\va}{VA}
        \DeclareSIUnit{\dBm}{dBm}
        \DeclareSIUnit{\pixel}{px} % Pixel

    \usepackage{gensymb} % Símbolos de unidades de medida

    \usepackage{mathtools}

        \DeclareFontFamily{U}{mathc}{}
        \DeclareFontShape{U}{mathc}{m}{it}%
        {<->s*[1.03] mathc10}{}
        \DeclareMathAlphabet{\mathcal}{U}{mathc}{m}{it}

        \DeclareMathOperator{\sHom}{\mathcal{H\mkern-3mu om}}
        \DeclareMathOperator{\sExt}{\mathcal{E\mkern-3mu xt}}
        \DeclareMathOperator{\sEnd}{\mathcal{E\mkern-3mu nd}}

    \usepackage{steinmetz} % Números complexos

%%%%%%%%%%%%%%%%%%%%%%%%%%%%%%%%%%%%%%%%%%%%%%%%%%%%%%%%%%%%%%%%%%%%%%%%%%%%%%%%

% Pacotes de opções químicas

    \usepackage{chemformula}
    \usepackage[version=3]{mhchem}

%%%%%%%%%%%%%%%%%%%%%%%%%%%%%%%%%%%%%%%%%%%%%%%%%%%%%%%%%%%%%%%%%%%%%%%%%%%%%%%%

% Opções adicionais para formatação

    \usepackage[normalem]{ulem} % Sublinados diversos

    \usepackage{framed} % Criar caixas inteligentes

    \usepackage{footnote} % Notas de rodapé

        \makeatletter % Para referenciar notas de rodapé
            \newcommand\footnoteref[1]{\protected@xdef\@thefnmark{\ref{#1}}\@footnotemark}
        \makeatother

    \usepackage{lscape} % Página em paisagem

    \usepackage{datetime2} % Formatação de datas

%%%%%%%%%%%%%%%%%%%%%%%%%%%%%%%%%%%%%%%%%%%%%%%%%%%%%%%%%%%%%%%%%%%%%%%%%%%%%%%%

% Formatação de tabelas

    \usepackage{tabularx} % Tabelas do tipo tabularx
    \usepackage{longtable} % Tabelas com várias páginas
    \usepackage{booktabs} % Formatação de tabelas como em livro
        \renewcommand{\arraystretch}{1.25} % Espaçamento entre linhas interno em tabelas
        %\renewcommand{\cellgape}{\Gape[4pt]} % Espaçamento de tabelas
    \usepackage{makecell} % formatação avançada para tabelas
    \usepackage{multirow} % Merge em tabelas
    \usepackage{multicol} % Texto em colunas na folha
    \usepackage{arydshln} % Draw dash-lines in array/tabular
    \usepackage{colortbl} % Linhas, colunas e celulas coloridas

    \usepackage{array} % opções especiais para alinhamento de tabelas
        \newcolumntype{L}[1]{>{\raggedright\let\newline\\\arraybackslash\hspace{0pt}}m{#1}}
        \newcolumntype{C}[1]{>{\centering\let\newline\\\arraybackslash\hspace{0pt}}m{#1}}
        \newcolumntype{R}[1]{>{\raggedleft\let\newline\\\arraybackslash\hspace{0pt}}m{#1}}

%%%%%%%%%%%%%%%%%%%%%%%%%%%%%%%%%%%%%%%%%%%%%%%%%%%%%%%%%%%%%%%%%%%%%%%%%%%%%%%%

% Pacotes para trabalhar com figuras

    \usepackage{graphicx} % Adição de imagens

    \usepackage{svg} % Adição de imagens no formato SVG

    \usepackage[outdir=./]{epstopdf} % Figuras em EPS convertidas em PDF

    \usepackage{pdfpages} % Adição de PDFs como páginas

    \usepackage[angle=0, text={}]{draftwatermark} % Adição de marca d'água
    % \usepackage[printwatermark]{xwatermark}

    \usepackage{tikz} % Desenhos
        \usetikzlibrary{through}
        \usetikzlibrary{shapes}
        \usetikzlibrary{shapes.geometric}
        \usetikzlibrary{trees}
        \usetikzlibrary{fit}
        \usetikzlibrary{patterns}
        \usetikzlibrary{calc}
        \usetikzlibrary{arrows}
        \usetikzlibrary{decorations}
        \usetikzlibrary{decorations.pathmorphing}
        \usetikzlibrary{positioning}

    \usepackage{pgfplots} % Desenho de gráficos
        \pgfdeclarelayer{background}    % declare background layer
        \pgfdeclarelayer{foreground}    % declare foreground layer
        \pgfsetlayers{background,main,foreground}  % set the order of the layers (main is the standard layer)
        \pgfplotsset{compat=1.14}
        \usepgfplotslibrary{fillbetween}

    \usepackage{pgf-pie} % Gráficos pizza

    \usepackage[RPvoltages]{circuitikz} % Desenhos de circuitos
        \ctikzset{bipoles/thickness=1}

    \usepackage{pgfplotstable}

        \pgfplotstableset{% global config, for example in the preamble
            assign column name/.style={
                /pgfplots/table/column name={\textbf{#1}} % Primeira linha em negrito
            },
            every first column/.style={
                column type/.add={l}{} % Primeira coluna alinhada a esquerda
            },
            string type, % A entrada é textual
            col sep=tab, % O arquivo é separado por tabs
            every head row/.style={before row=\toprule,after row=\midrule}, % Definições de linhas horizintais do cabeçalho
            every last row/.style={after row=\bottomrule}, % Definições de linhas horizontais do final
        }

    \usepackage{chemfig} % Desenho de moléculas

%%%%%%%%%%%%%%%%%%%%%%%%%%%%%%%%%%%%%%%%%%%%%%%%%%%%%%%%%%%%%%%%%%%%%%%%%%%%%%%%

% Utilitários adicionais para lidar com figuras e tabelas

    \usepackage{float} % posicionamento espacial

        % Criando ambiente float para códigos
        \newfloat{lstfloat}{htbp}{lop}
        \floatname{lstfloat}{Código}
        \def\lstfloatautorefname{Código} % needed for hyperref/auroref

    \usepackage{caption} % Comando \caption*
    % \captionsetup{skip=0.5em}

    \usepackage{subcaption} % Opções de subfiguras

%%%%%%%%%%%%%%%%%%%%%%%%%%%%%%%%%%%%%%%%%%%%%%%%%%%%%%%%%%%%%%%%%%%%%%%%%%%%%%%%

% Auxiliares gerais

\usepackage{lipsum} % Lorem ipsum

\usepackage{ifdraft} % Opções adicionais para o modo draft

\usepackage{csvsimple} % Carregar arquivos para o doc

    \newcommand{\expandItemsListDat}[1]{ % Expandir items de arquivo .dat
        \csvloop{
            file = {#1},
            no head,
            before line = \item,
            % after line =;
        }}

    \newcommand{\expandItemsListDatAspas}[1]{ % Expandir items de arquivo .dat
        \csvloop{
            file = {#1},
            no head,
            before line ={\item``},
            after line ={''}
        }}

\usepackage{etoolbox} % Toolbox of programming facilities

    % Remover espaçamentos que dividem capitulos nas listas de figuras e tabelas
    \makeatletter
        \patchcmd{\@chapter}{\addtocontents{lof}{\protect\addvspace{10\p@}}}{}{}{}
        \patchcmd{\@chapter}{\addtocontents{lot}{\protect\addvspace{10\p@}}}{}{}{}
    \makeatother

%%%%%%%%%%%%%%%%%%%%%%%%%%%%%%%%%%%%%%%%%%%%%%%%%%%%%%%%%%%%%%%%%%%%%%%%%%%%%%%%

% Referência cruzada e links

    \usepackage[hidelinks]{hyperref} % Links no documento

    \usepackage{nameref} % Referenciar entidades por nome

    \usepackage{titleref} % Referenciar títulos

    % Criação de lista de símbolos e acônimos

        \usepackage{acro}
            \acsetup{
                make-links = true,
                use-id-as-short = true,
                format/foreign = \emph,
                list/name={\centerline{Abreviaturas e Siglas}},
                list/template = longtable,
                templates/colspec={>{\bfseries}lp{.85\textwidth}}
            }

%%%%%%%%%%%%%%%%%%%%%%%%%%%%%%%%%%%%%%%%%%%%%%%%%%%%%%%%%%%%%%%%%%%%%%%%%%%%%%%%

% Configuração de contadores do documento

    \ifx\chapter\undefined\else % Artigo não tem chapter
        \ifx\letter\undefined % Carta não tem figure/table/equation
            \usepackage{chngcntr} % Muda os contadores de figuras, equações, etc
                \counterwithout{figure}{chapter} % Número de figura sem contar capítulo
                \counterwithout{table}{chapter} % Número de tabela sem contar capítulo
                \counterwithout{equation}{chapter} % Número de equação sem contar capítulo
        \fi
    \fi


%%%%%%%%%%%%%%%%%%%%%%%%%%%%%%%%%%%%%%%%%%%%%%%%%%%%%%%%%%%%%%%%%%%%%%%%%%%%%%%%

% controle de citação e referências bibliográfica

    \usepackage[
        backend=biber,
        style=ieee,
        citestyle=numeric,
        sorting=none,
        block=space
    ]{biblatex}
    % \usepackage[style=abnt-numeric, citestyle=numeric, sorting=none]{biblatex} %https://github.com/abntex/biblatex-abnt/issues/90

        \renewbibmacro*{name:andothers}{% Based on name:andothers from biblatex.def
            \ifboolexpr{%
                test {\ifnumequal{\value{listcount}}{\value{liststop}}}%
                and%
                test \ifmorenames%
            }{%
                \ifnumgreater{\value{liststop}}{1}%
                {\finalandcomma}%
                {}%
                \andothersdelim\bibstring[\emph]{andothers}%
            }{}%
        }

        % \appto\bibfont{\setlength{\emergencystretch}{.5em}} % Evitar wanings por espaçamento na bibliografia nos casos mais simples

%%%%%%%%%%%%%%%%%%%%%%%%%%%%%%%%%%%%%%%%%%%%%%%%%%%%%%%%%%%%%%%%%%%%%%%%%%%%%%%%

% Definições de macros especiais para capa e folha de rosto

    % Lista de nomes

        \newcommand{\name}[1]
        {
        	&{#1}\\
        }
        \newenvironment{names}[1]
        {
            \begin{table}[H]\flushleft
        		\begin{tabular}{>{\raggedleft}p{.45\linewidth} | >{\bf}p{.45\linewidth}}
        			\sf{#1}
        			}
                    	%args here
                    {
        		\end{tabular}
        	\end{table}
        }

    % Outras macros
        \providecommand{\keywords}[1]{\textbf{{Keywords:}} #1}
        \providecommand{\palavraschave}[1]{\textbf{{Palavras-chave:}} #1}

        \newcommand{\etal}{\emph{et al}.}
        \newcommand{\ie}{\emph{i}.\emph{e}.}
        \newcommand{\eg}{\emph{e}.\emph{g}.}

    \newcommand{\nomes}{}
    \newcommand{\grupo}{}
    \newcommand{\centro}{}
    \newcommand{\centroSigla}{}
    \newcommand{\disciplina}{}
    \newcommand{\codigoDisciplina}{}
    \newcommand{\titulo}{}
    \newcommand{\professor}{}
    \newcommand{\local}{}
    \newcommand{\data}{\number\year}
    \newcommand{\notaDeRosto}{}
    \newcommand{\agradecimentos}{}
    \newcommand{\cabecalho}
    {
        {\large%
        \textbf{UNIVERSIDADE FEDERAL DO ABC}}\\
        \expandafter\MakeUppercase\expandafter{\small\centro{}%
        \ifdefempty{\centro}{}{ - }%
        \centroSigla\\
        \codigoDisciplina{}%
        \ifdefempty{\codigoDisciplina}{}{ - }%
        \disciplina%
        }
    }

    \newcommand{\capaComLogo}{
        \begin{titlepage} \center
            \cabecalho

            \vspace{6em}

            \begin{center}
                \includegraphics[width=0.335\linewidth]{pictures/logo_ufabc}
            \end{center}

            \nomes
            \textbf{\grupo}

            \vfill

            {\large{\MakeUppercase{\textbf{\titulo}}}}

            \vspace{1.5cm}
    		\professor

    		\vfill
    		\vfill

    		{\local\\\data}
        \end{titlepage}
    }

    \newcommand{\capa}{
        \begin{titlepage} \center
            \cabecalho

            \vspace{6em}

            % \capaLogo

            \nomes

            \vfill

            {\large{\MakeUppercase{\textbf{\titulo}}}}

            % \\\vspace{1.5cm}
    		% \professor

    		\vfill
    		\vfill

    		{\local\\\data}
        \end{titlepage}
    }

    \newcommand{\folhaDeRosto}{
        \newpage
        \begin{titlepage}
    		\center
            \nomes
            % \vspace{5cm}
            \vfill
            {\large{\MakeUppercase{\textbf{\titulo}}}}
            \vfill
            \hfill\begin{minipage}{0.5\linewidth}\onehalfspacing
            \notaDeRosto
            \end{minipage}
            \vfill
    		{\local\\\data}
        \end{titlepage}
    }

    \newcommand{\folhaAgradecimentos}{%
        \newpage\pagestyle{clear}
        \chapter*{\centerline{Agradecimentos}}
        %
        \agradecimentos
    }

%%%%%%%%%%%%%%%%%%%%%%%%%%%%%%%%%%%%%%%%%%%%%%%%%%%%%%%%%%%%%%%%%%%%%%%%%%%%%%%%

% Configurações de estilos de páginas

\AtBeginDocument{%
    \renewcommand{\headrulewidth}{0pt}
    %
    \ifx\standalone\undefined
        \setlength{\headheight}{1.5em}
    \else
        \fancyhf{}
        \renewcommand{\headrulewidth}{0pt}
	    \pagestyle{plain}
    \fi
    %
    \ifx\letter\or\article\undefined
        \titleformat{\chapter}{\normalfont \titlefont \bfseries}{\chaptername\ \thechapter}{1em}{}
        % \titleformat{\chapter}{\normalfont \titlefont \bfseries}{\thechapter}{1em}{}
        %
        \renewcommand{\chaptermark}[1]{\markboth{#1}{}} %mostrar somente o nome do capítulo com \leftmark
    \fi
    %
    \pagestyle{fancy}
    \fancypagestyle{plain}{% Página padrão
        \renewcommand{\headrulewidth}{0pt}
	    % \pagenumbering{arabic}
    }
    \fancypagestyle{main}{% Página padrão
        \fancyhf{}
        \lhead{}
        \chead{}
        \rhead{}
        \lfoot{}
        \cfoot{\thepage}
        \rfoot{}
	    \pagenumbering{arabic}
        \renewcommand{\headrulewidth}{0pt}
    }
    %
    \fancypagestyle{toc}{% Página  do sumário
        \renewcommand{\headrulewidth}{0pt}
        \fancyhf{}
        \lhead{}
        \chead{}
        \rhead{}
        \lfoot{}
        \cfoot{\thepage}
        \rfoot{}
	    \pagenumbering{Roman}
    }
    %
    \fancypagestyle{clear}{% Página do sumário
        \fancyhf{}
        \lhead{}
        \chead{}
        \rhead{}
        \lfoot{}
        \cfoot{}
        \rfoot{}
	   % \pagenumbering{None}
    }
    %
    \fancypagestyle{letterCapitania}{% Página do modelo de carta
        \fancyhf{}
        \renewcommand{\headrulewidth}{0pt}
	    \setlength\headheight{70pt}
        \lhead{
            \textbf{Carta de Intenções}\\
            \textbf{\today}\\\vspace{1em}
            Chapa: \chapa
            \vfill}
        \chead{}
        \rhead{\includegraphics[height=0.925\headheight]{Templates/logo_rocket.eps}}
        \lfoot{}
        \cfoot{\thepage}
        \rfoot{}
	    \pagenumbering{arabic}
    }
}

%%%%%%%%%%%%%%%%%%%%%%%%%%%%%%%%%%%%%%%%%%%%%%%%%%%%%%%%%%%%%%%%%%%%%%%%%%%%%%%%

% http://bcc.ufabc.edu.br/documentos/normalizacao.pdf

\renewcommand{\nomes}
{
	\begin{table}[H]
		\centering
		\begin{tabular}{c}

			Heitor Rodrigues Savegnago \\

		\end{tabular}
	\end{table}
}

% \renewcommand{\centro}{Centro de Matemática, Computação e Cognição}
% \renewcommand{\centroSigla}{CMCC}
\renewcommand{\centro}{Centro de Engenharia, Modelagem e Ciências Sociais Aplicadas}
\renewcommand{\centroSigla}{CECS}

% \renewcommand{\disciplina}{Trabalho de Graduação I em Engenharia de Informação}
% \renewcommand{\codigoDisciplina}{ESTI902-17}

% \renewcommand{\disciplina}{Projeto de Graduação em Computação I}
% \renewcommand{\codigoDisciplina}{MCTA029-17}

% \renewcommand{\grupo}{Grupo 1}

\renewcommand{\titulo}{Análise de sistemas de geolocalização baseados em GPS e AoA para foguetes de sondagem atmosférica}

% \renewcommand{\professor}{Prof. Dr. Francisco de Assis Zampirolli}
\renewcommand{\professor}{Prof. Dr. Ivan Roberto de Santana Casella}

\renewcommand{\local}{Santo André, SP}

\renewcommand{\data}{2025}

\renewcommand{\notaDeRosto}
{
    Trabalho de Conclusão de Curso apresentado ao \centro{} da Universidade Federal do ABC como requisito parcial à obtenção do título de Bacharel em Engenharia de Informação.

    \vspace{1em}

    Orientador: \professor.
}

\newcommand{\impecavel}{%
{\color{slideBlue}m}%
{\color{slideCyan}a}%
{\color{slideTurquoise}r}%
{\color{slideGreen}a}%
{\color{slideYellow}v}%
{\color{slideOrange}i}%
{\color{slideRed}l}%
{\color{slidePink}h}%
{\color{slidePurple}o}%
{\color{slideBlue}s}%
{\color{slideCyan}o}%
}

\newcommand{\palette}{{\Huge
    {\color{slideBlue}$\blacksquare$}
    {\color{slideCyan}$\blacksquare$}
    {\color{slideTurquoise}$\blacksquare$}
    {\color{slideGreen}$\blacksquare$}
    {\color{slideYellow}$\blacksquare$}
    {\color{slideOrange}$\blacksquare$}
    {\color{slideRed}$\blacksquare$}
    {\color{slidePink}$\blacksquare$}
    {\color{slidePurple}$\blacksquare$}
}}

\lstset{
    language=matlab,
    morekeywords=[1]{deg2rad},
    morekeywords=[2]{ref_sin, ref_cos, isoctave, generate_fig, signal_r, argument_r},
}


% \newcommand{\x}{$\bullet$}





% \newcommand\pgfmathsinandcos[3]{%
%   \pgfmathsetmacro#1{sin(#3)}%
%   \pgfmathsetmacro#2{cos(#3)}%
% }
% \newcommand\LongitudePlane[3][current plane]{%
%   \pgfmathsinandcos\sinEl\cosEl{#2} % elevation
%   \pgfmathsinandcos\sint\cost{#3} % azimuth
%   \tikzset{#1/.style={cm={\cost,\sint*\sinEl,0,\cosEl,(0,0)}}}
% }

% \newcommand\LatitudePlane[3][current plane]{%
%   \pgfmathsinandcos\sinEl\cosEl{#2} % elevation
%   \pgfmathsinandcos\sint\cost{#3} % latitude
%   \pgfmathsetmacro\yshift{\RadiusSphere*\cosEl*\sint}
%   \tikzset{#1/.style={cm={\cost,0,0,\cost*\sinEl,(0,\yshift)}}} %
% }
% \newcommand\NewLatitudePlane[4][current plane]{%
%   \pgfmathsinandcos\sinEl\cosEl{#3} % elevation
%   \pgfmathsinandcos\sint\cost{#4} % latitude
%   \pgfmathsetmacro\yshift{#2*\cosEl*\sint}
%   \tikzset{#1/.style={cm={\cost,0,0,\cost*\sinEl,(0,\yshift)}}} %
% }
% \newcommand\DrawLongitudeCircle[2][1]{
%   \LongitudePlane{\angEl}{#2}
%   \tikzset{current plane/.prefix style={scale=#1}}
%    % angle of "visibility"
%   \pgfmathsetmacro\angVis{atan(sin(#2)*cos(\angEl)/sin(\angEl))} %
%   \draw[current plane] (\angVis:1) arc (\angVis:\angVis+180:1);
%   \draw[current plane,opacity=0.4] (\angVis-180:1) arc (\angVis-180:\angVis:1);
% }
% \newcommand\DrawLongitudeArc[4][black]{
%   \LongitudePlane{\angEl}{#2}
%   \tikzset{current plane/.prefix style={scale=1}}
%   \pgfmathsetmacro\angVis{atan(sin(#2)*cos(\angEl)/sin(\angEl))} %
%   \pgfmathsetmacro\angA{mod(max(\angVis,#3),360)} %
%   \pgfmathsetmacro\angB{mod(min(\angVis+180,#4),360} %
%   \draw[current plane,#1,opacity=0.4] (#3:\RadiusSphere) arc (#3:#4:\RadiusSphere);
%   \draw[current plane,#1]  (\angA:\RadiusSphere) arc (\angA:\angB:\RadiusSphere);
% }%
% \newcommand\DrawLatitudeCircle[2][1]{
%   \LatitudePlane{\angEl}{#2}
%   \tikzset{current plane/.prefix style={scale=#1}}
%   \pgfmathsetmacro\sinVis{sin(#2)/cos(#2)*sin(\angEl)/cos(\angEl)}
%   % angle of "visibility"
%   \pgfmathsetmacro\angVis{asin(min(1,max(\sinVis,-1)))}
%   \draw[current plane] (\angVis:1) arc (\angVis:-\angVis-180:1);
%   \draw[current plane,opacity=0.4] (180-\angVis:1) arc (180-\angVis:\angVis:1);
% }

% \newcommand\DrawLatitudeArc[4][black]{
%   \LatitudePlane{\angEl}{#2}
%   \tikzset{current plane/.prefix style={scale=1}}
%   \pgfmathsetmacro\sinVis{sin(#2)/cos(#2)*sin(\angEl)/cos(\angEl)}
%   % angle of "visibility"
%   \pgfmathsetmacro\angVis{asin(min(1,max(\sinVis,-1)))}
%   \pgfmathsetmacro\angA{max(min(\angVis,#3),-\angVis-180)} %
%   \pgfmathsetmacro\angB{min(\angVis,#4)} %
%   \draw[current plane,#1,opacity=0.4] (#3:\RadiusSphere) arc (#3:#4:\RadiusSphere);
%   \draw[current plane,#1] (\angA:\RadiusSphere) arc (\angA:\angB:\RadiusSphere);
% }

% %% document-wide tikz options and styles

% \tikzset{%
%   >=latex, % option for nice arrows
%   inner sep=0pt,%
%   outer sep=2pt,%
%   mark coordinate/.style={inner sep=0pt,outer sep=0pt,minimum size=3pt,
%     fill=black,circle}%
% }




\usepackage{tikz-3dplot}

%Angle Definitions
%-----------------

%set the plot display orientation
%synatax: \tdplotsetdisplay{\theta_d}{\phi_d}
% \tdplotsetmaincoords{65}{110}
% \tdplotsetmaincoords{60}{135}
\tdplotsetmaincoords{54.736}{135}
% \tdplotsetmaincoords{60}{45}

% there's got to be a better way to do this.
\newcommand{\Normalize}[3]
{
    \pgfmathsetmacro{\normyn}{sqrt(#1*#1+#2*#2+#3*#3)}
    \pgfmathsetmacro{\normx}{#1/\normyn}\pgfmathsetmacro{\normy}{#2/\normyn}\pgfmathsetmacro{\normz}{#3/\normyn}
}

% calculate the counterclockwise angle of a vector of length 1 in the rotated xy plane.
\newcommand{\toAngle}[3]
{
    \tdplottransformrotmain{1}{0}{0}
    \pgfmathsetmacro\xa{acos(\tdplotresx *#1 + \tdplotresy* #2 + \tdplotresz* #3)}
    \tdplottransformrotmain{0}{1}{0}
    \pgfmathsetmacro\ya{acos(\tdplotresx *#1 + \tdplotresy* #2 + \tdplotresz* #3)}
    \pgfmathsetmacro\normySum{round(\xa+\ya)}
    \pgfmathsetmacro\normyDiff{round(\xa-\ya )}
    \ifthenelse{\lengthtest{\normySum pt = 270pt}} {
        \pgfmathsetmacro\normyAngle{\ya+90}
    }{
        \ifthenelse{\lengthtest{\normyDiff pt = -90pt}} {
            \pgfmathsetmacro\normyAngle{360-\xa }
        }{
            \pgfmathsetmacro\normyAngle{\xa }
        }
    }
}



\newcommand{\drawArc}[9]
{
    \Normalize{#1}{#2}{#3}
    \pgfmathsetmacro{\pax}{\normx}\pgfmathsetmacro{\pay}{\normy}    \pgfmathsetmacro{\paz}{\normz}

    \Normalize{#4}{#5}{#6}
    \pgfmathsetmacro{\pbx}{\normx}\pgfmathsetmacro{\pby}{\normy}    \pgfmathsetmacro{\pbz}{\normz}

    % take the cross product and normalize it
    \tdplotcrossprod(\pax,\pay,\paz)(\pbx,\pby,\pbz)

    % calculate the rotation that maps the z axis onto the cross product
    \tdplotsetrotatedcoords{atan2(\tdplotresy,\tdplotresx)}{atan2(sqrt(\tdplotresx*\tdplotresx+\tdplotresy*\tdplotresy),\tdplotresz)}{0.0}

    % calculate the counterclockwise angles from the rotated x axis to each vector, then order them increasing.
    \toAngle{\pax}{\pay}{\paz}
    \pgfmathsetmacro\xangle{\normyAngle}

    \toAngle{\pbx}{\pby}{\pbz}
    \pgfmathsetmacro\yangle{\normyAngle}

    \ifthenelse{\lengthtest{\xangle pt < \yangle pt}} {
        \pgfmathsetmacro\first{\xangle}
        \pgfmathsetmacro\second{\yangle}
    }{
        \pgfmathsetmacro\first{\yangle}
        \pgfmathsetmacro\second{\xangle}
    }
    % draw the arc at radius R from the first angle to the second

    \tdplotdrawarc[tdplot_rotated_coords, cmyk_R, very  thick, -latex,]{(0,0,0)}{#7}{\first }{\second }{#8}{#9}
}






% \newcommand{\chapter}[1]{#1}

\bibliography{../reference.bib}

\DeclareAcronym{AoA}{
    % short = AoA,
    long = {Ângulo de Chegada},
    pdfcomment = {Ângulo de Chegada},
    foreign = \textit{Angle of Arrival}
}

\DeclareAcronym{BLE}{
    % short = AoA,
    long = {\textit{Bluetooth} de Baixo Consumo de Energia},
    pdfcomment = {Bluetooth de Baixo Consumo de Energia},
    foreign = \textit{Bluetooth Low Energy}
}

\DeclareAcronym{IoT}{
    % short = IoT,
    long = {Internet das Coisas},
    pdfcomment = {Internet das Coisas},
    foreign = \textit{Internet of Things}
}

\DeclareAcronym{NLoS}{
    % short = NLoS,
    long = {Sem Linha de visão},
    pdfcomment = {Sem Linha de visão},
    foreign = \textit{Non Line of Sight}
}

\DeclareAcronym{LoS}{
    % short = LoS,
    long = {Linha de visão},
    pdfcomment = {Linha de visão},
    foreign = \textit{Line of Sight}
}

\DeclareAcronym{LoRa}{
    % short = LoRa,
    long = {\textit{Longe Range}},
    pdfcomment = {Longe Range},
}

\DeclareAcronym{GPS}{
    % short = GPS,
    long = {Sistema de Posicionamento Global},
    pdfcomment = {Sistema de Posicionamento Global},
    foreign = \textit{Global Positioning System}
}

\DeclareAcronym{GNSS}{
    % short = GNSS,
    long = {Sistema Global de Navegação por Satélite},
    pdfcomment = {Sistema Global de Navegação por Satélite},
    foreign = \textit{Global Navigation Satellite System}
}

\DeclareAcronym{NaN}{
    long = {Número Indefinido ou Inválido},
    pdfcomment = {Número Indefinido ou Inválido},
    foreign = \textit{Not a Number}
}

\DeclareAcronym{RF}{
    % short = RF,
    long = {Radiofrequência},
    pdfcomment = {Radiofrequência},
    foreign = \textit{Radio Frequency}
}

\DeclareAcronym{FFT}{
    % short = FFT,
    long = {Transformada Rápida de Fourier},
    pdfcomment = {Transformada Rápida de Fourier},
    foreign = \textit{Fast Fourier Transform}
}

\DeclareAcronym{UCA}{
    % short = UCA,
    long = {Arranjo Circular Uniforme},
    pdfcomment = {Arranjo Circular Uniforme},
    foreign = \textit{Uniform Circular Array}
}

\DeclareAcronym{ULA}{
    % short = ULA,
    long = {Arranjo Linear Uniforme},
    pdfcomment = {Arranjo Linear Uniforme},
    foreign = \textit{Uniform Linear Array}
}

\DeclareAcronym{SDR}{
    % short = SDR,
    long = {Rádio Definido por \textit{Software}},
    pdfcomment = {Rádio Definido por \textit{Software}},
    foreign = \textit{Software-Defined Radio}
}

\DeclareAcronym{SNR}{
    % short = SNR,
    long = {Relação Sinal-Ruído},
    pdfcomment = {Relação Sinal-Ruído},
    foreign = \textit{Signal-Noise Ratio}
}


\DeclareAcronym{AWGN}{
    long = {Ruído Gaussiano Branco Aditivo},
    pdfcomment = {Ruído Gaussiano Branco Aditivo},
    foreign = \textit{Additive White Gaussian Noise}
}

\DeclareAcronym{ATT}{
    long = {Atenuação},
    pdfcomment = {Atenuação}
}

% \DeclareAcronym{GN}{
%     long = {Gauss-Newton},
%     pdfcomment = {Gauss-Newton}
% }

\DeclareAcronym{MUSIC}{
    long = {Classificação de Múltiplos Sinais},
    pdfcomment = {Classificação de Múltiplos Sinais},
    foreign = \textit{MUltiple SIgnal Classification}
}

\DeclareAcronym{GEO}{
    long = {Geométrico},
    pdfcomment = {Geométrico}
}

\DeclareAcronym{Ag}{
    short = {\ensuremath{\mathbf{A}_\text{g}}},
    long = {},
	list = {Coordenada geográfica A},
	pdfcomment = {Coordenada geográfica A},
    first-style = short,
    tag = symbols,
    sort = {0_Ag}
}

\DeclareAcronym{Bg}{
    short = {\ensuremath{\mathbf{B}_\text{g}}},
    long = {},
	list = {Coordenada geográfica B},
	pdfcomment = {Coordenada geográfica B},
    first-style = short,
    tag = symbols,
    sort = {0_Bg}
}

\DeclareAcronym{dAB}{
    short = {\ensuremath{d_{\mathbf{AB}}}},
    long = {},
	list = {Distância entre coordenadas geográficas A e B},
	pdfcomment = {Distância entre coordenadas geográficas A e B},
    first-style = short,
    tag = symbols,
    sort = {0_dAB}
}


\DeclareAcronym{Rterra}{
    short = {\ensuremath{R_{\text{Terra}}}},
    long = {},
	list = {Raio do planeta},
	pdfcomment = {Raio do planeta},
    first-style = short,
    tag = symbols,
    sort = {0_Rterra}
}

%%%%%%%%%%%%%%%%%%%%%%%%%%%%%%%%

\DeclareAcronym{betab}{
    short = {\ensuremath{\beta_{b}}},
    long = {},
	list = {Ângulo de \textit{bearing} relativo},
	pdfcomment = {Ângulo de \textit{bearing} relativo},
    first-style = short,
    tag = symbols,
    sort = {1_02_15betab}
}

\DeclareAcronym{thetaA}{
    short = {\ensuremath{\theta_{\mathbf{A}_\text{g}}}},
    long = {},
	list = {Longitude da coordenada geográfica A},
	pdfcomment = {Longitude da coordenada geográfica A},
    first-style = short,
    tag = symbols,
    sort = {1_08_15thetaA}
}


\DeclareAcronym{thetaB}{
    short = {\ensuremath{\theta_{\mathbf{B}_\text{g}}}},
    long = {},
	list = {Longitude da coordenada geográfica B},
	pdfcomment = {Longitude da coordenada geográfica B},
    first-style = short,
    tag = symbols,
    sort = {1_08_15thetaB}
}

\DeclareAcronym{phiA}{
    short = {\ensuremath{\phi_{\mathbf{A}_\text{g}}}},
    long = {},
	list = {Latitude da coordenada geográfica A},
	pdfcomment = {Latitude da coordenada geográfica A},
    first-style = short,
    tag = symbols,
    sort = {1_21_15_phiA}
}

\DeclareAcronym{phiB}{
    short = {\ensuremath{\phi_{\mathbf{B}_\text{g}}}},
    long = {},
	list = {Latitude da coordenada geográfica B},
	pdfcomment = {Latitude da coordenada geográfica B},
    first-style = short,
    tag = symbols,
    sort = {1_21_15_phiB}
}

\DeclareAcronym{Deltatheta}{
    short = {\ensuremath{\Delta_\theta}},
    long = {},
	list = {Diferença entre os ângulos \ac{thetaA} e \ac{thetaB}},
	pdfcomment = {Diferença entre os ângulos \texttheta{}[A] e \texttheta{}[B]},
    first-style = short,
    tag = symbols,
    sort = {1_04_01_08Deltatheta}
}

\DeclareAcronym{Deltaphi}{
    short = {\ensuremath{\Delta_\phi}},
    long = {},
	list = {Diferença entre os ângulos \ac{phiA} e \ac{phiB}},
	pdfcomment = {Diferença entre os ângulos \textphi{}[A] e \textphi{}[B]},
    first-style = short,
    tag = symbols,
    sort = {1_04_01_21Deltaphi}
}


%%%%%%%%%%%%%%%%%%%%%%%%%%%%%%%%

\DeclareAcronym{Ak}{
    short = {\ensuremath{A_k}},
    long = {},
	list = {\ac{k}-ésima antena do arranjo},
	pdfcomment = {k-ésima antena do arranjo},
    first-style = short,
    tag = symbols,
    sort = {0_Ak}
}

\DeclareAcronym{c}{
    short = {\ensuremath{c}},
    long = {},
	list = {Velocidad da luz no ar},
	pdfcomment = {Velocidad da luz no ar},
    first-style = short,
    tag = symbols,
    sort = {0_c}
}

\DeclareAcronym{D}{
    short = {\ensuremath{D_{\text{ant}}}},
    long = {},
	list = {Maior dimensão da antena emissora para cálculo da distância de Fraunhofer},
	pdfcomment = {Maior dimensão da antena emissora para cálculo da distância de Fraunhofer},
    first-style = short,
    tag = symbols,
    sort = {0_d_0D}
}

\DeclareAcronym{d}{
    short = {\ensuremath{d}},
    long = {},
	list = {Distância entre antenas},
	pdfcomment = {Distância entre antenas},
    first-style = short,
    tag = symbols,
    sort = {0_d_1d}
}

\DeclareAcronym{dFran}{
    short = {\ensuremath{d_\text{F}}},
    long = {},
	list = {Distância de Fraunhofer},
	pdfcomment = {Distância de Fraunhofer},
    first-style = short,
    tag = symbols,
    sort = {0_d_1dFran}
}

\DeclareAcronym{f}{
    short = {\ensuremath{f}},
    long = {},
	list = {Frequência do sinal \ac{w} de interesse},
	pdfcomment = {Frequência do sinal w de interesse},
    first-style = short,
    tag = symbols,
    sort = {0_f}
}

\DeclareAcronym{Ik}{
    short = {\ensuremath{I_k}},
    long = {},
	list = {Componente em fase do valor complexo do sinal recebido},
	pdfcomment = {Componente em fase do valor complexo do sinal recebido},
    first-style = short,
    tag = symbols,
    sort = {0_Ik}
}

\DeclareAcronym{imath}{
    short = {\ensuremath{\imath}},
    long = {},
	list = {Unidade imaginária, $\scriptstyle \sqrt{-1}$},
	pdfcomment = {Unidade imaginária, raiz(-1)},
    first-style = short,
    tag = symbols,
    sort = {0_imath}
}

\DeclareAcronym{k}{
    short = {\ensuremath{k}},
    long = {},
	list = {Índice das antenas do arranjo},
	pdfcomment = {Índice das antenas do arranjo},
    first-style = short,
    tag = symbols,
    sort = {0_k}
}


\DeclareAcronym{Nant}{
    short = {\ensuremath{N_\text{ant}}},
    long = {},
	list = {Número de antenas do arranjo},
	pdfcomment = {Número de antenas do arranjo},
    first-style = short,
    tag = symbols,
    sort = {0_Nant}
}

\DeclareAcronym{Qk}{
    short = {\ensuremath{Q_k}},
    long = {},
	list = {Componente em quadratura do valor complexo do sinal recebido},
	pdfcomment = {Componente em quadratura do valor complexo do sinal recebido},
    first-style = short,
    tag = symbols,
    sort = {0_Qk}
}

\DeclareAcronym{T}{
    short = {\ensuremath{T}},
    long = {},
	list = {Período do sinal \ac{w} de interesse},
	pdfcomment = {Período do sinal w de interesse},
    first-style = short,
    tag = symbols,
    sort = {0_T}
}

\DeclareAcronym{w}{
    short = {\ensuremath{w}},
    long = {},
	list = {Sinal de interesse para análise, incidente no arranjo de antenas},
	pdfcomment = {Sinal de interesse para análise, incidente no arranjo de antenas},
    first-style = short,
    tag = symbols,
    sort = {0_w}
}

\DeclareAcronym{xk}{
    short = {\ensuremath{x_{A_k}}},
    long = {},
	list = {Componente $x$ de coordenada para a antena \ac{Ak}},
	pdfcomment = {Componente x de coordenada para a antena A[k]},
    first-style = short,
    tag = symbols,
    sort = {0_xk}
}

\DeclareAcronym{yk}{
    short = {\ensuremath{y_{A_k}}},
    long = {},
	list = {Componente $y$ de coordenada para a antena \ac{Ak}},
	pdfcomment = {Componente y de coordenada para a antena A[k]},
    first-style = short,
    tag = symbols,
    sort = {0_yk}
}

\DeclareAcronym{Zk}{
    short = {\ensuremath{Z_k}},
    long = {},
	list = {Valor complexo do sinal recebido na antena \ac{Ak}},
	pdfcomment = {Valor complexo do sinal recebido na antena A[k]},
    first-style = short,
    tag = symbols,
    sort = {0_Zk}
}


%%%%%%%%%%%%%%%%%%%%%%%%%%%%%%%%

\DeclareAcronym{alphak}{
	short = {\ensuremath{\alpha_k}},
    long = {},
	list = {Ângulo formado pelo par de antenas \ac{Ak} e $A_{k+1}$ em relação à geometria do sistema},
	pdfcomment = {Ângulo formado pelo par de antenas A[k] e A[k+1] em relação à geometria do sistema},
    first-style = short,
    tag = symbols,
    sort = {1_01_10alphak}
}

\DeclareAcronym{betak}{
    short = {\ensuremath{\beta_{\pm k}}},
    long = {},
	list = {Par de ângulos simétricos calculados a partir do sinal incidente no par de antenas \ac{Ak} e $A_{k+1}$},
	pdfcomment = {Par de ângulos simétricos calculados a partir do sinal incidente no par de antenas A[k] e A[k+1]},
    first-style = short,
    tag = symbols,
    sort = {1_02_10betak}
}

\DeclareAcronym{DeltaPhi}{
    short = {\ensuremath{\Delta \Phi_{k}}},
    long = {},
	list = {Diferença de fase em par de antenas \ac{Ak} e $A_{k+1}$},
	pdfcomment = {Diferença de fase em par de antenas A[k] e A[k+1]},
    first-style = short,
    tag = symbols,
    sort = {1_04_00DeltaPhi}
}

\DeclareAcronym{delta}{
    short = {\ensuremath{\delta}},
    long = {},
	list = {Intervalo de quantização e de filtro para valores de \ac{Theta}},
	pdfcomment = {Intervalo de quantização e de filtro para valores de \textTheta\ },
    first-style = short,
    tag = symbols,
    sort = {1_04_10delta}
}

\DeclareAcronym{Theta}{
    short = {\ensuremath{\Theta}},
    long = {},
	list = {Conjunto de todos os valores \ac{thetak} aferidos},
	pdfcomment = {Conjunto de todos os valores \texttheta{}[\textpm{}k] aferidos},
    first-style = short,
    tag = symbols,
    sort = {1_08_00Theta}
}

\DeclareAcronym{ThetaQuanti}{
    short = {\ensuremath{\Theta_{\left\lfloor\bullet\right\rceil}}},
    long = {},
	list = {Valores de \ac{Theta} quantizados por \ac{delta}},
	pdfcomment = {Valores de \textTheta\ quantizados por \textdelta\ },
    first-style = short,
    tag = symbols,
    sort = {1_08_01ThetaQuanti}
}

\DeclareAcronym{ThetaFiltro}{
    short = {\ensuremath{\Theta_\text{F}}},
    long = {},
	list = {Valores de \ac{Theta} filtrados},
	pdfcomment = {Valores de \textTheta\ filtrados},
    first-style = short,
    tag = symbols,
    sort = {1_08_02ThetaFiltro}
}

\DeclareAcronym{thetaAoA}{
    short = {\ensuremath{\theta_{\text{\acs{AoA}}}}},
    long = {},
	list = {Valor do Ângulo de Chegada \acs{AoA}},
	pdfcomment = {Valor do Ângulo de Chegada (AoA)},
    first-style = short,
    tag = symbols,
    sort = {1_08_10thetaAoA}
}

\DeclareAcronym{thetak}{
    short = {\ensuremath{\theta_{\pm k}}},
    long = {},
	list = {Par de possíveis ângulos de \ac{thetaAoA} referentes ao par de antenas \ac{Ak} e $A_{k+1}$},
	pdfcomment = {Par de possíveis ângulos de \texttheta{}[AoA] referentes ao par de antenas A[k] e A[k+1]},
    first-style = short,
    tag = symbols,
    sort = {1_08_11thetak}
}

\DeclareAcronym{thetaMo}{
    short = {\ensuremath{\theta_\mathcal{M_o}}},
    long = {},
	list = {Moda estatística do conjunto \ac{ThetaQuanti}},
	pdfcomment = {Moda estatística do conjunto \texttheta\ quantizado},
    first-style = short,
    tag = symbols,
    sort = {1_08_12thetaMo}
}

\DeclareAcronym{lambda}{
    short = {\ensuremath{\lambda}},
    long = {},
	list = {Comprimento de onda do sinal \ac{w} de interesse},
	pdfcomment = {Comprimento de onda do sinal w de interesse },
    first-style = short,
    tag = symbols,
    sort = {1_11_10lambda}
}

\DeclareAcronym{rho}{
    short = {\ensuremath{\rho}},
    long = {},
	list = {Raio do polígono regular formador do arranjo de antenas},
	pdfcomment = {Raio do polígono regular formador do arranjo de antenas},
    first-style = short,
    tag = symbols,
    sort = {1_17_10rho}
}

\DeclareAcronym{Phik}{
    short = {\ensuremath{\Phi_k}},
    long = {},
	list = {Fase do sinal na antena \ac{k}},
	pdfcomment = {Fase do sinal na antena k},
    first-style = short,
    tag = symbols,
    sort = {1_21_00Phik}
}

\DeclareAcronym{omega}{
    short = {\ensuremath{\omega}},
    long = {},
	list = {Frequência angular do sinal \ac{w} de interesse},
	pdfcomment = {Frequência angular do sinal w de interesse},
    first-style = short,
    tag = symbols,
    sort = {1_24_10omega}
}





%%%%%%%%%%%%%%%%%%%%%%%%%%%%%%%%%%%%%%%%%%%%%%%%%%%%%%%%%%%%%%%%

\DeclareAcronym{round}{
    short = {\ensuremath{\left\lfloor\mathcal{h}\right\rceil}},
    long = {},
	list = {Operação arredondar, arredonda o valor de $\mathcal{h}$ para o inteiro mais próximo},
    first-style = short,
    tag = operators,
    sort = id
}

\DeclareAcronym{real}{
    short = {\ensuremath{\operatorname{\mathcal{Re}}\left(\mathcal{h}\right)}},
    long = {},
	list = {Parte real do valor complexo $\mathcal{h}$},
    first-style = short,
    tag = operators,
    sort = id
}

\DeclareAcronym{imag}{
    short = {\ensuremath{\operatorname{\mathcal{Im}}\left(\mathcal{h}\right)}},
    long = {},
	list = {Parte imaginária do valor complexo $\mathcal{h}$},
    first-style = short,
    tag = operators,
    sort = id
}

\DeclareAcronym{moda}{
    short = {\ensuremath{\operatorname{\mathcal{Mo}}\left(\mathcal{H}\right)}},
    long = {},
	list = {Operação estatística moda para o conjunto $\mathcal{H}$},
    first-style = short,
    tag = operators,
    sort = id
}

\DeclareAcronym{mean}{
    short = {\ensuremath{\widetilde{\mathcal{H}}}},
    long = {},
	list = {Operação estatística mediana para o conjunto $\mathcal{H}$},
    first-style = short,
    tag = operators,
    sort = id
}


\begin{document}
    \pagestyle{clear}
    \capa

    \cleardoublepage

    \folhaDeRosto

    \cleardoublepage

    \folhaAgradecimentos

    \cleardoublepage

    % \folhaDeAprovacao
    \begin{abstract}%
    %
        % \MexerDepois{Alterar}

O objetivo principal do projeto é a concepção de um sistema de localização relativa para um foguete de sondagem atmosférica.
Este sistema servirá para guiar um grupo de busca em campo na tarefa de localizar o veículo após o aterrizagem.


O direcionamento da busca seria feita utilizando o método de ângulo de chegada relativo ao sinal oriundo da telemetria do foguete.
O sinal será recebido por uma matriz de antenas, viabilizando os cálculos de ângulo.


O objetivo secundário do projeto é comparar a performance deste sistema com a performance de outro sistema de localização relativa, que utiliza coordenadas geográficas para os cálculos.
Este outro sistema se baseia em cálculos de azimute entre as coordenadas do veículo e do grupo de busca.

Por se tratar de um sistema com dados processados no veículo, se houverem problemas internos com o GPS de bordo, estes cálculos são inviabilizados.
Outro problema desde sistema é relativo à precisão da coordenada geográfica, fazendo com que, à certa distância, a precisão da direção perca sua confiabilidade.

Por outro lado, a versão utilizando o angulo de chegada não depende dos dados transmitidos no sinal, bem como poderá funcionar bem a distâncias mais curtas.






% , um sistema que "aponte" a direção a se seguir, e comparar a performance deste com um sistema que apontaria na mesma direção, mas baseado em cálculos de Azimuth relativo usando duas coordenadas de GPS, do foguete e da equipe de busca.
% Eu já construí esse sistema baseado em GPS, ele funciona até que bem, mas a taxa de atualização do GPS que eu tenho é bem ruim lenta, devo usar isso como parte da minha justificativa pra concepção do projeto.

Ao ser lançado, um foguete de sondagem atmosférica pode pousar em qualquer lugar dentro do seu raio de alcance, e realizar a busca pode se tornar um grande desafio sem uma estratégia de localização eficaz.
Muitos desses veículos contam com localização por \acs{GNSS}, como o \acs{GPS}, porém ainda dependem de um sistema de telemetria que garanta a correta transmissão das coordenadas geográficas à equipe de busca.
O presente trabalho considera o cenário onde a informação de localização não pôde ser decodificada, porém o sinal de \acs{RF} ainda é detectável, propondo a aferição de \acf{AoA} para determinar a direção de busca.
Foi construída a simulação de um sistema que, baseado na diferença de defasagens em uma malha circular de antenas, é capaz de determinar o \acs{AoA} do sinal \acs{RF} incidente.
Durante o desenvolvimento, foi almejada a compatibilidade com diferentes \textit{softwares} de resolução numérica, particularmente o GNU Octave e o MATLAB, o que se mostrou um desafio, considerando as limitações no uso de \textit{software} livre.
Foram consideradas malhas de antenas com três, cinco e sete antenas.
A geometria com menos antenas apresentou uma acurácia geral mais baixa comparada às demais.
As simulações contemplaram casos com diferentes níveis de interferência por ruído do tipo \acs{AWGN}, e consideraram cenários com e sem atenuação no sinal.
Em todas as simulações, o valor de R\textsuperscript{2} foi acima de \qty{75}{\percent} e, em média, acima de \qty{92}{\percent}.
% À medida de comparação, também foram simulados casos equivalentes utilizando o algoritmo de \acl{GN}, que demonstrou problemas em lidar com ruído e atenuação.
Esses resultados indicam que o método proposto se mostra eficaz em diferentes contextos.

\paragraph*{Palavras-chave:} \textit{Angle of Arrival}; Radiofrequência; Foguetes de Sondagem Atmosférica; Resolução Numérica; Telemetria; Localização.

%
    %
    \end{abstract}

    % \begin{otherlanguage}{english}\itshape
    %     \begin{abstract}%
    %     %
    %         Upon launch, an atmospheric sounding rocket can land anywhere within its range, and conducting the search can be a major challenge without an effective location strategy.
Many of these vehicles rely on \acs{GNSS} location, such as \acs{GPS}, but they still rely on a telemetry system to ensure the correct transmission of geographic coordinates to the search team.
This work considers a scenario where the location information could not be decoded, but the \acs{RF} signal is still detectable, proposing the Angle of Arrival (\acs{AoA}) measurement to determine the search direction.
A simulation was built for a system that, based on the phase shift difference in a circular antenna array, is capable of determining the \acs{AoA} of the incident \ac{RF} signal.
During development, compatibility with various numerical resolution software programs was sought, particularly GNU Octave and MATLAB, which proved challenging given the limitations of using free software.
Antenna arrays with three, five, and seven antennas were considered. The geometry with fewer antennas showed lower overall accuracy compared to the others.
The simulations considered cases with different levels of \acs{AWGN} noise interference and considered scenarios with and without signal attenuation.
In all simulations, the R\textsuperscript{2} value was above \qty{75}{\percent} and, on average, above \qty{92}{\percent}.
For comparison, equivalent cases were also simulated using the \acl{GN} algorithm, which demonstrated problems in dealing with noise and attenuation.
These results indicate that the proposed method is effective in different contexts.

\paragraph*{\textit{Keywords:}} Angle of Arrival; Radio Frequency; Atmospheric Sounding Rockets; Numerical Resolution; Telemetry; Location.
%
    %     %
    %     \end{abstract}
    % \end{otherlanguage}

    % \begingroup

    \cleardoublepage
    \pagestyle{toc}

    \listoffigures

    \listoftables

    \lstlistoflistings

    % \addcontentsline{toc}{chapter}{Abreviaturas e Siglas}
    \noindent\printacronyms[%
        display=all,
        exclude={symbols, operators}
    ]

    \noindent\printacronyms[%
        display=all,
        name={\centerline{Símbolos e Operadores}},
        include={symbols, operators}
    ]
    \cleardoublepage

    \tableofcontents % Sumário
    % \pagebreak
    \cleardoublepage

    % \endgroup

    \pagestyle{main}
    \chapter{Introdução}

\section{Motivação}

% foguetes de sondagem, o que são? onde vivem?
% como achar depois?
% Um GNSS simples serve?
% Opção de dois dispositivos GPS, mas funciona sempre?
% Proposta de sistema baseado em AoA


Foguetes de sondagem atmosférica são veículos aeroespaciais sub-orbitais utilizados para levar sensores e experimentos científicos a altos níveis atmosféricos, com o intuito de realizar estudos e análises relacionados às diversas condições ali presentes \cite{isro}.
Estes veículos geralmente utilizam motor de propelente sólido, de um ou dois estágios, e são equipados com sistemas de controle, telemetria e recuperação, além de transportarem o experimento científico, denominado carga-paga \cite{esa, sabbatini2014esa}.
Algumas das vantagens desses veículos são o baixo custo e a menor necessidade de alcance pra sistemas de telemetria e rastreio, tendo em vista que não entram em órbita \cite{nasa}.

No contexto de foguetes de sondagem, existem competições que fomentam o desenvolvimento e competitividade em equipes universitárias de foguetemodelismo \cite{esra}.
Algumas dessas competições tem grande parte de suas categorias definidas nas bases de foguetes de sondagem, com apogeu de voo entre \SI{1}{\kilo\metre} e \SI{10}{\kilo\metre} de altura acima do nível do solo.
Nestes casos, a sequência de operações normal do foguete, apresentada na \autoref{fig:conops}, consiste em: ignição do primeiro estágio do motor, decolagem, período propulsionado, término de queima do primeiro estágio, desacoplamento do primeiro estágio, ignição do segundo estágio, segundo período propulsionado, término de queima do segundo estágio, início do período inercial balístico, apogeu, detecção do apogeu pelos sistemas embarcados e liberação do paraquedas piloto, liberação do paraquedas principal a certa altitude e finalmente o pouso \cite{esa, sabbatini2014esa}.
Desacoplamento do primeiro estágio, ignição e fase propulsionada do segundo somente se aplicam a foguetes de dois estágios.

\begin{figure}[h]
    \centering
    \caption{Sequência operações normal para foguete de sondagem.}
    % \begin{center}
\begin{tikzpicture}

    % Variables


    \coordinate (bottomleft) at (-0.5,-0.5);
    \coordinate (topright) at (10.5,5);

    \def\coordref[#1](#2){%

        \coordinate(sysref) at (#2);

        \draw[#1, -latex] (sysref) ++(-0.4,-0.3) -- ++(0.9,0) node[midway, below]{$x$};
        \draw[#1, -latex] (sysref) ++(-0.3,-0.4) -- ++(0,0.9) node[midway, left]{$y$};
        \draw[#1, -latex] \centerarc(sysref)(-90:180:0.25);
        \draw[#1] (sysref) node{$+$}
    }

    % \draw[Red,dashed] (bottomleft) rectangle (topright);
    \clip (bottomleft) rectangle (topright);

    \coordinate (O) at (0,0);

    % \draw [help lines, dashed] (bottomleft) grid (topright); % desenha grid
    % \draw [red] (O) node[draw,cross out] {}; % marca pont(0,0)

    % \draw[Red] (4,2)
    %     node[draw, thick, shape=foguete, fill=cmyk_R!25, scale=4] {}
    %     node[draw, circle, inner sep=2pt] {}
    % ;

    % \shade[inner color=yellow,outer color=white] (6,0) rectangle +(2,1);

    % \draw
    %     (-0.5,0) -- (10.5,0)
    % ;


    \coordinate (left) at (-0.5,0);
    \coordinate (right) at (10.5,0);
    \coordinate (middle) at ($(left)!0.5!(right)$);
    \draw[thin, Black!50, path fading=west] (left) -- (middle);
    \draw[thin, Black!50, path fading=east] (middle) -- (right);
    % \shade[draw, inner color=Green!10,outer color=Red] (-0.5,0) -- (10.5,3);


    \draw[cmyk_K] (0,0) node[draw, shape=foguete, fill=cmyk_R, anchor=south, rotate=0] {};

    \draw[cmyk_K] (0.5,2) node[OrangeRed, rotate=170, anchor=south, inner sep=-2.5pt, xshift=0.25pt] {\Fire} node[draw, shape=foguete, fill=cmyk_R, anchor=south, rotate=-10] {};

    \draw[cmyk_K] (5,4) node[draw, shape=foguete, fill=cmyk_R, anchor=south, rotate=-95] {};

    \draw[cmyk_K] (9,0) node[draw, shape=foguete, fill=cmyk_R, anchor=east, rotate=-90] {};



    \begin{pgfonlayer}{background}    % select the background layer
        \clip (bottomleft) rectangle (10.5,0);
        \shade[inner color=SaddleBrown!50,outer color=White] (bottomleft) rectangle (10.5,0.5);
    \end{pgfonlayer}


\end{tikzpicture}
% \end{center}
    % \includegraphics{}
    \caption*{Fonte: Autor}
    \label{fig:conops}
\end{figure}

A partir do momento do pouso, o próximo objetivo nessas competições consiste em localizar o foguete, vários métodos podem ser empregados nessa situação, desde cores chamativas no veículo e paraquedas, até sinais sonoros.
Essas competições geralmente recomendam, e até exigem, a presença de um \ac{GNSS}, capaz de transmitir as coordenadas do veículo após o pouso para localização, como um \ac{GPS} \cite{irec}.

O processo de localização baseada em dados simples de \ac{GNSS}, latitude e longitude, pode se tornar mais complicado se o grupo de busca não tem certeza de como encontrar essas coordenadas.
Existem dispositivos de \ac{GNSS} portáteis, porém estes podem criar dificuldades na interface com os dados recebidos da telemetria do foguete.
Neste caso, seria possível desenvolver um dispositivo capaz de lidar diretamente com as informações de localização fornecidas pela telemetria e guiar o grupo de buscar na direção correta.

Os dados recebidos da telemetria ainda precisam de certo grau de confiabilidade para que sejam devidamente processados e tratados, o que pode ser um problema se o veículo está longe do grupo de busca ou o dispositivo de \ac{GNSS} a bordo não esteja apto a fornecer dados corretamente.
Nesse caso, ainda é possível buscar o foguete utilizando o próprio sinal da telemetria, independente dos dados transmitidos.







\section{Objetivos}

O principal objetivo deste trabalho é desenvolver e projetar um dispositivo portátil capaz de indicar a direção da origem de um sinal de \ac{RF} baseado em métodos de detecção de \ac{AoA}.

Como objetivo secundário, a análise comparativa com um sistema de utilidade semelhante, porém baseado inteiramente em coordenadas de \ac{GNSS}.

\section{Estrutura do documento}

\MexerDepois{Completar isso depois}

    \chapter{Revisão Bibliográfica}\label{cap:revbib}

Este capítulo apresenta um breve levantamento de trabalhos relacionados, que mostram a relevância do assunto abordado, bem como a fundamentação teórica utilizada ao longo do trabalho.
% Este capítulo apresenta a fundamentação teórica utilizada ao longo do trabalho, bem como um breve levantamento de trabalhos relacionados, que mostram a relevância do assunto abordado.

% \section{Fundamentação teórica}

A construção deste trabalho fundamentou-se em princípios teóricos, utilizando as bases de direcionamento por coordenadas geográficas, apresentada na \autoref{ssec:gnss}, e princípios de eletromagnetismo para estimar o \ac{AoA}, apresentados na \autoref{ssec:aoa}.

\subsection{Direcionamento por coordenadas geográficas}\label{ssec:gnss}

Coordenadas geográficas são definidas por dois valores, latitude e longitude, associadas a coordenadas esféricas referenciadas a partir do centro da terra, assumindo o raio da coordenada como o raio médio da superfície do planeta, cerca de $R_\text{Terra} = \SI{6371E3}{\metre}$ \cite{palomaguitarrara, chrisveness}.
A latitude equivale à componente polar $\phi$ centralizada na linha do equador, enquanto a longitude equivale à componente $\theta$ centralizada no meridiano de Greenwich \cite{palomaguitarrara, henriquefleming2003}.

Conhecendo as coordenadas de dois pontos distintos A e B, é possível determinar seu ângulo de \textit{bearing} \ac{betab} relativo, referente ao norte, ou seja, o ângulo da direção a se seguir partindo do ponto A para chegar ao ponto B, a partir da direção norte no ponto de origem A \cite{henriquefleming2003}.

Sendo \textcolor{Green}{\ac{Ag}} e \textcolor{Blue}{\ac{Bg}} duas coordenadas geográficas, \textcolor{Green}{\ac{phiA}} e \textcolor{Blue}{\ac{phiB}} suas respectivas latitudes, e \textcolor{Green}{\ac{thetaA}} e \textcolor{Blue}{\ac{thetaB}} suas respectivas longitudes, conforme ilustrado na \autoref{fig:globo}.

\begin{figure}[htbp]
    \centering
    \caption{Representação geométrica de distância e ângulo em relação ao norte entre coordenadas geográficas \textcolor{Green}{\ac{Ag}} e \textcolor{Blue}{\ac{Bg}}.}
    \input{../pictures/globo}
    % \includegraphics{../pictures/globo.pdf}
    \caption*{Fonte: Autor.}
    \label{fig:globo}
\end{figure}

Calculam-se \ac{Deltaphi} e \ac{Deltatheta} conforme Equações \ref{eq:Deltaphi} e \ref{eq:Deltatheta}, respectivamente.

    \begin{equation}\label{eq:Deltaphi}
        \ac{Deltaphi} = \textcolor{Blue}{\ac{phiB}} - \textcolor{Green}{\ac{phiA}}
    \end{equation}
    \begin{equation}\label{eq:Deltatheta}
        \ac{Deltatheta} = \textcolor{Blue}{\ac{thetaB}} - \textcolor{Green}{\ac{thetaA}}
    \end{equation}

Através da lei dos haversines é possível obter a distância mínima $d$ entre as coordenadas, sobre a superfície, e também o ângulo de \textit{Bearing} \textcolor{Red}{\ac{betab}} formado no vértice \textcolor{Green}{\ac{Ag}} do triângulo esférico $\mathbf{N}\textcolor{Green}{\ac{Ag}}\textcolor{Blue}{\ac{Bg}}$ \cite{chrisveness}.
Para o cálculo de distância, os ângulos devem ser tratados em radianos.

\begin{equation}
    X = \cos\left(\textcolor{Blue}{\ac{thetaB}}\right)\cdot \sin\left(\ac{Deltaphi}\right)
\end{equation}
\begin{equation}
    Y = \cos\left(\textcolor{Green}{\ac{thetaA}}\right)\cdot\sin\left(\textcolor{Blue}{\ac{thetaB}}\right) - \sin\left(\textcolor{Green}{\ac{thetaA}}\right) \cdot \cos\left(\textcolor{Green}{\ac{thetaB}}\right) \cdot \cos\left(\ac{Deltaphi}\right)
\end{equation}
\begin{equation}
    Z = \sin^2\left(\frac{\ac{Deltatheta}}{2}\right) + \cos\left(\textcolor{Blue}{\ac{thetaB}}\right) \cdot \cos\left(\textcolor{Green}{\ac{thetaA}}\right) \cdot \sin^2\left(\frac{\ac{Deltaphi}}{2}\right)
\end{equation}
\begin{equation}
    \textcolor{Red}{\ac{betab}} = \arctan\left(\frac{X}{Y}\right) - \frac{\pi}{2}
\end{equation}
\begin{equation}
    \textcolor{Red}{\ac{dAB}} = R_\text{Terra} \cdot 2 \cdot \arctan\left(\frac{\sqrt{Z}}{\sqrt{1-Z}}\right)
\end{equation}

O ângulo \textcolor{Red}{\ac{betab}} calculado aqui é referente à direção cardeal Norte, assim, uma equipe de busca equipada com uma bússola simples seria capaz de seguir a direção correta.
A \autoref{fig:bearing} apresenta a aplicação desenvolvida por \citeauthor{chrisveness}, capaz de calcular o ângulo de \textit{Bearing} entre duas coordenadas, note que, neste caso, o ângulo referido é relacionado à direção cardinal Leste \cite{chrisveness}.

\begin{figure}[htbp]
    \centering
    \caption{Cálculo do ângulo de \textit{Bearing} \textcolor{Red}{\ac{betab}} entre as coordenadas dos Campi Santo André e São Bernardo do Campo da UFABC.}
    \includegraphics[width=0.7\textwidth]{../pictures/bearing.png}
    \caption*{Fonte: \citeauthor{chrisveness} 2019 \cite{chrisveness}}
    \label{fig:bearing}
\end{figure}

\input{../chapter/2_2_2_AoA}

\section{Fundamentação teórica}

A construção deste trabalho fundamentou-se em princípios teóricos, utilizando as bases de direcionamento por coordenadas geográficas, apresentada na \autoref{ssec:gnss}, e princípios de eletromagnetismo para estimar o \ac{AoA}, apresentados na \autoref{ssec:aoa}.

\subsection{Direcionamento por coordenadas geográficas}\label{ssec:gnss}

Coordenadas geográficas são definidas por dois valores, latitude e longitude, associadas a coordenadas esféricas referenciadas a partir do centro da terra, assumindo o raio da coordenada como o raio médio da superfície do planeta, cerca de $R_\text{Terra} = \SI{6371E3}{\metre}$ \cite{palomaguitarrara, chrisveness}.
A latitude equivale à componente polar $\phi$ centralizada na linha do equador, enquanto a longitude equivale à componente $\theta$ centralizada no meridiano de Greenwich \cite{palomaguitarrara, henriquefleming2003}.

Conhecendo as coordenadas de dois pontos distintos A e B, é possível determinar seu ângulo de \textit{bearing} \ac{betab} relativo, referente ao norte, ou seja, o ângulo da direção a se seguir partindo do ponto A para chegar ao ponto B, a partir da direção norte no ponto de origem A \cite{henriquefleming2003}.

Sendo \textcolor{Green}{\ac{Ag}} e \textcolor{Blue}{\ac{Bg}} duas coordenadas geográficas, \textcolor{Green}{\ac{phiA}} e \textcolor{Blue}{\ac{phiB}} suas respectivas latitudes, e \textcolor{Green}{\ac{thetaA}} e \textcolor{Blue}{\ac{thetaB}} suas respectivas longitudes, conforme ilustrado na \autoref{fig:globo}.

\begin{figure}[htbp]
    \centering
    \caption{Representação geométrica de distância e ângulo em relação ao norte entre coordenadas geográficas \textcolor{Green}{\ac{Ag}} e \textcolor{Blue}{\ac{Bg}}.}
    \input{../pictures/globo}
    % \includegraphics{../pictures/globo.pdf}
    \caption*{Fonte: Autor.}
    \label{fig:globo}
\end{figure}

Calculam-se \ac{Deltaphi} e \ac{Deltatheta} conforme Equações \ref{eq:Deltaphi} e \ref{eq:Deltatheta}, respectivamente.

    \begin{equation}\label{eq:Deltaphi}
        \ac{Deltaphi} = \textcolor{Blue}{\ac{phiB}} - \textcolor{Green}{\ac{phiA}}
    \end{equation}
    \begin{equation}\label{eq:Deltatheta}
        \ac{Deltatheta} = \textcolor{Blue}{\ac{thetaB}} - \textcolor{Green}{\ac{thetaA}}
    \end{equation}

Através da lei dos haversines é possível obter a distância mínima $d$ entre as coordenadas, sobre a superfície, e também o ângulo de \textit{Bearing} \textcolor{Red}{\ac{betab}} formado no vértice \textcolor{Green}{\ac{Ag}} do triângulo esférico $\mathbf{N}\textcolor{Green}{\ac{Ag}}\textcolor{Blue}{\ac{Bg}}$ \cite{chrisveness}.
Para o cálculo de distância, os ângulos devem ser tratados em radianos.

\begin{equation}
    X = \cos\left(\textcolor{Blue}{\ac{thetaB}}\right)\cdot \sin\left(\ac{Deltaphi}\right)
\end{equation}
\begin{equation}
    Y = \cos\left(\textcolor{Green}{\ac{thetaA}}\right)\cdot\sin\left(\textcolor{Blue}{\ac{thetaB}}\right) - \sin\left(\textcolor{Green}{\ac{thetaA}}\right) \cdot \cos\left(\textcolor{Green}{\ac{thetaB}}\right) \cdot \cos\left(\ac{Deltaphi}\right)
\end{equation}
\begin{equation}
    Z = \sin^2\left(\frac{\ac{Deltatheta}}{2}\right) + \cos\left(\textcolor{Blue}{\ac{thetaB}}\right) \cdot \cos\left(\textcolor{Green}{\ac{thetaA}}\right) \cdot \sin^2\left(\frac{\ac{Deltaphi}}{2}\right)
\end{equation}
\begin{equation}
    \textcolor{Red}{\ac{betab}} = \arctan\left(\frac{X}{Y}\right) - \frac{\pi}{2}
\end{equation}
\begin{equation}
    \textcolor{Red}{\ac{dAB}} = R_\text{Terra} \cdot 2 \cdot \arctan\left(\frac{\sqrt{Z}}{\sqrt{1-Z}}\right)
\end{equation}

O ângulo \textcolor{Red}{\ac{betab}} calculado aqui é referente à direção cardeal Norte, assim, uma equipe de busca equipada com uma bússola simples seria capaz de seguir a direção correta.
A \autoref{fig:bearing} apresenta a aplicação desenvolvida por \citeauthor{chrisveness}, capaz de calcular o ângulo de \textit{Bearing} entre duas coordenadas, note que, neste caso, o ângulo referido é relacionado à direção cardinal Leste \cite{chrisveness}.

\begin{figure}[htbp]
    \centering
    \caption{Cálculo do ângulo de \textit{Bearing} \textcolor{Red}{\ac{betab}} entre as coordenadas dos Campi Santo André e São Bernardo do Campo da UFABC.}
    \includegraphics[width=0.7\textwidth]{../pictures/bearing.png}
    \caption*{Fonte: \citeauthor{chrisveness} 2019 \cite{chrisveness}}
    \label{fig:bearing}
\end{figure}

\input{../chapter/2_2_2_AoA}

\section{Trabalhos relacionados} \label{sec:trabalhos_relacionados}

Em seu trabalho, \citeauthor{horst2021localization} \cite{horst2021localization} analisa dois algoritmos de detecção de \ac{AoA}, realizando as análises em ambientes internos e utilizando arranjos de antenas.
O primeiro método analisado consiste em uma aproximação do ângulo, feita utilizando um \textit{software} fornecido pela Texas Instruments, fabricante do \textit{hardware} utilizado.
Já o segundo método, baseia-se na construção matemática do \ac{AoA} calculado pela diferença de fase instantânea do sinal entre as antenas do sistema, uma abordagem semelhante à proposta neste trabalho.
Os resultados obtidos indicam que o método de aproximação teve melhor acurácia nos valores de ângulo.

A proposta de \citeauthor{zeaiter:hal-03693641} \cite{zeaiter:hal-03693641} busca validar a performance da detecção de \ac{AoA} em ambiente fechado, realizando a análise em diferentes modulações, larguras de canal e fatores de espalhamento.
Também propõe que, ao combinar de seu algoritmo de localização de \ac{AoA} com a função de autocorrelação, é possível analisar os dados de dois sinais recebidos simultaneamente.

Outro trabalho de \citeauthor{zeaiter:hal-03932846} \cite{zeaiter:hal-03932846} consiste em uma aproximação do \ac{AoA} utilizando um método de autocorrelação em um sinal \ac{LoRa} de baixa potência.
Seu objetivo consiste em detectar o sinal \ac{LoRa} operando em transmissão de baixa potência, caso onde a vida útil da bateria do sistema transmissor é estendida.
O algoritmo apresentado busca picos de autocorrelação no sinal recebido, além de utilizar \ac{FFT} para denotá-los e melhorar a \ac{SNR}.
Quando um pico é detectado, o algoritmo é capaz de encontrar o \ac{AoA}.

% O trabalho de \citeauthor{aernouts2020combining} \cite{aernouts2020combining} combina o método de filtro de partículas às medidas TDoA e \ac{AoA} obtidas em ambiente urbano denso.
% A performance é analisada de maneira comparativa à estimativa de TDoA e a um trabalho anterior baseado em combinação de matrizes.
% Seus resultados indicam um erro médio estimado de \SI{199}{\metre} sem o \ac{AoA}.

\citeauthor{bnilam20172d} \cite{bnilam20172d} propõe uma técnica que, sem qualquer informação prévia de largura de banda, consegue estimar \ac{AoA} do sinal recebido.
O sistema proposto consiste em uma \ac{UCA} seguida de um filtro transversal, também utiliza de vetores especiais de largura de banda variável junto com um estimador de relação sinal-ruído térmico para determinar simultaneamente \ac{AoA} e largura de banda do sinal recebido.

Em outro trabalho, \citeauthor{bnilam2017adaptive} \cite{bnilam2017adaptive} estudam a possibilidade de estimar \ac{AoA} para transceptores de \ac{IoT} em ambiente interno.
Também propõe um modelo probabilístico adaptativo que opera no modelo de estimativa de \ac{AoA}, incrementando sua performance.
Seus resultados indicam que estes métodos superam a performance de modelos probabilísticos estáticos tradicionais, tanto em acurácia de localização quanto em estabilidade no valor obtido.

Neste trabalho, \citeauthor{bnilam2019low} \cite{bnilam2019low} propõe um dispositivo de baixo custo capaz de estimar o \ac{AoA}, de forma que seja viável sua utilização em dispositivos de \ac{IoT}.
O dispositivo consiste em uma conversão de vários \ac{SDR} individuais de baixo custo num único \ac{SDR} com múltiplos canais de \ac{RF}.
Seus resultados experimentais indicam que o dispositivo é capaz de estimar valores de \ac{AoA} de forma estável e acurada.

A proposta de \citeauthor{bnilam2020angle} \cite{bnilam2020angle} neste trabalho consiste em um novo algoritmo para determinação de \ac{AoA} chamado ANGLE (\textit{ANGular Location Estimation}), baseado em modelos probabilísticos para a resposta do sinal recebido.
Sua proposta ainda sugere duas versões do método, para o caso de amostragem única e de decomposição de subespaço, como utilizado no algoritmo MUSIC (\textit{MUltiple SIgnal Classification}).

\citeauthor{bnilam2020lora} \cite{bnilam2020lora} apresenta, neste trabalho, uma abordagem mais amigável para estimativa de \ac{AoA} em redes \ac{LoRa}.
O sistema proposto, denominado LoRay (\ac{LoRa} \textit{array}) é composto por \textit{hardware} e \textit{software} preparados para fazer a estimativa de \ac{AoA} em ambiente urbano, onde o sistema foi validado.
O hardware utilizado foi descrito em um trabalho anterior \cite{bnilam2019low}.
Este sistema apresentou resultados estáveis e acurados para estimativa de \ac{AoA} tanto nos casos \ac{LoS} quanto nos \ac{NLoS}.

% \citeauthor{steckel2018low} \cite{steckel2018low}

% \citeauthor{du2018long} \cite{du2018long}

Em seu trabalho, \citeauthor{niculescu2003ad} \cite{niculescu2003ad} propõe métodos para detecção de posição e orientação em cada nó de uma rede \textit{ad hoc}.
A proposta parte de possíveis problemas relacionados à utilização de \ac{GPS} em ambiente fechado

O trabalho de \citeauthor{Horst2025BTLEAoA} \cite{Horst2025BTLEAoA} é o mais recente dentre os levantados, e propõe estimar o \ac{AoA} utilizando do algoritmo de \acf{GN}.
O sistema proposto é baseado na tecnologia de \ac{BLE}, presente em alguns dispositivos \ac{IoT}, para a localização em ambiente fechado.
Essa proposta será adaptada para comparação com os resultados do presente trabalho.

    \chapter{Metodologia}\label{cap:metodologia}

Neste capítulo, são explorados os métodos utilizados para a construção do trabalho proposto.


\section{Simulação}

A construção da simulação partiu de uma abordagem físico-matemática, definindo o sinal \ac{w} como uma função de onda relativa ao tempo e ao espaço, analisando seus valores incidindo em cada antena \ac{Ak} e comparando as defasagens \ac{DeltaPhi} entre os diferentes pares de antenas.
Para simplificar a construção da simulação, foram utilizadas funções paramétricas, descritas na presente seção.

\subsection{Parâmetros envolvidos}

Com o objetivo de garantir a coerência entre as partes da simulação, vários parâmetros foram utilizados, definindo detalhes em relação às operações matemáticas e às formas de registro dos valores calculados.
Estes parâmetros são divididos entre os que recebem valores numéricos, booleanos ou matrizes numéricas.

Os parâmetros numéricos são:
\begin{itemize}
	\item \lstinline|amp_w|, amplitude desejada para o sinal;
	\item \lstinline|ang_w|, direção do emissor do sinal, equivalente ao ângulo \ac{thetaAoA} de chegada do sinal em relação à malha de antenas;
	\item \lstinline|angle_Z_A_x_B|, ângulo relativo \ac{betak} para par de antenas;
	\item \lstinline|d|, distância \ac{d} entre par de antenas da malha;
	\item \lstinline|choose_angle|, ângulo \ac{thetaAoA} final calculado pelo sistema;
	\item \lstinline|interval|, indica os limites para a geração de imagem da simulação;
	\item \lstinline|lambda_w|, comprimento de onda \ac{lambda};
	\item \lstinline|N_antenas|, quantidade \ac{Nant} de antenas da malha;
	\item \lstinline|omega_w|, frequência angular \ac{omega};
	\item \lstinline|phase_w|, fase $\phi$ do sinal no emissor;
	\item \lstinline|Rho|, raio \ac{rho} do polígono que dispõe as antenas na malha;
	\item \lstinline|r_w|, distância que o emissor de sinal está da coordenada $(0,~0)$ do sistema;
	\item \lstinline|range_step|, largura em graus do passo na simulação.
	\item \lstinline|resolution|, relativo à quantidade de pontos utilizados na aproximação numérica do cálculo de correlação;
	\item \lstinline|SNR|, valor da \ac{SNR} linear;
	\item \lstinline|SNR_dB|, valor da \ac{SNR} em \si{\decibel};
	\item \lstinline|t_w|, tempo $t$ associado ao instante de aferição do sinal;
	\item \lstinline|x_w| ou \lstinline|y_w|, coordenada $x$ ou $y$ no espaço para aferição do sinal \ac{w};
	\item \lstinline|Z_antenna|, \lstinline|Z_antenna_A| ou \lstinline|Z_antenna_B|, valor complexo, coordenada de antena;
	\item \lstinline|Z_phase_A| ou \lstinline|Z_phase_B|, valor complexo, fase \ac{Phik} de antena;
\end{itemize}

Os parâmetros booleanos são:
\begin{itemize}
	\item \lstinline|ATT|, indica se o sinal contará com atenuação por distância;
	\item \lstinline|C|, indica a utilização de componente cossenoidal na construção do sinal;
	\item \lstinline|CHG_PHI|, indica se a fase geral do sinal deve mudar ao longo da simulação;
	\item \lstinline|CHG_R|, indica se a distância do emissor do sinal deverá mudar ao longo da simulação;
	\item \lstinline|CHG_THETA|, indica se o ângulo de origem do sinal deverá mudar ao longo da simulação;
	\item \lstinline|NOISE|, indica se o sinal contará com ruído;
	\item \lstinline|S|, indica a utilização de componente senoidal na construção do sinal;
	\item \lstinline|S_DAT|, indica se os pontos gerados pela simulação deverão ser salvos;
	\item \lstinline|S_GIF|, indica se a imagem gerada pela simulação deverá ser salva;
\end{itemize}

Os parâmetros de matrizes numéricas são:
\begin{itemize}
	\item \lstinline|ant_array|, coordenadas das antenas da malha;
	\item \lstinline|delta_A_x_B_array|, contendo o ângulo \ac{thetak} calculado por $\ac{alphak} + \ac{betak}$ aferido para cada par de antenas da malha;
	\item \lstinline|delta_B_x_A_array|, contendo o ângulo \ac{thetak} calculado por $\ac{alphak} - \ac{betak}$ aferido para cada par de antenas da malha;
	\item \lstinline|Z_phase_array|, matriz de valores numéricos complexos, contendo o sinal complexo aferido para cada antena da malha;
	\item \lstinline|z_plot|, estado corrente do sinal no espaço, utilizado na geração de imagem da simulação;
	\item \lstinline|Z_x_array|, valores complexos, contendo a defasagem \ac{DeltaPhi} aferido para cada par de antenas na malha;
\end{itemize}


\subsection{Funções auxiliares}

% argument_r
A primeira função a ser definida é \lstinline|argument_r|, que opera como auxiliar para normalização de argumento para as funções trigonométricas utilizadas nas análises, garantindo coerência em frequência angular e coordenadas espaciais.
Seus argumentos são, respectivamente, \lstinline|x_w|, \lstinline|y_w|, \lstinline|t_w|, \lstinline|ang_w|, \lstinline|r_w|, \lstinline|phase_w|, \lstinline|lambda_w| e \lstinline|omega_w|.
O \autoref{cod:argument_r} apresenta uma versão simplificada da função \lstinline|argument_r| desenvolvida.

\begin{lstfloat}[htbp]
	\centering
	\lstinputlisting[
			basicstyle=\ttfamily\small\setstretch{1},
			label=cod:argument_r,
			caption={Função \lstinline|argument_r|, simplificada.}
		]{../code/argument_r_alt.m}
	\caption*{Fonte: Autor.}
\end{lstfloat}

% Carregar bibliotecas

% ref_sin e ref_cos
Para determinar a fase do sinal \ac{w}, incidente em cada antena \ac{Ak}, calcula-se a correlação deste sinal com sinais de referência seno e cossenos, fornecidos respectivamente pelas funções \lstinline|ref_sin| e \lstinline|ref_cos|.
As duas funções recebem os mesmos argumentos, e estes são, respectivamente, \lstinline|t_w| e \lstinline|omega_w|.
Ambos os casos utilizam a função \lstinline|argument_r| para garantir coerência de frequência com o sinal incidente.
Os Códigos \ref{cod:ref_cos} e \ref{cod:ref_sin} apresentam, respectivamente, versões simplificadas das funções \lstinline|ref_cos| e \lstinline|ref_sin| desenvolvidas.


\begin{lstfloat}[htbp]
	\centering
	\lstinputlisting[
			basicstyle=\ttfamily\small\setstretch{1},
			label=cod:ref_cos,
			caption={Função \lstinline|ref_cos|, simplificada.}
		]{../code/ref_cos_alt.m}
	\caption*{Fonte: Autor.}
\end{lstfloat}

\begin{lstfloat}[htbp]
	\centering
	\lstinputlisting[
			basicstyle=\ttfamily\small\setstretch{1},
			label=cod:ref_sin,
			caption={Função \lstinline|ref_sin|, simplificada.}
		]{../code/ref_sin_alt.m}
	\caption*{Fonte: Autor.}
\end{lstfloat}


% signal_r
A próxima função construída foi \lstinline|signal_r|, que calcula o valor do sinal \ac{w} numa coordenada $(x,~y)$ e um instante $t$.
Considera-se que o sinal é composto pela soma de seno e cosseno, e que são determinadas a distância e a direção de sua fonte emissora.
Também é possível definir amplitude e fase na origem, além da presença de atenuação e ruído do tipo \ac{AWGN}.
Seus argumentos são, respectivamente, \lstinline|x_w|, \lstinline|y_w|, \lstinline|t_w|, \lstinline|amp_w|, \lstinline|ang_w|, \lstinline|r_w|, \lstinline|phase_w|, \lstinline|lambda_w|, \lstinline|omega_w|, \lstinline|S|, \lstinline|C|, \lstinline|NOISE|, \lstinline|SNR_dB| e \lstinline|ATT|.
É utilizada a função \lstinline|argument_r| para garantir coerência de frequência entre as componentes e com os sinais de referência utilizados no cálculo de correlação.
Para implementação do ruído, foi utilizada a função \lstinline|awgn|, no GNU Octave, é necessária a biblioteca \textit{communications}, porém para o MATLAB, não é necessário carregar bibliotecas \cite{awgnOctave, awgnMATLAB}.
O \autoref{cod:signal_r} apresenta uma versão simplificada da função \lstinline|signal_r| desenvolvida.

\begin{lstfloat}[htbp]
	\centering
	\lstinputlisting[
			basicstyle=\ttfamily\small\setstretch{1},
			label=cod:signal_r,
			caption={Função \lstinline|signal_r|, simplificada.}
		]{../code/signal_r_alt.m}
	\caption*{Fonte: Autor.}
\end{lstfloat}

% phase_z
A função \lstinline|phase_z| calcula o valor complexo de fase \ac{Zk} para a antena \ac{Ak} através da correlação pelos sinais de seno e cosseno.
Seus argumentos são, respectivamente, \lstinline|t|, \lstinline|Z_antenna|, \lstinline|amp_w|, \lstinline|ang_w|, \lstinline|r_w|, \lstinline|phase_w|, \lstinline|lambda_w|, \lstinline|omega_w|, \lstinline|S|, \lstinline|C|, \lstinline|NOISE|, \lstinline|SNR_dB| e \lstinline|ATT|.
O \autoref{cod:phase_z} apresenta uma versão simplificada da função \lstinline|phase_z| desenvolvida.

\begin{lstfloat}[htbp]
	\centering
	\lstinputlisting[
			basicstyle=\ttfamily\small\setstretch{1},
			label=cod:phase_z,
			caption={Função \lstinline|phase_z|, simplificada.}
		]{../code/phase_z_alt.m}
	\caption*{Fonte: Autor.}
\end{lstfloat}

% dephase_A_to_B
O cálculo do valor complexo de defasagem \ac{DeltaPhi}, o ângulo relativo \ac{betak} e o ângulo \ac{alphak} entre um par de antenas é realizado pela função \lstinline|dephase_A_to_B|.
Seus argumentos são, respectivamente, \lstinline|Z_phase_A| e \lstinline|Z_phase_B|.
O \autoref{cod:dephase_A_to_B} apresenta uma versão simplificada da função \lstinline|dephase_A_to_B| desenvolvida.

\begin{lstfloat}[htbp]
	\centering
	\lstinputlisting[
			basicstyle=\ttfamily\small\setstretch{1},
			label=cod:dephase_A_to_B,
			caption={Função \lstinline|dephase_A_to_B|, simplificada.}
		]{../code/dephase_A_to_B_alt.m}
	\caption*{Fonte: Autor.}
\end{lstfloat}

% deltas_A_B
Os ângulos \ac{thetak} para um par de antenas são calculados pela função \lstinline|deltas_A_B|.
Seus argumentos são, respectivamente, \lstinline|angle_Z_A_x_B|, \lstinline|Z_antenna_A| e \lstinline|Z_antenna_B|.
O \autoref{cod:deltas_A_B} apresenta uma versão simplificada da função \lstinline|deltas_A_B| desenvolvida.

\begin{lstfloat}[htbp]
	\centering
	\lstinputlisting[
			basicstyle=\ttfamily\small\setstretch{1},
			label=cod:deltas_A_B,
			caption={Função \lstinline|deltas_A_B|, simplificada.}
		]{../code/deltas_A_B_alt.m}
	\caption*{Fonte: Autor.}
\end{lstfloat}


% isoctave
A última função auxiliar desenvolvida foi \lstinline|isoctave|, que confere se a corrente simulação está sendo executada no GNU Octave, retornando um valor binário e não recebe qualquer parâmetro.
O \autoref{cod:isoctave} apresenta uma versão simplificada da função \lstinline|isoctave| desenvolvida.

\begin{lstfloat}[htbp]
	\centering
	\lstinputlisting[
			basicstyle=\ttfamily\small\setstretch{1},
			label=cod:isoctave,
			caption={Função \lstinline|isoctave|, simplificada.}
		]{../code/isoctave_alt.m}
	\caption*{Fonte: Autor.}
\end{lstfloat}



\subsection{Função de cálculo para \acs{AoA}}

% calc_AoA_polygon
A primeira grande função desenvolvida foi \lstinline|calc_AoA|, que é responsável pelo cálculo geral da simulação.
Inicialmente são calculadas as coordenadas das \ac{Nant} antenas e, em sequência, os valores de fase do sinal incidente \ac{w} em cada antena \ac{Ak}, então as defasagens entre os pares de antenas e finalmente a seleção do valor mais provável para \ac{thetaAoA}.
Seus argumentos são, respectivamente, \lstinline|amp_w|, \lstinline|ang_w|, \lstinline|r_w|, \lstinline|phase_w|, \lstinline|lambda_w|, \lstinline|omega_w|, \lstinline|S|, \lstinline|C|, \lstinline|NOISE|, \lstinline|SNR_dB|, \lstinline|ATT|, \lstinline|resolution|, \lstinline|d| e \lstinline|N_antenas|.
Nessa função também são definidas três subfunções auxiliares \lstinline|phase_z|, \lstinline|dephase_A_to_B| e \lstinline|deltas_A_B|.
O \autoref{cod:calc_AoA} apresenta uma versão simplificada da função \lstinline|calc_AoA| desenvolvida.
A \autoref{fig:AoA:fluxograma} apresenta a sequências de operações realizadas pela função.

\begin{lstfloat}[htbp]
	\centering
	\lstinputlisting[
			basicstyle=\ttfamily\small\setstretch{1},
			label=cod:calc_AoA,
			caption={Função \lstinline|calc_AoA|, simplificada.}
		]{../code/calc_AoA_alt.m}
	\caption*{Fonte: Autor.}
\end{lstfloat}


\begin{figure}[htbp]
    \centering
    \caption{Fluxograma de operações da função \lstinline|calc_AoA|.}
    \begin{tikzpicture}[node distance=1.75cm]
    \node (s) [startstop] {Início};
    \node (a1) [below of=s, io] {Aferir fase nas antenas};
    \node (a2) [below of=a1, process] {Calcular defasagem};
    \node (a3) [below of=a2, process] {Ajustar ângulos por pares};
    \node (a4) [below of=a3, process] {Quantizar valores};
    \node (a5) [right of=s, process, node distance=6cm] {Calcular moda};
    \node (a6) [below of=a5, process] {Filtrar valores plausíveis};
    \node (a7) [below of=a6, process] {Calcular mediana};
    \node (a8) [below of=a7, process] {Retornar resultado};
    \node (f)  [below of=a8, startstop] {Fim};

    % \node [anchor=south west, font = {\scriptsize\bfseries}, Red] at (a4.east) {N};
    % \node [anchor=south east, font = {\scriptsize\bfseries}, Green] at (a4.west) {S};

	\node (x1) [below of = s, ghost] {};

    \node [fit=(x1)] (fita1) {}; \draw [niceBrace] ([yshift=2.5pt]fita1.south west) -- ([yshift=-2.5pt]fita1.north west);
    \node [fit=(a2)] (fita2) {}; \draw [niceBrace] ([yshift=2.5pt]fita2.south west) -- ([yshift=-2.5pt]fita2.north west);
    \node [fit=(a3)] (fita3) {}; \draw [niceBrace] ([yshift=2.5pt]fita3.south west) -- ([yshift=-2.5pt]fita3.north west);
    \node [fit=(a4)] (fita4) {}; \draw [niceBrace] ([yshift=2.5pt]fita4.south west) -- ([yshift=-2.5pt]fita4.north west);

    \node [fit=(a5)] (fita5) {}; \draw [niceBrace] ([yshift=-2.5pt]fita5.north east) -- ([yshift=2.5pt]fita5.south east);
    \node [fit=(a6)] (fita6) {}; \draw [niceBrace] ([yshift=-2.5pt]fita6.north east) -- ([yshift=2.5pt]fita6.south east);
    \node [fit=(a7)] (fita7) {}; \draw [niceBrace] ([yshift=-2.5pt]fita7.north east) -- ([yshift=2.5pt]fita7.south east);
    \node [fit=(a8)] (fita8) {}; \draw [niceBrace] ([yshift=-2.5pt]fita8.north east) -- ([yshift=2.5pt]fita8.south east);

	\node [left of=x1, auxBlock, anchor=east] {\ac{Zk}};
	\node [left of=a2, auxBlock, anchor=east] {\ac{DeltaPhi}};
	\node [left of=a3, auxBlock, anchor=east] {\ac{Theta}};
	\node [left of=a4, auxBlock, anchor=east] {\ac{ThetaQuanti}};

    \node [right of=a5, auxBlock] {\ac{thetaMo}};
    \node [right of=a6, auxBlock] {\ac{ThetaFiltro}};
    \node [right of=a7, auxBlock] {$\widetilde{\ac{ThetaFiltro}}$};
    \node [right of=a8, auxBlock] {\ac{thetaAoA}};

    \draw [arrow] (s) -- (a1);
    \draw [arrow] (a1) -- (a2);
    \draw [arrow] (a2) -- (a3);
    \draw [arrow] (a3) -- (a4);
    % \draw [arrow] (a4) -- (a5);
    \draw [arrow] (a4) -| ($(a4)!0.5!(a5)$) |- (a5);
    % \draw [arrow] (a4) -- ([xshift=-.5cm]a4.west) |- (a2);
    \draw [arrow] (a5) -- (a6);
    \draw [arrow] (a6) -- (a7);
    \draw [arrow] (a7) -- (a8);
    \draw [arrow] (a8) -- (f);
\end{tikzpicture}
    \label{fig:AoA:fluxograma}
    \caption*{Fonte: Autor.}
\end{figure}

\subsection{Função de geração saída visual}

% generate_fig_polygon

A segunda grande função desenvolvida foi \lstinline|generate_fig|, que constrói a animação de saída da simulação, formada por dois gráficos.
O primeiro gráfico, à esquerda nas animações geradas, apresenta a disposição das antenas, os valores de fase para cada uma delas, os valores de defasagem entre os pares de antenas, todos os possíveis valores de \ac{thetak}, e finalmente o valor real e o escolhido para \ac{thetaAoA}.
O segundo gráfico, à direita nas animações geradas, apresenta a disposição das antenas e uma representação do sinal \ac{w} no espaço exibido.
Os valores exibidos são calculados pela função \lstinline|calc_AoA|.
Seus argumentos são, respectivamente, \lstinline|z_plot|, \lstinline|x_w|, \lstinline|y_w|, \lstinline|ang_w|, \lstinline|lambda_w|, \lstinline|interval|, \lstinline|Rho|, \lstinline|choose_angle|, \lstinline|ant_array|, \lstinline|Z_phase_array|, \lstinline|Z_x_array|, \lstinline|delta_A_x_B_array| e \lstinline|delta_B_x_A_array|.
A \autoref{fig:example:simul_POLY_3_R_50} ilustra os gráficos gerados pela função \lstinline|generate_fig|.

\begin{figure}[htbp]
	\centering
	\caption{Exemplo de quadro da animação de saída da função \lstinline|generate_fig|.}
	\includegraphics[width=0.9\textwidth]{../pictures/simul_POLY_3_R_50.png}
	\label{fig:example:simul_POLY_3_R_50}
	\caption*{Fonte: Autor, saída gráfica disponível em \href{https://github.com/HeckRodSav/TG/blob/main/documentation/pictures/POLY_3/simul_POLY_3_R_50.gif}{\underline{GitHub}}.}
\end{figure}

\subsection{Função geral da simulção}

% Definir função base xyt e variáveis

Finalmente a função responsável por juntar todas as partes é \lstinline|w_xyt|, a base para a simulação, ela invoca as funções \lstinline|calc_AoA| e \lstinline|generate_fig| com os devidos parâmetros, além de garantir que os arquivos gerados sejam salvos corretamente.
Seus argumentos são, respectivamente, \lstinline|NOISE|, \lstinline|ATT|, \lstinline|CHG_PHI|, \lstinline|CHG_R|, \lstinline|CHG_THETA|, \lstinline|S_GIF|, \lstinline|S_DAT|, \lstinline|SNR|, \lstinline|range_step| e \lstinline|N_antenas|.
Essa função também invoca a simulação do algoritmo de Gauss-Newton para as mesmas condições.

\section{Simulação do Algoritmo de Gauss-Newton}

Para análise comparatória, também foi desenvolvida uma simulação para o método de \ac{AoA} utilizando o algoritmo de Gauss-Newton, adaptando a proposta de \citeauthor{Horst2025BTLEAoA} \cite{Horst2025BTLEAoA}, referida na \autoref{sec:trabalhos_relacionados}.
Nesta adaptação, cada receptor é formado por um par de antenas, separados pela mesma distância \ac{d} utilizada na presente proposta, e o sistema como um todo utiliza \ac{Nant} receptores dispostos de maneira linear.
Considerando que a presente análise é planar, a componente de terceira dimensão foi desconsiderada.
O método realiza 5 iterações para convergir os resultados em cada ponto de análise.

O código desenvolvido está nos anexos, na \autoref{apdx:codigo:gn}.


    \chapter{Resultados}



\begin{figure}[H]
    \centering
    \input{../pictures/simul_POLY_3_R_50.tex}
    \caption{Gráfico}
\end{figure}

\begin{figure}[H]
    \centering
    \input{../pictures/simul_POLY_3_R_50_SNR_1_ATT.tex}
    \caption{Gráfico}
\end{figure}

\begin{figure}[H]
    \centering
    \input{../pictures/simul_POLY_3_R_50_SNR_1.tex}
    \caption{Gráfico}
\end{figure}

    \chapter{Conclusão}\label{cap:conclusao}

Foguetes de sondagem atmosférica podem pousar em qualquer lugar dentro do raio de alcance do voo, e recuperá-los pode ser inviável sem uma estratégia de localização eficaz.
Uma estratégia muito utilizada é a localização por \ac{GNSS}, por exemplo, o \ac{GPS}, contudo, esta ainda depende que a equipe de busca tenha acesso às próprias coordenadas geográficas e comunicação efetiva com o sistema embarcado do foguete.

O presente trabalho propõe a utilização de um sistema de localização baseada no sinal \ac{RF} emitido pelo veículo, e não pela informação contida nesse sinal, analisando as diferenças de defasagem do sinal incidente em uma malha de antenas, e assim calculando o \ac{AoA} deste sinal.

Durante a revisão bibliográfica, fundamentou-se a teoria aplicada nessa proposta.
Partindo de uma abordagem físico-matemática para analisar o sinal e a forma que a defasagem entre antenas pode ser utilizada para calcular a direção do emissor.
Também foram listadas algumas propostas que atuam de forma semelhante, analisando o sinal incidente em uma malha de antenas, que demonstra a relevância do método.
Além disso, foi realizado um breve levantamento sobre o método de direcionamento por coordenadas geográficas e o ângulo de \textit{bearing}, que guia uma equipe de busca ao veículo almejado.

Com base na fundamentação físico-matemática, foi desenvolvida uma simulação com o sinal de \ac{RF} incidente em uma malha de antenas.
Considerando que o foguete esteja em solo, assumiu-se um espaço de duas dimensões, porém ainda mantendo a possibilidade do veículo, emissor do sinal, se mover livremente em relação ao sistema de antenas.
As simulações foram construídas a partir dessas possibilidades, com o veículo circulando o sistema de antenas e se aproximando.

As simulações realizadas utilizaram geometrias de três, cinco e sete antenas.
A escolha dessas quantidades deu-se por questões geométricas, pois polígonos regulares com uma contagem par de lados terão lados paralelos, enquanto polígonos de lados ímpares não apresentam essa propriedade.
Os valores de R\textsuperscript{2} para as configurações simuladas foram todos acima de \qty{75}{\percent}, o que indica grande acurácia na modelagem proposta.
A comparação entre as geometrias estudadas indicou que o sistema com três antenas teve uma acurácia média menor que as geometrias com mais antenas.

% Comparando com os resultados válidos do algoritmo de \ac{GN}, os valores obtidos pelo sistema proposto se mostraram mais assertivos, particularmente em casos com ruído e atenuação.
% Possívelmente com mais iterações o algoritmo seria capaz de convegir para valores mais próximos aos corretos, porém os erros numéricos presentes podem ter corroborado com a divergência dos resultados.

As limitações impostas pelo uso de \textit{software} livre fizeram com que fossem utilizados métodos diferentes dos levantados na revisão bibliográfica, porém o método estatístico proposto se mostrou eficaz nos testes realizados.
Outra limitação foi relacionada à compatibilidade do código escrito, já que a sintaxe e algumas funções do MATLAB têm algumas diferenças em relação às equivalentes do GNU Octave.

Em conclusão, o trabalho aqui proposto se mostrou eficaz na determinação do \ac{AoA} para um sinal incidente em uma malha de antenas, garantindo um valor de R\textsuperscript{2} acima de \qty{75}{\percent} em todos os casos e valor médio de R\textsuperscript{2} acima de \qty{92}{\percent}.

Para trabalhos futuros, é possível analisar outras disposições de antenas no arranjo.
Apesar da formulação atual optar por polígonos regulares por simplicidade, a matemática utilizada calcula os ângulos de cada par de antenas individualmente, o que viabiliza outras disposições de antenas, que respeitem a distância entre antenas de um par.
Outras possibilidades englobam analisar mais classes de ruídos e até problemas de propagação multicaminho.
Além disso, a construção de um dispositivo eletrônico capaz de realizar a aferição de fase em uma malha de antenas poderá corroborar no levantamento de outros problemas a serem analisados e também validar a presente proposta.


    % \nocite{*}
    \printbibliography[heading=bibintoc, title={Referências}]

    \clearpage
    \phantomsection % Corrigir posição do link no sumário
    \addcontentsline{toc}{chapter}{Apêndices}
    \appendix

    % \chapter{Códigos desenvolvidos para simulação}

O sistema desenvolvido e os arquivos-fonte deste relatório estão disponíveis \href{https://github.com/HeckRodSav/TG}{\underline{neste repositório no GitHub}}.

\section{Simulação de direcionamento GNSS}

\lstinputlisting{../code/azimuth.m}

\section{Simulação de AoA}

\lstinputlisting{../code/xyt.m}

\lstinputlisting{../code/generate_fig.m}

\lstinputlisting{../code/signal_r.m}

\lstinputlisting{../code/ref_cos.m}

\lstinputlisting{../code/ref_sin.m}

\lstinputlisting{../code/argument_r.m}

\lstinputlisting{../code/isoctave.m}

\end{document}


% . Introdução

%     - Introdução do cenário de estudo e apresentação/motivação do tema do trabalho neste contexto

% . Objetivos principais e secundarios

% . Revisão bibliográfica de trabalhos correlatos

%     - 3 a 4 trabalhos e 1 paragrafo por trabalho
%     - Ao final, mostrar que o tema escolhido tem relevancia de estudo etc

% . Fundamentos do tema

%     - Pode ser quebrado em mais de um capítulo (por exemplo, capítulo sobre lançamento de foguetes, tecnicas de monitiramento e controle,  sistemas embarcados etc)
%     - Descrição das tecnologias e técnicas que serão estudadas e analisadas

% . Análise de resultados

%     - Eu acho muito importante no TG1 já começar as análises e obter/apresentar algum resultado preliminar do estudo.

% . Conclusão

% .  Anexos

% . Referências bibliográficas