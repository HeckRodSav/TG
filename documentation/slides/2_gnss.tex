\section{GNSS}
\subsection{Coordenadas Geográficas}
    \begin{frame}{Coordenadas Geográficas}
        \begin{columns}
            \begin{column}{0.55\textwidth}
                \centering
                \input{../pictures/globo_B}
                % \includegraphics[width=0.95\textwidth]{../pictures/globo.pdf}
            \end{column}
            \begin{column}{0.45\textwidth}
                \begin{itemize}[<+(-3)->]\addtolength{\itemsep}{0.5\baselineskip}
                    \item \textcolor{Green}{$\mathbf{A}_\text{g}$} e \textcolor{Blue}{$\mathbf{B}_\text{g}$} são coordenadas geográficas
                    \item \textcolor{Green}{$\theta_{\mathbf{A}_\text{g}}$} e \textcolor{Blue}{$\theta_{\mathbf{B}_\text{g}}$} são latitudes
                    \item \textcolor{Green}{$\phi_{\mathbf{A}_\text{g}}$} e \textcolor{Blue}{$\phi_{\mathbf{B}_\text{g}}$} são longitudes
                    \item Conhecendo essas informações, é possível determinar o ângulo \textcolor{Red}{$\beta_b$} relativo entre as duas coordenadas
                    \item Também é possível determinar sua distância \textcolor{Red}{$d_\mathbf{AB}$}
                \end{itemize}
            \end{column}
        \end{columns}
    \end{frame}

% \subsection{Cálculo de Bearing}
%     \begin{frame}{Cálculo de Bearing}
%         % \begin{multicols}{2}
%             \begin{equation*}
%                 \Delta_\phi = \textcolor{Blue}{\phi_{\mathbf{B}_\text{g}}} - \textcolor{Green}{\phi_{\mathbf{A}_\text{g}}}
%             \end{equation*}
%             \begin{equation*}
%                 \Delta_\theta = \textcolor{Blue}{\theta_{\mathbf{B}_\text{g}}} - \textcolor{Green}{\theta_{\mathbf{A}_\text{g}}}
%             \end{equation*}
%             \begin{equation*}
%                 X = \cos\left(\textcolor{Blue}{\theta_{\mathbf{B}_\text{g}}}\right)\cdot \sin\left(\Delta_\phi\right)
%             \end{equation*}
%             \begin{equation*}
%                 Y = \cos\left(\textcolor{Green}{\theta_{\mathbf{A}_\text{g}}}\right)\cdot\sin\left(\textcolor{Blue}{\theta_{\mathbf{B}_\text{g}}}\right) - \sin\left(\textcolor{Green}{\theta_{\mathbf{A}_\text{g}}}\right) \cdot \cos\left(\textcolor{Green}{\theta_{\mathbf{B}_\text{g}}}\right) \cdot \cos\left(\Delta_\phi\right)
%             \end{equation*}
%             \begin{equation*}
%                 Z = \sin^2\left(\frac{\Delta_\theta}{2}\right) + \cos\left(\textcolor{Blue}{\theta_{\mathbf{B}_\text{g}}}\right) \cdot \cos\left(\textcolor{Green}{\theta_{\mathbf{A}_\text{g}}}\right) \cdot \sin^2\left(\frac{\Delta_\phi}{2}\right)
%             \end{equation*}
%             \begin{equation*}
%                 \textcolor{Red}{\beta_b} = \arctan\left(\frac{X}{Y}\right) - \frac{\pi}{2}
%             \end{equation*}
%             \begin{equation*}
%                 \textcolor{Red}{d_\mathbf{AB}} = R_\text{Terra} \cdot 2 \cdot \arctan\left(\frac{\sqrt{Z}}{\sqrt{1-Z}}\right)
%             \end{equation*}
%         % \end{multicols}
%     \end{frame}

\begin{frame}{Coordenadas Geográficas}
	\begin{itemize}[<+->]\addtolength{\itemsep}{0.5\baselineskip}
		\item É necessário decodificar os dados recebidos pela telemetria do foguete;
		\item A precisão da busca fica dependente da precisão do GNSS;
		\item O método baseia-se na ideia de que a equipe de busca tem acesso à própria coordenada geográfica;
		\item Num espaço sem acesso à internet, obter tal coordenada pode ser um problema;
		\item Se o sinal da telemetria for fraco, pode se tornar inviável utilizar este método.
	\end{itemize}
\end{frame}
