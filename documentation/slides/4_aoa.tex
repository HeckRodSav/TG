\section{AoA}

    \begin{frame}{\textit{Angle of Arrival}}
        % % \resizebox{!}{0.7\textheight}{%
\begin{circuitikz}[american, voltage shift=0.5, line width=0.5, every node/.style={font = {\footnotesize\bfseries}}]

    \def\wavelength{3.5}
    \pgfmathsetmacro\d{0.5*\wavelength}

    \def\antennaAngle{20}
    \pgfmathsetmacro\signalAngle{\antennaAngle+40}

    \def\closeRange{9}
    \def\farRange{\closeRange+13}

	\def\NAntennas{3}
	\pgfmathsetmacro\AngleAntennas{360/\NAntennas}
	\def\ShiftAngleAntennas{-90}

	\pgfmathsetmacro\RhoAntennas{\d/(2*sin(180/\NAntennas))}

    \def\centerarc(#1)(#2:#3:#4)% Syntax: [draw options] (center) (initial angle:final angle:radius)
    { ($(#1)+({#4*cos(#2)},{#4*sin(#2)})$) arc (#2:#3:#4) }

    \def\coordref[#1](#2){%

        \coordinate(sysref) at (#2);

        \draw[#1, -latex] (sysref) ++(-0.4,-0.3) -- ++(0.9,0) node[midway, below]{$x$};
        \draw[#1, -latex] (sysref) ++(-0.3,-0.4) -- ++(0,0.9) node[midway, left]{$y$};
        \draw[#1, -latex] \centerarc(sysref)(-90:180:0.25);
        \draw[#1] (sysref) node{$+$}
    }

    \coordinate (bottomleft) at (-3.5,-1);
    \coordinate (topright) at (3.5,5);


    % \draw[Red,dashed] (bottomleft) rectangle (topright);
    \clip (bottomleft) rectangle (topright);

    \coordinate (O) at (0,0);
    \coordinate (sourceAntenna) at (\signalAngle:\closeRange*\wavelength);
    % \draw [help lines, dashed] (bottomleft) grid (topright); % desenha grid
    % \draw [red] (O) node[draw,cross out] {}; % marca pont(0,0)

    % Circulo de antenas
	% \draw[densely dotted, opacity=0.25] (O) ++(90:\RhoAntennas) circle (\RhoAntennas);

    % Linhas do sinal de fundo
    \foreach \x [evaluate={\y=int((\x+\closeRange));\z=int((\x+\closeRange));}] in {-3,...,3} {
        \draw [black!75, very thin]
        (sourceAntenna) ++ (\signalAngle:-\z*\wavelength)
            % node[anchor=west, font = {\footnotesize\bfseries}]{$\y\lambda$}
        ($(sourceAntenna) + (\signalAngle:-\z*\wavelength) + ({10*cos(\signalAngle+90)},{10*sin(\signalAngle+90)})$)
            --
        ($(sourceAntenna) + (\signalAngle:-\z*\wavelength) - ({10*cos(\signalAngle+90)},{10*sin(\signalAngle+90)})$)
        % \draw [gray, thin] (sourceAntenna) circle (\z)
        ;
    }

    % Antenas
    \draw[thick, cmyk_R] (O) node[dinantenna] (A00) {} ;
    % \draw[thick, cmyk_G, opacity=0.75] (O) ++(60:\d) node[dinantenna] (A0d) {} node [below] {$A_{k+2}$};
    \draw[thick, cmyk_B] (O) ++(\antennaAngle:\d) node[dinantenna] (Ad0) {} ;

    \draw[very thin, Black!50, -latex] % Desenha eixo X
        (-3,0) -- (3,0) node[below left] {$x$}
    ;

    % Ângulo alpha entre antenas
    \draw[thin, cmyk_M]
        \centerarc(O)(0:\antennaAngle:0.3)
        node [above, inner sep=3pt] {$\alpha$}
    ;


    % Desenha senoide de fundo
    \draw[Goldenrod, domain=-8:8, samples=100]
        (A00) ++(\signalAngle+90:0.5*\wavelength) coordinate(signalAux)
        plot[shift={(signalAux)}, rotate=\signalAngle]({\x},{cos(\x * pi * 2 / \wavelength r)})
    ;

    % Direção do sinal
    \draw[very thick, dashed, -latex, Goldenrod]
        % (A00) ++(1.5*\d,0) ++ (\signalAngle:-0.5*\d) -- coordinate(angleArrow) ++(\signalAngle:\d)
        (A00) ++(-2,0) ++ (\signalAngle:-0.25*\d) -- coordinate(angleArrow) ++ (\signalAngle:0.5*\d) --++(\signalAngle:0.25*\d)
    ;
    % Angulo Theta do sinal
    \draw[thin]
        (angleArrow) ++ (0.4, 0) node [below,inner sep=2pt] {$\theta_\text{\ac{AoA}}$}
        \centerarc(angleArrow)(0:\signalAngle:0.4)
    ;

    % Triangulo retângulo + quadradinho
    \draw[Black]

        (A00) --++($({\signalAngle-90}:{\d*sin(\signalAngle-\antennaAngle)})$) coordinate (pontoTriangulo) -- (Ad0) -- (A00)

        (pontoTriangulo)
          ++(\signalAngle:0.125)
        --++(\signalAngle+90:0.125)
        --++(\signalAngle+180:0.125)
    ;

    % Arco do angulo beta
    \draw[thin, Purple]
        (Ad0) ++ (180+\antennaAngle:0.4) node[above, inner sep=3pt] {$\beta$}
        \centerarc(Ad0)(180+\antennaAngle:180+\signalAngle:0.4)
    ;

    % Distânci d entre antenas
    \draw[latex-latex]
        ($(A00)+(0,1)$) -- ($(Ad0)+(0,1)$) node [midway, fill=white, circle, inner sep=1pt] {$d$}
    ;

    \newcommand\CircleRadius{3cm}
    %   \draw (0,0) circle (\CircleRadius);
    % special method of noting the position of a point
    \coordinate (P) at (50:\CircleRadius);

\end{circuitikz}
% }


        \centering\includegraphics{../pictures/AoA_1_new}
    \end{frame}

    \begin{frame}{\textit{Angle of Arrival}}
        % \input{../pictures/AoA_2}
        \centering\includegraphics{../pictures/AoA_2_new}
    \end{frame}

    \begin{frame}{\textit{Angle of Arrival}}
        % \input{../pictures/AoA_3}
        \centering\includegraphics{../pictures/AoA_3_new}
    \end{frame}

    \begin{frame}{\textit{Angle of Arrival}}
        % \input{../pictures/AoA_4}
        \centering\includegraphics{../pictures/AoA_4_new}
    \end{frame}

    \begin{frame}{Distância $d$}
                \begin{columns}
            \begin{column}{0.45\textwidth}
                \centering \vfill
                \input{../pictures/AoA_0}\vfill
                \visible<5->{\input{../pictures/AoA_0_fail}}\vfill
                \visible<6->{\input{../pictures/AoA_0_ok}}\vfill
            \end{column}
            \begin{column}{0.55\textwidth}
                \begin{itemize}[<+(-1)->]
                    \item Toma-se uma \textcolor{cmyk_B}{antena} como referência
                    \item Posiciona-se uma segunda \textcolor{Red}{antena} a uma distância determinada
                    \item Analisando a defasagem entre as antenas, é possível determinar o ângulo de incidência do sinal
                    \item Se a distância for maior que $\sfrac{\lambda}{2}$, haverá conflito de defasagem
                    \item Se for menor, há perda de resolução
                    \item Adota-se a distância de $d = \sfrac{\lambda}{2}$
                \end{itemize}
            \end{column}
        \end{columns}
    \end{frame}

\subsection{Malha de antenas}
    \begin{frame}{Geometria da malha de antenas}

        \begin{equation*}
            \rho = \frac{d}{2\cdot \sin\left(\displaystyle\frac{\pi}{N_\text{ant}}\right)}
        \end{equation*}

        \begin{equation*}
            k = \left\{1, 2, \dotsc, N_\text{ant}\right\}
        \end{equation*}

        \begin{equation*}
            \textcolor{cmyk_B}{A_k} =
            \rho
            \cdot \exp\left(\imath\cdot k \cdot \frac{2\pi}{N_\text{ant}}\right) =
            \left( \operatorname{\mathcal{Re}}\left( \textcolor{cmyk_B}{A_k} \right), ~\operatorname{\mathcal{Im}}\left( \textcolor{cmyk_B}{A_k} \right) \right) =
            \left( x_{A_k}, ~ y_{A_k} \right)
        \end{equation*}

        \begin{equation*}
            \textcolor{cmyk_M}{\alpha_k} = \arg\left( \textcolor{cmyk_B}{A_k} - \textcolor{cmyk_R}{A_{k+1}} \right)
        \end{equation*}
    \end{frame}

    \begin{frame}{Três antenas}
        \centering%
        % \fbox{%
            % \scalebox{0.75}%
            % \resizebox{0.5\textwidth}{!}
            % {\input{../pictures/antennas_3}}%
            \includegraphics[scale=0.8]{../pictures/antennas_3.pdf}%
        % }
        % \vfill
    \end{frame}
    \begin{frame}{Cinco antenas}
        \centering%
        % \fbox{%
            % \scalebox{0.75}%
            % \resizebox{0.5\textwidth}{!}
            % {\input{../pictures/antennas_5}}%
            \includegraphics[scale=0.8]{../pictures/antennas_5.pdf}%
        % }
        % \vfill
    \end{frame}
    \begin{frame}{Sete antenas}
        \centering%
        % \fbox{%
            % \scalebox{0.75}%
            % \resizebox{0.5\textwidth}{!}
            % {\input{../pictures/antennas_7}}%
            \includegraphics[scale=0.8]{../pictures/antennas_7.pdf}%
        % }
        % \vfill
    \end{frame}

\section{Calculo de fase}
    \begin{frame}{Geometria do sistema}
        \begin{equation*}
            T = \frac{2\pi}{\omega} = \frac{1}{f}
        \end{equation*}

        \begin{equation*}
            I_k =
            \int\limits_0^{T} \cos\left(\omega \cdot\tau\right)
            \cdot w\left( x_{A_k}, ~y_{A_k}, ~\tau \right) \partial \tau
        \end{equation*}

        \begin{equation*}
            Q_k =
            \int\limits_0^{T} \sin\left(\omega\cdot\tau\right)
            \cdot w\left( x_{A_k}, ~y_{A_k}, ~\tau \right) \partial \tau
        \end{equation*}

        \begin{equation*}
            \textcolor{cmyk_B}{Z_k} =
            \frac{\omega}{\pi}\cdot\left(I_k + \imath Q_k\right)
        \end{equation*}

        \begin{equation*}
            \Delta_\Phi =
            \textcolor{cmyk_B}{\Phi_k} - \textcolor{cmyk_R}{\Phi_{k+1}} =
            \arg\left(\textcolor{cmyk_B}{Z_k}\right) - \arg\left(\textcolor{cmyk_R}{Z_{k+1}}\right) =
            \arg\left(\textcolor{cmyk_B}{Z_k} \cdot \overline{\textcolor{cmyk_R}{Z_{k+1}}}\right)
        \end{equation*}

        \begin{equation*}
            \textcolor{Purple}{\beta_{\pm k}} = \arccos\left(\frac{\cancel{\lambda}}{\cancel{d}}\cdot\frac{\Delta_\Phi}{\cancel{2}\pi}\right)
        \end{equation*}

    \end{frame}

    \begin{frame}{Geometria do sistema}
        \centering\includegraphics{../pictures/AoA_geometria.pdf}
    \end{frame}

    \begin{frame}{Determinar AoA}
        \begin{equation*}
            \theta_{\pm k} = \textcolor{cmyk_M}{\alpha_k}\pm \textcolor{Purple}{\beta_{\pm k}}
        \end{equation*}

        \begin{equation*}
            \Theta = \left\{\theta_{\pm k} ~\middle\vert~ \forall k\right\}
        \end{equation*}

        \begin{equation*}
            \delta = \frac{\pi}{2 \cdot \left( 1 + N_\text{ant} \right)}
        \end{equation*}

        \begin{equation*}
            \Theta_{\left\lfloor\bullet\right\rceil} =
            \left\{\left\lfloor\frac{\theta}{\delta}\right\rceil\cdot\delta ~\middle\vert~ \forall \theta \in \Theta  \right\}
        \end{equation*}

        \begin{equation*}
            \theta_\mathcal{M_o} = \operatorname{\mathcal{M_o}}\left( \Theta_{\left\lfloor\bullet\right\rceil}  \right)
        \end{equation*}

        \begin{equation*}
            \Theta_\text{F} = \left\{\theta \in \Theta  ~\middle\vert~
            \theta_\mathcal{M_o} - \delta \leq \theta \leq \theta_\mathcal{M_o} + \delta\right\}
        \end{equation*}

        \begin{equation*}
            \theta_\text{AoA} = \widetilde{\Theta_\text{F}}
        \end{equation*}
    \end{frame}

\section{Simulação}
    \begin{frame}{Definição de onda}
        % \begin{multicols}{2}
            \begin{equation*}
                r_0 = r \cdot \lambda
            \end{equation*}

            \begin{equation*}
                x_0 = r_0 \cdot \cos(\theta)
            \end{equation*}

            \begin{equation*}
                y_0 = r_0 \cdot \sin(\theta)
            \end{equation*}

            \begin{equation*}
                \mathcal{aux}_{\text{\lstinline|argument_r|}}(x, y, t, \theta, r, \phi, \lambda, \omega) =
                \frac{2\pi}{\lambda} \cdot \sqrt{(y - y_0)^2+(x - x_0)^2} + \omega \cdot t + \phi
            \end{equation*}

            \begin{equation*}
                w(x, y, t, \theta, r, \phi, \lambda, \omega) = \frac{\sin\left(\mathcal{aux}\right)+ \cos\left(\mathcal{aux}\right)}{\sqrt{2}}
            \end{equation*}
        % \end{multicols}
    \end{frame}


    \begin{frame}

        \centering
        \href{https://github.com/HeckRodSav/TG/blob/main/documentation/pictures/POLY_3/simul_POLY_3_R_50.gif}{\includegraphics[width=\textwidth]{../pictures/simul_POLY_3_R_50.png}}

    \end{frame}

    % \begin{frame}{\href{https://drive.google.com/file/d/1-ep5hH8TSrnHU_m9b4AFmUJlB_hGR41m/view?usp=drive_link}{Análise de vetores complexos}}
    %     % \begin{multicols}{2}
    %         \begin{equation*}
    %             C(x,y) = \int_0^T w(x,y,t) \cdot \cos(k(x,y,t)) \partial t
    %         \end{equation*}

    %         \begin{equation*}
    %             S(x,y) = \int_0^T w(x,y,t) \cdot \sin(k(x,y,t)) \partial t
    %         \end{equation*}

    %         \begin{equation*}
    %             Z(x,y) = 2\cdot(S + \imath C)
    %         \end{equation*}

    %         \begin{equation*}
    %             \Delta_{x,y} = \arg(\textcolor{Green}{Z_{0,0}}\cdot Z^*_{x,y})
    %         \end{equation*}

    %         \begin{equation*}
    %             \text{componente}_{x,y} = -\frac{\Delta_{x,y}}{\pi}\cdot\frac{\cancel{\lambda}}{\cancel{d \cdot 2}} = -\frac{\Delta_{x,y}}{\pi}
    %         \end{equation*}
    %         % \end{multicols}
    % \end{frame}