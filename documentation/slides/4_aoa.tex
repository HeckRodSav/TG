\section{Posicionamento por AoA}

    \begin{frame}{Estimar defasagem}
                \begin{columns}
            \begin{column}{0.45\textwidth}
                \centering \vfill
                \input{../pictures/AoA_0}\vfill
                \visible<5->{\input{../pictures/AoA_0_fail}}\vfill
                \visible<6->{\input{../pictures/AoA_0_ok}}\vfill
            \end{column}
            \begin{column}{0.55\textwidth}
                \begin{itemize}[<+(-1)->]
                    \item Toma-se uma \textcolor{Green}{antena} como referência
                    \item Posiciona-se uma segunda \textcolor{Red}{antena} a uma distância determinada
                    \item Analisando a defasagem entre as antenas, é possível determinar o ângulo de incidência do sinal
                    \item Se a distância for maior que $\sfrac{\lambda}{2}$, haverá conflito de defasagem
                    \item Se for menor, há perda de resolução
                    \item Adota-se a distância de $d = \sfrac{\lambda}{2}$
                \end{itemize}
            \end{column}
        \end{columns}
    \end{frame}

    \begin{frame}{\textit{Angle of Arrival}}
        % \resizebox{!}{0.7\textheight}{%
\begin{circuitikz}[american, voltage shift=0.5, line width=0.5, every node/.style={font = {\footnotesize\bfseries}}]

    \def\wavelength{3.5}
    \pgfmathsetmacro\d{0.5*\wavelength}

    \def\antennaAngle{20}
    \pgfmathsetmacro\signalAngle{\antennaAngle+40}

    \def\closeRange{9}
    \def\farRange{\closeRange+13}

	\def\NAntennas{3}
	\pgfmathsetmacro\AngleAntennas{360/\NAntennas}
	\def\ShiftAngleAntennas{-90}

	\pgfmathsetmacro\RhoAntennas{\d/(2*sin(180/\NAntennas))}

    \def\centerarc(#1)(#2:#3:#4)% Syntax: [draw options] (center) (initial angle:final angle:radius)
    { ($(#1)+({#4*cos(#2)},{#4*sin(#2)})$) arc (#2:#3:#4) }

    \def\coordref[#1](#2){%

        \coordinate(sysref) at (#2);

        \draw[#1, -latex] (sysref) ++(-0.4,-0.3) -- ++(0.9,0) node[midway, below]{$x$};
        \draw[#1, -latex] (sysref) ++(-0.3,-0.4) -- ++(0,0.9) node[midway, left]{$y$};
        \draw[#1, -latex] \centerarc(sysref)(-90:180:0.25);
        \draw[#1] (sysref) node{$+$}
    }

    \coordinate (bottomleft) at (-3.5,-1);
    \coordinate (topright) at (3.5,5);


    % \draw[Red,dashed] (bottomleft) rectangle (topright);
    \clip (bottomleft) rectangle (topright);

    \coordinate (O) at (0,0);
    \coordinate (sourceAntenna) at (\signalAngle:\closeRange*\wavelength);
    % \draw [help lines, dashed] (bottomleft) grid (topright); % desenha grid
    % \draw [red] (O) node[draw,cross out] {}; % marca pont(0,0)

    % Circulo de antenas
	% \draw[densely dotted, opacity=0.25] (O) ++(90:\RhoAntennas) circle (\RhoAntennas);

    % Linhas do sinal de fundo
    \foreach \x [evaluate={\y=int((\x+\closeRange));\z=int((\x+\closeRange));}] in {-3,...,3} {
        \draw [black!75, very thin]
        (sourceAntenna) ++ (\signalAngle:-\z*\wavelength)
            % node[anchor=west, font = {\footnotesize\bfseries}]{$\y\lambda$}
        ($(sourceAntenna) + (\signalAngle:-\z*\wavelength) + ({10*cos(\signalAngle+90)},{10*sin(\signalAngle+90)})$)
            --
        ($(sourceAntenna) + (\signalAngle:-\z*\wavelength) - ({10*cos(\signalAngle+90)},{10*sin(\signalAngle+90)})$)
        % \draw [gray, thin] (sourceAntenna) circle (\z)
        ;
    }

    % Antenas
    \draw[thick, cmyk_R] (O) node[dinantenna] (A00) {} ;
    % \draw[thick, cmyk_G, opacity=0.75] (O) ++(60:\d) node[dinantenna] (A0d) {} node [below] {$A_{k+2}$};
    \draw[thick, cmyk_B] (O) ++(\antennaAngle:\d) node[dinantenna] (Ad0) {} ;

    \draw[very thin, Black!50, -latex] % Desenha eixo X
        (-3,0) -- (3,0) node[below left] {$x$}
    ;

    % Ângulo alpha entre antenas
    \draw[thin, cmyk_M]
        \centerarc(O)(0:\antennaAngle:0.3)
        node [above, inner sep=3pt] {$\alpha$}
    ;


    % Desenha senoide de fundo
    \draw[Goldenrod, domain=-8:8, samples=100]
        (A00) ++(\signalAngle+90:0.5*\wavelength) coordinate(signalAux)
        plot[shift={(signalAux)}, rotate=\signalAngle]({\x},{cos(\x * pi * 2 / \wavelength r)})
    ;

    % Direção do sinal
    \draw[very thick, dashed, -latex, Goldenrod]
        % (A00) ++(1.5*\d,0) ++ (\signalAngle:-0.5*\d) -- coordinate(angleArrow) ++(\signalAngle:\d)
        (A00) ++(-2,0) ++ (\signalAngle:-0.25*\d) -- coordinate(angleArrow) ++ (\signalAngle:0.5*\d) --++(\signalAngle:0.25*\d)
    ;
    % Angulo Theta do sinal
    \draw[thin]
        (angleArrow) ++ (0.4, 0) node [below,inner sep=2pt] {$\theta_\text{\ac{AoA}}$}
        \centerarc(angleArrow)(0:\signalAngle:0.4)
    ;

    % Triangulo retângulo + quadradinho
    \draw[Black]

        (A00) --++($({\signalAngle-90}:{\d*sin(\signalAngle-\antennaAngle)})$) coordinate (pontoTriangulo) -- (Ad0) -- (A00)

        (pontoTriangulo)
          ++(\signalAngle:0.125)
        --++(\signalAngle+90:0.125)
        --++(\signalAngle+180:0.125)
    ;

    % Arco do angulo beta
    \draw[thin, Purple]
        (Ad0) ++ (180+\antennaAngle:0.4) node[above, inner sep=3pt] {$\beta$}
        \centerarc(Ad0)(180+\antennaAngle:180+\signalAngle:0.4)
    ;

    % Distânci d entre antenas
    \draw[latex-latex]
        ($(A00)+(0,1)$) -- ($(Ad0)+(0,1)$) node [midway, fill=white, circle, inner sep=1pt] {$d$}
    ;

    \newcommand\CircleRadius{3cm}
    %   \draw (0,0) circle (\CircleRadius);
    % special method of noting the position of a point
    \coordinate (P) at (50:\CircleRadius);

\end{circuitikz}
% }


    \end{frame}

    \begin{frame}{\textit{Angle of Arrival}}
        \input{../pictures/AoA_2}
    \end{frame}

    \begin{frame}{\textit{Angle of Arrival}}
        \input{../pictures/AoA_3}
    \end{frame}

    \begin{frame}{\textit{Angle of Arrival}}
        \input{../pictures/AoA_4}
    \end{frame}

    \section{Matriz de antenas}
    \begin{frame}{\textit{Angle of Arrival}}
            % \resizebox{!}{0.7\textheight}{%
    \begin{circuitikz}[american, voltage shift=0.5, line width=0.5,every node/.style={font = {\footnotesize\bfseries}}]

        \def\wavelength{4}
        \def\d{0.5*\wavelength}


        \def\antennaAngle{240}
        \def\closeRange{9}
        \def\farRange{\closeRange+13}

        \def\centerarc[#1](#2)(#3:#4:#5)(#6)(#7)% Syntax: [draw options] (center) (initial angle:final angle:radius)
        { \draw[#1] ($(#2)+({#5*cos(#3)},{#5*sin(#3)})$) arc (#3:#4:#5) node[midway,anchor=#7] {#6}; }


        \coordinate (O) at (0,0);
        \coordinate (antenna) at (\antennaAngle:\closeRange*\wavelength);
        % \draw [help lines, dashed] (-5,-3) grid (5,3); % desenha grid
        % \draw [red] (O) node[draw,cross out] {}; % marca pont(0,0) 
        
        % \draw (-6.8,-4) rectangle (6.8,4);
        \clip (-6.8,-4) rectangle (6.8,4);

        % \draw[thick]
        %     (antenna) node[dinantenna]{}
        % ;
        
        \foreach \x [evaluate={\y=int((\x+\closeRange));\z=int((\x+\closeRange)*\wavelength);}] in {-3,...,3} {
            \draw [black, thin] 
            (antenna) ++ (\antennaAngle:-\z)
                % node[anchor=west, font = {\footnotesize\bfseries}]{$\y\lambda$}
            ($(antenna) + (\antennaAngle:-\z) + ({10*cos(\antennaAngle+90)},{10*sin(\antennaAngle+90)})$)
                -- 
            ($(antenna) + (\antennaAngle:-\z) - ({10*cos(\antennaAngle+90)},{10*sin(\antennaAngle+90)})$);
            % \draw [gray, thin] (antenna) circle (\z);
        }
        
        \draw[thick]
            (0,0)  node[Green, dinantenna] (A00) {}
            (0,\d) node[Blue,  dinantenna] (A0d) {}
            (\d,0) node[Red,   dinantenna] (Ad0) {}
        ;

        \draw[very thick, dashed, -latex]
            (A00) ++(-\d,0) coordinate(aux) ++(\antennaAngle:0.5*\d) -- ++(\antennaAngle:-\d)
        ;

        
        % \draw[Goldenrod, domain=-8:8, samples=100] plot[shift={(aux)}, rotate=\antennaAngle]({\x},{sin(\x * pi * 2 / \wavelength r)});

        \draw[thin, Red, opacity=0.5]
            (A00) ++ ($({\antennaAngle-90}:{\d*sin(\antennaAngle)})$) -- (Ad0) -- (A00)

            ($({\antennaAngle-90}:{\d*sin(\antennaAngle)})$) 
              ++(\antennaAngle+180:0.25)
            --++(\antennaAngle-90:0.25)
            --++(\antennaAngle:0.25)
        ;


        \centerarc[thin, Red, opacity=0.5](A00)(\antennaAngle+90:360:0.4)($\beta_{0d}$)(north)

        
        \draw[thin, Blue, opacity=0.5]
            (A00) ++ ($({\antennaAngle+90}:{\d*cos(\antennaAngle)})$) -- (A0d) -- (A00)

            ($({\antennaAngle+90}:{\d*cos(\antennaAngle)})$) 
              ++(\antennaAngle+180:0.25)
            --++(\antennaAngle+90:0.25)
            --++(\antennaAngle:0.25)    
        ;

        \centerarc[thin, Blue, opacity=0.5](A00)(\antennaAngle-90:90:0.4)($\beta_{d0}$)(north east)

        \draw[latex-latex]
            ($(A00)+(0,1)$) -- ($(Ad0)+(0,1)$) node [midway, fill=white] {$d$}
        ;
        


        % \foreach \x in {0,60,...,300} {
        %     \draw[thick] (\x:1 cm) -- (\x + 60:1 cm);
            
        %     \draw (\x + 30:1.732 cm) node[Gray, circ]{};
        %     \draw[Gray, dashed] (\x:1 cm) -- ++(\x: 0.9cm);
        %     \draw[Gray, dotted]
        %     %     % (\x:1 cm) arc (\x+240:\x+180:1cm)
        %         (\x:1 cm) arc [start angle=\x+120, delta angle=110, radius=1cm]
        %         (\x:1 cm) arc [start angle=\x+120, delta angle=-50, radius=1cm]
        %     ;
        % }
    
        % \draw (0,0) node [circ] {} node [below left,font={\scriptsize\bfseries}] {BS};
        % \draw[thick, densely dotted] (0,0) circle (1cm);
        
        % \draw[-latex] (0,0) -- (0:1cm) node[midway, below] {$R_c$};
        % \draw[-latex] (0,0) -- (90:0.866cm) node[midway, left] {$R$};
            
    \end{circuitikz}
  % }


    \end{frame}

\subsection{Cálculo de Ângulo de chegada}
    \begin{frame}{Definição de onda}
        % \begin{multicols}{2}
            \begin{equation*}
                r = r_\omega \cdot \lambda
            \end{equation*}

            \begin{equation*}
                x_0 = r \cdot \cos(\theta_\omega)
            \end{equation*}

            \begin{equation*}
                y_0 = r \cdot \sin(\theta_\omega)
            \end{equation*}

            \begin{equation*}
                k(x, y, t, \theta_\omega, r_\omega, \lambda, \omega) = \frac{2\pi}{\lambda} \cdot \sqrt{(y - y_0)^2+(x - x_0)^2} + \omega \cdot t
            \end{equation*}

            \begin{equation*}
                w(x, y, t, \theta_\omega, r_\omega, \lambda, \omega) = \sin(k + \phi)+ \cos(k + \phi)
            \end{equation*}
        % \end{multicols}
    \end{frame}

    \begin{frame}{\href{https://drive.google.com/file/d/1-ep5hH8TSrnHU_m9b4AFmUJlB_hGR41m/view?usp=drive_link}{Análise de vetores complexos}}
        % \begin{multicols}{2}
            \begin{equation*}
                C(x,y) = \int_0^T w(x,y,t) \cdot \cos(k(x,y,t)) \partial t
            \end{equation*}

            \begin{equation*}
                S(x,y) = \int_0^T w(x,y,t) \cdot \sin(k(x,y,t)) \partial t
            \end{equation*}

            \begin{equation*}
                Z(x,y) = 2\cdot(S + \imath C)
            \end{equation*}

            \begin{equation*}
                \Delta_{x,y} = \arg(\textcolor{Green}{Z_{0,0}}\cdot Z^*_{x,y})
            \end{equation*}

            \begin{equation*}
                \text{componente}_{x,y} = -\frac{\Delta_{x,y}}{\pi}\cdot\frac{\cancel{\lambda}}{\cancel{d \cdot 2}} = -\frac{\Delta_{x,y}}{\pi}
            \end{equation*}
            % \end{multicols}
    \end{frame}