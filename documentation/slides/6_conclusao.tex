\section{Conclusão}

\begin{frame}{Conclusão}
	\begin{itemize}[<+->]\addtolength{\itemsep}{0.5\baselineskip}
		\item É necessário ter uma boa estratégia para localizar um foguete de sondagem;
		\item Para utilizar um GNSS, é necessário decodificar o sinal recebido;
		% \item Se não for possível decodificar, ainda é possível determinar AoA;
		\item Mesmo com o sinal detectável, ainda é possível ter problemas para decodificá-lo;
		\item Analisando a defasagem do sinal entre pares de antenas torna possível estimar o AoA;
		\item Cada par de antenas gera dois possíveis ângulos candidatos;
		\item A proposta utilizando votação se mostrou eficaz.
	\end{itemize}
\end{frame}

\begin{frame}{Conclusão}
	\begin{itemize}[<+->]\addtolength{\itemsep}{0.5\baselineskip}
		\item Dispor as antenas da malha nos vértices de um polígono regular simplificou os cálculos;
		\item Polígonos com quantidade par de lados sempre terão lados paralelos;
		\item Mais pares na malha levaram a maior precisão nos valores obtidos;
		\item O valor mínimo de R\textsuperscript{2} obtido foi superior a \qty{75}{\percent};
		\item O valor médio de R\textsuperscript{2} obtido foi superior a \qty{92}{\percent};
		\item Houveram problemas relacionados ao uso de \textit{software} livre e a manter a compatibilidade entre o GNU Octave e o MATLAB;.
	\end{itemize}
\end{frame}

\begin{frame}{Melhorias futuras}
	\begin{itemize}[<+->]\addtolength{\itemsep}{0.5\baselineskip}
		\item Analisar diferentes disposições de antenas na malha, já que os cálculos não exigem a disposição poligonal;
		\item Contemplar outros tipos de ruído pode acrescentar ao modelo;
		\item Construir um \textit{hardware} capaz de aferir a defasagem em uma malha de antenas poderá contribuir para simulações mais completas.
	\end{itemize}
\end{frame}