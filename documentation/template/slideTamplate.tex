%%%%%%%%%%%%%%%%%%%%%%%%%%%%%%%%%%%%%%%%%%%%%%%%%%%%%%%%%%%%%%%%%%%%%%%%%%%%%%%%
% TEMPLATE PARA SLIDES
% UFABC Rocket Design
%
% Lista de revisões: (data ISO - nome)
%
% 2021-04-01 - Heitor Rodrigues Savegnago
% 2021-04-14 - Heitor Rodrigues Savegnago
% 2021-04-16 - Heitor Rodrigues Savegnago
% 2021-04-21 - Heitor Rodrigues Savegnago
% 2021-04-28 - Heitor Rodrigues Savegnago
% 2021-05-05 - Heitor Rodrigues Savegnago
% 2021-05-12 - Heitor Rodrigues Savegnago
% 2021-05-13 - Heitor Rodrigues Savegnago
% 2023-09-22 - Heitor Rodrigues Savegnago
% 2024-12-12 - Heitor Rodrigues Savegnago
% 2025-07-13 - Heitor Rodrigues Savegnago
%
%%%%%%%%%%%%%%%%%%%%%%%%%%%%%%%%%%%%%%%%%%%%%%%%%%%%%%%%%%%%%%%%%%%%%%%%%%%%%%%%

% Formatação geral do documento

\usepackage[
    orientation=landscape,
    size=custom,
    width=16,
    height=12,
    scale=0.5,
    debug
]{beamerposter}

% Fonte
    %\renewcommand{\rmdefault}{phv} % Arial
    \usepackage[lighttt]{lmodern} % Fonte Latin Modern
        \renewcommand{\familydefault}{\sfdefault} % Estilo global Sans Serif

    % Subistituição de fonte para casos de erro
    \DeclareFontFamilySubstitution{TS1}{aer}{lmr}
    \DeclareFontFamilySubstitution{TS1}{aett}{lmtt}
    \DeclareFontFamilySubstitution{TS1}{aess}{lmss}

% Correções de espaçamentos
    \usepackage{indentfirst} % Parágrafos indentados

% Opções para tópicos de enumerate

    % \usepackage[shortlabels]{enumitem} % Selecionar formato em itemize
    % Incompatível com opções de only

        % \setlength{\itemsep}{1em}
    % Conflita com [<+->] em itemize
        % \let\tempone\itemize
        % \let\temptwo\enditemize
        % \renewenvironment{itemize}{\tempone\addtolength{\itemsep}{0.5\baselineskip}}{\temptwo}

        % \let\tempthree\enumerate
        % \let\tempfour\endenumerate
        % \renewenvironment{enumerate}{\tempthree\addtolength{\itemsep}{0.5\baselineskip}}{\tempfour}

% Texto geral em distribuição justificada
    \usepackage{ragged2e}
        \apptocmd{\frame}{}{\justifying}{}



%%%%%%%%%%%%%%%%%%%%%%%%%%%%%%%%%%%%%%%%%%%%%%%%%%%%%%%%%%%%%%%%%%%%%%%%%%%%%%%%

% Codificação de carecteres de entrada

\usepackage{ae} % "Almost European"
\usepackage[T1]{fontenc} % Caracteres especiais
\usepackage[utf8]{inputenc} % Caracteres especiais

%%%%%%%%%%%%%%%%%%%%%%%%%%%%%%%%%%%%%%%%%%%%%%%%%%%%%%%%%%%%%%%%%%%%%%%%%%%%%%%%

% Configuração de linguagem padrão do documento

\usepackage[english, main=brazil]{babel} % Detalhes automáticos em Português

    \AtBeginDocument{\renewcommand{\contentsname}{\centerline{Sumário}}}
    \AtBeginDocument{\renewcommand{\bibname}{Referências Bibliográficas}}
% 	\AtBeginDocument{\renewcommand{\figurename}{Figura}}
% 	\AtBeginDocument{\renewcommand{\tablename}{Tabela}}

\usepackage{textcomp} % Suporte de caracteres especiais

\usepackage{csquotes} % Opções de citação

%%%%%%%%%%%%%%%%%%%%%%%%%%%%%%%%%%%%%%%%%%%%%%%%%%%%%%%%%%%%%%%%%%%%%%%%%%%%%%%%


% Configurações de core


\usepackage{transparent} % Opções de transparência

% \usepackage{xcolor} % Define cores
% O pacote xcolor já é carregado pelo beamer

    \definecolor{0DF}{HTML}{00DDFF}%
    \definecolor{0FD}{HTML}{00FFDD}%
    \definecolor{DF0}{HTML}{DDFF00}%
    \definecolor{FD0}{HTML}{FFDD00}%
    \definecolor{F0D}{HTML}{FF00DD}%
    \definecolor{D0F}{HTML}{DD00FF}%

    \definecolor{0BD}{HTML}{00BBDD}%
    \definecolor{0DB}{HTML}{00DDBB}%
    \definecolor{BD0}{HTML}{BBDD00}%
    \definecolor{DB0}{HTML}{DDBB00}%
    \definecolor{D0B}{HTML}{DD00BB}%
    \definecolor{B0D}{HTML}{BB00DD}%

    \definecolor{09B}{HTML}{0099BB}%
    \definecolor{0B9}{HTML}{00BB99}%
    \definecolor{9B0}{HTML}{99BB00}%
    \definecolor{B90}{HTML}{BB9900}%
    \definecolor{B09}{HTML}{BB0099}%
    \definecolor{90B}{HTML}{9900BB}%

    \definecolor{079}{HTML}{007799}%
    \definecolor{097}{HTML}{009977}%
    \definecolor{790}{HTML}{779900}%
    \definecolor{970}{HTML}{997700}%
    \definecolor{907}{HTML}{990077}%
    \definecolor{709}{HTML}{770099}%

    \definecolor{057}{HTML}{005577}%
    \definecolor{075}{HTML}{007755}%
    \definecolor{570}{HTML}{557700}%
    \definecolor{750}{HTML}{775500}%
    \definecolor{705}{HTML}{770055}%
    \definecolor{507}{HTML}{550077}%

    \definecolor{035}{HTML}{003355}%
    \definecolor{053}{HTML}{005533}%
    \definecolor{350}{HTML}{335500}%
    \definecolor{530}{HTML}{553300}%
    \definecolor{503}{HTML}{550033}%
    \definecolor{305}{HTML}{330055}%

    \definecolor{013}{HTML}{001133}%
    \definecolor{031}{HTML}{003311}%
    \definecolor{130}{HTML}{113300}%
    \definecolor{310}{HTML}{331100}%
    \definecolor{301}{HTML}{330011}%
    \definecolor{103}{HTML}{110033}%

    \definecolor{rgb_R}{rgb}{1,0,0}%
    \definecolor{rgb_G}{rgb}{0,1,0}%
    \definecolor{rgb_B}{rgb}{0,0,1}%
    \definecolor{rgb_M}{rgb}{1,0,1}%
    \definecolor{rgb_Y}{rgb}{1,1,0}%
    \definecolor{rgb_C}{rgb}{0,1,1}%
    \definecolor{rgb_W}{rgb}{1,1,1}%
    \definecolor{rgb_K}{rgb}{0,0,0}%

    \definecolor{cmyk_C}{cmyk}{1,0,0,0}%
    \definecolor{cmyk_M}{cmyk}{0,1,0,0}%
    \definecolor{cmyk_Y}{cmyk}{0,0,1,0}%
    \definecolor{cmyk_G}{cmyk}{1,0,1,0}%
    \definecolor{cmyk_B}{cmyk}{1,1,0,0}%
    \definecolor{cmyk_R}{cmyk}{0,1,1,0}%
    \definecolor{cmyk_K}{cmyk}{1,1,1,1}%
    \definecolor{cmyk_W}{cmyk}{0,0,0,0}%

    \definecolor{antena_3_1}{hsb}{0.33,1,1}%
    \definecolor{antena_3_2}{hsb}{0.66,1,1}%
    \definecolor{antena_3_3}{hsb}{1.00,1,1}%

    \definecolor{antena_4_1}{hsb}{0.25,1,1}%
    \definecolor{antena_4_2}{hsb}{0.50,1,1}%
    \definecolor{antena_4_3}{hsb}{0.75,1,1}%
    \definecolor{antena_4_4}{hsb}{1.00,1,1}%

    \definecolor{antena_5_1}{hsb}{0.2,1,1}%
    \definecolor{antena_5_2}{hsb}{0.4,1,1}%
    \definecolor{antena_5_3}{hsb}{0.6,1,1}%
    \definecolor{antena_5_4}{hsb}{0.8,1,1}%
    \definecolor{antena_5_5}{hsb}{1.0,1,1}%

    \definecolor{antena_6_1}{hsb}{0.17,1,1}%
    \definecolor{antena_6_2}{hsb}{0.33,1,1}%
    \definecolor{antena_6_3}{hsb}{0.50,1,1}%
    \definecolor{antena_6_4}{hsb}{0.67,1,1}%
    \definecolor{antena_6_5}{hsb}{0.83,1,1}%
    \definecolor{antena_6_6}{hsb}{1.00,1,1}%

    \definecolor{antena_7_1}{hsb}{0.14,1,1}%
    \definecolor{antena_7_2}{hsb}{0.29,1,1}%
    \definecolor{antena_7_3}{hsb}{0.43,1,1}%
    \definecolor{antena_7_4}{hsb}{0.57,1,1}%
    \definecolor{antena_7_5}{hsb}{0.71,1,1}%
    \definecolor{antena_7_6}{hsb}{0.86,1,1}%
    \definecolor{antena_7_7}{hsb}{1.00,1,1}%

    \definecolor{strs}			{rgb}	{0.9,	0.2,	0	}%
    \definecolor{coments}		{rgb}	{0,		0.5,	0	}%
    \definecolor{backcode}		{rgb}	{0.3,	0,		0.2	}%

    \definecolor{slideBlue}{rgb}{0,0,0.6}%
    \definecolor{slideCyan}{rgb}{0,.3,.6}%
    \definecolor{slideTurquoise}{rgb}{0,.6,.3}%
    \definecolor{slideGreen}{rgb}{.3,.6,0}%
    \definecolor{slideYellow}{rgb}{.7,.6,0}%
    \definecolor{slideOrange}{rgb}{0.8,.3,0}%
    \definecolor{slideRed}{rgb}{.7,0,0}%
    \definecolor{slidePink}{rgb}{.8,0,.4}%
    \definecolor{slidePurple}{rgb}{.6,0,.8}%

    \definecolor{UFABCRDblue}{RGB}{8, 18, 77} % Cor do logo da rocket
    % \definecolor{UFABCRDblue}{cmyk}{0.90, 0.77, 0, 0.70} % Cor do logo da rocket

    % \newcommand{\MexerDepois}[1]{{\huge\color{F0D}#1}}
    \newcommand{\MexerDepois}[1]{
            \vspace*{2em}
            {\huge\color{F0D}#1}
            \vspace*{2em}}
        \newcommand{\mexer}[1]{{\color{F0D}#1}}

    \newcommand{\showcolors}%Mostra tabela de cores
    {{\ttfamily
            {\color{0DF}$\overset{\text{\tiny 0DF}}{\blacksquare}$}
            {\color{0BD}$\overset{\text{\tiny 0BD}}{\blacksquare}$}
            {\color{09B}$\overset{\text{\tiny 09B}}{\blacksquare}$}
            {\color{079}$\overset{\text{\tiny 079}}{\blacksquare}$}
            {\color{057}$\overset{\text{\tiny 057}}{\blacksquare}$}
            {\color{035}$\overset{\text{\tiny 035}}{\blacksquare}$}
            {\color{013}$\overset{\text{\tiny 013}}{\blacksquare}$}
            \\
            {\color{0FD}$\overset{\text{\tiny 0FD}}{\blacksquare}$}
            {\color{0DB}$\overset{\text{\tiny 0DB}}{\blacksquare}$}
            {\color{0B9}$\overset{\text{\tiny 0B9}}{\blacksquare}$}
            {\color{097}$\overset{\text{\tiny 097}}{\blacksquare}$}
            {\color{075}$\overset{\text{\tiny 075}}{\blacksquare}$}
            {\color{053}$\overset{\text{\tiny 053}}{\blacksquare}$}
            {\color{031}$\overset{\text{\tiny 031}}{\blacksquare}$}
            \\
            {\color{DF0}$\overset{\text{\tiny DF0}}{\blacksquare}$}
            {\color{BD0}$\overset{\text{\tiny BD0}}{\blacksquare}$}
            {\color{9B0}$\overset{\text{\tiny 9B0}}{\blacksquare}$}
            {\color{790}$\overset{\text{\tiny 790}}{\blacksquare}$}
            {\color{570}$\overset{\text{\tiny 570}}{\blacksquare}$}
            {\color{350}$\overset{\text{\tiny 350}}{\blacksquare}$}
            {\color{130}$\overset{\text{\tiny 130}}{\blacksquare}$}
            \\
            {\color{FD0}$\overset{\text{\tiny FD0}}{\blacksquare}$}
            {\color{DB0}$\overset{\text{\tiny DB0}}{\blacksquare}$}
            {\color{B90}$\overset{\text{\tiny B90}}{\blacksquare}$}
            {\color{970}$\overset{\text{\tiny 970}}{\blacksquare}$}
            {\color{750}$\overset{\text{\tiny 750}}{\blacksquare}$}
            {\color{530}$\overset{\text{\tiny 530}}{\blacksquare}$}
            {\color{310}$\overset{\text{\tiny 310}}{\blacksquare}$}
            \\
            {\color{F0D}$\overset{\text{\tiny F0D}}{\blacksquare}$}
            {\color{D0B}$\overset{\text{\tiny D0B}}{\blacksquare}$}
            {\color{B09}$\overset{\text{\tiny B09}}{\blacksquare}$}
            {\color{907}$\overset{\text{\tiny 907}}{\blacksquare}$}
            {\color{705}$\overset{\text{\tiny 705}}{\blacksquare}$}
            {\color{503}$\overset{\text{\tiny 503}}{\blacksquare}$}
            {\color{301}$\overset{\text{\tiny 301}}{\blacksquare}$}
            \\
            {\color{D0F}$\overset{\text{\tiny D0F}}{\blacksquare}$}
            {\color{B0D}$\overset{\text{\tiny B0D}}{\blacksquare}$}
            {\color{90B}$\overset{\text{\tiny 90B}}{\blacksquare}$}
            {\color{709}$\overset{\text{\tiny 709}}{\blacksquare}$}
            {\color{507}$\overset{\text{\tiny 507}}{\blacksquare}$}
            {\color{305}$\overset{\text{\tiny 305}}{\blacksquare}$}
            {\color{103}$\overset{\text{\tiny 103}}{\blacksquare}$}
    }}

%%%%%%%%%%%%%%%%%%%%%%%%%%%%%%%%%%%%%%%%%%%%%%%%%%%%%%%%%%%%%%%%%%%%%%%%%%%%%%%%

% Configurações do pacote listings para adição de código no documento

\usepackage{listings}%Configura layout para mostrar codigos a partir de arquivo
    \AtBeginDocument{\renewcommand{\lstlistingname}{Código}}


    \lstdefinelanguage{JavaScript}{
        keywords={typeof, new, true, false, catch, function, return, null, catch, switch, var, if, in, while, do, else, case, break},
        ndkeywords={class, export, boolean, throw, implements, import, this},
        % ndkeywordstyle=\color{darkgray}\bfseries,
        identifierstyle=\color{black},
        sensitive=true,
        comment=[l]{//},
        morecomment=[s]{/*}{*/},
        morestring=[b]',
        morestring=[b]"
    }

    \lstset{% Configurando layout para mostrar códigos C++
        language=[11]C++,
        basicstyle=\ttfamily\scriptsize,
        backgroundcolor=\color{backcode!5},
        stringstyle=\color{strs},
        commentstyle=\color{coments},
        keywordstyle=[1]\itshape\color{079},
        keywordstyle=[2]\color{907},
        keywordstyle=[3]\color{097},
        keywordstyle=[4]\bfseries\color{790},
        keywordstyle=[5]\color{709},
        keywordstyle=[6]\color{970},
        morekeywords=[1]{byte},
        morekeywords=[2]{},
        morekeywords=[3]{uint8_t, size_t, type},
        morekeywords=[4]{},
        numbers=left,
        numberstyle=\ttfamily\tiny,
        escapeinside={§}{§},
        tabsize=2,
        extendedchars=true,
        showspaces=false,
        showstringspaces=false,
        numberbychapter=false,
        emptylines=1,
        frame=L,
        firstnumber=auto,
        breaklines=true,
        breakautoindent=true,
        captionpos=t,
        float=htbp,
        xleftmargin=2em,
        inputencoding=utf8,
        %texcl=true,
        upquote=true,
        literate=%
            {á}{{\'a}}1 {à}{{\`a}}1 {ä}{{\"a}}1 {â}{{\^a}}1 {ã}{{\~a}}1 {å}{{\r{a}}}1
            {Á}{{\'A}}1 {À}{{\`A}}1 {Ä}{{\"A}}1 {Â}{{\^A}}1 {Ã}{{\~A}}1 {Å}{{\r{A}}}1
            {é}{{\'e}}1 {è}{{\`e}}1 {ë}{{\"e}}1 {ê}{{\^e}}1 {ẽ}{{\~e}}1
            {É}{{\'E}}1 {È}{{\`E}}1 {Ë}{{\"E}}1 {Ê}{{\^E}}1 {Ẽ}{{\~E}}1
            {í}{{\'i}}1 {ì}{{\`i}}1 {ï}{{\"i}}1 {î}{{\^i}}1 {ĩ}{{\~i}}1
            {Í}{{\'I}}1 {Ì}{{\`I}}1 {Ï}{{\"I}}1 {Î}{{\^I}}1 {Ĩ}{{\~I}}1
            {ó}{{\'o}}1 {ò}{{\`o}}1 {ö}{{\"o}}1 {ô}{{\^o}}1 {õ}{{\~o}}1 {ő}{{\H{o}}}1
            {Ó}{{\'O}}1 {Ò}{{\`O}}1 {Ö}{{\"O}}1 {Ô}{{\^O}}1 {Õ}{{\~O}}1 {Ő}{{\H{O}}}1
            {ú}{{\'u}}1 {ù}{{\`u}}1 {ü}{{\"u}}1 {û}{{\^u}}1 {ũ}{{\~u}}1 {ű}{{\H{u}}}1
            {Ú}{{\'U}}1 {Ù}{{\`U}}1 {Ü}{{\"U}}1 {Û}{{\^U}}1 {Ũ}{{\~U}}1 {Ű}{{\H{U}}}1
            {œ}{{\oe}}1 {Œ}{{\OE}}1 {æ}{{\ae}}1 {Æ}{{\AE}}1 {ß}{{\ss}}1
            {ç}{{\c{c}}}1 {Ç}{{\c{C}}}1
            {ñ}{{\~n}}1 {Ñ}{{\~N}}1
            {ø}{{\o}}1 {Ø}{{\O}}1
            {⁰}{{\textsuperscript{0}}}1
            {¹}{{\textsuperscript{1}}}1
            {²}{{\textsuperscript{2}}}1
            {³}{{\textsuperscript{3}}}1
            {⁴}{{\textsuperscript{4}}}1
            {⁵}{{\textsuperscript{5}}}1
            {⁶}{{\textsuperscript{6}}}1
            {⁷}{{\textsuperscript{7}}}1
            {⁸}{{\textsuperscript{8}}}1
            {⁹}{{\textsuperscript{9}}}1
            {°}{{\textdegree}}1
            {€}{{\euro}}1 {£}{{\pounds}}1
            {«}{{\guillemotleft}}1 {»}{{\guillemotright}}1
            {¿}{{?`}}1 {¡}{{!`}}1
    }

    \newcommand{\coda}[1]{{\color{057}\lstinline|#1|}}
    \newcommand{\code}[1]{{\color{075}\lstinline|#1|}}
    \newcommand{\codi}[1]{{\color{570}\lstinline|#1|}}
    \newcommand{\codo}[1]{{\color{750}\lstinline|#1|}}
    \newcommand{\codu}[1]{{\color{705}\lstinline|#1|}}
    \newcommand{\codw}[1]{{\color{507}\lstinline|#1|}}
    \newcommand{\codGuide}{
        \begin{center}
            \large{\coda{A}\\\code{E}\\\codi{I}\\\codo{O}\\\codu{U}\\\codw{W}

            \showcolors}
        \end{center}
    }

%%%%%%%%%%%%%%%%%%%%%%%%%%%%%%%%%%%%%%%%%%%%%%%%%%%%%%%%%%%%%%%%%%%%%%%%%%%%%%%%

% Adição de caracteres

\usepackage[euler]{textgreek} % Caracteres gregos

\usepackage{pmboxdraw} % Caracteres unicode

%%%%%%%%%%%%%%%%%%%%%%%%%%%%%%%%%%%%%%%%%%%%%%%%%%%%%%%%%%%%%%%%%%%%%%%%%%%%%%%%

% Pacotes de opções matemáticas

\usepackage{amsmath, amssymb, xfrac, cancel} % símbolos matemáticos

\usepackage{siunitx} % Comando \SI para unidades de medida
    \sisetup{locale = FR} % Utilizar virgular para marcação decimal
    \sisetup{separate-uncertainty = true}
    \sisetup{exponent-product=\ensuremath{\cdot}}
    \sisetup{separate-uncertainty=true}
    \sisetup{multi-part-units=single}
    \sisetup{group-separator = {}}
    \sisetup{detect-all}
    \newcommand{\infinity}{\ensuremath{\infty}}
    \robustify{\infinity}
    % \DeclareCommandCopy{\infty}{\infinity}
    \sisetup{
        input-digits = 0123456789\infty\infinity
    }

    \DeclareSIUnit{\nothing}{{\relax}}
    \DeclareSIUnit{\var}{VAR}
    \DeclareSIUnit{\va}{VA}
    \DeclareSIUnit{\dBm}{dBm}
    \DeclareSIUnit{\pixel}{px} % Pixel

\usepackage{gensymb} % Símbolos de unidades de medida

\usepackage{mathtools}

    \DeclareFontFamily{U}{mathc}{}
    \DeclareFontShape{U}{mathc}{m}{it}%
    {<->s*[1.03] mathc10}{}
    \DeclareMathAlphabet{\mathcal}{U}{mathc}{m}{it}

    \DeclareMathOperator{\sHom}{\mathcal{H\mkern-3mu om}}
    \DeclareMathOperator{\sExt}{\mathcal{E\mkern-3mu xt}}
    \DeclareMathOperator{\sEnd}{\mathcal{E\mkern-3mu nd}}

\usepackage{steinmetz} % Números complexos

%%%%%%%%%%%%%%%%%%%%%%%%%%%%%%%%%%%%%%%%%%%%%%%%%%%%%%%%%%%%%%%%%%%%%%%%%%%%%%%%

% Pacotes de opções químicas

\usepackage{chemformula}
\usepackage[version=3]{mhchem}

%%%%%%%%%%%%%%%%%%%%%%%%%%%%%%%%%%%%%%%%%%%%%%%%%%%%%%%%%%%%%%%%%%%%%%%%%%%%%%%%

% Formatação de tabelas

\usepackage{tabularx} % Tabelas do tipo tabularx
\usepackage{booktabs} % Formatação de tabelas como em livro
\usepackage{makecell} % formatação avançada para tabelas
\usepackage{multicol} % Texto em colunas na folha

%%%%%%%%%%%%%%%%%%%%%%%%%%%%%%%%%%%%%%%%%%%%%%%%%%%%%%%%%%%%%%%%%%%%%%%%%%%%%%%%

% Pacotes para trabalhar com figuras

\usepackage{graphicx} % Adição de imagens

\usepackage{svg} % Adição de imagens no formato SVG

\usepackage[outdir=./]{epstopdf} % Figuras em EPS convertidas em PDF

\usepackage{pdfpages} % Adição de PDFs como páginas

\usepackage[angle=0, text={}]{draftwatermark} % Adição de marca d'água
% \usepackage[printwatermark]{xwatermark}

\usepackage{tikz} % Desenhos
    \usetikzlibrary{through}
    \usetikzlibrary{shapes}
    \usetikzlibrary{shapes.geometric}
    \usetikzlibrary{trees}
    \usetikzlibrary{fit}
    \usetikzlibrary{patterns}
    \usetikzlibrary{calc}
    \usetikzlibrary{arrows}
    \usetikzlibrary{decorations}
    \usetikzlibrary{decorations.pathmorphing}
    \usetikzlibrary{positioning}

\usepackage{pgfplots} % Desenho de gráficos
    \pgfdeclarelayer{background}    % declare background layer
    \pgfdeclarelayer{foreground}    % declare foreground layer
    \pgfsetlayers{background,main,foreground}  % set the order of the layers (main is the standard layer)
    \pgfplotsset{compat=1.14}
    \usepgfplotslibrary{fillbetween}

\usepackage{pgf-pie} % Gráficos pizza

\usepackage[RPvoltages]{circuitikz} % Desenhos de circuitos
    \ctikzset{bipoles/thickness=1}

\usepackage{pgfplotstable}

    \pgfplotstableset{% global config, for example in the preamble
        assign column name/.style={
            /pgfplots/table/column name={\textbf{#1}} % Primeira linha em negrito
        },
        every first column/.style={
            column type/.add={l}{} % Primeira coluna alinhada a esquerda
        },
        string type, % A entrada é textual
        col sep=tab, % O arquivo é separado por tabs
        every head row/.style={before row=\toprule,after row=\midrule}, % Definições de linhas horizintais do cabeçalho
        every last row/.style={after row=\bottomrule}, % Definições de linhas horizontais do final
    }

\usepackage{chemfig} % Desenho de moléculas

%%%%%%%%%%%%%%%%%%%%%%%%%%%%%%%%%%%%%%%%%%%%%%%%%%%%%%%%%%%%%%%%%%%%%%%%%%%%%%%%

% Utilitários adicionais para lidar com figuras e tabelas

% \usepackage{float} % posicionamento espacial

\usepackage{caption} % Comando \caption*

\usepackage{subcaption} % Opções de subfiguras

%%%%%%%%%%%%%%%%%%%%%%%%%%%%%%%%%%%%%%%%%%%%%%%%%%%%%%%%%%%%%%%%%%%%%%%%%%%%%%%%

% Auxiliares gerais

\usepackage{lipsum} % Lorem ipsum
% \usepackage{rotating}
\usepackage{ifdraft} % Opções adicionais para o modo draft

\usepackage{csvsimple} % Carregar arquivos para o doc

\newcommand{\expandItemsListDat}[1]{ % Expandir items de arquivo .dat
    \csvloop{
        file = {#1},
        no head,
        before line = \item,
        % after line =;
    }}

\newcommand{\expandItemsListDatAspas}[1]{ % Expandir items de arquivo .dat
    \csvloop{
        file = {#1},
        no head,
        before line ={\item``},
        after line ={''}
    }}

%%%%%%%%%%%%%%%%%%%%%%%%%%%%%%%%%%%%%%%%%%%%%%%%%%%%%%%%%%%%%%%%%%%%%%%%%%%%%%%%

% Referência cruzada e links

% \usepackage[hidelinks]{hyperref} % Links no documento
% O pacote hyperref já é carregado pelo beamer

%%%%%%%%%%%%%%%%%%%%%%%%%%%%%%%%%%%%%%%%%%%%%%%%%%%%%%%%%%%%%%%%%%%%%%%%%%%%%%%%

% controle de citação e referências bibliográfica

\usepackage[
    backend=biber,
    style=ieee,
    citestyle=numeric,
    sorting=none,
    block=space
]{biblatex}

%%%%%%%%%%%%%%%%%%%%%%%%%%%%%%%%%%%%%%%%%%%%%%%%%%%%%%%%%%%%%%%%%%%%%%%%%%%%%%%%

% Definições de macros especiais

% Exibir ou ocultar o logo
    \newif\ifmostrarlogo
    \mostrarlogofalse

    \newcommand{\localLogo}{template/logo_ufabc.png}

    \logo{\ifmostrarlogo\includegraphics[width=0.1\textwidth]{\localLogo}\vspace{0.826\textheight}\fi}

% Ocultar slides da navegação
\makeatletter
    \let\beamer@writeslidentry@miniframeson=\beamer@writeslidentry
    \def\beamer@writeslidentry@miniframesoff{%
      \expandafter\beamer@ifempty\expandafter{\beamer@framestartpage}{}% does not happen normally
      {%else
        % removed \addtocontents commands
        \clearpage\beamer@notesactions%
      }
    }
    \newcommand*{\miniframeson}{\let\beamer@writeslidentry=\beamer@writeslidentry@miniframeson}
    \newcommand*{\miniframesoff}{\let\beamer@writeslidentry=\beamer@writeslidentry@miniframesoff}
\makeatother

    \newcommand{\etal}{\emph{et al}.}
    \newcommand{\ie}{\emph{i}.\emph{e}.}
    \newcommand{\eg}{\emph{e}.\emph{g}.}

\newcommand{\nomes}{}
\newcommand{\grupo}{}
\newcommand{\centro}{}
\newcommand{\centroSigla}{}
\newcommand{\disciplina}{}
\newcommand{\codigoDisciplina}{}
\newcommand{\titulo}{}
\newcommand{\professor}{}
\newcommand{\local}{}
\newcommand{\data}{\number\year}
\newcommand{\notaDeRosto}{}
\newcommand{\agradecimentos}{}

%%%%%%%%%%%%%%%%%%%%%%%%%%%%%%%%%%%%%%%%%%%%%%%%%%%%%%%%%%%%%%%%%%%%%%%%%%%%%%%%


% Configurações de estilos de páginas

% Selecionando tema de apresentação
    \usetheme{Dresden}

% Ajustando as cores do template
    % \usecolortheme[named=UFABCRDblue]{structure} % Cor principal

    % \setbeamercolor{palette primary}{bg=slideCyan,fg=white}
    % \setbeamercolor{palette secondary}{bg=slideTurquoise,fg=white}
    % \setbeamercolor{palette tertiary}{bg=slideGreen,fg=white}
    % \setbeamercolor{palette quaternary}{bg=slideYellow,fg=white}
    % \setbeamercolor{structure}{fg=UFABCRDblue} % itemize, enumerate, etc
    % \setbeamercolor{section in toc}{fg=UFABCRDblue} % TOC sections
    % \setbeamercolor{block title}{bg=UFABCRDblue,fg=white}
    % \setbeamercolor{block body}{bg=UFABCRDblue!10,fg=black}
    % \setbeamercolor{block title alerted}{bg=slideRed,fg=white}
    % \setbeamercolor{block body alerted}{bg=slideRed!10,fg=black}
    % \setbeamercolor{block title example}{bg=slideGreen,fg=white}
    % \setbeamercolor{block body example}{bg=slideGreen!10,fg=black}

% Remover o menu de navegação semi transparente inferior
    \setbeamertemplate{navigation symbols}{}

% Outras definições do template
    % \institute{UFABC Rocket Design}
    \titlegraphic{\includegraphics[height=0.2\textheight]{\localLogo}}


    % Exibir número no cando tos slides
    \newcommand{\frameofframes}{/}
    \newcommand{\setframeofframes}[1]{\renewcommand{\frameofframes}{#1}}

    \setframeofframes{de}
    \makeatletter
    \setbeamertemplate{footline}
      {%
        \begin{beamercolorbox}[colsep=1.5pt]{upper separation line foot}
        \end{beamercolorbox}
        \begin{beamercolorbox}[ht=2.5ex,dp=1.125ex,%
          leftskip=.3cm,rightskip=.3cm plus1fil]{author in head/foot}%
          \leavevmode{\usebeamerfont{author in head/foot}\insertshortauthor}%
          \hfill%
          {\usebeamerfont{institute in head/foot}\usebeamercolor[fg]{institute in head/foot}\insertshortinstitute}%
        \end{beamercolorbox}%
        \begin{beamercolorbox}[ht=2.5ex,dp=1.125ex,%
          leftskip=.3cm,rightskip=.3cm plus1fil]{title in head/foot}%
          {\usebeamerfont{title in head/foot}\insertshorttitle}%
          \hfill%
          {\usebeamerfont{frame number}\usebeamercolor[fg]{frame number}\insertframenumber~\frameofframes~\inserttotalframenumber}
        \end{beamercolorbox}%
        \begin{beamercolorbox}[colsep=1.5pt]{lower separation line foot}
        \end{beamercolorbox}
      }
    \makeatother

    % Comando auxiliar para exibir cores do tema
    \newcommand{\coresDoTema}{
        \begin{frame}{Beamer-Color}
            \scriptsize
            \begin{multicols}{3}
                \begin{itemize}
                    \item {\usebeamercolor[fg]{normal text} $\blacksquare$ normal text }\\
                    \item {\usebeamercolor[bg]{palette primary} $\blacksquare$ palette primary }\\
                    \item {\usebeamercolor[bg]{palette secondary} $\blacksquare$ palette secondary }\\
                    \item {\usebeamercolor[bg]{palette tertiary} $\blacksquare$ palette tertiary }\\
                    \item {\usebeamercolor[bg]{palette quaternary} $\blacksquare$ palette quaternary }\\
                    \item {\usebeamercolor[fg]{palette sidebar primary} $\blacksquare$ palette sidebar primary }\\
                    \item {\usebeamercolor[fg]{palette sidebar secondary} $\blacksquare$ palette sidebar secondary }\\
                    \item {\usebeamercolor[fg]{palette sidebar tertiary} $\blacksquare$ palette sidebar tertiary }\\
                    \item {\usebeamercolor[fg]{palette sidebar quaternary} $\blacksquare$ palette sidebar quaternary }\\
                    \item {\usebeamercolor[fg]{section title} $\blacksquare$ section title }\\
                    \item {\usebeamercolor[fg]{section name} $\blacksquare$ section name }\\
                    \item {\usebeamercolor[fg]{subsection title} $\blacksquare$ subsection title }\\
                    \item {\usebeamercolor[fg]{subsection name} $\blacksquare$ subsection name }\\
                    \item {\usebeamercolor[fg]{part title} $\blacksquare$ part title }\\
                    \item {\usebeamercolor[fg]{part name} $\blacksquare$ part name }\\
                    \item {\usebeamercolor[fg]{title} $\blacksquare$ title }\\
                    \item {\usebeamercolor[fg]{block title} $\blacksquare$ block title }\\
                    \item {\usebeamercolor[fg]{block body} $\blacksquare$ block body }\\
                    \item {\usebeamercolor[fg]{block title alerted} $\blacksquare$ block title alerted }\\
                    \item {\usebeamercolor[fg]{block body alerted} $\blacksquare$ block body alerted }\\
                    \item {\usebeamercolor[fg]{block title example} $\blacksquare$ block title example }\\
                    \item {\usebeamercolor[fg]{block body example} $\blacksquare$ block body example }\\
                    \item {\usebeamercolor[fg]{structure} $\blacksquare$ structure }\\
                    \item {\usebeamercolor[fg]{local structure} $\blacksquare$ local structure }\\
                    \item {\usebeamercolor[fg]{tiny structure} $\blacksquare$ tiny structure }\\
                    \item {\usebeamercolor[fg]{item} $\blacksquare$ item }\\
                    \item {\usebeamercolor[bg]{item projected} $\blacksquare$ item projected }\\
                    \item {\usebeamercolor[fg]{subitem} $\blacksquare$ subitem }\\
                    \item {\usebeamercolor[bg]{subitem projected} $\blacksquare$ subitem projected }\\
                    \item {\usebeamercolor[fg]{example text} $\blacksquare$ example text }\\
                    \item {\usebeamercolor[fg]{titlelike} $\blacksquare$ titlelike }\\
                    \item {\usebeamercolor[fg]{separation line} $\blacksquare$ separation line }\\
                    \item {\usebeamercolor[fg]{upper separation line head} $\blacksquare$ upper separation line head}\\
                    \item {\usebeamercolor[fg]{middle separation line head} $\blacksquare$ middle separation line head}\\
                    \item {\usebeamercolor[fg]{lower separation line head} $\blacksquare$ lower separation line head}\\
                    \item {\usebeamercolor[fg]{upper separation line foot} $\blacksquare$ upper separation line foot}\\
                    \item {\usebeamercolor[fg]{middle separation line foot} $\blacksquare$ middle separation line foot}\\
                    \item {\usebeamercolor[fg]{lower separation line foot} $\blacksquare$ lower separation line foot}\\
                    \item {\usebeamercolor[fg]{footnote} $\blacksquare$ footnote}\\
                    \item {\usebeamercolor[fg]{footnote mark} $\blacksquare$ footnote mark}\\
                \end{itemize}
            \end{multicols}
        \end{frame}
    }
