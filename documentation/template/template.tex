%%%%%%%%%%%%%%%%%%%%%%%%%%%%%%%%%%%%%%%%%%%%%%%%%%%%%%%%%%%%%%%%%%%%%%%%%%%%%%%%
% TEMPLATE PARA DOCUMENTOS
% UFABC Rocket Design
%
% Para atualiza sua versão, acesse o link abaixo e depois volte aqui para atualizar
%
% https://www.overleaf.com/read/xbbzdqfgqmvp#ee6791
%
% Lista de revisões: (data ISO - nome)
%
% 2020-12-17 - Heitor Rodrigues Savegnago
% 2021-01-17 - Heitor Rodrigues Savegnago
% 2021-01-23 - Heitor Rodrigues Savegnago
% 2021-04-21 - Heitor Rodrigues Savegnago
% 2021-12-23 - Heitor Rodrigues Savegnago
% 2022-02-02 - Heitor Rodrigues Savegnago
% 2022-04-19 - Heitor Rodrigues Savegnago
% 2022-11-02 - Heitor Rodrigues Savegnago
% 2023-05-24 - Heitor Rodrigues Savegnago
% 2023-08-06 - Heitor Rodrigues Savegnago
% 2023-08-18 - Heitor Rodrigues Savegnago
% 2023-09-06 - Heitor Rodrigues Savegnago
% 2023-09-27 - Heitor Rodrigues Savegnago
% 2023-11-29 - Heitor Rodrigues Savegnago
% 2024-05-13 - Heitor Rodrigues Savegnago
% 2024-07-28 - Heitor Rodrigues Savegnago
% 2024-08-14 - Heitor Rodrigues Savegnago
% 2024-12-12 - Heitor Rodrigues Savegnago
%
%%%%%%%%%%%%%%%%%%%%%%%%%%%%%%%%%%%%%%%%%%%%%%%%%%%%%%%%%%%%%%%%%%%%%%%%%%%%%%%%

% Formatação geral do documento

    % Configuração de margens
    \ifx\standalone\undefined
        \usepackage[
            a4paper,
            twoside,
            top = 3cm,
            bottom = 2cm,
            left = 3cm,
            right = 2cm
        ]{geometry} % Margens do documento
    \fi

    % Fonte
        %\renewcommand{\rmdefault}{phv} % Arial
        \usepackage[lighttt]{lmodern} % Fonte Latin Modern
            % \renewcommand{\familydefault}{\sfdefault} % Estilo global Sans Serif

        % Subistituição de fonte para casos de erro
        \DeclareFontFamilySubstitution{TS1}{aer}{lmr}
        \DeclareFontFamilySubstitution{TS1}{aett}{lmtt}
        \DeclareFontFamilySubstitution{TS1}{aess}{lmss}

    % Espaçamento entre linhas
        \usepackage{setspace} % Espaçamento do texto
            \renewcommand{\baselinestretch}{1.15}

    % Correções de espaçamentos
        \usepackage{indentfirst} % Parágrafos indentados
            \setlength{\parindent}{1cm} % Define espaço de paragrafo como 1cm

        \raggedbottom % Corrige espaçamento de paragrafo

        \usepackage{microtype} % Ajusta detalhes menores nos espaçamentos da página para ficar mais agradável

    % Decorações da página
        \usepackage{fancyhdr} % Cabeçalhos e rodapés em páginas

    % Formatação de títulos
    	\newcommand{\titlefont}{\fontsize{18}{20}}
    	\newcommand{\sectionfont}{\fontsize{16}{20}}
    	\newcommand{\subsectionfont}{\fontsize{14}{20}}
    	\newcommand{\subsubsectionfont}{\fontsize{12}{20}}

        \usepackage{titlesec} % Configurações para seção
            \ifx\article\undefined
                \titleformat{\chapter}{\normalfont \titlefont \bfseries}{\thechapter}{1em}{}
                \titlespacing*{\chapter}{0pt}{3.5ex plus 1ex minus .2ex}{2.3ex plus .2ex}
            \fi
            \titleformat{\section}{\normalfont \sectionfont \bfseries}{\thesection}{1em}{}
            \titleformat{\subsection}{\normalfont \subsectionfont \bfseries}{\thesubsection}{1em}{}
            \titleformat{\subsubsection}{\normalfont \subsubsectionfont \bfseries}{\thesubsubsection}{1em}{}

        \usepackage{titletoc} % Configurações para sumário e listas de figuras e tabelas

            % Essas configurações estão modeladas para alinhar todas as numerações
            % à esquerda e títulos também alinhados à esquerda em outra margem

            % Configurações para sumário estilo ABNT
            \titlecontents{chapter} % Qual nível se refere
                [3.5em] % Espaço à esquerda, entre margem e título
                {\bigskip} % A cima da linha (geralmente espaçamento vertical)
                {\normalfont\normalsize\contentslabel[\bfseries\thecontentslabel]{3.5em}\bfseries\uppercase} % Formato da linha com numeração
                {\hspace*{-3.5em}\bfseries\uppercase} % Formato da linha sem numeração
                {\bfseries\dotfill\contentspage} % Formato do preenchimento entre título e número da página

            \titlecontents{section} % Qual nível se refere
                [3.5em] % Espaço à esquerda, entre margem e título
                {\smallskip} % A cima da linha (geralmente espaçamento)
                {\normalfont\normalsize\contentslabel[\bfseries\thecontentslabel]{3.5em}\bfseries} % Formato da linha com numeração
                {\hspace*{-3.5em}\bfseries} % Formato da linha sem numeração
                {\bfseries\dotfill\contentspage} % Formato do preenchimento entre título e número da página

            \dottedcontents{subsection}[3.5em]{}{3.5em}{0.44em}
            \dottedcontents{subsubsection}[3.5em]{}{3.5em}{0.44em}

            % Configuração para listas de figuras e tabelas
            \dottedcontents{figure}[2em]{\smallskip}{2em}{0.44em}
            \dottedcontents{table}[2em]{\smallskip}{2em}{0.44em}

    % Alteração de limite de níveis no sumário

        \setcounter{tocdepth}{4} % Níveis exibidos no sumário
        \setcounter{secnumdepth}{4} % Nível de números exibidos

    % Opções para tópicos de enumerate

        \usepackage[shortlabels]{enumitem} % Selecionar formato em itemize

%%%%%%%%%%%%%%%%%%%%%%%%%%%%%%%%%%%%%%%%%%%%%%%%%%%%%%%%%%%%%%%%%%%%%%%%%%%%%%%%

% Codificação de carecteres de entrada

    \usepackage{ae} % "Almost European"
    \usepackage[T1]{fontenc} % Caracteres especiais
    \usepackage[utf8]{inputenc} % Caracteres especiais

    \usepackage{fontawesome} % Símbolos especiais

%%%%%%%%%%%%%%%%%%%%%%%%%%%%%%%%%%%%%%%%%%%%%%%%%%%%%%%%%%%%%%%%%%%%%%%%%%%%%%%%

% Configuração de linguagem padrão do documento

    \usepackage[english, main=brazil]{babel} % Detalhes automáticos em Português
        % \selectlanguage{brazil}

        \AtBeginDocument{\renewcommand{\contentsname}{\centerline{Sumário}}}
        \AtBeginDocument{\renewcommand{\bibname}{Referências Bibliográficas}}
        \AtBeginDocument{\renewcommand{\listfigurename}{\centerline{Figuras}}}
        \AtBeginDocument{\renewcommand{\listtablename}{\centerline{Tabelas}}}
        \AtBeginDocument{\renewcommand{\lstlistlistingname}{\centerline{Códigos}}}
    % 	\AtBeginDocument{\renewcommand{\figurename}{Figura}}
    % 	\AtBeginDocument{\renewcommand{\tablename}{Tabela}}

    \usepackage{textcomp} % Suporte de caracteres especiais

    \usepackage{csquotes} % Opções de citação

%%%%%%%%%%%%%%%%%%%%%%%%%%%%%%%%%%%%%%%%%%%%%%%%%%%%%%%%%%%%%%%%%%%%%%%%%%%%%%%%

% Configurações de cores

    \usepackage{transparent} % Opções de transparência

    \usepackage[svgnames]{xcolor} % Define cores

        \definecolor{0DF}{HTML}{00DDFF}%
        \definecolor{0FD}{HTML}{00FFDD}%
        \definecolor{DF0}{HTML}{DDFF00}%
        \definecolor{FD0}{HTML}{FFDD00}%
        \definecolor{F0D}{HTML}{FF00DD}%
        \definecolor{D0F}{HTML}{DD00FF}%

        \definecolor{0BD}{HTML}{00BBDD}%
        \definecolor{0DB}{HTML}{00DDBB}%
        \definecolor{BD0}{HTML}{BBDD00}%
        \definecolor{DB0}{HTML}{DDBB00}%
        \definecolor{D0B}{HTML}{DD00BB}%
        \definecolor{B0D}{HTML}{BB00DD}%

        \definecolor{09B}{HTML}{0099BB}%
        \definecolor{0B9}{HTML}{00BB99}%
        \definecolor{9B0}{HTML}{99BB00}%
        \definecolor{B90}{HTML}{BB9900}%
        \definecolor{B09}{HTML}{BB0099}%
        \definecolor{90B}{HTML}{9900BB}%

        \definecolor{079}{HTML}{007799}%
        \definecolor{097}{HTML}{009977}%
        \definecolor{790}{HTML}{779900}%
        \definecolor{970}{HTML}{997700}%
        \definecolor{907}{HTML}{990077}%
        \definecolor{709}{HTML}{770099}%

        \definecolor{057}{HTML}{005577}%
        \definecolor{075}{HTML}{007755}%
        \definecolor{570}{HTML}{557700}%
        \definecolor{750}{HTML}{775500}%
        \definecolor{705}{HTML}{770055}%
        \definecolor{507}{HTML}{550077}%

        \definecolor{035}{HTML}{003355}%
        \definecolor{053}{HTML}{005533}%
        \definecolor{350}{HTML}{335500}%
        \definecolor{530}{HTML}{553300}%
        \definecolor{503}{HTML}{550033}%
        \definecolor{305}{HTML}{330055}%

        \definecolor{013}{HTML}{001133}%
        \definecolor{031}{HTML}{003311}%
        \definecolor{130}{HTML}{113300}%
        \definecolor{310}{HTML}{331100}%
        \definecolor{301}{HTML}{330011}%
        \definecolor{103}{HTML}{110033}%

        \definecolor{rgb_R}{rgb}{1,0,0}%
        \definecolor{rgb_G}{rgb}{0,1,0}%
        \definecolor{rgb_B}{rgb}{0,0,1}%
        \definecolor{rgb_M}{rgb}{1,0,1}%
        \definecolor{rgb_Y}{rgb}{1,1,0}%
        \definecolor{rgb_C}{rgb}{0,1,1}%
        \definecolor{rgb_W}{rgb}{1,1,1}%
        \definecolor{rgb_K}{rgb}{0,0,0}%

        \definecolor{cmyk_C}{cmyk}{1,0,0,0}%
        \definecolor{cmyk_M}{cmyk}{0,1,0,0}%
        \definecolor{cmyk_Y}{cmyk}{0,0,1,0}%
        \definecolor{cmyk_G}{cmyk}{1,0,1,0}%
        \definecolor{cmyk_B}{cmyk}{1,1,0,0}%
        \definecolor{cmyk_R}{cmyk}{0,1,1,0}%
        \definecolor{cmyk_K}{cmyk}{1,1,1,1}%
        \definecolor{cmyk_W}{cmyk}{0,0,0,0}%

        \definecolor{strs}			{rgb}	{0.9,	0.2,	0	}%
        \definecolor{coments}		{rgb}	{0,		0.5,	0	}%
        \definecolor{backcode}		{rgb}	{0.3,	0,		0.2	}%

        \newcommand{\MexerDepois}[1]{
            \vspace*{2em}
            {\huge\color{F0D}#1}
            \vspace*{2em}}
        \newcommand{\mexer}[1]{{\color{F0D}#1}}

        \newcommand{\showcolors}%Mostra tabela de cores
        {{\ttfamily
        		{\color{0DF}$\overset{\text{\tiny 0DF}}{\blacksquare}$}
        		{\color{0BD}$\overset{\text{\tiny 0BD}}{\blacksquare}$}
        		{\color{09B}$\overset{\text{\tiny 09B}}{\blacksquare}$}
        		{\color{079}$\overset{\text{\tiny 079}}{\blacksquare}$}
        		{\color{057}$\overset{\text{\tiny 057}}{\blacksquare}$}
        		{\color{035}$\overset{\text{\tiny 035}}{\blacksquare}$}
        		{\color{013}$\overset{\text{\tiny 013}}{\blacksquare}$}
        		\\
        		{\color{0FD}$\overset{\text{\tiny 0FD}}{\blacksquare}$}
        		{\color{0DB}$\overset{\text{\tiny 0DB}}{\blacksquare}$}
        		{\color{0B9}$\overset{\text{\tiny 0B9}}{\blacksquare}$}
        		{\color{097}$\overset{\text{\tiny 097}}{\blacksquare}$}
        		{\color{075}$\overset{\text{\tiny 075}}{\blacksquare}$}
        		{\color{053}$\overset{\text{\tiny 053}}{\blacksquare}$}
        		{\color{031}$\overset{\text{\tiny 031}}{\blacksquare}$}
        		\\
        		{\color{DF0}$\overset{\text{\tiny DF0}}{\blacksquare}$}
        		{\color{BD0}$\overset{\text{\tiny BD0}}{\blacksquare}$}
        		{\color{9B0}$\overset{\text{\tiny 9B0}}{\blacksquare}$}
        		{\color{790}$\overset{\text{\tiny 790}}{\blacksquare}$}
        		{\color{570}$\overset{\text{\tiny 570}}{\blacksquare}$}
        		{\color{350}$\overset{\text{\tiny 350}}{\blacksquare}$}
        		{\color{130}$\overset{\text{\tiny 130}}{\blacksquare}$}
        		\\
        		{\color{FD0}$\overset{\text{\tiny FD0}}{\blacksquare}$}
        		{\color{DB0}$\overset{\text{\tiny DB0}}{\blacksquare}$}
        		{\color{B90}$\overset{\text{\tiny B90}}{\blacksquare}$}
        		{\color{970}$\overset{\text{\tiny 970}}{\blacksquare}$}
        		{\color{750}$\overset{\text{\tiny 750}}{\blacksquare}$}
        		{\color{530}$\overset{\text{\tiny 530}}{\blacksquare}$}
        		{\color{310}$\overset{\text{\tiny 310}}{\blacksquare}$}
        		\\
        		{\color{F0D}$\overset{\text{\tiny F0D}}{\blacksquare}$}
        		{\color{D0B}$\overset{\text{\tiny D0B}}{\blacksquare}$}
        		{\color{B09}$\overset{\text{\tiny B09}}{\blacksquare}$}
        		{\color{907}$\overset{\text{\tiny 907}}{\blacksquare}$}
        		{\color{705}$\overset{\text{\tiny 705}}{\blacksquare}$}
        		{\color{503}$\overset{\text{\tiny 503}}{\blacksquare}$}
        		{\color{301}$\overset{\text{\tiny 301}}{\blacksquare}$}
        		\\
        		{\color{D0F}$\overset{\text{\tiny D0F}}{\blacksquare}$}
        		{\color{B0D}$\overset{\text{\tiny B0D}}{\blacksquare}$}
        		{\color{90B}$\overset{\text{\tiny 90B}}{\blacksquare}$}
        		{\color{709}$\overset{\text{\tiny 709}}{\blacksquare}$}
        		{\color{507}$\overset{\text{\tiny 507}}{\blacksquare}$}
        		{\color{305}$\overset{\text{\tiny 305}}{\blacksquare}$}
        		{\color{103}$\overset{\text{\tiny 103}}{\blacksquare}$}
        }}

%%%%%%%%%%%%%%%%%%%%%%%%%%%%%%%%%%%%%%%%%%%%%%%%%%%%%%%%%%%%%%%%%%%%%%%%%%%%%%%%

% Configurações do pacote listings para adição de código no documento

    \usepackage{listings}%Configura layout para mostrar codigos a partir de arquivo
        \AtBeginDocument{\renewcommand{\lstlistingname}{Código}}


        \lstdefinelanguage{JavaScript}{
            keywords={typeof, new, true, false, catch, function, return, null, catch, switch, var, if, in, while, do, else, case, break},
            ndkeywords={class, export, boolean, throw, implements, import, this},
            % ndkeywordstyle=\color{darkgray}\bfseries,
            identifierstyle=\color{black},
            sensitive=true,
            comment=[l]{//},
            morecomment=[s]{/*}{*/},
            morestring=[b]',
            morestring=[b]"
        }

        \lstset{% Configurando layout para mostrar códigos C++
            language=[11]C++,
            basicstyle=\ttfamily\small\setstretch{1},
            backgroundcolor=\color{backcode!5},
            stringstyle=\color{strs},
            commentstyle=\color{coments},
            keywordstyle=[1]\itshape\color{079},
            keywordstyle=[2]\color{907},
            keywordstyle=[3]\color{097},
            keywordstyle=[4]\bfseries\color{790},
            keywordstyle=[5]\color{709},
            keywordstyle=[6]\color{970},
            morekeywords=[1]{byte},
            morekeywords=[2]{},
            morekeywords=[3]{uint8_t, size_t, type},
            morekeywords=[4]{},
            numbers=left,
            numberstyle=\tiny,
            escapeinside={§}{§},
            tabsize=2,
            extendedchars=true,
            showspaces=false,
            showstringspaces=false,
            numberbychapter=false,
            emptylines=1,
            frame=L,
            firstnumber=auto,
            breaklines=true,
            breakautoindent=true,
            captionpos=t,
            float=htbp,
            xleftmargin=2em,
            inputencoding=utf8,
            %texcl=true,
            upquote=true,
            literate=%
                {á}{{\'a}}1 {à}{{\`a}}1 {ä}{{\"a}}1 {â}{{\^a}}1 {ã}{{\~a}}1 {å}{{\r{a}}}1
                {Á}{{\'A}}1 {À}{{\`A}}1 {Ä}{{\"A}}1 {Â}{{\^A}}1 {Ã}{{\~A}}1 {Å}{{\r{A}}}1
                {é}{{\'e}}1 {è}{{\`e}}1 {ë}{{\"e}}1 {ê}{{\^e}}1 {ẽ}{{\~e}}1
                {É}{{\'E}}1 {È}{{\`E}}1 {Ë}{{\"E}}1 {Ê}{{\^E}}1 {Ẽ}{{\~E}}1
                {í}{{\'i}}1 {ì}{{\`i}}1 {ï}{{\"i}}1 {î}{{\^i}}1 {ĩ}{{\~i}}1
                {Í}{{\'I}}1 {Ì}{{\`I}}1 {Ï}{{\"I}}1 {Î}{{\^I}}1 {Ĩ}{{\~I}}1
                {ó}{{\'o}}1 {ò}{{\`o}}1 {ö}{{\"o}}1 {ô}{{\^o}}1 {õ}{{\~o}}1 {ő}{{\H{o}}}1
                {Ó}{{\'O}}1 {Ò}{{\`O}}1 {Ö}{{\"O}}1 {Ô}{{\^O}}1 {Õ}{{\~O}}1 {Ő}{{\H{O}}}1
                {ú}{{\'u}}1 {ù}{{\`u}}1 {ü}{{\"u}}1 {û}{{\^u}}1 {ũ}{{\~u}}1 {ű}{{\H{u}}}1
                {Ú}{{\'U}}1 {Ù}{{\`U}}1 {Ü}{{\"U}}1 {Û}{{\^U}}1 {Ũ}{{\~U}}1 {Ű}{{\H{U}}}1
                {œ}{{\oe}}1 {Œ}{{\OE}}1 {æ}{{\ae}}1 {Æ}{{\AE}}1 {ß}{{\ss}}1
                {ç}{{\c{c}}}1 {Ç}{{\c{C}}}1
                {ñ}{{\~n}}1 {Ñ}{{\~N}}1
                {ø}{{\o}}1 {Ø}{{\O}}1
                {⁰}{{\textsuperscript{0}}}1
                {¹}{{\textsuperscript{1}}}1
                {²}{{\textsuperscript{2}}}1
                {³}{{\textsuperscript{3}}}1
                {⁴}{{\textsuperscript{4}}}1
                {⁵}{{\textsuperscript{5}}}1
                {⁶}{{\textsuperscript{6}}}1
                {⁷}{{\textsuperscript{7}}}1
                {⁸}{{\textsuperscript{8}}}1
                {⁹}{{\textsuperscript{9}}}1
                {°}{{\textdegree}}1
                {€}{{\euro}}1 {£}{{\pounds}}1
                {«}{{\guillemotleft}}1 {»}{{\guillemotright}}1
                {¿}{{?`}}1 {¡}{{!`}}1
        }

        \newcommand{\coda}[1]{{\color{057}\lstinline|#1|}}
        \newcommand{\code}[1]{{\color{075}\lstinline|#1|}}
        \newcommand{\codi}[1]{{\color{570}\lstinline|#1|}}
        \newcommand{\codo}[1]{{\color{750}\lstinline|#1|}}
        \newcommand{\codu}[1]{{\color{705}\lstinline|#1|}}
        \newcommand{\codw}[1]{{\color{507}\lstinline|#1|}}
        \newcommand{\codGuide}{
            \begin{center}
                \large{\coda{A}\\\code{E}\\\codi{I}\\\codo{O}\\\codu{U}\\\codw{W}

                \showcolors}
            \end{center}
        }

    % Configurações adicinais para o pacote titletoc
        \contentsuse{lstlisting}{lol}
        \dottedcontents{lstlisting}[2em]{\smallskip}{2em}{0.44em}

%%%%%%%%%%%%%%%%%%%%%%%%%%%%%%%%%%%%%%%%%%%%%%%%%%%%%%%%%%%%%%%%%%%%%%%%%%%%%%%%

% Adição de caracteres

    \usepackage[euler]{textgreek} % Caracteres gregos

    \usepackage{pmboxdraw} % Caracteres unicode

%%%%%%%%%%%%%%%%%%%%%%%%%%%%%%%%%%%%%%%%%%%%%%%%%%%%%%%%%%%%%%%%%%%%%%%%%%%%%%%%

% Pacotes de opções matemáticas

    \usepackage{amsmath, amssymb, xfrac, cancel} % símbolos matemáticos

    \usepackage{siunitx} % Comando \SI para unidades de medida
        \sisetup{locale = FR} % Utilizar virgular para marcação decimal
        \sisetup{separate-uncertainty = true}
        \sisetup{exponent-product=\ensuremath{\cdot}}
        \sisetup{separate-uncertainty=true}
        \sisetup{multi-part-units=single}
        \sisetup{group-separator = {}}
        \sisetup{detect-all}

        \DeclareSIUnit{\nothing}{{\relax}}
        \DeclareSIUnit{\var}{VAR}
        \DeclareSIUnit{\va}{VA}
        \DeclareSIUnit{\dBm}{dBm}
        \DeclareSIUnit{\pixel}{px} % Pixel

    \usepackage{gensymb} % Símbolos de unidades de medida

    \usepackage{mathtools}

        \DeclareFontFamily{U}{mathc}{}
        \DeclareFontShape{U}{mathc}{m}{it}%
        {<->s*[1.03] mathc10}{}
        \DeclareMathAlphabet{\mathcal}{U}{mathc}{m}{it}

        \DeclareMathOperator{\sHom}{\mathcal{H\mkern-3mu om}}
        \DeclareMathOperator{\sExt}{\mathcal{E\mkern-3mu xt}}
        \DeclareMathOperator{\sEnd}{\mathcal{E\mkern-3mu nd}}

    \usepackage{steinmetz} % Números complexos

%%%%%%%%%%%%%%%%%%%%%%%%%%%%%%%%%%%%%%%%%%%%%%%%%%%%%%%%%%%%%%%%%%%%%%%%%%%%%%%%

% Pacotes de opções químicas

    \usepackage{chemformula}
    \usepackage[version=3]{mhchem}

%%%%%%%%%%%%%%%%%%%%%%%%%%%%%%%%%%%%%%%%%%%%%%%%%%%%%%%%%%%%%%%%%%%%%%%%%%%%%%%%

% Opções adicionais para formatação

    \usepackage[normalem]{ulem} % Sublinados diversos

    \usepackage{framed} % Criar caixas inteligentes

    \usepackage{footnote} % Notas de rodapé

        \makeatletter % Para referenciar notas de rodapé
            \newcommand\footnoteref[1]{\protected@xdef\@thefnmark{\ref{#1}}\@footnotemark}
        \makeatother

    \usepackage{lscape} % Página em paisagem

    \usepackage{datetime2} % Formatação de datas

%%%%%%%%%%%%%%%%%%%%%%%%%%%%%%%%%%%%%%%%%%%%%%%%%%%%%%%%%%%%%%%%%%%%%%%%%%%%%%%%

% Formatação de tabelas

    \usepackage{tabularx} % Tabelas do tipo tabularx
    \usepackage{longtable} % Tabelas com várias páginas
    \usepackage{booktabs} % Formatação de tabelas como em livro
        \renewcommand{\arraystretch}{1.25} % Espaçamento entre linhas interno em tabelas
        %\renewcommand{\cellgape}{\Gape[4pt]} % Espaçamento de tabelas
    \usepackage{makecell} % formatação avançada para tabelas
    \usepackage{multirow} % Merge em tabelas
    \usepackage{multicol} % Texto em colunas na folha
    \usepackage{arydshln} % Draw dash-lines in array/tabular
    \usepackage{colortbl} % Linhas, colunas e celulas coloridas

    \usepackage{array} % opções especiais para alinhamento de tabelas
        \newcolumntype{L}[1]{>{\raggedright\let\newline\\\arraybackslash\hspace{0pt}}m{#1}}
        \newcolumntype{C}[1]{>{\centering\let\newline\\\arraybackslash\hspace{0pt}}m{#1}}
        \newcolumntype{R}[1]{>{\raggedleft\let\newline\\\arraybackslash\hspace{0pt}}m{#1}}

%%%%%%%%%%%%%%%%%%%%%%%%%%%%%%%%%%%%%%%%%%%%%%%%%%%%%%%%%%%%%%%%%%%%%%%%%%%%%%%%

% Pacotes para trabalhar com figuras

    \usepackage{graphicx} % Adição de imagens

    \usepackage{svg} % Adição de imagens no formato SVG

    \usepackage[outdir=./]{epstopdf} % Figuras em EPS convertidas em PDF

    \usepackage{pdfpages} % Adição de PDFs como páginas

    \usepackage[angle=0, text={}]{draftwatermark} % Adição de marca d'água
    % \usepackage[printwatermark]{xwatermark}

    \usepackage{tikz} % Desenhos
        \usetikzlibrary{through}
        \usetikzlibrary{shapes}
        \usetikzlibrary{shapes.geometric}
        \usetikzlibrary{trees}
        \usetikzlibrary{fit}
        \usetikzlibrary{patterns}
        \usetikzlibrary{calc}
        \usetikzlibrary{arrows}
        \usetikzlibrary{decorations}
        \usetikzlibrary{decorations.pathmorphing}
        \usetikzlibrary{positioning}

    \usepackage{pgfplots} % Desenho de gráficos
        \pgfdeclarelayer{background}    % declare background layer
        \pgfdeclarelayer{foreground}    % declare foreground layer
        \pgfsetlayers{background,main,foreground}  % set the order of the layers (main is the standard layer)
        \pgfplotsset{compat=1.14}
        \usepgfplotslibrary{fillbetween}

    \usepackage{pgf-pie} % Gráficos pizza

    \usepackage[RPvoltages]{circuitikz} % Desenhos de circuitos
        \ctikzset{bipoles/thickness=1}

    \usepackage{pgfplotstable}

        \pgfplotstableset{% global config, for example in the preamble
            assign column name/.style={
                /pgfplots/table/column name={\textbf{#1}} % Primeira linha em negrito
            },
            every first column/.style={
                column type/.add={l}{} % Primeira coluna alinhada a esquerda
            },
            string type, % A entrada é textual
            col sep=tab, % O arquivo é separado por tabs
            every head row/.style={before row=\toprule,after row=\midrule}, % Definições de linhas horizintais do cabeçalho
            every last row/.style={after row=\bottomrule}, % Definições de linhas horizontais do final
        }

    \usepackage{chemfig} % Desenho de moléculas

%%%%%%%%%%%%%%%%%%%%%%%%%%%%%%%%%%%%%%%%%%%%%%%%%%%%%%%%%%%%%%%%%%%%%%%%%%%%%%%%

% Utilitários adicionais para lidar com figuras e tabelas

    \usepackage{float} % posicionamento espacial

        % Criando ambiente float para códigos
        \newfloat{lstfloat}{htbp}{lop}
        \floatname{lstfloat}{Código}
        \def\lstfloatautorefname{Código} % needed for hyperref/auroref

    \usepackage{caption} % Comando \caption*
    % \captionsetup{skip=0.5em}

    \usepackage{subcaption} % Opções de subfiguras

%%%%%%%%%%%%%%%%%%%%%%%%%%%%%%%%%%%%%%%%%%%%%%%%%%%%%%%%%%%%%%%%%%%%%%%%%%%%%%%%

% Auxiliares gerais

\usepackage{lipsum} % Lorem ipsum

\usepackage{ifdraft} % Opções adicionais para o modo draft

\usepackage{csvsimple} % Carregar arquivos para o doc

    \newcommand{\expandItemsListDat}[1]{ % Expandir items de arquivo .dat
        \csvloop{
            file = {#1},
            no head,
            before line = \item,
            % after line =;
        }}

    \newcommand{\expandItemsListDatAspas}[1]{ % Expandir items de arquivo .dat
        \csvloop{
            file = {#1},
            no head,
            before line ={\item``},
            after line ={''}
        }}

\usepackage{etoolbox} % Toolbox of programming facilities

    % Remover espaçamentos que dividem capitulos nas listas de figuras e tabelas
    \makeatletter
        \patchcmd{\@chapter}{\addtocontents{lof}{\protect\addvspace{10\p@}}}{}{}{}
        \patchcmd{\@chapter}{\addtocontents{lot}{\protect\addvspace{10\p@}}}{}{}{}
    \makeatother

%%%%%%%%%%%%%%%%%%%%%%%%%%%%%%%%%%%%%%%%%%%%%%%%%%%%%%%%%%%%%%%%%%%%%%%%%%%%%%%%

% Referência cruzada e links

    \usepackage[hidelinks]{hyperref} % Links no documento

    \usepackage{nameref} % Referenciar entidades por nome

    \usepackage{titleref} % Referenciar títulos

    % Criação de lista de símbolos e acônimos

        \usepackage{acro}
            \acsetup{
                make-links = true,
                use-id-as-short = true,
                format/foreign = \emph,
                list/name={\centerline{Abreviaturas e Siglas}},
                list/template = longtable,
                templates/colspec={>{\bfseries}lp{.85\textwidth}}
            }

%%%%%%%%%%%%%%%%%%%%%%%%%%%%%%%%%%%%%%%%%%%%%%%%%%%%%%%%%%%%%%%%%%%%%%%%%%%%%%%%

% Configuração de contadores do documento

    \ifx\chapter\undefined\else % Artigo não tem chapter
        \ifx\letter\undefined % Carta não tem figure/table/equation
            \usepackage{chngcntr} % Muda os contadores de figuras, equações, etc
                \counterwithout{figure}{chapter} % Número de figura sem contar capítulo
                \counterwithout{table}{chapter} % Número de tabela sem contar capítulo
                \counterwithout{equation}{chapter} % Número de equação sem contar capítulo
        \fi
    \fi


%%%%%%%%%%%%%%%%%%%%%%%%%%%%%%%%%%%%%%%%%%%%%%%%%%%%%%%%%%%%%%%%%%%%%%%%%%%%%%%%

% controle de citação e referências bibliográfica

    \usepackage[
        backend=biber,
        style=ieee,
        citestyle=numeric,
        sorting=none,
        block=space
    ]{biblatex}
    % \usepackage[style=abnt-numeric, citestyle=numeric, sorting=none]{biblatex} %https://github.com/abntex/biblatex-abnt/issues/90

        \renewbibmacro*{name:andothers}{% Based on name:andothers from biblatex.def
            \ifboolexpr{%
                test {\ifnumequal{\value{listcount}}{\value{liststop}}}%
                and%
                test \ifmorenames%
            }{%
                \ifnumgreater{\value{liststop}}{1}%
                {\finalandcomma}%
                {}%
                \andothersdelim\bibstring[\emph]{andothers}%
            }{}%
        }

        % \appto\bibfont{\setlength{\emergencystretch}{.5em}} % Evitar wanings por espaçamento na bibliografia nos casos mais simples

%%%%%%%%%%%%%%%%%%%%%%%%%%%%%%%%%%%%%%%%%%%%%%%%%%%%%%%%%%%%%%%%%%%%%%%%%%%%%%%%

% Definições de macros especiais para capa e folha de rosto

    % Lista de nomes

        \newcommand{\name}[1]
        {
        	&{#1}\\
        }
        \newenvironment{names}[1]
        {
            \begin{table}[H]\flushleft
        		\begin{tabular}{>{\raggedleft}p{.45\linewidth} | >{\bf}p{.45\linewidth}}
        			\sf{#1}
        			}
                    	%args here
                    {
        		\end{tabular}
        	\end{table}
        }

    % Outras macros
        \providecommand{\keywords}[1]{\textbf{{Keywords:}} #1}
        \providecommand{\palavraschave}[1]{\textbf{{Palavras-chave:}} #1}

        \newcommand{\etal}{\emph{et al}.}
        \newcommand{\ie}{\emph{i}.\emph{e}.}
        \newcommand{\eg}{\emph{e}.\emph{g}.}

    \newcommand{\nomes}{}
    \newcommand{\grupo}{}
    \newcommand{\centro}{}
    \newcommand{\centroSigla}{}
    \newcommand{\disciplina}{}
    \newcommand{\codigoDisciplina}{}
    \newcommand{\titulo}{}
    \newcommand{\professor}{}
    \newcommand{\local}{}
    \newcommand{\data}{\number\year}
    \newcommand{\notaDeRosto}{}
    \newcommand{\agradecimentos}{}
    \newcommand{\cabecalho}
    {
        {\large%
        \textbf{UNIVERSIDADE FEDERAL DO ABC}}\\
        \expandafter\MakeUppercase\expandafter{\small\centro{}%
        \ifdefempty{\centro}{}{ - }%
        \centroSigla\\
        \codigoDisciplina{}%
        \ifdefempty{\codigoDisciplina}{}{ - }%
        \disciplina%
        }
    }

    \newcommand{\capaComLogo}{
        \begin{titlepage} \center
            \cabecalho

            \vspace{6em}

            \begin{center}
                \includegraphics[width=0.335\linewidth]{pictures/logo_ufabc}
            \end{center}

            \nomes
            \textbf{\grupo}

            \vfill

            {\large{\MakeUppercase{\textbf{\titulo}}}}

            \vspace{1.5cm}
    		\professor

    		\vfill
    		\vfill

    		{\local\\\data}
        \end{titlepage}
    }

    \newcommand{\capa}{
        \begin{titlepage} \center
            \cabecalho

            \vspace{6em}

            % \capaLogo

            \nomes

            \vfill

            {\large{\MakeUppercase{\textbf{\titulo}}}}

            % \\\vspace{1.5cm}
    		% \professor

    		\vfill
    		\vfill

    		{\local\\\data}
        \end{titlepage}
    }

    \newcommand{\folhaDeRosto}{
        \newpage
        \begin{titlepage}
    		\center
            \nomes
            % \vspace{5cm}
            \vfill
            {\large{\MakeUppercase{\textbf{\titulo}}}}
            \vfill
            \hfill\begin{minipage}{0.5\linewidth}\onehalfspacing
            \notaDeRosto
            \end{minipage}
            \vfill
    		{\local\\\data}
        \end{titlepage}
    }

    \newcommand{\folhaAgradecimentos}{%
        \newpage\pagestyle{clear}
        \chapter*{\centerline{Agradecimentos}}
        %
        \agradecimentos
    }

%%%%%%%%%%%%%%%%%%%%%%%%%%%%%%%%%%%%%%%%%%%%%%%%%%%%%%%%%%%%%%%%%%%%%%%%%%%%%%%%

% Configurações de estilos de páginas

\AtBeginDocument{%
    \renewcommand{\headrulewidth}{0pt}
    %
    \ifx\standalone\undefined
        \setlength{\headheight}{1.5em}
    \else
        \fancyhf{}
        \renewcommand{\headrulewidth}{0pt}
	    \pagestyle{plain}
    \fi
    %
    \ifx\letter\or\article\undefined
        \titleformat{\chapter}{\normalfont \titlefont \bfseries}{\chaptername\ \thechapter}{1em}{}
        % \titleformat{\chapter}{\normalfont \titlefont \bfseries}{\thechapter}{1em}{}
        %
        \renewcommand{\chaptermark}[1]{\markboth{#1}{}} %mostrar somente o nome do capítulo com \leftmark
    \fi
    %
    \pagestyle{fancy}
    \fancypagestyle{plain}{% Página padrão
        \renewcommand{\headrulewidth}{0pt}
	    % \pagenumbering{arabic}
    }
    \fancypagestyle{main}{% Página padrão
        \fancyhf{}
        \lhead{}
        \chead{}
        \rhead{}
        \lfoot{}
        \cfoot{\thepage}
        \rfoot{}
	    \pagenumbering{arabic}
        \renewcommand{\headrulewidth}{0pt}
    }
    %
    \fancypagestyle{toc}{% Página  do sumário
        \renewcommand{\headrulewidth}{0pt}
        \fancyhf{}
        \lhead{}
        \chead{}
        \rhead{}
        \lfoot{}
        \cfoot{\thepage}
        \rfoot{}
	    \pagenumbering{Roman}
    }
    %
    \fancypagestyle{clear}{% Página do sumário
        \fancyhf{}
        \lhead{}
        \chead{}
        \rhead{}
        \lfoot{}
        \cfoot{}
        \rfoot{}
	   % \pagenumbering{None}
    }
    %
    \fancypagestyle{letterCapitania}{% Página do modelo de carta
        \fancyhf{}
        \renewcommand{\headrulewidth}{0pt}
	    \setlength\headheight{70pt}
        \lhead{
            \textbf{Carta de Intenções}\\
            \textbf{\today}\\\vspace{1em}
            Chapa: \chapa
            \vfill}
        \chead{}
        \rhead{\includegraphics[height=0.925\headheight]{Templates/logo_rocket.eps}}
        \lfoot{}
        \cfoot{\thepage}
        \rfoot{}
	    \pagenumbering{arabic}
    }
}

%%%%%%%%%%%%%%%%%%%%%%%%%%%%%%%%%%%%%%%%%%%%%%%%%%%%%%%%%%%%%%%%%%%%%%%%%%%%%%%%